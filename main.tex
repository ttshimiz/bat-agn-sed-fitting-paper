% Paper to publish our results from the IR SED fitting of the
% BAT AGN.
% Basic LaTex structure from MNRAS template on Overleaf
\documentclass[fleqn, usenatbib]{mnras}
\usepackage{newtxtext,newtxmath}
% Depending on your LaTeX fonts installation, you might get better results with one of these:
%\usepackage{mathptmx}
%\usepackage{txfonts}

\usepackage[T1]{fontenc}
\usepackage{ae,aecompl}


%%%%% AUTHORS - PLACE YOUR OWN PACKAGES HERE %%%%%

% Only include extra packages if you really need them. Common packages are:
\usepackage{graphicx}	% Including figure files
\usepackage{amsmath}	% Advanced maths commands
\usepackage{amssymb}	% Extra maths symbols
\usepackage{amsfonts}
\usepackage[flushleft]{threeparttable}
\usepackage{booktabs}
\usepackage{pdflscape}
\usepackage{caption}
\usepackage{morefloats}
%%%%%%%%%%%%%%%%%%%%%%%%%%%%%%%%%%%%%%%%%%%%%%%%%%

%%%%% AUTHORS - PLACE YOUR OWN COMMANDS HERE %%%%%

% Please keep new commands to a minimum, and use \newcommand not \def to avoid
% overwriting existing commands. Example:
%\newcommand{\pcm}{\,cm$^{-2}$}	% per cm-squared
\newcommand{\herschel}{\emph{Herschel}}
\newcommand{\swift}{\textit{Swift}}
\newcommand{\msun}{M$_{\sun}$}
\newcommand{\mstar}{$M_{\mathrm{star}}$}
\newcommand{\lsun}{L$_{\sun}$}
\newcommand{\mdust}{$M_{\rm dust}$}
\newcommand{\tdust}{$T_{\rm dust}$}
%%%%%%%%%%%%%%%%%%%%%%%%%%%%%%%%%%%%%%%%%%%%%%%%%%

%%%%%%%%%%%%%%%%%%% TITLE PAGE %%%%%%%%%%%%%%%%%%%

% Title of the paper, and the short title which is used in the headers.
% Keep the title short and informative.
\title[\emph{Herschel}-BAT Sample: AGN-SF Connection]{\emph{Herschel} far-infrared photometry of the Swift Burst Alert
Telescope active galactic nuclei sample of the local universe--III. Global star-forming properties and the connection to nuclear activity\thanks{{\it Herschel} is an ESA space observatory with science instruments provided by European-led Principal Investigator consortia and with important participation from NASA.}}

% The list of authors, and the short list which is used in the headers.
% If you need two or more lines of authors, add an extra line using \newauthor
\author[T. T. Shimizu]{T. Taro Shimizu$^{1}$\thanks{Email: tshimizu@astro.umd.edu}, Richard F. Mushotzky$^1$, Marcio Mel\'endez$^1$, Michael J. Koss$^{2}$, \newauthor Amy J. Barger$^{3,4,5}$, and Lennox L. Cowie$^{5}$\\
$^{1}$Department of Astronomy, University of Maryland, College Park, MD 20742, USA\\
$^{2}$Institute for Astronomy, Department of Physics, ETH Zurich, Wolfgang-Pauli-Strasse 27, CH-8093 Zurich, Switzerland\\
$^{3}$Department of Astronomy, University of Wisconsin-Madison, 475 N. Charter Street, Madison, WI 53706, USA\\
$^{4}$Department of Physics and Astronomy, University of Hawaii, 2505 Correa Road, Honolulu, HI 96822, USA\\
$^{5}$Institute for Astronomy, University of Hawaii, 2680 Woodlawn Drive, Honolulu, HI 96822, USA}

% These dates will be filled out by the publisher
\date{Accepted XXX. Received YYY; in original form ZZZ}

% Enter the current year, for the copyright statements etc.
\pubyear{2016}

% Don't change these lines
\begin{document}
\label{firstpage}
\pagerange{\pageref{firstpage}--\pageref{lastpage}}
\maketitle

% Abstract of the paper
\begin{abstract}

\end{abstract}

% Select between one and six entries from the list of approved keywords.
% Don't make up new ones.
\begin{keywords}
galaxies: active -- galaxies: Seyfert -- infrared: galaxies -- galaxies: star formation -- galaxies: evolution 
\end{keywords}

%%%%%%%%%%%%%%%%%%%%%%%%%%%%%%%%%%%%%%%%%%%%%%%%%%

%%%%%%%%%%%%%%%%% BODY OF PAPER %%%%%%%%%%%%%%%%%%

\section{Introduction}
Active galactic nuclei (AGN), accreting supermassive black holes at the centers of galaxies, are energetically capable of affecting the large-scale evolution of their host galaxies. Indirect evidence for this occurring over the lifespan of a galaxy lies in the observed relationships between the supermassive black hole mass and various host galaxy properties such as the stellar velocity dispersion \citep{Magorrian:1998qv, Ferrarese:2000gf, Gebhardt:2000xy, Gultekin:2009ul, McConnell:2013fk} and bulge mass \citep{Marconi:2003ve, Haring:2004ly, McConnell:2013fk}. More detailed observations over a larger range in host galaxy properties reveals the connection between supermassive black hole growth and galaxy growth is likely more complicated \citep{Kormendy:2013fj} especially when considering bulgeless or pseudobulge galaxies. Nevertheless, these correlations have motivated many studies attempting to understand the coevolution of black holes and galaxies.

The prevailing theory linking the host galaxy and its supermassive black hole involves negative AGN feedback whereby the energy released by the AGN both shuts down star formation in the host galaxy as well as accretion onto the supermassive black hole \citep[see][for reviews]{Fabian:2012dp, Alexander:2012fr}. This results in a form of self-regulation where the growth of the galaxy through star formation is inherently linked to the growth of the supermassive black hole through accretion. Evidence backing this theory seemed to be found in the observation that the SFR density and accretion rate density of the universe evolve similarly with redshift  \citep{Boyle:1998vn, Franceschini:1999ys, Silverman:2008zl, Silverman:2009lq, Aird:2010fr, Merloni:2013zr}. Further, cosmological simulations of the universe appear to require some form of AGN feedback to be able to reproduce the high mass end of the $z\sim0$ galaxy luminosity function \citep[e.g][]{Croton:2006kx, Bower:2006gf}. 

Naturally, researchers turned to studies of star formation in AGN host galaxies looking for direct relationships between the strength of the AGN and SFR.  However, an issue in performing these studies involves the measurement of the host galaxy SFR. Emission line, ultra-violet (UV), and mid-infrared (MIR) indicators of the SFR that are extensively used for normal, non-AGN galaxies suffer from varying degrees of contamination in the presence of an AGN. This was solved with the advent of far-infrared (FIR; $\lambda = 40-500$ \micron) telescopes, including the \textit{Infrared Astronomical Satellite} (\textit{IRAS}), the \textit{Infrared Space Observatory} (\textit{ISO}), the \textit{Spitzer Space Telescope}, and most recently the \herschel{} \textit{Space Observatory} \citep{Pilbratt:2010rz}. FIR emission remains the ideal waveband to study star formation in AGN host galaxies for three reasons: (1) the rapidly declining spectral energy distribution (SED) associated with AGN heated dust longwards of 20 \micron{} \citep{Fritz:2006yq, Netzer:2007ve, Mullaney:2011yq, Shi:2014qr} ensures little AGN contamination (2) thermal emission of large grains heated by the ionizing emission of recently formed massive stars \citep{Lonsdale-Persson:1987uq, Devereux:1990yq} creates a strong FIR ``bump'' and provides a reliable measure of the recent (10--100 Myr) SFR and (3) FIR emission is free from extinction which can be a large uncertainty in the calculation of the SFR from UV and optical tracers. 

   

Even though the FIR provides a reliable SFR tracer, individual investigations into the connection between AGN activity and star-formation have produced conflicting and even contradictory results. Some studies \citep{Page:2012mz, Barger:2015ly} reported lowered SFRs for the highest luminosity AGN, indicative of a suppression of star formation due to AGN feedback. Others have found weak or flat relationships \citep{Diamond-Stanic:2012rw, Mullaney:2012gf, Rovilos:2012wd, Rosario:2012fr, Azadi:2015ve, Stanley:2015qy} especially at more moderate AGN luminosities. Finally, some find an overall positive correlation \citep{Lutz:2008ph, Netzer:2009lr, Rovilos:2012wd, Rosario:2012fr, Chen:2013uq, Dai:2015gf}. \citet{Rosario:2012fr} provided evidence that at low AGN luminosity the relationship is flat and only at higher luminosities a positive correlation emerges that matches \citet{Lutz:2008ph} and \citet{Netzer:2009lr} when major mergers begin to dominate and drive both enhanced star formation and AGN activity at the same time \citep{Sanders:1988fk, Hopkins:2006ys}. 

A large complicating factor is the vast difference in timescales for the two competing processes. \citet{Hickox:2014yq} reconciled many of the seemingly contradictory results by recognizing that short-term (compared to the $\sim$ 100 Myr timescale for star formation) AGN variability could wash out any long term relationship between AGN activity and star formation. Just by assuming a linear relationship between the AGN luminosity and SFR and a probability distribution for observing an instantaneous AGN luminosity, \citet{Hickox:2014yq} was able to match the redshift evolution and flattening at low AGN luminosity of the AGN-SFR relationship found in \citet{Rosario:2012fr}. 

Many of the previously mentioned studies possibly suffer from one or more problems that include small number statistics either due to a small area survey or low sensitivity or an inaccurate scaling from monochromatic luminosities to a total SFR. Further, \citet{Stanley:2015qy} showed the importance of including information contained in non-detections for analyzing the connection between AGN luminosity and SFR, ultimately finding a very flat relationship that agreed with the model of \citet{Hickox:2014yq}.

Beyond measurements of the SFR, extending the SED to the FIR provides a wealth of information on the host galaxy properties. The \herschel{} \textit{Space Observatory}  provided a great advancement in FIR observations which in the past was plagued by low sensitivity and poor angular resolution. With the SPIRE instrument, SEDs were extended into the submillimeter regime which are crucial for determining not only the overall IR luminosity but also the dominant dust temperatures and dust masses of galaxies. Dust masses, in particular, are quite uncertain without the inclusion of photometry longward of 100 \micron{} \citep{Gordon:2010ix}.

In this paper, we aim to test these findings on the AGN-star formation relationship in the low redshift universe. We have performed a \herschel{} survey of a sample of AGN from the \textit{Swift} Burst Alert Telescope 58 month catalog which selected sources over the entire sky in the 14--195 keV energy range. At this high of energy, we are nearly unbiased towards any host galaxy properties including SFR as well as obscuration below a hydrogen column density ($N_{\rm H}$) of 10$^{24}$ cm$^{-2}$. With 5 band imaging with \herschel{} along with archival photometry from the \textit{Wide-field Infrared Survey Explorer} (\textit{WISE}), we have constructed MIR-FIR SEDs for over 300 AGN. Because of the low-redshift nature of our sources, every AGN is easily able to be identified and there is no concern over source confusion. These SEDs not only allow us to calculate SFRs but also to test varying SED decomposition methods as models as well as measure other properties of the host galaxies such as the dust mass and dust temperature to compare with non-AGN samples. With a high detection rate ($>80$ per cent at 70, 160, and 250 \micron), our measured properties from the SEDs are well constrained, removing much of the uncertainty due to censoring.

The paper is organized as follows: in Section 2 we briefly describe the \herschel-BAT AGN sample; in Section 3 we detail our \textit{WISE} and \herschel{} datasets. Section 4 introduces our two non-AGN comparison samples. Section 5 outlines the three models and our fitting procedure, and Section 6 presents our results. Sections 6 and 7 discuss the implications of our results, compares with other studies and concludes the paper. Throughout, we use a cosmology with $H_{0}=70$ km s$^{-1}$ Mpc$^{-1}$, $\Omega_{\rm M}=0.3$, and $\Omega_{\Lambda} = 0.7$ to calculate luminosity distances from redshifts. 

\section{\herschel-BAT Sample}
Because \swift/BAT continuously monitors the entire sky in the energy range 14--195 keV for gamma ray bursts, it simultaneously provides an all-sky survey at ultra high X-ray energies. This allows for the creation of complete catalogs with increasing sensitivity the longer \swift/BAT remains in operations. Given the extreme environments necessary to produce strong 14--195 keV emission, the majority of sources in the \swift/BAT catalogues are AGN.

We chose 313 AGN from the parent sample of $\sim720$ AGN detected in the 58 month catalogue\footnote{\url{https://swift.gsfc.nasa.gov/results/bs58mon}} after imposing a redshift cutoff of $z<0.05$ and excluding Blazars and BL Lac objects to form our \herschel-BAT AGN sample. With a mean redshift of $\left<z\right> = 0.025$, our AGN sample provides a comprehensive view of the properties of AGN host galaxies in the local universe. Our selection at ultra high X-ray energies further removes biases and selection effects due to host galaxy contamination and obscuration \citep{Mushotzky:2004gf} that can influence samples at other wavelengths.   

The demographics of our sample are nearly evenly split between unabsorbed (Type I, 43 per cent) and absorbed AGN (Type II, 53 per cent) with the remaining 4 per cent (5 objects) either a Low-Ionization Nuclear Emission-line Region (LINER) or unclassified. For a complete listing of our sample with names, luminosity distances, redshifts, and AGN type, we point the reader to \citet{Melendez:2014yu} or \citet{Shimizu:2016qy}.

\section{Data}
\subsection{\herschel{} Photometry}
293 of the \herschel{}-BAT AGN were observed with \herschel{} as part of a Cycle 1 open time program (OT1\_rmushotz\_1, PI: Richard Mushotzky). The remaining 20 sources were part of the other programs with public archival data. \herschel{} observed all 313 AGN using both the Photoconductor Array Camera and Spectrometer \citep[PACS;][]{Poglitsch:2010fp} and Spectral and Photometric Imaging Receiver \citep[SPIRE;][]{Griffin:2010sf} producing images in five wavebands: 70, 160, 250, 350, and 500 \micron.

Detailed descriptions of the data reduction process and photometric flux extraction can be found in \citet{Melendez:2014yu} for PACS and \citet{Shimizu:2016qy} for SPIRE. The following is a short description of the flux extraction procedure. We measured fluxes at each waveband directly from the images using aperture photometry with a concentric annulus to define the local background. We applied aperture corrections for sources where we used a point source aperture as defined in the respective PACS and SPIRE data reduction guides. Fluxes for sources that were unresolved at all three wavebands in SPIRE were determined using the SPIRE Timeline Fitter within the \herschel{} Interactive Processing Environment. All fluxes have a signal-to-noise ratio of at least 5, otherwise we provided a 5$\sigma$ upper limit. 

\subsection{\textit{WISE} Photometry}
To extend the SEDs into the mid-infrared (MIR), we supplemented our \herschel{} data with archival \textit{Wide-field Infrared Survey Explorer} \citep[WISE;][]{Wright:2010fk} photometry. WISE performed a broadband all-sky survey at 3.4 (W1), 4.6 (W2), 12 (W3), and 22 (W3) \micron{} with angular resolution comparable to \herschel/PACS at 70 \micron{} for W1, W2, and W3 and 160 \micron{} for W4. We queried the AllWISE catalog through the NASA/IPAC Infrared Science Archive\footnote{\url{http://irsa.ipac.caltech.edu/frontpage/}} to search for coincident sources within 6". Counterparts were found for all but one AGN (Mrk 3) at every waveband. The catalog only contained W1 and W2 fluxes for Mrk 3 due to differences in the depth of coverage for W1/W2 and W3/W4.

The AllWISE catalog provides magnitudes determined using multiple extraction methods. We consider only the profile-fitting magnitudes (\textit{w}N\textit{mpro} where N is 1, 2, 3, or 4) and the elliptical aperture magnitudes (\textit{w}N\textit{gmag}). Profile-fitting magnitudes were determined by fitting the position dependent point spread function using deblending procedures when necessary to decompose overlapping sources. The \textit{w}N\textit{mpro} magnitudes therefore are only relevant for unresolved sources.

If a WISE source is associated with a source in the \textit{Two Micron All Sky Survey} (2MASS) Extended Source Catalog (XSC), then \textit{w}N\textit{gmag} magnitudes were also measured using an elliptical aperture with the same shape from the XSC and sizes scaled given the larger WISE beam. Thus, \textit{w}N\textit{gmag} magnitudes are more appropriate for extended sources. For details of all of the WISE magnitude measurements we point the reader to the All-Sky Release Explanatory Supplement\footnote{\url{http://wise2.ipac.caltech.edu/docs/release/allsky/expsup/}}.

Given the low-redshift nature of our sample, using only the \textit{w}N\textit{mpro} magnitudes would severely underestimate the flux for large extended sources. To decide which magnitude to include in the SED for each source, we used the reduced $\chi^{2}$ value (\textit{w}N\textit{rchi2}) from the profile-fitting. If \textit{w}N\textit{rchi2}$<3$ then we chose the \textit{w}N\textit{mpro} magnitude, otherwise \textit{w}N\textit{gmag} was chosen. 

%\subsection{\emph{Spitzer}/IRS spectra}
%To supplement the IR photometry, we searched the Cornell Atlas of Spitzer IRS Sources (CASSIS) (REFERENCE) for low-resolution Spitzer/IRS spectra. These MIR spectra span the 5-30 \micron{} wavelength range and can provide independent measures of the SFR and AGN fraction through the strength of emission features from different species such as polycyclic aromatic hydrocarbons (PAH), NeII, NeIII, and OIV. We can use the luminosity and equivalent width (EqW) of these features to test the reliability and accuracy of the properties from the broadband SED fitting. 
%
%We can also test whether AGN heating has an effect on the dust composition of galaxies. Some studies (REFERENCES) have suggested that the hard ionizing radiation from the AGN has destroyed PAHs while others report no effect from the AGN. With our large sample covering a range of AGN luminosity and an independent measure of the SFR, we can definitively determine whether the AGN has an effect on the PAH abundance of galaxies.
%
%We found 120/313 AGN on CASSIS and downloaded their reduced spectra using the default extraction method. Due to the different sizes of the Short-Low (SL) slit and Long-Low (LL) slit, the continuum of each can be offset mainly from extended emission not captured by the SL slit. Therefore spectral orders were stitched together by fitting a line to the ends of each order and scaling to match the overlapping regions. SL2 and LL2 were first matched and scaled to SL1 and LL1 respectively. Then the combined SL1/SL2 spectra was matched to the combined LL1/LL2 spectra to produce the final full Spitzer/IRS low resolution spectra. 

\section{Comparison Samples}
To test whether the AGN has any effect on the star-forming properties of their host galaxies, we need samples of galaxies which contain little evidence for nuclear activity. These samples also need to occupy the same redshift range to mitigate against evolutionary effects and have been observed with nearly the same instruments so the properties can be compared on an equal level. Therefore we chose to use the \herschel{} \textit{Reference Survey} \cite[HRS;][]{Boselli:2010fj} and the \textit{Key Insights on Nearby Galaxies: a Far-Infrared Survey with Herschel} \cite[KINGFISH;][]{Kennicutt:2011vn} galaxies as our comparison samples. In the following subsections we briefly describe these two surveys. For more detail, we direct the reader to the original survey publications.

\subsection{\herschel{} Reference Survey}
The HRS was a guaranteed time key \herschel{} program dedicated to studying the dust content of ``normal'' galaxies. The 323 galaxy sample is volume limited ($15< D < 25$ Mpc) to avoid distance effects and K-band flux limited to avoid selection effects due to dust and provide a representative population of local galaxies. The size of the HRS as well as the local nature make it an ideal sample to compare to the \herschel-BAT galaxies.

The HRS galaxies were imaged using both PACS and SPIRE, however the 100 \micron{} filter was used instead of the 70 \micron{} filter. \citet{Cortese:2014qq} and \citet{Ciesla:2012lq} measured the PACS and SPIRE flux densities respectively using similar techniques as the \herschel-BAT galaxies. WISE 12 and 22 \micron{} photometry for HRS were provided in \citet{Ciesla:2014qy}. The available data for the HRS SEDs are nearly identical as our sample. 

The only issue in comparing the HRS galaxies to the \herschel-BAT AGN concerns the stellar mass (\mstar) distribution. Because the near-infrared is most strongly effected by the mass of the older stellar population, the HRS K-band selection produces a \mstar{} distribution that is representative of the naturally occurring \mstar{} distribution. However, many recent studies have shown that AGN prefer high \mstar{} galaxies, a feature that is also found in the BAT AGN \citep{Koss:2011vn}. Figure~\ref{fig:mstar_dist} displays the \mstar{} distributions for the \herschel-BAT and comparison samples. The \herschel-BAT galaxies have an average \mstar{} of 10.6 \msun, whereas the average \mstar{} for HRS is 9.8 \msun, nearly a ten-fold difference. This will be important to account for in interpreting results from a comparison between the two samples.

\begin{figure}
\includegraphics[width=\columnwidth]{figures/stellar_mass_comp}
\caption{Histograms and Kernel Density Estimates (KDE) of the \mstar{} distribution for the BAT AGN (blue), HRS (green), and KINGFISH (red) samples. The BAT AGN probe a higher \mstar{} galaxy population than non-AGN samples. A color version is available in the online publication. \label{fig:mstar_dist}}
\end{figure}

\subsection{KINGFISH Sample}
While the HRS galaxies were selected homogeneously, the KINGFISH galaxies are a heterogeneous sample selected to span the broad range of star-forming properties and ISM environments from dwarf galaxies to massive star-forming galaxies. Only 61 galaxies comprise the sample, 57 of which were part of the \textit{Spitzer} Infrared Nearby Galaxies Survey \citep[SINGS;][]{Kennicutt:2003wk}.

The KINGFISH galaxies were imaged at every \herschel{} waveband (70, 100, 160, 250, 350, and 500 \micron) using aperture photometry with flux densities provided by \citet{Dale:2012dq}. To cover the MIR portion of the SED, we used the \textit{Spitzer}/IRAC and MIPS 8 and 24 \micron{} data from \citet{Dale:2007fk}. Because the MIR data is from the SINGS survey, this reduces the KINGFISH sample from 61 to 57 galaxies. 

Various miscellaneous properties about each KINGFISH galaxy, including \mstar, were compiled from \citet{Kennicutt:2011vn}. We removed all galaxies with $\log($\mstar$)<8.0$ \msun{} because all are dwarf galaxies that are not represented in the \herschel-BAT sample. The final \mstar{} distribution for the KINGFISH comparison sample is shown in Figure~\ref{fig:mstar_dist}. While the KINGFISH galaxies are located at slightly higher masses (mean $\log($\mstar$)=9.9$ \msun), the \herschel-BAT sample is still biased significantly towards higher stellar masses. 

%An advantage of the KINGFISH sample is the availability of \textit{Spitzer}/IRS low-resolution spectra which were analyzed by \citet{Smith:2007lr}. The MIR emission feature measurements will be important for comparing their fluxes and luminosities to the FIR SED measured properties and assessing the effect of the AGN. 

\section{The IR SEDs of $z=0$ AGN}
Before comparing and fitting, we begin with a qualitative look at the SEDs of the 313 AGN. Figure~\ref{fig:seds} shows the 3--500 \micron{} SED for every source in the \herschel-BAT sample. We display the W1 and W2 photometry for completeness even though these are not used in the SED fitting. Immediately noticeable is the varied SED shapes. Some of the SEDs (e.g. 2MASX J07595347+2323241, Cen A, ESO 005-G004, NGC 4051, NGC 4138, NGC 6814) feature a very prominent FIR bump that is recognizable in nearly all star-forming galaxies and is indicative of thermally heated dust from recently formed massive stars as well as cirrus emission heated by an older population. The general shape is usually well fit by a blackbody modified by a frequency dependent optical depth \citep[e.g.][]{Calzetti:2000fk, Smith:2012fj, Bianchi:2013jk, Symeonidis:2013fe, Cortese:2014qq}.

These star formation dominated objects represent one extreme end of the shapes we observe. The other extreme are SEDs whose shapes seem to peak short-ward of 70 \micron{} and display rapidly falling emission with increasing wavelength. Examples include 2MASX J06561197-4919499, 2MASX J210990996-0940147, ESO 103-035, IC 4329A, MCG -05-23-016, and Mrk 335. Emission in the MIR dominates these SEDs and is likely associated with hotter dust heated by the AGN. MIR colors are commonly used to select AGN samples because they display strong red colors compared to non-AGN galaxies \citep{Lacy:2004uq, Donley:2012qy, Stern:2012mz} due to the dust in the obscuring torus heated by the optical and UV emission form the accretion disk. Monochromatic MIR luminosities, especially for sources dominated by an unresolved central component, show near linear correlations with the X-ray luminosity \citep[e.g][]{Lutz:2004fj, Gandhi:2009kx, Asmus:2012qy} providing more evidence that much of the MIR in AGN host galaxies is associated with the AGN. 

The remaining \herschel-BAT AGN display SEDs somewhere in between MIR-dominated and FIR-dominated as a result of competing contributions between the AGN and star formation. The varied shapes of the SED emphasizes the need for SED decomposition to accurately determine the IR luminosity associated with either star formation or the AGN. 

If star formation suppression occurs at high AGN luminosity then we expect the IR SEDs for the most X-ray luminous sources to resemble the MIR-dominated SEDs just as \citet{Barger:2015ly} observed. To test this, we binned the SEDs by logarithmic 14--195 keV luminosity. Five bins ( $\log\,L_{\rm 14-195\,keV} > 42.5$, $\log\,L_{\rm 14-195\,keV} = 42.5-43.0$, $43.0-43.5$, $43.5-44.0$, and $\log\,L_{\rm 14-195\,keV} > 44.0$) were chosen with 0.5 dex widths to ensure enough sources occupied each one. The number of sources in each bin is 23, 39, 94, 116, and 34. 

We then calculated the median normalized flux within each luminosity bin where we have normalized the SED using either the 12 \micron{} or 160 \micron{} flux. We used the ASURV \citep{Feigelson:1985lr} package which applies the principles of survival analysis to astronomical data. Information contained in the upper limits can be then included in the measurement of statistical properties of samples without biasing results towards brighter sources that are more likely to be detected. Specifically, ASURV calculates the non-parametric Kaplan-Meier product-limit (KMPL) estimator for a sample distribution. The KMPL estimator is an estimate of the survival function which is simply 1-CDF (cumulative distribution function). Using the KMPL, we calculate for each waveband the median normalized flux ratio and estimate the uncertainty using the 16th and 84th percentiles. 

Figure~\ref{fig:avg_sed} shows the average IR SED for sources in each X-ray luminosity bin. The SEDs in the left panel are normalized to the 12 \micron{} flux and the SEDs in the right panel are normalized to the 160 \micron{} flux. The reason for doing this process twice is to look at the change in the SED with $L_{\rm 14-195\,keV}$ when it is normalized to a waveband that is highly correlated with $L_{\rm 14-195\,keV}$ and a waveband that is more correlated with the SFR.

For 12 \micron{} flux normalized SEDs, the FIR wavebands change the most compared to the MIR ones. There is more and more relative emission in the MIR with increasing AGN luminosity, as expected. Because we have normalized by 12 \micron{}, which closely follows $L_{\rm 14-195\,keV}$, it is not surprising that for the wavebands more closely related to star formation, the SED decreases. As the AGN luminosity increases, 12 \micron{} will increase along with it, reducing the ratios in the bands not related to the AGN. So while the left panel seems to indicate that at higher AGN luminosities, star formation is being suppressed, it is really the normalization we have used that is causing this effect.

Instead in the right panel, we have normalized the SEDs with the 160 \micron{} flux, which tracks closest to star formation. Now we see as AGN luminosity is increased, there is no longer any change in the FIR wavebands, but large increases in the MIR portion of the SED, indicating the AGN component is becoming more dominant. The effect can also be seen at the very shortest wavelengths, where for the lowest three X-ray luminosity bins, the 3.4/4.6 color is blue similar to non-AGN galaxies, whereas at the highest bins, the 3.4/4.6 color is red indicative of an AGN-dominated SED \citep{Stern:2012mz}. 

Just by inspecting the average SED as a function of X-ray luminosity, it seems the shape is being driven more by an increasing MIR hot dust component. There does not seem to be any indication that star formation is being suppressed at the highest luminosities. However, to definitively test this we need to decompose the SED into star-forming and AGN components to calculate accurate SFRs.

\begin{figure*}
\centering
\includegraphics[width=\textwidth]{figures/avg_sed_binned_lx}
\caption{Average IR SEDs for the \herschel-BAT AGN binned by $L_{\rm 14-195\,keV}$. In the left panel we normalized the SEDs by their 12 \micron{} flux while in the right panel we normalized by 160 \micron{} flux. \label{fig:avg_sed}}
\end{figure*}

\section{SED Fitting}

%%%%%%%%%%%%%%%%%%%%% TABLE OF C12 PARAMETERS %%%%%%%%%%%%%%%%%%%%%%%%%%
%\begin{landscape}
\begin{table*}
\centering
\begin{threeparttable}
\captionsetup{font=small,labelfont=bf,labelsep=period}
\caption{Best Fit C12 Model Parameters, Luminosities, and AGN Fractions \label{tab:c12_params}}
\begin{tabular}{lcccccccc}
\toprule 
Name & $\log\,M_{\rm dust}$ &$T_{\rm dust}$ & $\alpha$ & $\lambda_{\rm c}$ & $\log\,L_{\rm IR}$ & $\log\,L_{\rm SF}$ & $\log\,L_{\rm AGN,IR}$ &$f_{\rm AGN}$ \\
 & [\msun] & [K] &  & [\micron] & [\lsun] & [\lsun] & [\lsun] & \\
\midrule
1RXSJ044154.5-082639 & $6.80_{-0.14}^{+0.21}$ & $27.98_{-4.21}^{+2.45}$ & $1.70_{-0.34}^{+0.46}$ &$47.71_{-13.96}^{+18.85}$ & $10.37_{-0.04}^{+0.03}$ & $9.99_{-0.23}^{+0.11}$ & $9.99_{-0.23}^{+0.11}$ & $0.58_{-0.14}^{+0.18}$ \\
1RXSJ045205.0+493248 & $7.47_{-0.10}^{+0.12}$ & $23.27_{-1.61}^{+0.99}$ & $1.44_{-0.46}^{+0.59}$ &$39.95_{-13.09}^{+14.82}$ & $10.48_{-0.04}^{+0.03}$ & $10.19_{-0.08}^{+0.04}$ & $10.19_{-0.08}^{+0.04}$ & $0.50_{-0.10}^{+0.10}$ \\
2E1739.1-1210 & $7.55_{-0.19}^{+0.22}$ & $25.11_{-3.00}^{+2.11}$ & $1.46_{-0.37}^{+0.51}$ &$45.22_{-14.16}^{+16.42}$ & $10.83_{-0.04}^{+0.03}$ & $10.46_{-0.14}^{+0.07}$ & $10.46_{-0.14}^{+0.07}$ & $0.58_{-0.11}^{+0.12}$ \\
2MASSJ07594181-3843560 & $<6.58$ & ... & $1.08_{-0.46}^{+0.51}$ &$39.20_{-5.56}^{+7.93}$ & $<10.63$ & $<9.63$ & $>10.59$ & $>0.90$ \\
2MASSJ17485512-3254521 & $<6.14$ & ... & $1.45_{-0.47}^{+0.56}$ &$44.52_{-16.37}^{+18.07}$ & $<9.42$ & $<9.05$ & $>8.99$ & $>0.51$ \\
2MASXJ00253292+6821442 & $6.16_{-0.18}^{+0.39}$ & $25.34_{-5.06}^{+2.79}$ & $1.46_{-0.40}^{+0.53}$ &$45.40_{-15.93}^{+16.38}$ & $9.63_{-0.05}^{+0.04}$ & $9.11_{-0.18}^{+0.10}$ & $9.11_{-0.18}^{+0.10}$ & $0.70_{-0.11}^{+0.11}$ \\
2MASXJ01064523+0638015 & $6.81_{-0.44}^{+0.69}$ & $21.56_{-7.50}^{+9.09}$ & $1.86_{-0.42}^{+0.53}$ &$44.58_{-9.10}^{+10.57}$ & $10.47_{-0.05}^{+0.04}$ & $<9.95$ & $>10.24$ & $>0.66$ \\
2MASXJ01073963-1139117 & $7.66_{-0.06}^{+0.09}$ & $25.52_{-1.96}^{+1.27}$ & $2.05_{-0.49}^{+0.58}$ &$41.53_{-10.35}^{+20.19}$ & $10.87_{-0.03}^{+0.03}$ & $10.62_{-0.10}^{+0.07}$ & $10.62_{-0.10}^{+0.07}$ & $0.44_{-0.12}^{+0.13}$ \\
2MASXJ03305218+0538253 & $6.76_{-0.33}^{+0.69}$ & $28.20_{-8.76}^{+6.57}$ & $2.46_{-0.61}^{+0.78}$ &$33.77_{-9.67}^{+8.47}$ & $10.81_{-0.06}^{+0.06}$ & $9.99_{-0.28}^{+0.23}$ & $9.99_{-0.28}^{+0.23}$ & $0.85_{-0.13}^{+0.10}$ \\
2MASXJ03342453-1513402 & $7.43_{-0.04}^{+0.05}$ & $26.61_{-1.09}^{+0.61}$ & $1.69_{-0.44}^{+0.58}$ &$43.00_{-16.10}^{+24.14}$ & $10.59_{-0.03}^{+0.03}$ & $10.51_{-0.05}^{+0.03}$ & $10.51_{-0.05}^{+0.03}$ & $0.18_{-0.10}^{+0.13}$ \\\bottomrule
\end{tabular}
\begin{tablenotes}
\item Notes -- \textit{Column 1:} Name of the source. \textit{Column 2:} Log of the dust mass in solar units. \textit{Column 3:} Dust temperature for the MBB component. '...' indicates the dust temperature was fixed at 23 K. \textit{Column 4:} Slope of the powerlaw component. \textit{Column 5:} Turnover wavelength. \textit{Column 6:} Log of the total infrared luminosity from 8--1000 \micron{} in solar units. \textit{Column 7:} Log of the MBB component luminosity in solar units. \textit{Column 8:} Log of the powerlaw component luminosity in solar units. \textit{Column 9:} Fractional contribution of the AGN to the total infrared luminosity calculated using Equations~\ref{eq:c12_lagnIR}~and~\ref{eq:c12_fagn}. The full version of this table is available in the online publication.
\end{tablenotes}
\end{threeparttable}
\end{table*}
%\end{landscape}
%%%%%%%%%%%%%%%%%%%%%%%%%%%%%%%%%%%%%%%%%%%%%%%%%%%%%%%%%%%%%%%%

Many models and templates exist in the literature to fit the broadband SEDs of galaxies. We chose a model that allowed us to both decompose the SED into star-forming and AGN components as well as provide estimates on the dust temperature and dust mass.

\subsection{Casey 2012 Model}
One of the most widely used models for fitting the FIR SED of galaxies is a single modified blackbody (MBB). The simple model consists of a normal, single temperature blackbody that represents isotropic dust emission combined with a frequency dependent opacity given that dust is not a perfect blackbody.In the optically thin limit, the opacity can be approximated as a powerlaw, $\tau_{\nu}=(\nu/\nu_{0})^{\beta}$. The form of the single modified blackbody for the flux density at each frequency is then

\begin{equation}
S(\nu) \propto \nu^{\beta}B_{\nu}(T_{\mathrm{d}})
\end{equation}

\noindent where $\beta$ is the spectral emissivity index and $B_{\nu}(T_{\mathrm{d}})$ is the standard Planck blackbody function for an object with temperature $T_{\mathrm{d}}$. This simple model has been shown to fit well the prominent FIR bumps for large samples of star-forming galaxies  and provides estimates of the dust temperature, dust mass, and SFR \cite[e.g.][]{Calzetti:2000fk, Bianchi:2013jk, Cortese:2014qq}.

To calculate the dust mass, we must assume a particular dust absorption coefficient, $\kappa_{0}$ at a particular frequency, $\nu_{0}$. For this work, we assume $\kappa_{0}=0.192\,\mathrm{m^{2}\,kg^{-1}}$ and $\nu_{0}=857\,\mathrm{GHz}$ (i.e. 350 \micron) from \citet{Draine:2003gd}. However, as \citet{Bianchi:2013jk} shows, by assuming a specific $\kappa_{0}$, we must also fix the spectral emissivity index to the value used to measure $\kappa_{0}$. In this work, we fix $\beta=2.0$ to match the spectral emissivity index used by \citet{Draine:2003gd}. The final full form of the single MBB model is then

\begin{equation}\label{eq:greybody}
S_{\mathrm{MBB}}(\nu) = \frac{M_{\mathrm{d}}\kappa_{0}}{D_{\mathrm{L}}^2}\left(\frac{\nu}{\nu_{0}}\right)^{\beta}\frac{2h\nu^{3}}{c^{2}}\frac{1}{e^{{h\nu/kT_{\mathrm{d}}}}-1}
\end{equation}

\noindent where $M_{\mathrm{d}}$ is the dust mass, $D_{\mathrm{L}}$ is the luminosity distance, $c$ is the speed of light, $h$ is the Planck constant, and $k$ is the Boltzmann constant. The two free parameters then are $M_{\mathrm{d}}$ and $T_{\mathrm{d}}$, the dust mass and dust temperature respectively.

The simple assumption that dust emission in the IR can be modeled with a single temperature blackbody works well for ``normal'' star-forming galaxies. However for galaxies with large amounts of hot dust either due to a compact starburst or central AGN, this assumption can quickly break down. To account for this hot dust we also fit our sample using the model described in \citet[][hereafter C12]{Casey:2012jl} which is the combination of a single MBB and an exponentially cutoff powerlaw. The C12 model takes the form

\begin{equation}\label{eq:casey}
S_{\mathrm{C12}}(\nu) = N_{\mathrm{pl}}\left(\frac{\nu}{\nu_{\mathrm{c}}}\right)^{-\alpha}e^{-\nu_{\mathrm{c}}/\nu}S_{\mathrm{MBB}}(\nu)
\end{equation}

\noindent where $\nu_{\mathrm{c}}$ represents the turnover frequency and $N_{\mathrm{PL}}$ is a normalization constant. C12 illustrated using the \textit{Great Origins All-Sky LIRG Survey} (GOALS) sample that this model provides better estimates of the cold dust temperature, dust mass, and IR luminosity compared to both a single temperature modified blackbody and template libraries.

The C12 model introduces three more free parameters ($N_{\mathrm{PL}}$, $\alpha$, and $\nu_{\mathrm{c}}$), however, within the implementation used by C12, $N_{\mathrm{PL}}$ and $\nu_{\mathrm{c}}$ are tied to the normalization of the modified blackbody component and dust temperature to produce a smoothly varying SED and reduce the number of free parameters from five to three. 

But, after early tests using this initial setup, we found that fixing $N_{\mathrm{PL}}$ and $\nu_{\mathrm{c}}$ as a function of the other parameters produced unreliable fits. This is because AGN SEDs from the MIR to FIR are not as smooth as those seen in (U)LIRGS, likely due to the disconnect between star-formation and AGN heating. Within starbursting galaxies both the hot and cold dust are related through the same heating process, i.e. star formation, while the much of the MIR emission in AGN host galaxies originates from dust around the AGN with no strong connection to global star formation in the galaxy. Therefore, we chose to leave both $N_{\mathrm{PL}}$ and $\nu_{\mathrm{c}}$ as free parameters resulting in a total of five for the entire model.

We describe in detail within Appendix~\ref{sec:sed_bayes} the exact fitting method used to estimate each parameter. In short we use a Bayesian framework with a Monte Carlo Markov Chain (MCMC) to probe the posterior probability distribution function (PDF). We marginalize the resulting posterior PDF for each parameter and use the median as our best-fit value and the 16th and 84th percentiles to define our lower and upper uncertainties, respectively. 

We fit all 313 of the \herschel-BAT AGN using the C12 model and report the best fitting parameters in Table~\ref{tab:c12_params}. We also fit the both the HRS and KINGFISH galaxies using the exact same model so we can accurately compare the star-forming properties between non-AGN and AGN host galaxies. 

\subsection{Luminosities and AGN Fractions}
In addition to the parameters associated with the model, we also calculated several luminosities and an AGN fractions. For the this work, we define three luminosities: 1) $L_{\rm IR}$ will represent the total infrared luminosity determined by integrating the full SED from 8--1000 \micron. 2) $L_{\rm SF}$ will represent the 8--1000 \micron{} luminosity due to star formation in the host galaxy. 3) $L_{\rm AGN, IR}$ will represent the 8--1000 \micron luminosity due to the AGN-heated dust. 

For the C12 model, we also measured $L_{\rm IR}$ by integrating the best fit total SED from 8--1000 \micron. However, we have decomposed the SED into two components: the powerlaw (PL) and MBB component. If we assumed the MIR powerlaw component is completely dominated by the AGN for all sources then we could just take the ratio of the luminosities. This is not the case though as some sources have much of their MIR emission originating in the host galaxy. Therefore we must correct for host galaxy MIR emission associated with star formation.

We make this correction using the results of the C12 modeling for the HRS and KINGFISH samples. Since all of the galaxies in these samples have either no or low-luminosity AGN, the MIR emission is primarily the result of stochastically heated grains near star-forming regions. In Figure~\ref{fig:lmbb_lpl_ratio}, we show the distribution of $L_{\rm MBB}/L_{\rm PL}$, the ratio of the MBB component luminosity to the MIR powerlaw component. The ratios for the HRS and KINGFISH samples are both narrowly distributed around a single value, indicating that for non-AGN galaxies the energy contained in the PL and MBB components are tightly connected. Combining both the HRS and KINGFISH samples, we find a median $\log(L_{\rm MBB}/L_{\rm PL})=0.48 \pm 0.11$, which transforms to $L_{\rm PL,SF} \approx \frac{1}{3}L_{\rm MBB}$ where we now indicate the contribution to the PL component from star formation as  $L_{\rm PL,SF}$. We can calculate the AGN contribution for the \herschel-BAT AGN then by assuming the star-forming emission follows the same ratio.

\begin{align}
L_{\rm AGN,\,IR} &= L_{\rm PL} - L_{\rm PL, SF} = L_{\rm PL} - \frac{1}{3}L_{\rm MBB} \label{eq:c12_lagnIR} \\
f_{\rm AGN} &= L_{\rm AGN, \, IR}/L_{\rm IR} \label{eq:c12_fagn} \\
L_{\rm SF} &= L_{\rm MBB} + L_{\rm PL,SF} =  \frac{4}{3}L_{\rm MBB} \label{eq:c12_lsf}
\end{align}

 The uncertainty on the correction factor leads to an uncertainty on $f_{\rm AGN}$ using this method which we estimate by measuring $f_{\rm AGN}$ for the combined HRS and KINGFISH sample. Figure~\ref{fig:fagn_nonAGN} shows the distribution of $f_{\rm AGN}$ for our non-AGN sample. As expected the distribution is centered around 0. We estimate the spread of the distribution by calculating the standard deviation, finding an uncertainty of 0.1. We apply 0.1 as the minimum error on $f_{\rm AGN}$  as well as the minimum $f_{\rm AGN}$ we can measure. Any $f_{\rm AGN}$ below 0.1 are converted to upper limits with a value of 0.1. Further we also assume that 0.9 is the maximum $f_{\rm AGN}$ able to be detected so any measured $f_{\rm AGN}$ above 0.9 are converted to lower limits. 

\begin{figure}
\includegraphics[width=\columnwidth]{figures/lmbb-lpl-ratio}
\caption{Histograms and Kernel Density Estimates (KDE) of the $L_{\rm MBB}/L_{\rm PL}$, distribution for the BAT AGN (blue), HRS (green), and KINGFISH (red) samples. The non-AGN samples have a narrowly distributed ratio whereas the BAT AGN span a wide range due to the AGN contribution. A color version of this figure is available in the online publication. \label{fig:lmbb_lpl_ratio}}
\end{figure}

\begin{figure}
\includegraphics[width=\columnwidth]{figures/fagn_nonAGN}
\caption{$f_{\rm AGN}$ distribution for the HRS and KINGFISH galaxies. The standard deviation of this distribution quantifies the uncertainty on $f_{\rm AGN}$ associated with the correction factor used to calculate $f_{\rm AGN}$. A color version of this figure is available in the online publication. \label{fig:fagn_nonAGN}}
\end{figure}

\section{Results and Discussion}
\subsection{$M_{\rm dust}$, $T_{\rm dust}$, and SFR of local AGN host galaxies}

%%%%%%%%%%%%%%%%%%%%%%%% FIGURE: Mdust, Tdust, SFR COMPARISON %%%%%%%%%%%
\begin{figure*}
\includegraphics[width=\textwidth]{figures/mdust_tdust_sfr_comparison}
\caption{ Comparison of $M_{\rm dust}$, $T_{\rm dust}$, and SFR between the \herschel-BAT AGN (blue), HRS (green), and KINGFISH (red) samples. The top row compares the absolute values of these properties while the bottom row compares the properties normalized by $M_{\rm star}$. A color version of this figure is available in the online publication. \label{fig:mdust_tdust_sfr_comp}}
\end{figure*}
%%%%%%%%%%%%%%%%%%%%%%%%%%%%%%%%%%%%%%%%%%%%%%%%%%%%%%%%%

%%%%%%%%%%%%%%%%%%%%%%% TABLE: Mdust, Tdust, SFR Comparison %%%%%%%%%%%%%
\begin{table*}
\centering
\begin{threeparttable}
\captionsetup{font=small,labelfont=bf,labelsep=period}
\caption{Mean \mdust, \tdust, and SFR \label{tab:mean_mdust_tdust_sfr}}
\begin{tabular}{lccccccccc}
\toprule 
Sample & $\log$ \mdust & \tdust & $\log$ SFR & $\log$ \mdust/\mstar & \tdust/$\log$ \mstar & $\log$ SFR/\mstar \\
 & [\msun]  & [K]  &  [\msun{} yr$^{-1}$] &    &   & [yr$^{-1}$]\\
\midrule
\herschel-BAT & 7.20$\pm$0.05 & 23.5$\pm$0.3 & 0.11$\pm$0.04 & -3.22$\pm$0.05 & 2.21$\pm$0.03 & -10.35$\pm$0.06 \\ 
HRS & 6.70$\pm$0.04 & 21.5$\pm$0.2 & -0.58$\pm$0.04 & -3.14$\pm$0.07 & 2.21$\pm$0.02 & -10.40$\pm$0.05 \\
KINGFISH & 6.80$\pm$0.10 & 21.7$\pm$0.5 & -0.40$\pm$0.10 & -2.60$\pm$0.10 & 2.25$\pm$0.05 & -9.90$\pm$0.10 \\
\bottomrule
\end{tabular}
\begin{tablenotes}
\item blah blah blah
\end{tablenotes}
\end{threeparttable}
\end{table*}
%%%%%%%%%%%%%%%%%%%%%%%%%%%%%%%%%%%%%%%%%%%%%%%%%%%%%%%%%

Figure~\ref{fig:mdust_tdust_sfr_comp} compares the distribution of \mdust, \tdust, and SFR for our three samples of galaxies. Table~\ref{tab:mean_mdust_tdust_sfr} displays the mean and standard error on the mean for these three properties and each sample. We calculated the SFR using the conversion from 8--1000 \micron{} star-forming luminosity provided in  \citet{Murphy:2011rt}:

\begin{equation}\label{eq:sfr_ir}
\rm{SFR} = \frac{L_{\rm SF}}{2.57\times10^{43}}
\end{equation}

While sources with only upper limits are not shown in Figure~\ref{fig:mdust_tdust_sfr_comp}, we incorporate them into our calculation of the mean. 

From just a comparison of \mdust, \tdust, and SFR, we find that the \herschel-BAT AGN have higher dust masses than the HRS sample, slightly higher dust temperatures than both non-AGN samples, and higher SFRs especially compared to HRS. Their dust mass seems to be the same as the KINGFISH sample. However, as we show in Figure~\ref{fig:mstar_dist}, each sample has differing \mstar{} distributions which could cause wrong interpretations when comparing properties that are correlated with \mstar. All three of \mdust, \tdust, and SFR are strongly correlated with \mstar, so we must normalize the three properties to account trends with it.

In the lower row of Figure~\ref{fig:mdust_tdust_sfr_comp} and last three columns of Table~\ref{tab:mean_mdust_tdust_sfr} we show the distribution and mean values of \mdust, \tdust, and SFR after normalization to \mstar. The results change significantly after accounting for the high \mstar{} distribution of the \herschel-BAT AGN. Whereas before the \herschel-BAT AGN seemed to have higher \mdust, \tdust, and SFR compared to the HRS and KINGFISH samples, normalized to \mstar, they now are either equal to or lower than the non-AGN samples. All three samples now have nearly the same \tdust{} distribution, indicating the increased dust temperatures were simply a result of the increased stellar masses. There is no evidence for the AGN heating large scale dust in the galaxy to higher temperatures. 

Particularly interesting is the significant difference between the KINGFISH and \herschel-BAT AGN. The HRS sample was mass selected so it includes both star-forming and quiescent galaxies while the KINGFISH sample was mainly selected as a star-forming sample. Compared to star-forming galaxies, the AGN host galaxies have on average a 0.6 dex (factor of 4) decrease in their dust-to-stellar mass ratio and a 0.45 dex (factor of 3) decrease in their specific SFR (sSFR = SFR/\mstar). In \citet{Shimizu:2015xo}, we extensively examined whether our AGN sample showed decreased specific SFR compared to the so-called ``Main Sequence'' of star formation. We concluded, using a multitude of comparison samples that AGN host galaxies do exhibit reduced levels of star formation for their given stellar mass, a possible indication that AGN feedback has occurred in these galaxies. 
 
In this Paper, we find that not only is the specific SFR decreased in AGN host galaxies, but \mdust{} is also decreased. Assuming a constant dust-to-gas ratio, a reduced \mdust{} would indicate AGN host galaxies on average are gas deficient compared to star-forming galaxies. This would explain the reduced star formation since neutral gas provides the material for molecular gas from which stars form. There are two ways the AGN could have played a role in reducing the gas fraction of the galaxy. One is through physically blowing the gas out with AGN-driven winds. The other is by heating the halo and shutting off accretion from the intergalactic medium. Discerning between the two processes however is beyond the scope of this paper.

To determine whether the \herschel-BAT AGN properties are similar to the HRS and KINGFISH galaxies, we run the Peto-Prentice two-sample test. The Peto-Prentice (P-P) two sample test is comparable to the Kolmogorov-Smirnov (K-S) test which calculates the probability that two separate samples were drawn from the same parent population. The P-P test, however, allows for the inclusion of censored data. 

We input \mstar{} normalized properties for the \herschel-BAT AGN and run tests to compare the distribution with both the HRS and KINGFISH galaxies. For \mdust/\mstar{}, we find probabilities $<<0.01$ of the \herschel-BAT AGN being drawn from the same distribution as the HRS and KINGFISH galaxies. In contrast, we find probabilities of 0.99 and 0.66 that the \herschel-BAT AGN \tdust/\mstar{} distribution is the same as the one for the HRS and KINGFISH samples, respectively. Finally, we find probabilities of 0.20 and $<<0.01$ comparing the sSFR distributions between the \herschel-BAT AGN and HRS and KINGFISH samples. The standard cutoff probability to declare two samples being drawn from the same parent distribution is 0.05. 

A probability of 0.20 between the sSFR distributions of the HRS and \herschel-BAT AGN indicates they are drawn from the same distribtution. However, the HRS sample contains galaxies that have been stripped of their gas due to infall into a cluster. The ram-pressure stripping likely quenched SFR within the galaxy. If we remove these galaxies by only selecting gas-rich HRS galaxies, the probability drops to $<<0.01$, the same as the comparison between the \herschel-BAT AGN and the KINGFISH galaxies.

Our survival analysis two-sample tests confirm what our previous visual and basic statistical analysis suggested. The \mdust/\mstar{} and sSFR distributions of AGN host galaxies are significantly different than those of normal star-forming galaxies while the \tdust{} distributions are identical after accounting for \mstar. 

\subsection{IR colors as a predictor of $f_{\rm AGN}$}

\begin{figure*}
\includegraphics[width=\textwidth]{figures/agnfrac_vs_flux_ratio}
\caption{\label{fig:agnfrac_vs_flux_ratio}}
\end{figure*}

We utilize our broad coverage of the IR SED for AGN host galaxies to test the viability of different IR colors for predicting the AGN IR luminosity fraction (i.e. $f_{\rm AGN}$). For large surveys and/or high redshift studies, coverage from 12--500 \micron{} might not be available or possible.

In Figure~\ref{fig:agnfrac_vs_flux_ratio}, we plot six MIR/FIR flux ratios (12/70, 12/160, 12/250, 22/70, 22/160, 22/250) against the measured $f_{\rm AGN}$ for the \herschel-BAT AGN. The flux ratios are all in log units to better visualize the relationships and sources with upper or lower limits on $f_{\rm AGN}$ are not shown for clarity. We fit a linear model between each flux ratio and $f_{\rm AGN}$ with the following form

\begin{equation}
f_{\rm AGN} = m\log(F_{1}/F_{2}) + b + \epsilon_{\rm int}
\end{equation}

\noindent where $F_{1}/F_{2}$ represents each flux ratio. To calculate the best fitting parameters, we used a \textsc{python} implementation of \textsc{linmix\_err}\footnote{\url{https://github.com/jmeyers314/linmix}} \citep{Kelly:2007lr}, a Bayesian method for linear regression that incorporates errors in both independent and dependent variables as well as non-detections. \citet{Kelly:2007lr} showed through simulations that \textsc{linmix\_err} outperforms other popular linear regression methods such as ordinary least squares and \textsc{fit\_exy}. \textsc{linmix\_err} outputs marginalized posterior probability distributions for the slope, intercept, and intrinsic scatter between the two variables as well as the correlation coefficient. Our reported best fit parameters will be the median of the marginalized posterior probability distribution with an uncertainty equal to the standard deviation.

Table~\ref{tab:flux_ratio_fagn_linreg} outlines the best fit parameters for each relationship. While each flux ratio is highly correlated with $f_{\rm AGN}$ (all $\rho > 0.75$), the best predictor of $f_{\rm AGN}$ is the 22/160 ratio with, $\rho = 0.97$. However, both flux ratios involving the 160 \micron{} flux seem to perform equally well with similar slopes, small intrinsic scatter and high $\rho$. The worst predictors are the flux ratios involving the 250 \micron{} flux with $\rho$ only equal to 0.75. 

The ratios involving the 160 \micron{} flux work better as a predictor because the AGN component of the IR SED rarely contributes any emission to the total flux of the galaxy compared to the star-forming component. While this is also true of the 250 \micron{} flux, the 160 \micron{} flux is nearer to the peak of the MBB component, thus, more closely related to the total luminosity. The 250 \micron{} flux is located on the Rayleigh Jeans side of the blackbody and varies much more strongly as a function of the dust temperature due to the rapidly declining shape of the SED long-ward of the peak. Even though the total 70 \micron{} emission can have a large fraction of its emission contributed by the AGN (see next section), it still remains a decent predictor of $f_{\rm AGN}$ with $\rho = 0.82$ and $\rho=0.89$ for the 12/70 and 22/70 \micron{} ratios. 

%%%%%%%%%%%%%%%%%% TABLE OF LINEAR REGRESSION B/W FLUX RATIOS AND FAGN %%%%%%%%%%%%%%%%%%%%%%%%%%
\begin{table}
\begin{threeparttable}
\captionsetup{font=small,labelfont=bf,labelsep=period}
\caption{Linear Regression Between Flux Ratios and $f_{\rm AGN}$\label{tab:flux_ratio_fagn_linreg}}
\begin{tabular}{lcccc}
\toprule 
$F_{1}/F_{2}$ & $m$ & $b$  & $\sigma_{\rm int}$ & $\rho$  \\
\midrule
12/70      & $0.51\pm0.03$ & $0.96\pm0.03$ & $0.017\pm0.002$ & $0.82\pm0.03$ \\
12/160    & $0.46\pm0.02$ & $0.97\pm0.02$ & $0.004\pm0.002$ & $0.95\pm0.02$ \\
12/250    & $0.33\pm0.03$ & $0.77\pm0.02$ & $0.017\pm0.003$ & $0.75\pm0.04$ \\
22/70      & $0.64\pm0.03$ & $0.82\pm0.02$ & $0.011\pm0.002$ & $0.89\pm0.02$ \\
22/160    & $0.46\pm0.02$ & $0.80\pm0.02$ & $0.002\pm0.001$ & $0.97\pm0.02$ \\
22/250    & $0.32\pm0.03$ & $0.64\pm0.02$ & $0.017\pm0.003$ & $0.75\pm0.05$ \\
\bottomrule
\end{tabular}
\end{threeparttable}
\end{table}
%%%%%%%%%%%%%%%%%%%%%%%%%%%%%%%%%%%%%%%%%%%%%%%%%%%%%%%%%%%%%%%%%%%%%%%%%%%%%%

\subsection{The AGN contribution to the 70 \micron{} emission}
Many studies examining the relationship between nuclear activity and star formation rely on single monochromatic luminosities as an indicator of star formation \cite[e.g.][]{Netzer:2007ve, Netzer:2009lr, Rosario:2012fr}. The most widely used indicator has been the 60 \micron{} luminosity given its availability from the all-sky \textit{IRAS} survey. 

With our \herschel-BAT sample and SED modeling we can test how often and at what level AGN emission contributes to different wavelength bands. We test the AGN contribution to the 70 \micron{} emission. Based on visual inspection of all of the SED fits, we found this waveband is the only \herschel{} band with a noticeable AGN contribution. All other longer wavelength bands were dominated by the MBB component. Further this band is very close to the \textit{IRAS} 60 \micron{} band that is used often as an indicator of star-formation in AGN host galaxies.

To determine the AGN contribution to the 70 \micron{} emission, we first estimated the PL component associated with star formation. We assumed the shape of the star-forming PL component followed the average power law slope ($\alpha=0.0$ and turnover wavelength ($\lambda_{\rm c} = 50$ \micron) from the HRS sample. We then adjusted the normalization such that the integrated luminosity would equal $1/3L_{\rm MBB}$. We combined the estimated star forming PL component with the measured MBB component to form each source's total star-forming component. The 70 \micron{} AGN contribution was then calculated as the excess 70 \micron{} emission leftover after subtracting the star-forming contribution from the total 70 \micron{} emission from the best-fit SED. We denote the fraction contributed by the AGN as $f_{\rm AGN, 70}$.

Figure~\ref{fig:70um_agn_contribution}, \textit{left} shows the distribution of $f_{\rm AGN, 70}$. We find that for 45 per cent of the sample,  $f_{\rm AGN, 70} < 0.2$ but for 30 per cent of the sample $f_{\rm AGN, 70} < 0.5$. Using the 70 \micron{} luminosity as a single SFR indicator would overestimate values by at least a factor of 2 for almost one-third of an AGN sample. 

In the right panel of Figure~\ref{fig:70um_agn_contribution} we show $f_{\rm AGN, 70}$ as a function of the 14--195 keV luminosity. The grey points indicate the individual points while the blue points with error bars are binned averages with a 68 per cent confidence interval. At low AGN luminosity, it is clear we suffer from incompleteness indicated by the low number of grey points and large uncertainties on the binned points. At high AGN luminosity ($\log\,L_{\rm 14-195\,keV} > 43.0$), $f_{\rm AGN, 70}$ increases monotonically as expected. As the X-ray luminosity increases, the AGN IR component should increase such that more of the 70 \micron{} emission is due to the AGN. However, there is an incredible amount of scatter especially at high AGN luminosity where $f_{\rm AGN, 70}$ ranges almost uniformly between 0 and 1. $f_{\rm AGN, 70}$ is a function of not only the AGN IR component but also the star-forming component and while at high AGN luminosity it is more likely to find galaxies with high $f_{\rm AGN, 70}$, its still just as likely to find a galaxy with enough star formation to overwhelm any AGN contribution.

Our main conclusion from this section, though, is to be careful when using monochromatic luminosities as an indicator of the SFR, especially near 70 \micron. At longer wavelengths, we find single luminosities are quite reliable in regards to the AGN contribution.

\begin{figure*}
\includegraphics{figures/pacs_AGN_contribution}
\caption{\label{fig:70um_agn_contribution}}
\end{figure*}


\subsection{Comparison between Type 1s and 2}

%%%%%%%%%%%%%%%%%% FIGURES COMPARING TYPE 1s and TYPE 2s %%%%%%%%%%%%%%%%%%%%%%%%%%
\begin{figure*}
\includegraphics[width=\textwidth]{figures/agn_type_comparison}
\caption{Comparison of $M_{\rm dust}$, $T_{\rm dust}$, SFR, MIR powerlaw slope ($\alpha$), turnover wavelength ($\lambda_{\rm c}$), and $L_{\rm AGN,IR}$ between the \herschel-BAT AGN Type 1 (blue) and Type 2 (red) samples. Type 1 and Type 2 AGN have identical distributions in all properties and parameters except for $\alpha$ and $L_{\rm AGN,IR}$. A color version of this figure is available in the online publication. \label{fig:agn_type_comp}}
\end{figure*}

\begin{figure}
\includegraphics[width=\columnwidth]{figures/fagn_ir_seyferts}
\caption{\label{fig:frac_agn_seyferts}}
\end{figure}

%%%%%%%%%%%%%%%%%%%%%%%%%%%%%%%%%%%%%%%%%%%%%%%%%%%%%%%%%%%%%%%%%%%%%%%%%%

%%%%%%%%%%%%%%%%%%%%%%% TABLE: Comparison of Types 1 and Types 2%%%%%%%%%%%%%%%%%%%%%%%%%%%%%
\begin{table*}
\centering
\begin{threeparttable}
\captionsetup{font=small,labelfont=bf,labelsep=period}
\caption{Comparison of Star-Forming and SED Properties of Type 1 and Type 2 AGN \label{tab:agn_type_comp}}
\begin{tabular}{lccccccccc}
\toprule 
 & $\log$ \mdust & \tdust & $\log$ SFR & $\alpha$ & $\lambda_{\rm c}$ & $\log\,L_{\rm AGN,IR}$ \\
 & [\msun]  & [K]  &  [\msun{} yr$^{-1}$] &    &  \micron & [ergs s${^-1}$]\\
\midrule
Type 1 & 7.08$\pm$0.08 & 23.2$\pm$0.4 & -0.01$\pm$0.07 & 1.46$\pm$0.04 & 48.7$\pm$0.8 & 10.00$\pm$0.06 \\ 
Type 2 & 7.33$\pm$0.05 & 23.5$\pm$0.4 & 0.21$\pm$0.04 & 1.73$\pm$0.04 & 49.4$\pm$0.7 & 9.71$\pm$0.07 \\
Peto-Prentice Probability & 0.04 & 0.73 & 0.02 & $2.6\times10^{-5}$ & 0.73 & 0.003 \\
\bottomrule
\end{tabular}
\begin{tablenotes}
\item blah blah blah
\end{tablenotes}
\end{threeparttable}
\end{table*}
%%%%%%%%%%%%%%%%%%%%%%%%%%%%%%%%%%%%%%%%%%%%%%%%%%%%%%%%%%%%%%%%%%%%%%%%%

In this Section, we compare the properties of unabsorbed (Type 1) vs absorbed (Type 2) AGN. According to the unified model \citep{Antonucci:1993os,Urry:1995il}, Type 1 and Type 2 AGN are two manifestations of the same object. The only difference between the two is the viewing angle towards the central AGN. Due to a dusty structure, lines of sight at low angle are obscured causing the broad line region to be hidden while high angle lines of sight allow an unobstructed view of both the accretion disk and broad line region. Type 1 AGN therefore exhibit broad emission lines and low hydrogen absorption while the opposite is true of Type 2 AGN. 

Because the differences in AGN types is related to the central obscuring structure, global star-forming properties of Type 1 and Type 2 AGN should be the same. In the top row of Figure~\ref{fig:agn_type_comp}, we compare the distribution of \mdust, \tdust, and SFR between Type 1s and Type 2s. Visually the distributions look identical. However, after including sources with upper limits on \mdust{} and SFR, we find differences between Type 1s and Type 2s. In Table~\ref{tab:agn_type_comp}, we show the average of these properties as well as the Peto-Prentice probability that Type 1s and Type 2s are drawn from the same distribution. The only property with a probability greater than 0.05 is \tdust, meaning that the cold dust temperature associated with star formation is the same in Type 1s and Type 2s. 

Surprisingly, however, are the results of the Peto-Prentice test for \mdust{} and SFR. While Figure~\ref{agn_type_comp} shows Type 1s and Type 2s having identical distributions, survival analysis indicates that Type 1s are skewed more strongly towards galaxies with lower \mdust{} and SFR. Type 1s have 0.25 dex lower average \mdust and 0.22 dex lower average SFR. This difference is caused by the population of sources with only upper limits on both \mdust{} and SFR. 25/35 of the AGN with upper limits are classified as Type 1. Further, the actual values of the upper limits for the Type 1 AGN compared to Type 2 AGN. Of the 25 Type 1 AGN with upper limits, 5 have $\log\,$\mdust $< 6.0$ and $\log\,$SFR $<-1.0$ compared to none of the Type 2 AGN. 

These Type 1 AGN with little to no dust and no recent star formation seem to form a separate sample away from the main population of AGN host galaxies. If we restrict the samples to only those sources with $\log\,$\mdust $> 6.0$ and $\log\,$SFR $>-1.0$, the Peto-Prentice probability changes to 0.25 for \mdust{} and 0.09 for the SFR. The new probabilities are both above the standard threshold of 0.05, indicating they originate from the same parent population. These anomalous Type 1 AGN could be an interesting subsample for future study, but our main result is that for the bulk of Type 1 and Type 2 AGN, their host galaxies display the same star-forming properties in agreement with the unified model.

Regarding the AGN related properties of the IR SED, we see differences between Type 1 and Type 2 AGN even before including sources with upper limits. Both the Type 1 $\alpha$ and $L_{\rm AGN,IR}$ distributions are visually different from Type 2 AGN with Type 2 AGN having higher $\alpha$ and lower $L_{\rm AGN,IR}$. Survival analysis and calculation of their means confirm the visual differences as shown in Table~\ref{tab:agn_type_comp}. Type 1 AGN have an average $\alpha$ of 1.46 compared to 1.73 for Type AGN and an average $\log\,L_{\rm AGN,IR}$ of 10.00 compared to 9.71. 

Steeper power-law slopes for Type 2 AGN has been observed previously in studies of AGN using low resolution \textit{Spitzer}/IRS spectra. Both \citet{Buchanan:2006cq} and \citet{Wu:2009pt} reported flatter MIR spectral indices for Type 1 AGN. Explanations for the differences included a greater contribution of a starburst component for Type 2 AGN which show red MIR spectra \citet{Brandl:2006kx}. The difference in slopes can also be explained by a changing opacity from shorter to longer wavelengths. If the optical depth increases with wavelength and shorter wavelength emission originates near the inner edge of the ``torus'', then Type 2 AGN would also show steeper MIR slopes. At the present time we cannot make strong conclusions as to which is the correct explanation. We plan to analyze this more closely in a future publication which will combine the \herschel{} photometry with \textit{Spitzer}/IRS spectra to better constrain the starburst component of MIR spectra. 

The difference in $\log\,L_{\rm AGN,IR}$ between Type 1 and Type 2 AGN can be explained by the different X-ray luminosity distribution of Type 1 and Type 2 \herschel-BAT AGN. It's been well documented  \citep[e.g.][]{Burlon:2011pi, Melendez:2014yu} that within the \swift/BAT AGN sample, Type 2 AGN are more prevalent at lower luminosities compared to Type 1 AGN. Given the strong correlation between the X-ray and IR luminosity of AGN, Type 2s should exhibit lower $\log\,L_{\rm AGN,IR}$.

The combination of lower $L_{\rm AGN,IR}$ and relatively similar SFR distributions produces a strong difference in the $f_{\rm AGN}$ distribution for Type 2s and Type 1s as shown in Figure~\ref{fig:frac_agn_seyferts}. Type 2s have much lower $f_{\rm AGN}$ compared to Type 1 AGN which agrees with \citet{Melendez:2008pd} who found that Type 2s have colder 25/60 \micron{} colors. Our main finding, though, is that the lower $f_{\rm AGN}$ is not indicative of larger SFRs in Type 2 AGN but rather lower AGN luminosities. 

\subsection{The correlation between nuclear activity and star formation}
Our main goal in this Paper is to assess the observational evidence for a connection between AGN and star formation in their host galaxy. To accomplish this, we analyze the relationship between the AGN strength and SFR. We assume the 14--195 keV luminosity ($L_{\rm 14--195 keV}$) linearly probes the bolometric AGN luminosity as shown by \citet{Winter:2012yq}. The 8--1000 \micron{} luminosity, after subtracting the contribution from AGN heated dust, measures the SFR using Equation~\ref{eq:sfr_ir}.

Figure~\ref{fig:sfr_lbat_correlation} plots the correlation between the SFR and $L_{\rm 14-195 keV}$ for the \herschel-BAT AGN. As before, blue colors indicate Type 1 AGN and red colors indicate Type 2 AGN.  The explicit model we assume is 

\begin{equation}
\log {\rm SFR} = m\log L_{\rm 14-195\,keV} + b + \epsilon_{\rm int}
\end{equation}

\noindent where $m$ is slope, $b$ is the intercept and $\epsilon_{\rm int}$ represents an intrinsic scatter between the two variables and is assumed to be a Gaussian random variable with mean 0 and standard deviation of $\sigma_{\rm int}$. 

%%%%%%%%%%%%%%%%%%%%%%% FIGURE: Correlation Between LBAT and SFR %%%%%%%%%%%%%%%%%%%%%%%%%%%%%
\begin{figure*}
\includegraphics{figures/sfr_lbat_correlation}
\caption{Correlation between AGN luminosity as probed by $L_{\rm 14-195\,keV}$ and SFR. Error bars on $L_{\rm14-195\,keV}$ correspond to the 90 percentile confidence interval from the 58 month BAT catalog. Error bars on SFR correspond to the 68 percentile confidence interval from our SED fitting. Downward pointing triangles plot 95 percent confidence upper limits on the SFR. Blue colors represent Type 1 AGN while red colors represent Type 2 AGN. The solid lines with shading plot the best fit lines and 95 percentile confidence interval from \textsc{linmix\_err}. No strong relationship exists between the AGN strength and global star formation. A color version of this figure is available in the online publication.\label{fig:sfr_lbat_correlation}}
\end{figure*}

\begin{figure*}
\includegraphics{figures/sfr_lbat_correlation_split_main_seqeunce}
\caption{\label{fig:sfr_lbat_correlation_ms}}
\end{figure*}
%%%%%%%%%%%%%%%%%%%%%%%%%%%%%%%%%%%%%%%%%%%%%%%%%%%%%%%%%%%%%%%%%%%%%%%%%%%

To see if there are differences in the relationship between Type 1s and Type 2s, we tested the correlation using the whole \herschel-BAT sample, only Type 1s, and only Type 2s. The black line and shaded region represents the best-fit line and 95 percent confidence interval for the whole sample while the blue and red lines and shaded regions represent the relationship for Type 1s and Type 2s respectively. In Table~\ref{tab:lbat_sfr_correlation}, we outline the best fit slope, intrinsic scatter, and correlation coefficient and their uncertainties, determined by calculating the median and standard deviation of the posterior probability distributions. 

None of the three samples show a strong correlation between the SFR and 14--195 keV luminosity. With slopes and correlation coefficients between 0.25 and 0.35, even though both are inconsistent with a slope and correlation coefficient of 0, the evidence for a connection between the strength of the AGN and global star formation is quite weak. 

Type 1 AGN do show a slightly stronger correlation than the whole sample and Type 2 AGN which is consistent with our findings in \citet{Melendez:2014yu} and \citet{Shimizu:2016qy} using the monochromatic \herschel{} luminosities. However, the significance of the differences are not more than 2$\sigma$ therefore we must conclude that Type 1 AGN show the same relationship between SFR and 14--195 keV luminosity.

%%%%%%%%%%%%%%%%%%%%%%% TABLE: AGN-SFR Connection%%%%%%%%%%%%%%%%%%%%%%%%%%%%%
\begin{table}
\begin{threeparttable}
\captionsetup{font=small,labelfont=bf,labelsep=period}
\caption{Linear Regression Between $L_{\rm14-195\,keV}$ and SFR \label{tab:lbat_sfr_correlation}}
\begin{tabular}{lcccc}
\toprule 
 Sample & $m$ & $b$  & $\sigma_{\rm int}$ & $\rho$  \\
\midrule
All        & $0.25\pm0.05$ & $-10.7\pm2.3$ & $0.25\pm0.02$ & $0.28\pm0.06$ \\
Type 1 &  $0.34\pm0.09$ & $-14.7\pm3.9$ & $0.24\pm0.04$ & $0.35\pm0.09$ \\ 
Type 2 &  $0.25\pm0.07$ & $-10.4\pm3.1$ & $0.26\pm0.03$ & $0.27\pm0.08$ \\
NMS      & $0.27\pm0.09$ & $-11.3\pm3.8$ & $0.16\pm0.03$ & $0.38\pm0.12$\\
BMS   & $0.43\pm0.11$ & $-18.9\pm5.0$ & $0.19\pm0.05$ & $0.52\pm0.13$ \\
\bottomrule
\end{tabular}
\end{threeparttable}
\end{table}
%%%%%%%%%%%%%%%%%%%%%%%%%%%%%%%%%%%%%%%%%%%%%%%%%%%%%%%%%%%%%%%%%%%%%%%%%

As shown in \citet{Shimizu:2015xo}, the \herschel-BAT AGN span a large range in $\Delta\log$(SFR) which we defined as the logarithmic difference between the observed SFR and the SFR expected if the source lived on the star-forming main sequence with its stellar mass. Since AGN don't exclusively live either on the main sequence or off of it, its possible the large scatter we observe in Figure~\ref{fig:sfr_lbat_correlation} is due to the correlation changing as a function of $\Delta\log$(SFR). 

To investigate this possibility, we split the sample up into galaxies ``Near the Main Sequence'' (NMS) and ``Below the Main Sequence'' (BMS). NMS sources correspond to $|\Delta\log$(SFR)$|< 0.64$ dex. These are galaxies that are within 2$\sigma$ of our measured main sequence in \citet{Shimizu:2015xo}. BMS galaxies have $\Delta\log$(SFR)$< -0.64$ dex and are more than 2$\sigma$ below the main sequence. 

In Figure~\ref{fig:sfr_lbat_correlation_ms} we plot the split sample with NMS galaxies in blue and BMS galaxies in red. The total number of galaxies in both samples is significantly reduced (119 compared to 313 for the entire sample) because only a subsample of the whole \herschel-BAT AGN sample have reliable stellar masses. Using the same method as before, we calculated a best fit linear relation for both the NMS and BMS subsamples. The parameters are listed in Table~\ref{tab:lbat_sfr_correlation} and we plot the best fit line with a 95 percent confidence interval as blue and red lines with shading in Figure~\ref{fig:sfr_lbat_correlation_ms}. For reference, we also plot the relationship measured using the whole sample as a black line with grey shading. 

The NMS sample doesn't show much change compared to the relationship measured using the entire sample with only a 0.02 change in the slope and 0.1 change in the correlation coefficient. There is a slight increase in the normalization as expected since we are using the upper half of the sample. For the BMS sample however, the slope has increased significantly by 0.18 and the correlation coefficient increased by 0.24. This is a possible indication that the strength of the relationship between AGN activity and star formation increases as galaxies become more quenched. Confirming this intriguing possibility will require more AGN at both low and high luminosity with reliable masses which we are lacking in our sample. At the very least it seems plausible that the large scatter in the overall SFR-$L_{14-195\,keV}$ can be explained by the differing locations in comparison to the star forming main sequence.

Our measured AGN-SFR relationship agrees mainly with the previous studies that found a flat or weak relationship \citep{Diamond-Stanic:2012rw, Mullaney:2012gf, Azadi:2015ve, Stanley:2015qy}. The scatter in the relationship is high covering nearly 2 orders of magnitude in SFR for a given AGN luminosity. Two different but not mutually exclusive explanations have been proposed to explain the lack of relationship seen between the strength of the AGN and SFR.0 \citet{Diamond-Stanic:2012rw} suggest that the AGN only influences star formation in the nuclear regions of galaxies and find that when restricting their measurements to only the inner 1kpc, a stronger relationship appears. \citet{Esquej:2014vl}, probing even smaller scales ($r < 100$ pc) using MIR interferometry, also find a nearly linear relationship between the nuclear SFR and SMBH accretion rate. This is  in agreement with numerical simulations \citep{Hopkins:2010kx} that predict an increasingly linear relationship as the star formation size scale decreases. Given our SFRs are measured globally over the whole galaxy, the size scale difference can easily explain our sub-linear relationship. It could also explain the small differences we see when splitting the sample up by the location on the main sequence. If the BMS AGN have reduced star formation only on large scales, while the NMS have larger amounts of extended star formation, then our measurements would naturally be exploring different sizes. The reduced sizes in the BMS AGN would imply a stronger relationship, exactly what we observe in Figure~\ref{fig:sfr_lbat_correlation_ms}. Measurement of the FIR sizes and morphologies will be the subject of a future publication and we will explore the hypothesis that the star-forming region size is dependent on location with respect to the main sequence.

The second explanation involves the varying timescales associated with star formation and accretion onto the SMBH. Measuring the SFR with from the IR luminosity results in an average SFR over nearly 100 Myr \citep{Kennicutt:2012it} while the X-ray luminosity is more aligned with the instantaneous AGN luminosity especially given the observations that the X-ray emission likely originates very near to the SMBH \citep[e.g][]{Chen:2011lr}. Therefore, if the AGN luminosity can vary over 100 Myr, while the SFR is relatively stable, this will cause any intrinsic relationship between the two to disappear. \citet{Hickox:2014yq} explored this using a simple model for the Eddington ratio distribution for an AGN and a linear relationship between the AGN luminosity and SFR, finding that a powerlaw Schechter function matches the observed weak AGN-SFR relationship, rather than a lognormal distribution which produces too strong of a relationship. 


\section{Conclusions}

\section*{Acknowledgements}
This publication makes use of data products from the Wide-field Infrared Survey Explorer, which is a joint project of the University of California, Los Angeles, and the Jet Propulsion Laboratory/California Institute of Technology, and NEOWISE, which is a project of the Jet Propulsion Laboratory/California Institute of Technology. WISE and NEOWISE are funded by the National Aeronautics and Space Administration.
%%%%%%%%%%%%%%%%%%%%%%%%%%%%%%%%%%%%%%%%%%%%%%%%%%

%%%%%%%%%%%%%%%%%%%% REFERENCES %%%%%%%%%%%%%%%%%%

% The best way to enter references is to use BibTeX:

\bibliographystyle{mnras}
\bibliography{my_bib} % if your bibtex file is called example.bib

%%%%%%%%%%%%%%%%%%%%%%%%%%%%%%%%%%%%%%%%%%%%%%%%%%

%%%%%%%%%%%%%%%%% APPENDICES %%%%%%%%%%%%%%%%%%%%%

\appendix

\section{SED Parameter Estimation}\label{sec:sed_bayes}
%We used two different fitting methods to fit the SEDs of the galaxies: 1) a Bayesian framework including Monte Carlo Markov Chain (MCMC) analysis for parameter estimation and 2) maximum likelihood optimization. For the C12 model, we chose the Bayesian framework, and for the template based models (DecompIR and D14) we chose maximum likelihood optimization. The main reason for using two different methods to determine the best fit model SED is that the Bayesian framework does not work well with a set of discrete model templates, especially when the number of templates is low. Further, the only parameter in the template fitting that needs to be optimized is the normalization of the template to fit the observed SED. With a uniform prior on the one parameter, Bayesian methods essentially are reduced to maximum likelihood optimization.
To find the best fitting parameters in our SED modeling, instead of standard least squares analysis, we used a Bayesian framework along with Monte Carlo Markov Chains to probe the posterior probability distribution functions for each parameter. The Bayesian framework allows for robust estimates of the uncertainty and for explicit statements about prior knowledge of the parameters. It also makes it relatively easy to include information contained in the undetected photometry of the SED.

\subsection{Likelihood Representation}
The likelihood defines the probability of observing a set of data given a specific model. In SED fitting, this translates to the combined probability of measuring all the photometric data points in the observed SED given a model for the SED (whether based on templates or analytic models). The total likelihood can then be expressed as the product of the probabilities of observing each single photometric point:

\begin{equation}\label{eq:likelihood}
\mathcal{L}(F|M) = \prod_{i}P(F_i|M)
\end{equation}

\noindent where $F$ is the set of photometric fluxes, $F_i$ and $M$ is the model. For our analysis, we assume the probability of our observations follows a Gaussian distribution with mean equal to $M$ and standard deviations equal to the measurement errors, $\sigma_{i}$.

\begin{equation}\label{eq:detected_prob}
P(F_i|M) = \frac{1}{\sqrt{2\pi\sigma_{i}^{2}}}\mathrm{exp}\left(\frac{-(F_i - M)^2}{2\sigma_i^2}\right)
\end{equation}

Equation~\ref{eq:detected_prob} only defines the probability for detected fluxes. To use the information contained in the undetected photometry, $U_i$, we define a different probability under the assumption that all of the upper limits are 5$\sigma$. 

\begin{align}\label{eq:undetected_prob}
P(U_i|M) &= \int_{-\infty}^{U_i}\frac{1}{\sqrt{2\pi\sigma_{i}^{2}}}\mathrm{exp}\left(\frac{-(x - M)^2}{2\sigma_i^2}\right)dx \notag\\
&= \frac{1}{2}\left(1 + \mathrm{erf}\left[\frac{U_i - M}{\sigma_i\sqrt{2}}\right]\right) 
\end{align}

\noindent where $\sigma_i = \frac{U_i}{5}$ and erf is the standard error function. For numerical accuracy and simplicity, it is customary to minimize the negative log-likelihood. Supposing we have $N$ total SED points with $D$ detections and $D-N$ non-detections then the total negative log-likelihood combining Equations~\ref{eq:likelihood},~\ref{eq:detected_prob},~and~\ref{eq:undetected_prob} is:

\begin{align}\label{eq:log_like}
-\log\mathcal{L} = \frac{1}{2}&\sum_{i=0}^{D}\left[\log(2\pi\sigma_i^2) - \left(\frac{F_i - M}{\sigma_i}\right)^2\right] + \notag \\
&\sum_{j=0}^{D-N}\log\left[1 + \mathrm{erf}\left(\frac{U_j - M}{\sigma_j\sqrt{2}}\right)\right]
\end{align}

It is important to recognize here how $M$ is calculated, no matter whether it represents a template or analytic model. Each data point in an SED is the observer-frame flux density measured over a defined wavelength range. Therefore, to determine the model flux densities we first redshifted the full rest-frame model SED into the observer frame using the known redshifts of all of our sources. This observer-frame SED was then convolved with each instrument filter transmission curve to produce model flux densities that can be accurately compared to the observed ones.

\subsection{Bayesian MCMC Analysis}
Within the Bayesian framework, the important probability is the probability of the model given the data at hand, i.e. the most probable SED model given the observed fluxes. This probability can be determined using Bayes theorem:

\begin{equation}\label{eq:bayes}
P(M|F) = \frac{P(F|M)P(M)}{P(F)}
\end{equation}

\noindent and is known as the posterior probability distribution. $P(F|M)$ is proportional to the likelihood (Equation~\ref{eq:likelihood}), $P(M)$ codifies our prior knowledge about the model, and $P(F)$ is the model evidence and can be disregarded as a simple normalization term. 

The reader may notice that assuming a flat prior, $P(M) \propto 1$, reduces Equation~\ref{eq:bayes} to $P(M|F) \propto \mathcal{L}$ verifying our use of maximum likelihood in determining the best template models.

For the C12 model, we used flat priors for the dust temperature and dust mass. We placed conservative limits on both the dust temperature and $\log M_{\mathrm dust}$ to be between 1 and 100 K and 1 and 10 \msun respectively. Within the powerlaw component for the C12 model, we also used flat priors for the powerlaw slope between -5 and 5 and the log of the normalization between -10 and 10. Based on previous work attempting to measure the intrinsic AGN SED and modeling the dusty torus, we expect the SED to turnover anywhere in the range between 20-70 \micron. Therefore we imposed a Gaussian prior centered at 45 \micron{} with a standard deviation of 20 \micron. We found that imposing this prior resulted in better and more realistic fits to the SEDs.

We used the \textsc{PYTHON} package \textsc{emcee} \citep{Foreman-Mackey:2013lr} to perform MCMC and sample the posterior probability distribution function (Equation~\ref{eq:bayes}). \textsc{emcee} runs an implementation of the Affine-Invariant MCMC sampler from Goodman \& Weare 2010. Instead of one single MCMC chain, it samples the posterior PDF with multiple ``walkers'', each with their own chain. For our analysis, we used 50 walkers that each produced a 1000 step chain. To allow for each chain to stabilize and move away from the initial guesses for the parameters, we imposed a 200 step ``burn-in''. In total, this resulted in 40000 steps to define the full posterior PDF. 

To determine the best fit parameters, we first marginalized the posterior PDF over all other parameters and then calculated the median. All quoted uncertainties represent the 68\% confidence interval determined from the 16th and 84th percentile of the marginalized posterior PDF.

For sources with less than four detections in their SED, we fixed the dust temperature to 25 K since the peak of the FIR bump is no longer constrained. In these cases, we only obtained upper limits on the dust mass by calculating the upper 95th percentile of the marginalized PDF.

\section{Comparison between different models}
In this section we compare the results for the total luminosity, IR AGN luminosity, and star forming luminosity between the C12 model and two other models to decompose the SED. 

\subsection{DecompIR model}
Besides analytic models, another popular method is the use of template SEDs. Templates are constructed based on well-sampled SEDs of large samples of galaxies and usually parameterized according to a known property such as infrared luminosity.

For galaxies known to host an AGN, recent studies have turned to the \textsc{DecompIR} \citep{Mullaney:2011yq} templates. \textsc{DecompIR} consists of five host galaxy templates that span the IR color and luminosity range of the original \citet{Brandl:2006kx} starburst galaxy templates. \citet{Mullaney:2011yq} constructed the AGN templates based on a subsample of the \swift/BAT AGN which had AGN dominated \textit{Spitzer}/IRS spectra determined by the equivalent width of the 11.3 \micron{} feature being $<0.03$ \micron. The \textit{Spitzer}/IRS spectra were combined with \textit{IRAS} photometry at 60 and 100 \micron to define the ``intrinsic'' AGN SED from 6--100 \micron. 

\citet{Mullaney:2011yq} created three different AGN templates: one based only on high AGN luminosity objects, low AGN luminosity objects, and a median of the entire sample. For this work, we only consider the median AGN template given our SEDs only contain two points in the MIR where AGN-related emission is expected to dominate. 

\subsection{Dale et al 2014 model}
The third model we chose to test on our sample is the semi-empirical templates from \citet[][hereafter D14]{Dale:2014yq}. These templates also contain two components, one for dust emission in the host galaxy and one for the AGN. The host galaxy components were built from an updated version of the \citet{Dale:2002ty} model. Each component represents an SED produced using a different value of $\alpha_{\mathrm SF}$, which is the powerlaw slope of the intensity distribution for the interstellar radiation field that is heating the dust. These SEDs contain a mixture of emission from PAHs, small stochastically heated grains, and thermally radiating large grains.

For the AGN component, D14 chose the median SED of the Palomar-Green quasars from \citet{Shi:2013vn} citing the care with which any star-forming component was removed and the prominence of several AGN related MIR features such as the [OIV] fine structure line and the broad 10 and 18 \micron{} silicate emission bumps. At long wavelengths the AGN template falls as a blackbody.

Instead of two separate templates for the AGN and host galaxy, D14 provided a single set of templates based on different combinations of $\alpha_{\mathrm SF}$ and $f_{\mathrm AGN}$, the fractional contribution of the AGN to the 5--20 \micron{} emission. In total there are 1365 templates that range in $\alpha_{\mathrm SF}=0.0625-4.0$ in 0.0625 intervals and $f_{\mathrm AGN} = 0-1$ in 0.05 intervals.

\subsection{SED Fitting for the template models}
For each template in a model set, Equation~\ref{eq:log_like} was minimized to determine the best fit normalization. We then chose the normalized template with the lowest $-\log\mathcal{L}$ as the best fitting model for a source. For the DecompIR set, this meant first simultaneously optimizing over a normalization for the AGN component and each host galaxy component, then choosing the combined template that resulted in the minimum $-\log\mathcal{L}$. For D14, this meant calculating $-\log\mathcal{L}$ over the entire set of $\alpha$ and $f_{AGN}$ templates.

Uncertainties using the maximum likelihood method were determined by generating 1000 simulated SEDs for each source. These data points in the simulated SEDs were calculated by assuming each detected point followed a Gaussian distribution with mean equal to the observed flux density and a standard deviation equal to the measured error. Each of the simulated SEDs were re-fit using the same method. The standard deviation on the set of best fit parameters from the simulated SEDs then was used as the uncertainty. In this way both statistical and systematic errors can be taken into account in assessing the reliability of our best fit parameters. 

\subsection{$L_{\rm IR}$ Comparison}
The measured property that should be most model independent is the total infrared luminosity, $L_{\rm IR}$. Assuming each model was able to fit well the broadband SEDs and reproduce the observed photometry, $L_{\rm IR}$ is simply the total integrated energy underneath the SED, irregardless of how the SED is decomposed. In Figure~\ref{fig:lir_total_comp}, we plot the correlations between each of the $L_{\rm IR}$ in the off-diagonal subplots as well as the single distributions for each one. 

The median $\log\,L_{\rm IR}$ for the C12, DecompIR, and D14 models are 10.43, 10.44, and 10.63 \lsun respectively and each model showing a spread of 0.5 dex. Based on the medians as well as Figure~\ref{fig:lir_total_comp}, the C12 and DecompIR models best agree. The average logarithmic difference between them is only -0.02 dex while the average difference between the D14 and C12 and DecompIR models is 0.2 dex, indicating a factor of 1.6 increase in $L_{\rm IR}$ between D14 and the other two models. This is due to the shape of the template SEDs for D14. D14's AGN template is flatter at shorter wavelengths causing the increased luminosity. 

The strict set of templates available in the D14 model also likely causes the increased scatter between D14 and the two others. Whereas DecompIR is composed of two templates that can freely float to match the observed SED and the C12 model is analytical allowing for large flexibility, the D14 model just contains a fixed set of templates which might not be able to match the observed SED as well as the other two models.

Despite these differences, overall there is great agreement, as expected, between the measurement of $L_{\rm IR}$ for the three models. Given the 0.2 dex difference between D14 and the C12 and DecompIR models, we can conservatively assign this value as the total uncertainty on all $L_{\rm IR}$. This uncertainty overwhelms the individual statistical uncertainties which are $\sim0.03$ dex.

%%%%%%%%%%%%%%%%%%%%%%%% FIGURE: LIR COMPARISON %%%%%%%%%%%
\begin{figure*}
\includegraphics[width=\textwidth]{figures/lir_total-comparison}
\caption{Correlations and KDEs of $L_{\rm IR}$ from the three SED models.  A color version of this figure is available in the online publication. \label{fig:lir_total_comp}}
\end{figure*}
%%%%%%%%%%%%%%%%%%%%%%%%%%%%%%%%%%%%%%%%%%%%%%%%%%

\subsection{$f_{\rm AGN}$ Comparison}
Where the models begin to disagree more, is in the actual decomposition of the SED. In particular, we compare $f_{\rm AGN}$, the fraction of the 8--1000 \micron{} luminosity due to AGN-heated dust. Calculating $f_{\rm AGN}$ for the template models is relatively easy compared to corrections we had to make for the C12 model. For the DecompIR template set, $L_{\rm IR}$, $L_{\rm SF}$, and $L_{\rm AGN, IR}$ were simply calculated from the best fit total, host galaxy, and AGN SEDs. For the D14 model set, $L_{\rm IR}$ was measured from the best fitting template and $L_{\rm SF}$ and $L_{\rm AGN, IR}$ were calculated based on the best fit $f_{\rm AGN}$.\footnote{\citet{Dale:2014yq} only provides $f_{\rm AGN}$ calculated between 5--15 \micron, however D. Dale graciously provided the authors with $f_{\rm AGN}$ calculated between 8--1000 \micron{} through private communication in August 2015.} Uncertainties on all of these luminosities were determined using the same Monte Carlo method used to determine the uncertainties on the best fitting parameters.

Figure~\ref{fig:agn_frac_comp} displays the relationship between each model's $f_{\rm AGN}$ in the off-diagonal plots as well as the distribution of $f_{\rm AGN}$ in the diagonal plots. All three models are in rough agreement of $f_{\rm AGN}$. The three off-diagonal plots show a strong correlation between the models, albeit with large scatter. We find Pearson correlation coefficients of 0.81, 0.83, and 0.92 for C12 vs. DecompIR, C12 vs. D14, and DecompIR vs. D14 respectively. The two template based models agree the best between each other. This is likely due to the fixed shape of the AGN component SED for these models and the correction factor we had to use to calculate the C12 based $f_{\rm AGN}$. 

Overall all three models agree on $f_{\rm AGN}$ within a scatter of 0.15. The scatter around a 1--1 correlation (shown by the dashed black line) between each model is 0.15, 0.13, and 0.10 for C12 vs. DecompIR, C12 vs. D14, and DecompIR vs. D14 respectively. Because this is comparable to the statistical uncertainty we determined through the individual model-fitting, we add this in quadrature to the uncertainties in Tables~\ref{tab:c12_params},~\ref{tab:decompir_params},~and~\ref{tab:d14_params}. Interestingly in terms of following a 1--1 correlation, C12 and DecompIR compare the best even though between these two models the scatter is largest. We suspect this is due to the broad agreement in the shape of the AGN SED where both models have the AGN rising as a powerlaw with a slope $\sim2$ and peaking around 40 \micron{}. The D14 model has a flatter MIR slope that peaks closer to 20 \micron leading to the systematic offset seen in Figure~\ref{fig:agn_frac_comp}. The offset between D14 and the other two models is $\sim$0.1. 

The 15 per cent uncertainty between the three models represents our general lack of knowledge about the details of decomposing broadband SEDs of AGN host galaxies. The estimated $f_{\rm AGN}$ is highly dependent on the assumed models of both the host galaxy and AGN, and currently at best we can only constrain to within 15 per cent. This influences then the calculations of both the SFR (based on $L_{\rm SF}$) and the IR portion of AGN luminosity ($L_{\rm AGN, IR}$). Studies relying on calculating the SFR of AGN host galaxies using the infrared need to take these discrepancies into account or else risk over-interpreting results based on broad SED decomposition.

For our AGN population, we find almost a uniform distribution of $f_{\rm AGN}$ independent of the model used. Most AGN have between 20 and 80 per cent of their IR emission coming from AGN heated dust. This means that the IR SED of an AGN host galaxy can wildly vary from being completely dominated by star formation related emission to being completely dominated by AGN related emission. 

%%%%%%%%%%%%%%%%%%%%%%%% FIGURE: AGN FRACTION COMPARISON %%%%%%%%%%%
\begin{figure*}
\includegraphics{figures/agn_frac-comparison}
\caption{Correlations and KDEs of $f_{\rm AGN}$ from the three SED models.  A color version of this figure is available in the online publication. \label{fig:agn_frac_comp}}
\end{figure*}
%%%%%%%%%%%%%%%%%%%%%%%%%%%%%%%%%%%%%%%%%%%%%%%%%%

\section{SED Figures}

%\begin{figure*}
%\centering
%\includegraphics[width=\textwidth]{figures/sedfig1}
%\caption{\label{fig:seds}}
%\end{figure*}
\begin{figure*}
\centering
\includegraphics[width=\textwidth]{figures/sedfig1}
\caption{12--500 \micron{} SEDs for all of the \herschel-BAT sample. Black points plot the observed flux densities with open red circles indicating 5$\sigma$ upper limits. The solid blue line and shaded region shows the best-fit C12 model with a 95 percent confidence interval. The red squares are the model flux densities after convolving the best-fit model SED with each instrument's transmission curve. The orange and green dashed lines show the best-fit PL and MBB components, respectively. \label{fig:seds}}
\end{figure*}

\begin{figure*}
\centering
\includegraphics[width=\textwidth]{figures/sedfig2}
\caption{}
\end{figure*}

\begin{figure*}
\centering
\includegraphics[width=\textwidth]{figures/sedfig3}
\caption{}
\end{figure*}

\begin{figure*}
\centering
\includegraphics[width=\textwidth]{figures/sedfig4}
\caption{}
\end{figure*}

\begin{figure*}
\centering
\includegraphics[width=\textwidth]{figures/sedfig5}
\caption{}
\end{figure*}

\begin{figure*}
\centering
\includegraphics[width=\textwidth]{figures/sedfig6}
\caption{}
\end{figure*}

\begin{figure*}
\centering
\includegraphics[width=\textwidth]{figures/sedfig7}
\caption{}
\end{figure*}

\begin{figure*}
\centering
\includegraphics[width=\textwidth]{figures/sedfig8}
\caption{}
\end{figure*}

\begin{figure*}
\centering
\includegraphics[width=\textwidth]{figures/sedfig9}
\caption{}
\end{figure*}

\begin{figure*}
\centering
\includegraphics[width=\textwidth]{figures/sedfig10}
\caption{}
\end{figure*}

\begin{figure*}
\centering
\includegraphics[width=\textwidth]{figures/sedfig11}
\caption{}
\end{figure*}

\begin{figure*}
\centering
\includegraphics[width=\textwidth]{figures/sedfig12}
\caption{}
\end{figure*}

\begin{figure*}
\centering
\includegraphics[width=\textwidth]{figures/sedfig13}
\caption{}
\end{figure*}

\begin{figure*}
\centering
\includegraphics[width=\textwidth]{figures/sedfig14}
\caption{}
\end{figure*}

\begin{figure*}
\centering
\includegraphics[width=\textwidth]{figures/sedfig15}
\caption{}
\end{figure*}

\begin{figure*}
\centering
\includegraphics[width=\textwidth]{figures/sedfig16}
\caption{}
\end{figure*}

\begin{figure*}
\centering
\includegraphics[width=\textwidth]{figures/sedfig17}
\caption{}
\end{figure*}

\begin{figure*}
\centering
\includegraphics[width=\textwidth]{figures/sedfig18}
\caption{}
\end{figure*}

\begin{figure*}
\centering
\includegraphics[width=\textwidth]{figures/sedfig19}
\caption{}
\end{figure*}

\begin{figure*}
\centering
\includegraphics[width=\textwidth]{figures/sedfig20}
\caption{}
\end{figure*}

\begin{figure*}
\centering
\includegraphics[width=\textwidth]{figures/sedfig21}
\caption{}
\end{figure*}

\begin{figure*}
\centering
\includegraphics[width=\textwidth]{figures/sedfig22}
\caption{}
\end{figure*}

\begin{figure*}
\centering
\includegraphics[width=\textwidth]{figures/sedfig23}
\caption{}
\end{figure*}

\begin{figure*}
\centering
\includegraphics[width=\textwidth]{figures/sedfig24}
\caption{}
\end{figure*}

\begin{figure*}
\centering
\includegraphics[width=\textwidth]{figures/sedfig25}
\caption{}
\end{figure*}

\begin{figure*}
\centering
\includegraphics[width=\textwidth]{figures/sedfig26}
\caption{}
\end{figure*}

%\section{Tables of Best Fit Parameters}
%%%%%%%%%%%%%%%%%%%%%% TABLE OF DECOMPIR PARAMETERS %%%%%%%%%%%%%%%%%%%%%%%%%%
%\begin{table*}
%\centering
%\begin{threeparttable}
%\captionsetup{font=small,labelfont=bf,labelsep=period}
%\caption{Best Fit DecompIR Model Parameters, Luminosities, and AGN Fractions \label{tab:decompir_params}}
%\begin{tabular}{lccccc}
%\toprule 
%Name & Host Galaxy & $\log\,L_{\rm IR}$ & $\log\,L_{\rm SF}$ & $\log\,L_{\rm AGN, IR}$ & $f_{\rm AGN}$ \\
% & Template & [\lsun] & [\lsun] & [\lsun] & \\
%\midrule
%1RXSJ044154.5-082639 & SB2 & $10.39_{-0.02}^{+0.03}$ & $10.05_{-0.04}^{+0.08}$ & $10.12_{-0.08}^{+0.06}$ &$0.54_{-0.09}^{+0.05}$ \\
%1RXSJ045205.0+493248 & SB1 & $10.50_{-0.02}^{+0.02}$ & $10.10_{-0.04}^{+0.03}$ & $10.28_{-0.05}^{+0.05}$ &$0.60_{-0.04}^{+0.04}$ \\
%2E1739.1-1210 & SB1 & $10.89_{-0.03}^{+0.02}$ & $10.43_{-0.05}^{+0.04}$ & $10.71_{-0.06}^{+0.04}$ &$0.65_{-0.04}^{+0.04}$ \\
%2MASSJ07594181-3843560 & SB5 & $<10.24$ & $<8.94$ & $>10.21$ &$>0.95$ \\
%2MASSJ17485512-3254521 & SB1 & $<9.53$ & $<9.25$ & $>9.06$ &$>0.42$ \\
%2MASXJ00253292+6821442 & SB1 & $9.68_{-0.02}^{+0.02}$ & $9.01_{-0.04}^{+0.04}$ & $9.58_{-0.04}^{+0.03}$ &$0.79_{-0.03}^{+0.02}$ \\
%2MASXJ01064523+0638015 & SB3 & $10.48_{-0.03}^{+0.04}$ & $9.69_{-0.34}^{+0.07}$ & $10.40_{-0.04}^{+0.08}$ &$0.84_{-0.04}^{+0.09}$ \\
%2MASXJ01073963-1139117 & SB5 & $10.90_{-0.02}^{+0.02}$ & $10.70_{-0.02}^{+0.02}$ & $10.47_{-0.09}^{+0.05}$ &$0.37_{-0.05}^{+0.03}$ \\
%2MASXJ03305218+0538253 & SB1 & $10.81_{-0.03}^{+0.02}$ & $9.70_{-0.08}^{+0.18}$ & $10.78_{-0.05}^{+0.03}$ &$0.92_{-0.04}^{+0.01}$ \\
%2MASXJ03342453-1513402 & SB5 & $10.65_{-0.02}^{+0.02}$ & $10.53_{-0.01}^{+0.02}$ & $10.04_{-0.11}^{+0.07}$ &$0.24_{-0.05}^{+0.03}$ \\
%\bottomrule
%\end{tabular}
%\begin{tablenotes}
%\item Notes -- \textit{Column 1:} Name of the source. \textit{Column 2:} Best fit host galaxy template \textit{Column 3:} Log of the total infrared luminosity from 8--1000 \micron{} in solar units. \textit{Column 4:} Log of the MBB component luminosity in solar units. \textit{Column 5:} Log of the powerlaw component luminosity in solar units. \textit{Column 6:} Fractional contribution of the AGN to the total infrared luminosity. The full version of this table is available in the online publication.
%\end{tablenotes}
%\end{threeparttable}
%\end{table*}
%%%%%%%%%%%%%%%%%%%%%%%%%%%%%%%%%%%%%%%%%%%%%%%%%%%%%%%%%%%%%%%%%
%
%%%%%%%%%%%%%%%%%%%%%% TABLE OF D14 PARAMETERS %%%%%%%%%%%%%%%%%%%%%%%%%%
%\begin{table*}
%\centering
%\begin{threeparttable}
%\captionsetup{font=small,labelfont=bf,labelsep=period}
%\caption{Best Fit D14 Model Parameters, Luminosities, and AGN Fractions \label{tab:d14_params}}
%\begin{tabular}{lcccccc}
%\toprule 
%Name & $\alpha$ & $f_{\rm AGN, MIR}$ & $\log\,L_{\rm IR}$ & $\log\,L_{\rm SF}$ & $\log\,L_{\rm AGN, IR}$ & $f_{\rm AGN}$ \\
% &  &  & [\lsun] & [\lsun] & [\lsun] & \\
%\midrule
%1RXSJ044154.5-082639 & $1.6250$ & $0.85$ & $10.63_{-0.06}^{+0.02}$ & $10.36_{-0.03}^{+0.02}$ & $10.29_{-0.14}^{+0.04}$ &$0.47_{-0.07}^{+0.02}$ \\
%1RXSJ045205.0+493248 & $2.2500$ & $0.85$ & $10.81_{-0.08}^{+0.01}$ & $10.41_{-0.01}^{+0.03}$ & $10.58_{-0.14}^{+0.02}$ &$0.60_{-0.09}^{+0.01}$ \\
%2E1739.1-1210 & $2.0000$ & $0.85$ & $11.12_{-0.05}^{+0.04}$ & $10.77_{-0.02}^{+0.02}$ & $10.87_{-0.10}^{+0.06}$ &$0.56_{-0.06}^{+0.02}$ \\
%2MASSJ07594181-3843560 & $1.5000$ & $1.00$ & $<10.95$ & $<9.65$ & $>10.93$ &$>0.95$ \\
%2MASSJ17485512-3254521 & $1.0000$ & $0.95$ & $<9.72$ & $<9.44$ & $>9.18$ &$>0.39$ \\
%2MASXJ00253292+6821442 & $1.9375$ & $0.90$ & $9.93_{-0.03}^{+0.02}$ & $9.47_{-0.02}^{+0.02}$ & $9.75_{-0.03}^{+0.03}$ &$0.65_{-0.01}^{+0.01}$ \\
%2MASXJ01064523+0638015 & $1.5000$ & $0.95$ & $10.77_{-0.02}^{+0.03}$ & $10.22_{-0.03}^{+0.03}$ & $10.63_{-0.02}^{+0.04}$ &$0.72_{-0.01}^{+0.02}$ \\
%2MASXJ01073963-1139117 & $1.8750$ & $0.70$ & $11.03_{-0.03}^{+0.05}$ & $10.86_{-0.01}^{+0.03}$ & $10.53_{-0.11}^{+0.11}$ &$0.32_{-0.06}^{+0.06}$ \\
%2MASXJ03305218+0538253 & $1.5625$ & $0.95$ & $11.07_{-0.02}^{+0.03}$ & $10.49_{-0.04}^{+0.03}$ & $10.93_{-0.03}^{+0.04}$ &$0.73_{-0.02}^{+0.02}$ \\
%2MASXJ03342453-1513402 & $1.8125$ & $0.50$ & $10.73_{-0.03}^{+0.04}$ & $10.66_{-0.03}^{+0.02}$ & $9.93_{-0.09}^{+0.19}$ &$0.16_{-0.02}^{+0.06}$ \\
%\bottomrule
%\end{tabular}
%\begin{tablenotes}
%\item Notes -- \textit{Column 1:} Name of the source. \textit{Column 2:} Best fit slope of the intensity distribution for the interstellar radiation field. \textit{Column 3:}  Best fit MIR AGN fraction. \textit{Column 4:} Log of the total infrared luminosity from 8--1000 \micron{} in solar units. \textit{Column 5:} Log of the MBB component luminosity in solar units. \textit{Column 6:} Log of the powerlaw component luminosity in solar units. \textit{Column 7:} Fractional contribution of the AGN to the total infrared luminosity. The full version of this table is available in the online publication.\end{tablenotes}
%\end{threeparttable}
%\end{table*}
%%%%%%%%%%%%%%%%%%%%%%%%%%%%%%%%%%%%%%%%%%%%%%%%%%%%%%%%%%%%%%%%%

%%%%%%%%%%%%%%%%%%%%%%%%%%%%%%%%%%%%%%%%%%%%%%%%%%


% Don't change these lines
\bsp	% typesetting comment
\label{lastpage}
\end{document}