% Paper to publish our results from the IR SED fitting of the
% BAT AGN.
% Basic LaTex structure from MNRAS template on Overleaf
\documentclass[fleqn, usenatbib]{mnras}
\usepackage{newtxtext,newtxmath}
% Depending on your LaTeX fonts installation, you might get better results with one of these:
%\usepackage{mathptmx}
%\usepackage{txfonts}

\usepackage[T1]{fontenc}
\usepackage{ae,aecompl}


%%%%% AUTHORS - PLACE YOUR OWN PACKAGES HERE %%%%%

% Only include extra packages if you really need them. Common packages are:
\usepackage{graphicx}	% Including figure files
\usepackage{amsmath}	% Advanced maths commands
\usepackage{amssymb}	% Extra maths symbols
\usepackage{amsfonts}
\usepackage[flushleft]{threeparttable}
\usepackage{booktabs}
\usepackage{longtable}
\usepackage{pdflscape}
\usepackage{caption}
\usepackage{morefloats}
%%%%%%%%%%%%%%%%%%%%%%%%%%%%%%%%%%%%%%%%%%%%%%%%%%

%%%%% AUTHORS - PLACE YOUR OWN COMMANDS HERE %%%%%

% Please keep new commands to a minimum, and use \newcommand not \def to avoid
% overwriting existing commands. Example:
%\newcommand{\pcm}{\,cm$^{-2}$}	% per cm-squared
\newcommand{\herschel}{\emph{Herschel}}
\newcommand{\swift}{\textit{Swift}}
\newcommand{\msun}{M$_{\sun}$}
\newcommand{\mstar}{$M_{\mathrm{star}}$}
\newcommand{\lsun}{L$_{\sun}$}
\newcommand{\mdust}{$M_{\rm dust}$}
\newcommand{\tdust}{$T_{\rm dust}$}
%%%%%%%%%%%%%%%%%%%%%%%%%%%%%%%%%%%%%%%%%%%%%%%%%%

%%%%%%%%%%%%%%%%%%% TITLE PAGE %%%%%%%%%%%%%%%%%%%

% Title of the paper, and the short title which is used in the headers.
% Keep the title short and informative.
\title[\emph{Herschel}-BAT Sample: AGN-SF Connection]{\emph{Herschel} far-infrared photometry of the Swift Burst Alert
Telescope active galactic nuclei sample of the local universe--III. Global star-forming properties and the lack of a connection to nuclear activity\thanks{{\it Herschel} is an ESA space observatory with science instruments provided by European-led Principal Investigator consortia and with important participation from NASA.}}

% The list of authors, and the short list which is used in the headers.
% If you need two or more lines of authors, add an extra line using \newauthor
\author[T. T. Shimizu]{T. Taro Shimizu$^{1}$\thanks{Email: tshimizu@astro.umd.edu}, Richard F. Mushotzky$^1$, Marcio Mel\'endez$^{1,2,3}$, Michael J. Koss$^{4}$, \newauthor Amy J. Barger$^{5,6,7}$, and Lennox L. Cowie$^{7}$\\
$^{1}$Department of Astronomy, University of Maryland, College Park, MD 20742, USA\\
$^{2}$NASA Goddard Space Flight Center,  Greenbelt, MD  20771, USA\\
$^{3}$KBRwyle Science, Technology and Engineering Group, 1290 Hercules Avenue  Houston, TX  77058, USA\\ 
$^{4}$Institute for Astronomy, Department of Physics, ETH Zurich, Wolfgang-Pauli-Strasse 27, CH-8093 Zurich, Switzerland\\
$^{5}$Department of Astronomy, University of Wisconsin-Madison, 475 N. Charter Street, Madison, WI 53706, USA\\
$^{6}$Department of Physics and Astronomy, University of Hawaii, 2505 Correa Road, Honolulu, HI 96822, USA\\
$^{7}$Institute for Astronomy, University of Hawaii, 2680 Woodlawn Drive, Honolulu, HI 96822, USA}

% These dates will be filled out by the publisher
\date{Accepted XXX. Received YYY; in original form ZZZ}

% Enter the current year, for the copyright statements etc.
\pubyear{2016}

% Don't change these lines
\begin{document}
\label{firstpage}
\pagerange{\pageref{firstpage}--\pageref{lastpage}}
\maketitle

% Abstract of the paper
\begin{abstract}
We combine the \herschel{} \textit{Space Observatory} PACS and SPIRE photometry with archival \textit{WISE} photometry to construct the spectral energy distributions (SED) for over 300 local ($z < 0.05$), ultra-hard X-ray (14--195 keV) selected active galactic nuclei (AGN) from the \swift{} \textit{Burst Alert Telescope} (BAT) 58 month catalogue. Using a simple analytical model that combines an exponentially cut-off powerlaw with a single temperature modified blackbody, we decompose the SEDs into a host-galaxy and AGN component. We calculate dust masses, dust temperatures, and star-formation rates (SFR) for our entire sample and compare them to a stellar mass-matched sample of local non-AGN galaxies. We find AGN host galaxies have systematically higher dust masses, dust temperatures, and SFRs due to the higher prevalence of late-type galaxies to host an AGN, in agreement with previous studies of the \swift/BAT AGN. We provide a scaling to convert X-ray luminosities into 8--1000 \micron{} AGN luminosities, as well as determine the best mid-to-far IR colors for identifying AGN dominated galaxies in the IR regime. We find that for nearly 30 per cent of our sample, the 70 \micron{} emission contains a significant contribution from the AGN ($> 0.5$), especially at higher luminosities ($L_{14-195\,\rm{keV}} > 10^{42.5}$ ergs s$^{-1}$). Finally, we measure the local SFR-AGN luminosity relationship, finding a slope of 0.18, large scatter (0.37 dex), and no evidence for an upturn at high AGN luminosity. We conclude with a discussion on the implications of our results within the context of galaxy evolution with and without AGN feedback.
\end{abstract}

% Select between one and six entries from the list of approved keywords.
% Don't make up new ones.
\begin{keywords}
galaxies: active -- galaxies: Seyfert -- infrared: galaxies -- galaxies: star formation -- galaxies: evolution 
\end{keywords}

%%%%%%%%%%%%%%%%%%%%%%%%%%%%%%%%%%%%%%%%%%%%%%%%%%

%%%%%%%%%%%%%%%%% BODY OF PAPER %%%%%%%%%%%%%%%%%%

\section{Introduction}
Ever since the discoveries that the large scale properties of galaxies are related to the mass of the supermassive black holes (SMBH) they host \citep{Magorrian:1998qv, Ferrarese:2000gf, Gebhardt:2000xy, Marconi:2003ve, Haring:2004ly, Gultekin:2009ul, McConnell:2013fk, Kormendy:2013fj}, studies have intensely focused on finding the link between the evolution and growth of both the galaxy and SMBH. Theoretical arguments \citep[eg.][]{Silk:1998qf}, as well as cosmological simulations  \citep[e.g][]{Croton:2006kx, Bower:2006gf} support the idea that SMBHs, while accreting material in their active galactic nuclei (AGN) phase, can affect both their own growth and the growth of their host galaxy through a mechanism that shuts down accretion and star formation (i.e. negative feedback). 

%Active galactic nuclei (AGN), accreting supermassive black holes at the centers of galaxies, are energetically capable of affecting the large-scale evolution of their host galaxies. Indirect evidence for this occurring over the lifespan of a galaxy lies in the observed relationships between the supermassive black hole mass and various host galaxy properties such as the stellar velocity dispersion \citep{Magorrian:1998qv, Ferrarese:2000gf, Gebhardt:2000xy, Gultekin:2009ul, McConnell:2013fk} and bulge mass \citep{Marconi:2003ve, Haring:2004ly, McConnell:2013fk, Kormendy:2013fj}. %More detailed observations over a larger range in host galaxy properties reveals the connection between supermassive black hole growth and galaxy growth is likely complicated \citep{Kormendy:2013fj} especially when considering bulgeless or pseudobulge galaxies. Nevertheless, these correlations have motivated many studies attempting to understand the coevolution of black holes and galaxies.

%The prevailing theory linking the host galaxy and its supermassive black hole involves negative AGN feedback whereby the energy released by the AGN both shuts down star formation in the host galaxy, as well as accretion onto the supermassive black hole \citep[see][for reviews]{Fabian:2012dp, Alexander:2012fr}. This results in a form of self-regulation where the growth of the galaxy through star formation is inherently linked to the growth of the supermassive black hole through accretion. 

Early evidence backing this theory seemed to be found in the observation that the SFR density and accretion rate density of the universe evolve similarly with redshift  \citep{Boyle:1998vn, Franceschini:1999ys, Silverman:2008zl, Silverman:2009lq, Aird:2010fr, Merloni:2013zr}. Further, AGN host galaxies were found to predominantly lie in the ``green valley''  of the color-magnitude diagram, in between the blue, star-forming galaxies and ``red-and-dead'' quiescent ones \citep{Martin:2007yg, Nandra:2007rz, Silverman:2008zl, Schawinski:2010zr, Koss:2011vn}, suggesting that nuclear activity plays a role in transitioning its host galaxy from star-forming to quiescence. 

While suggestive, optical colors can be strongly affected by extinction due to dust that might move a galaxy from ``blue'' to ''green'' \citep{Cowie:2008qy, Cardamone:2010uq}. Other measures of the SFR including emission line, ultra-violet (UV), and mid-infrared (MIR) luminosities that are extensively used for normal, non-AGN galaxies suffer from varying degrees of contamination in the presence of an AGN, especially unabsorbed ($N_{\rm H} < 10^{22}$ cm$^{-2}$) Type I AGN.

%Naturally, researchers turned to studies of star formation in AGN host galaxies looking for direct relationships between the strength of the AGN and SFR.  However, an issue in performing these studies involves the measurement of the host galaxy SFR. Emission line, ultra-violet (UV), and mid-infrared (MIR) indicators of the SFR that are extensively used for normal, non-AGN galaxies suffer from varying degrees of contamination in the presence of an AGN. 

This problem was solved with the advent of far-infrared (FIR; $\lambda = 40-500$ \micron) telescopes, including the \textit{Infrared Astronomical Satellite} (\textit{IRAS}), the \textit{Infrared Space Observatory} (\textit{ISO}), the \textit{Spitzer Space Telescope}, and most recently the \herschel{} \textit{Space Observatory} \citep{Pilbratt:2010rz}. FIR emission remains the ideal waveband to study star formation in AGN host galaxies for two reasons: (1) the rapidly declining spectral energy distribution (SED) associated with AGN heated dust in the FIR regime \citep{Fritz:2006yq, Netzer:2007ve, Mullaney:2011yq, Shi:2014qr} ensures little AGN contamination, and (2) thermal emission of large grains heated by the ionizing emission of recently formed massive stars \citep{Lonsdale-Persson:1987uq, Devereux:1990yq} creates a strong FIR ``bump'' and provides a reliable measure of the recent (10--100 Myr) SFR \citep{Kennicutt:1998kx}.

Even though the FIR can provide a reliable SFR tracer, individual investigations into the connection between AGN activity and star-formation have produced conflicting and even contradictory results. Some studies \citep{Page:2012mz, Barger:2015ly} have reported lowered SFRs for the highest luminosity AGN, indicative of a suppression of star formation due to AGN feedback. Others have found weak or flat relationships \citep{Diamond-Stanic:2012rw, Mullaney:2012gf, Rovilos:2012wd, Rosario:2012fr, Azadi:2015ve, Stanley:2015qy}, especially at more moderate AGN luminosities. Finally, some find an overall positive correlation \citep{Lutz:2008ph, Netzer:2009lr, Rovilos:2012wd, Rosario:2012fr, Chen:2013uq, Dai:2015gf}. %\citet{Rosario:2012fr} provided evidence that at low AGN luminosity the relationship is flat and only at higher luminosities does a positive correlation emerge that matches the one found by \citet{Lutz:2008ph} and \citet{Netzer:2009lr}. They hypothesize that at high AGN luminosities and high SFRs major mergers begin to dominate the evolution and drive both enhanced star formation and AGN activity at the same time \citep{Sanders:1988fk, Hopkins:2006ys}. 

%A large complicating factor is the vast difference in timescales for the two competing processes. \citet{Hickox:2014yq} reconciled many of the seemingly contradictory results by recognizing that short-term (compared to the $\sim$ 100 Myr timescale for star formation) AGN variability could wash out any long term relationship between AGN activity and star formation. Just by assuming a linear relationship between the AGN luminosity and SFR and a probability distribution for observing an instantaneous AGN luminosity, \citet{Hickox:2014yq} was able to match the redshift evolution and flattening at low AGN luminosity of the AGN-SFR relationship found in \citet{Rosario:2012fr}. 

Many of the studies may however suffer from one or more problems that include small number statistics either due to a small area survey or low sensitivity, an inaccurate scaling from monochromatic luminosities to a total SFR, and source confusion due to the relatively large beam size of \herschel, especially at longer wavelengths. Further, as \citet{Barger:2015ly} pointed out, stacking and averaging analyses can overestimate the level of star formation taking place in the bulk of the X-ray sample because the true SFR distribution is unknown. 

% \citet{Stanley:2015qy} showed the importance of including information contained in non-detections for analyzing the connection between AGN luminosity and SFR, ultimately finding a very flat relationship that agreed with the model of \citet{Hickox:2014yq}.

%Beyond measurements of the SFR, extending the SED to the FIR provides a wealth of information on the host galaxy properties. The \herschel{} \textit{Space Observatory}  provided a great advancement in FIR observations which in the past was plagued by low sensitivity and poor angular resolution. With the SPIRE instrument, SEDs were extended into the submillimeter regime which are crucial for determining not only the overall IR luminosity but also the dominant dust temperatures and dust masses of galaxies. Dust masses, in particular, are quite uncertain without the inclusion of photometry longward of 100 \micron{} \citep{Gordon:2010ix}.

In this paper, we examine the AGN-star formation relationship in the low redshift universe. We have performed a \herschel{} survey of a sample of AGN from the \textit{Swift} Burst Alert Telescope 58 month catalogue, which selected sources over the entire sky in the 14--195 keV energy range. At this high of energy, we are nearly unbiased with respect to any host galaxy properties including SFR, as well as obscuration below a hydrogen column density ($N_{\rm H}$) of 10$^{24}$ cm$^{-2}$. The \swift/BAT AGN sample has been extensively studied ever since the release of the first catalogues. Most relevant to our work, \citet{Koss:2011vn} performed an optical survey of over 100 \swift/BAT AGN using combined \textit{Sloan Digital Sky Survey} (SDSS) and \textit{Kitt Peak National Observatory} imaging to study their host galaxy properties. They found that the \swift/BAT AGN are dominated by massive (\mstar $> 10^{9.5}$ \msun) spirals with bluer colors than a mass-matched sample of non-AGN galaxies. \citet{Koss:2011vn} further found that AGN showed enhanced FIR luminosities, using 90 \micron{} photometry from \textit{Akari}. 

With 5 band imaging with \herschel{} along with archival photometry from the \textit{Wide-field Infrared Survey Explorer} (\textit{WISE}), we have constructed MIR-FIR SEDs for over 300 AGN, allowing us to accurately measure the SFR in a robust manner. Because of the low-redshift nature of our sources, every AGN is easily able to be identified and there is no concern over source confusion. These SEDs not only allow us to calculate SFRs but also to test varying SED decomposition methods and models, as well as measure other properties of the host galaxies, such as the dust mass and dust temperature, to compare with non-AGN samples. With a high detection rate (95, 83, 86, 72, and 46 per cent at 70, 160, 250, 350, and 500 \micron), our measured properties from the SEDs are well constrained, removing much of the uncertainty due to censoring. 

The paper is organized as follows: in Section 2 we briefly describe the \herschel-BAT AGN sample; in Section 3 we detail our \textit{WISE} and \herschel{} datasets. Section 4 introduces our non-AGN comparison sample. Section 5 qualitatively examines the SEDs and determines the average SED as a function of AGN luminosity, while Section 6 outlines our fitting methods. Sections 7 and 8 then discusse our results, compares with other studies, and concludes the paper. Throughout, we use a cosmology with $H_{0}=70$ km s$^{-1}$ Mpc$^{-1}$, $\Omega_{\rm M}=0.3$, and $\Omega_{\Lambda} = 0.7$ to calculate luminosity distances from redshifts. 

\section{\herschel-BAT Sample}
Because \swift/BAT continuously monitors the entire sky in the energy range 14--195 keV for gamma ray bursts, it simultaneously provides an all-sky survey at ultra high X-ray energies. This allows for the creation of complete catalogues with increasing sensitivity the longer \swift/BAT remains in operation. Given the extreme environments necessary to produce strong 14--195 keV emission, the majority of sources in the \swift/BAT catalogues are AGN.

We chose 313 AGN from the parent sample of $\sim720$ AGN detected in the 58 month catalogue\footnote{\url{https://swift.gsfc.nasa.gov/results/bs58mon}} after imposing a redshift cutoff of $z<0.05$  and excluding Blazars and BL Lac objects to form our \herschel-BAT AGN sample. With a mean redshift of $\left<z\right> = 0.025$, our AGN sample provides a comprehensive view of the properties of AGN host galaxies in the local universe. Our selection at ultra high X-ray energies further removes biases and selection effects due to host galaxy contamination and obscuration \citep{Mushotzky:2004gf} that can influence samples at other wavelengths.   

The demographics of our sample are nearly evenly split between Type I (43 per cent) and Type II (53 per cent) with the remaining 4 per cent (5 objects) either a Low-Ionization Nuclear Emission-line Region (LINER) or unclassified. The \herschel-BAT sample spans nearly four orders of magnitude in 14--195 keV luminosity ($10^{41} < L_{\rm 14-195\,keV} < 10^{45}$ ergs s$^{-1}$) allowing for a robust determination of the connection between AGN strength and SFR. For a complete listing of our sample with names, luminosity distances, redshifts, and AGN type, we point the reader to \citet{Melendez:2014yu} or \citet{Shimizu:2016qy}.

\section{Data}
\subsection{\herschel{} Photometry}
293 of the \herschel{}-BAT AGN were observed with \herschel{} as part of a Cycle 1 open time program (OT1\_rmushotz\_1, PI: Richard Mushotzky). The remaining 20 sources were part of other programs with public archival data. \herschel{} observed all 313 AGN using both the Photoconductor Array Camera and Spectrometer \citep[PACS;][]{Poglitsch:2010fp} and Spectral and Photometric Imaging Receiver \citep[SPIRE;][]{Griffin:2010sf} producing images in five wavebands: 70, 160, 250, 350, and 500 \micron.

Detailed descriptions of the data reduction process and photometric flux extraction can be found in \citet{Melendez:2014yu} for PACS and \citet{Shimizu:2016qy} for SPIRE. The following is a short description of the flux extraction procedure. We measured fluxes at each waveband directly from the images using aperture photometry with a concentric annulus to define the local background. We applied aperture corrections for sources where we used a point source aperture as defined in the respective PACS and SPIRE data reduction guides. Fluxes for sources that were unresolved at all three wavebands in SPIRE were determined using the SPIRE Timeline Fitter within the \herschel{} Interactive Processing Environment. All fluxes have a signal-to-noise ratio of at least 5, otherwise we provided a 5$\sigma$ upper limit. 

In \citet{Shimizu:2016qy}, using the SPIRE flux ratios, we found 6 radio-loud objects, which have a large contribution to their FIR emission from synchrotron radiation due to their jets. These sources are Pictor A, 3C 111.0, 3C 120, 2MASX J23272195+1524375, PKS2331-240, and [HB89] 0241+622. For this work we remove these sources from all analysis; however, we still provide figures of their SED fits in Appendix D and list their best-fit parameters in Table~\ref{tab:c12_params}. 

Also in \citet{Shimizu:2016qy}, we flagged objects with a ``D'' that likely had strong contamination due to a nearby source. For these sources, we treat the SPIRE fluxes as upper limits. For a small number of very nearby galaxies with large angular size (NGC 2655, NGC 3718, NGC 4939, NGC 4941, NGC 5033, and NGC 6300), we realized that the PACS FOV of our observations did not encompass the entirety of the galaxy leading to possible underestimations of the 70 and 160 \micron{} flux densities. Therefore, for this work, we chose to conservatively increase the uncertainty to 50 per cent.

\subsection{\textit{WISE} Photometry}
To extend the SEDs into the MIR, we supplemented our \herschel{} data with archival \textit{Wide-field Infrared Survey Explorer} \citep[WISE;][]{Wright:2010fk} photometry. WISE performed a broadband all-sky survey at 3.4 (W1), 4.6 (W2), 12 (W3), and 22 (W4) \micron{} with angular resolution comparable to \herschel/PACS at 70 \micron{} for W1, W2, and W3 and 160 \micron{} for W4. We queried the AllWISE catalogue through the NASA/IPAC Infrared Science Archive\footnote{\url{http://irsa.ipac.caltech.edu/frontpage/}} to search for coincident sources within 6\arcsec. Counterparts were found for all but one AGN (Mrk 3) at every waveband. The catalogue only contained W1 and W2 fluxes for Mrk 3 due to differences in the depth of coverage for W1/W2 and W3/W4, therefore Mrk 3 is not included as part of the sample in this work.

The AllWISE catalogue provides magnitudes determined using multiple extraction methods. We consider only the profile-fitting magnitudes (\textit{w}N\textit{mpro} where N is 1, 2, 3, or 4) and the elliptical aperture magnitudes (\textit{w}N\textit{gmag}). Profile-fitting magnitudes were determined by fitting the position dependent point spread function using deblending procedures when necessary to decompose overlapping sources. The \textit{w}N\textit{mpro} magnitudes therefore are only relevant for unresolved sources.

If a WISE source is associated with a source in the \textit{Two Micron All Sky Survey} (2MASS) Extended Source Catalog (XSC), then \textit{w}N\textit{gmag} magnitudes were also measured using an elliptical aperture with the same shape from the XSC and sizes scaled given the larger WISE beam. Thus, \textit{w}N\textit{gmag} magnitudes are more appropriate for extended sources. For details of all of the WISE magnitude measurements we point the reader to the All-Sky Release Explanatory Supplement\footnote{\url{http://wise2.ipac.caltech.edu/docs/release/allsky/expsup/}}.

Given the low-redshift nature of our sample, {\color{red}294/313 galaxies are in the 2MASS XSC and} using only the \textit{w}N\textit{mpro} magnitudes would severely underestimate the flux for many of them. To decide which magnitude to include in the SED for each source, we used the reduced $\chi^{2}$ value (\textit{w}N\textit{rchi2}) from the profile-fitting. If \textit{w}N\textit{rchi2}$<3$ then we chose the \textit{w}N\textit{mpro} magnitude, otherwise \textit{w}N\textit{gmag} was chosen. For 55/313 sources, we used the W1 profile-fitting magnitudes; for 129/313 sources, we used the W2 profile-fitting magnitudes; for 210/313 sources, we used the W3 profile-fitting magnitudes; and for 267/313 sources, we used the W4 profile-fitting magnitudes. {\color{red}The increasing numbers of sources for which we use the profile-fitting magnitudes is due to the increasing FWHM of the WISE point spread function which changes from 6\arcsec at W1 to 12\arcsec at W4.}

\section{Comparison Sample}
To test whether the AGN has any effect on the star-forming properties of their host galaxies, we need samples of galaxies which contain little evidence for nuclear activity. These samples also need to occupy the same redshift range to mitigate against evolutionary effects and have been observed with nearly the same instruments so the properties can be compared on an equal level. Three low-redshift samples exist that satisfy these constraints. They are the  \herschel{} \textit{Reference Survey} \cite[HRS;][]{Boselli:2010fj}, \textit{Key Insights on Nearby Galaxies: a Far-Infrared Survey with Herschel} \cite[KINGFISH;][]{Kennicutt:2011vn}, and the \herschel{} \textit{Stripe 82 Survey} \citep[HerS;][]{Viero:2014jk}. While the KINGFISH sample has been observed at all of the same wavelengths as the \herschel-BAT AGN, the sample was selected in a heterogeneous manner and only contains 61 galaxies, which does not allow for stellar mass matching. The HerS sample is numerous and covers a large range in mass, but was only observed with SPIRE. Further, the catalog was built based on SPIRE detections, which imposes a selection effect associated with the SFR. We therefore chose the HRS as our comparison sample to the \herschel-BAT AGN.

The HRS was a guaranteed time key \herschel{} program dedicated to studying the dust content of ``normal'' galaxies. The 323 galaxy sample is volume limited ($15< D < 25$ Mpc) to avoid distance effects and K-band flux limited to avoid selection effects due to dust and provide a representative population of local galaxies. The size of the HRS, as well as the local nature make it an ideal sample to compare to the \herschel-BAT galaxies.

The HRS galaxies also were imaged using both PACS and SPIRE, although the 100 \micron{} filter was used instead of the 70 \micron{} filter. \citet{Cortese:2014qq} and \citet{Ciesla:2012lq} measured the PACS and SPIRE flux densities respectively using similar techniques as the \herschel-BAT galaxies. WISE 12 and 22 \micron{} photometry for HRS were provided in \citet{Ciesla:2014qy}. The available data for the HRS SEDs are nearly identical to our sample. 

The only issue in comparing the HRS galaxies to the \herschel-BAT AGN concerns the stellar mass (\mstar) distribution. Because the near-infrared is most strongly effected by the mass of the older stellar population, the HRS K-band selection produces a \mstar{} distribution that is representative of the naturally occurring \mstar{} distribution. However, many recent studies have shown that AGN prefer high \mstar{} galaxies \citep[e.g.][]{Schawinski:2010zr,Xue:2010fj}, a feature that is also found in the BAT AGN \citep{Koss:2011vn}. Figure~\ref{fig:mstar_dist} displays the \mstar{} distributions for the \herschel-BAT and HRS sample (dashed line) represented using a Kernel Density Estimate (KDE)\footnote{\color{red}Instead of the standard histogram, we choose to visualize distributions in this work with KDEs. KDEs are constructed by first representing each data point in the sample by a specific kernel which are then summed together. We use a Gaussian kernel with a width according to ``Scott's Rule''  \citep{Scott:1992xy}, $width=N^{-1/5}$, where $N$ is the number of data points. The appearance of a histogram can change based on the number of bins, bin width, and bin centers whereas a KDE is only dependent on the kernel width.} The \herschel-BAT galaxies have an average \mstar{} of 10.6 \msun, whereas the average \mstar{} for HRS is 9.8 \msun, nearly a ten-fold difference. 

To account for possible stellar mass effects, we decided to produce a mass-matched HRS sample in the following way. Stellar masses for the \herschel-BAT AGN were calculated based on the AGN-subtracted \textit{ugriz} photometry from \citet{Koss:2011vn} and the $g-i$ calibrated stellar mass relation from \citet{Zibetti:2009jf}. This is the same method that stellar masses were calculated for the HRS sample, given in \citet{Cortese:2012fj}. For each of the 122 \herschel-BAT AGN which have stellar mass estimates, we chose all HRS galaxies which have a stellar mass within 0.15 dex of the \herschel-BAT mass. We then randomly selected an HRS galaxy from this pool. Because high-mass galaxies are relatively rare in the HRS sample, some matches are duplicated. We find that 97/122 of the matched HRS galaxies are unique, representing 80 per cent of the sample, so the duplicating effects should not be large. The red solid line in Figure~\ref{fig:mstar_dist} now shows the mass-matched HRS sample, which overlaps with the \herschel-BAT mass distribution. Throughout the rest of this paper, we refer to the HRS mass-matched sample as simply the HRS sample unless otherwise noted.
 
\begin{figure}
\includegraphics[width=\columnwidth]{figures/stellar_mass_comp_hrs_match}
\caption{Kernel Density Estimates (KDE) of the \mstar{} distribution for the BAT AGN (blue), full HRS sample (red), and mass-matched HRS sample (red dashed). The \herschel-BAT AGN probe a higher \mstar{} galaxy population than the full samples; however, with mass-matching we can reproduce the stellar-mass distribution of the \herschel-BAT AGN with non-AGN. A color version is available in the online publication. \label{fig:mstar_dist}}
\end{figure}

%\subsection{\emph{Spitzer}/IRS spectra}
%To supplement the IR photometry, we searched the Cornell Atlas of Spitzer IRS Sources (CASSIS) (REFERENCE) for low-resolution Spitzer/IRS spectra. These MIR spectra span the 5-30 \micron{} wavelength range and can provide independent measures of the SFR and AGN fraction through the strength of emission features from different species such as polycyclic aromatic hydrocarbons (PAH), NeII, NeIII, and OIV. We can use the luminosity and equivalent width (EqW) of these features to test the reliability and accuracy of the properties from the broadband SED fitting. 
%
%We can also test whether AGN heating has an effect on the dust composition of galaxies. Some studies (REFERENCES) have suggested that the hard ionizing radiation from the AGN has destroyed PAHs while others report no effect from the AGN. With our large sample covering a range of AGN luminosity and an independent measure of the SFR, we can definitively determine whether the AGN has an effect on the PAH abundance of galaxies.
%
%We found 120/313 AGN on CASSIS and downloaded their reduced spectra using the default extraction method. Due to the different sizes of the Short-Low (SL) slit and Long-Low (LL) slit, the continuum of each can be offset mainly from extended emission not captured by the SL slit. Therefore spectral orders were stitched together by fitting a line to the ends of each order and scaling to match the overlapping regions. SL2 and LL2 were first matched and scaled to SL1 and LL1 respectively. Then the combined SL1/SL2 spectra was matched to the combined LL1/LL2 spectra to produce the final full Spitzer/IRS low resolution spectra. 

%\subsection{KINGFISH Sample}
%While the HRS galaxies were selected homogeneously, the KINGFISH galaxies are a heterogeneous sample selected to span the broad range of star-forming properties and ISM environments from dwarf galaxies to massive star-forming galaxies. Only 61 galaxies comprise the sample, 57 of which were part of the \textit{Spitzer} Infrared Nearby Galaxies Survey \citep[SINGS;][]{Kennicutt:2003wk}.

%The KINGFISH galaxies were imaged at every \herschel{} waveband (70, 100, 160, 250, 350, and 500 \micron) using aperture photometry with flux densities provided by \citet{Dale:2012dq}. To cover the MIR portion of the SED, we used the \textit{Spitzer}/IRAC and MIPS 8 and 24 \micron{} data from \citet{Dale:2007fk}. Because the MIR data is from the SINGS survey, this reduces the KINGFISH sample from 61 to 57 galaxies. 

%Various miscellaneous properties about each KINGFISH galaxy, including \mstar, were compiled from \citet{Kennicutt:2011vn}. We removed all galaxies with $\log($\mstar$)<8.0$ \msun{} because all are dwarf galaxies that are not represented in the \herschel-BAT sample. The final \mstar{} distribution for the KINGFISH comparison sample is shown in Figure~\ref{fig:mstar_dist}. While the KINGFISH galaxies are located at slightly higher masses (mean $\log($\mstar$)=9.9$ \msun), the \herschel-BAT sample is still biased significantly towards higher stellar masses. 

%An advantage of the KINGFISH sample is the availability of \textit{Spitzer}/IRS low-resolution spectra which were analyzed by \citet{Smith:2007lr}. The MIR emission feature measurements will be important for comparing their fluxes and luminosities to the FIR SED measured properties and assessing the effect of the AGN. 

\section{The IR SEDs of $z=0$ AGN}\label{sec:ir_seds}

Before fitting the SEDs and comparing to a non-AGN sample, we begin with a qualitative look at the SEDs of the 313 AGN. Figure~\ref{fig:seds} shows the 12--500 \micron{} SED for every source in the \herschel-BAT sample. Immediately noticeable is the varied SED shapes. Some of the SEDs (e.g. 2MASX J07595347+2323241, Cen A, ESO 005-G004, NGC 4051, NGC 4138, NGC 6814) feature a very prominent FIR bump that is recognizable in nearly all star-forming galaxies and is indicative of thermally heated dust from recently formed massive stars, as well as cirrus emission heated by an older population. The general shape is usually well fit by a blackbody modified by a frequency dependent optical depth \citep[e.g.][]{Calzetti:2000fk, Smith:2012fj, Bianchi:2013jk, Symeonidis:2013fe, Cortese:2014qq}.

These star formation dominated objects represent one extreme end of the shapes we observe. The other extreme are SEDs whose shapes seem to peak short-ward of 70 \micron{} and display rapidly falling emission with increasing wavelength. Examples include 2MASX J06561197-4919499, 2MASX J210990996-0940147, ESO 103-035, IC 4329A, MCG -05-23-016, and Mrk 335. Emission in the MIR dominates these SEDs and is likely associated with hotter dust heated by the AGN. MIR colors are commonly used to select AGN samples because they display strong red (i.e. increasing SED with increasing wavelength shortward of 70 \micron) colors compared to non-AGN galaxies \citep{Lacy:2004uq, Donley:2012qy, Stern:2012mz} due to the dust in the obscuring torus heated by the optical and UV emission from the accretion disk. Monochromatic MIR luminosities, especially for sources dominated by an unresolved central component, show near linear correlations with the X-ray luminosity \citep[e.g][]{Lutz:2004fj, Gandhi:2009kx, Asmus:2012qy} providing more evidence that much of the MIR in AGN host galaxies is associated with the AGN. 

The remaining \herschel-BAT AGN display SEDs somewhere in between MIR-dominated and FIR-dominated as a result of competing contributions between the AGN and star formation. The varied shapes of the SED emphasizes the need for SED decomposition, especially at shorter wavelengths, to accurately determine the IR luminosity associated with either star formation or the AGN. 

If star formation suppression occurs at high AGN luminosity then we expect the IR SEDs for the most X-ray luminous sources to resemble the MIR-dominated SEDs, as \citet{Barger:2015ly} observed. To test this, we binned the SEDs by logarithmic 14--195 keV luminosity. Five bins ( $\log\,L_{\rm 14-195\,keV} > 42.5$, $\log\,L_{\rm 14-195\,keV} = 42.5-43.0$, $43.0-43.5$, $43.5-44.0$, and $\log\,L_{\rm 14-195\,keV} > 44.0$) were chosen with 0.5 dex widths to ensure enough sources occupied each one. The number of sources in each bin is 22, 39, 94, 116, and 35. 

We then calculated the median luminosity density {\color{red}for each waveband in the SED} within each X-ray luminosity bin . We used the ASURV \citep{Feigelson:1985lr} package, which applies the principles of survival analysis to astronomical data. Information contained in the upper limits can be then included in the measurement of statistical properties of samples without biasing results towards brighter sources that are more likely to be detected. Specifically, ASURV calculates the non-parametric Kaplan-Meier product-limit (KMPL) estimator for a sample distribution. The KMPL estimator is an estimate of the survival function, which is simply 1$-$CDF (cumulative distribution function). Using the KMPL, we calculate for each waveband the median luminosity density (50th percentile) and estimate the uncertainty using the 16th and 84th percentiles. 

Figure~\ref{fig:avg_sed} shows the median IR SED for sources in each X-ray luminosity bin. From visual inspection, the median SED as a function of X-ray luminosity seems to be driven more by an increasing MIR hot dust component. There does not seem to be any indication that star formation is being suppressed at the highest luminosities. The wavebands that show the largest change with X-ray luminosity are the MIR bands. At wavelengths longer than 160 \micron, the SED shows no significant change, consistent with the correlations we measured in \citet{Melendez:2014yu} and \citet{Shimizu:2016qy}. If we make the assumption that the longer wavelengths are mainly associated with star-formation, then the lack of change indicates star-formation{\color{red}, on timescales of $\sim$100 Myr,} is unrelated to X-ray luminosity within the range probed by our sample. However, to definitively test this we need to decompose the SED into star-forming and AGN components to accurately calculate SFRs.

\begin{figure*}
\centering
\includegraphics[width=0.75\textwidth]{figures/avg_sed_binned_lx_lnu}
\caption{Median IR SEDs in units of $10^{30}$ ergs s$^{-1}$ Hz$^{-1}$ for the \herschel-BAT AGN binned by $L_{\rm 14-195\,keV}$. Each point in the SEDs is the median $L_{\nu}$ for the sources in the bin including upper limits calculated using the Kaplan-Meier product-limit estimator. The error bars represent the 68 per cent confidence interval determined from the 16th and 84th percentiles. SEDs have been slightly offset by 0.02 dex in wavelength so the error bars can be clearly shown. \label{fig:avg_sed}}
\end{figure*}

\section{SED Fitting}\label{sec:sed_fit}

%%%%%%%%%%%%%%%%%%%%% TABLE OF C12 PARAMETERS %%%%%%%%%%%%%%%%%%%%%%%%%%
%\begin{landscape}
\begin{table*}
\centering
\begin{threeparttable}
\captionsetup{font=small,labelfont=bf,labelsep=period}
\caption{Best Fit C12 Model Parameters, Luminosities, and AGN Fractions \label{tab:c12_params}}
\begin{tabular}{lcccccccc}
\toprule 
Name & $\log\,M_{\rm dust}$ &$T_{\rm dust}$ & $\alpha$ & $\lambda_{\rm c}$ & $\log\,L_{\rm IR}$ & $\log\,L_{\rm SF}$ & $\log\,L_{\rm AGN,IR}$ &$f_{\rm AGN}$ \\
 & [\msun] & [K] &  & [\micron] & [\lsun] & [\lsun] & [\lsun] & \\
\midrule
1RXSJ044154.5-082639 & $6.80_{-0.14}^{+0.21}$ & $27.98_{-4.21}^{+2.45}$ & $1.70_{-0.34}^{+0.46}$ &$47.71_{-13.96}^{+18.85}$ & $10.37_{-0.04}^{+0.03}$ & $9.99_{-0.23}^{+0.11}$ & $9.99_{-0.23}^{+0.11}$ & $0.58_{-0.14}^{+0.18}$ \\
1RXSJ045205.0+493248 & $7.47_{-0.10}^{+0.12}$ & $23.27_{-1.61}^{+0.99}$ & $1.44_{-0.46}^{+0.59}$ &$39.95_{-13.09}^{+14.82}$ & $10.48_{-0.04}^{+0.03}$ & $10.19_{-0.08}^{+0.04}$ & $10.19_{-0.08}^{+0.04}$ & $0.50_{-0.10}^{+0.10}$ \\
2E1739.1-1210 & $7.55_{-0.19}^{+0.22}$ & $25.11_{-3.00}^{+2.11}$ & $1.46_{-0.37}^{+0.51}$ &$45.22_{-14.16}^{+16.42}$ & $10.83_{-0.04}^{+0.03}$ & $10.46_{-0.14}^{+0.07}$ & $10.46_{-0.14}^{+0.07}$ & $0.58_{-0.11}^{+0.12}$ \\
2MASSJ07594181-3843560 & $<6.58$ & ... & $1.08_{-0.46}^{+0.51}$ &$39.20_{-5.56}^{+7.93}$ & $<10.63$ & $<9.63$ & $>10.59$ & $>0.90$ \\
2MASSJ17485512-3254521 & $<6.14$ & ... & $1.45_{-0.47}^{+0.56}$ &$44.52_{-16.37}^{+18.07}$ & $<9.42$ & $<9.05$ & $>8.99$ & $>0.51$ \\
2MASXJ00253292+6821442 & $6.16_{-0.18}^{+0.39}$ & $25.34_{-5.06}^{+2.79}$ & $1.46_{-0.40}^{+0.53}$ &$45.40_{-15.93}^{+16.38}$ & $9.63_{-0.05}^{+0.04}$ & $9.11_{-0.18}^{+0.10}$ & $9.11_{-0.18}^{+0.10}$ & $0.70_{-0.11}^{+0.11}$ \\
2MASXJ01064523+0638015 & $6.81_{-0.44}^{+0.69}$ & $21.56_{-7.50}^{+9.09}$ & $1.86_{-0.42}^{+0.53}$ &$44.58_{-9.10}^{+10.57}$ & $10.47_{-0.05}^{+0.04}$ & $<9.95$ & $>10.24$ & $>0.66$ \\
2MASXJ01073963-1139117 & $7.66_{-0.06}^{+0.09}$ & $25.52_{-1.96}^{+1.27}$ & $2.05_{-0.49}^{+0.58}$ &$41.53_{-10.35}^{+20.19}$ & $10.87_{-0.03}^{+0.03}$ & $10.62_{-0.10}^{+0.07}$ & $10.62_{-0.10}^{+0.07}$ & $0.44_{-0.12}^{+0.13}$ \\
2MASXJ03305218+0538253 & $6.76_{-0.33}^{+0.69}$ & $28.20_{-8.76}^{+6.57}$ & $2.46_{-0.61}^{+0.78}$ &$33.77_{-9.67}^{+8.47}$ & $10.81_{-0.06}^{+0.06}$ & $9.99_{-0.28}^{+0.23}$ & $9.99_{-0.28}^{+0.23}$ & $0.85_{-0.13}^{+0.10}$ \\
2MASXJ03342453-1513402 & $7.43_{-0.04}^{+0.05}$ & $26.61_{-1.09}^{+0.61}$ & $1.69_{-0.44}^{+0.58}$ &$43.00_{-16.10}^{+24.14}$ & $10.59_{-0.03}^{+0.03}$ & $10.51_{-0.05}^{+0.03}$ & $10.51_{-0.05}^{+0.03}$ & $0.18_{-0.10}^{+0.13}$ \\\bottomrule
\end{tabular}
\begin{tablenotes}
\item Notes -- \textit{Column 1:} Name of the source. \textit{Column 2:} Log of the dust mass in solar units. \textit{Column 3:} Dust temperature for the MBB component. '...' indicates the dust temperature was fixed at 23 K. \textit{Column 4:} Slope of the powerlaw component. \textit{Column 5:} Turnover wavelength. \textit{Column 6:} Log of the total infrared luminosity from 8--1000 \micron{} in solar units. \textit{Column 7:} Log of the MBB component luminosity in solar units. \textit{Column 8:} Log of the powerlaw component luminosity in solar units. \textit{Column 9:} Fractional contribution of the AGN to the total infrared luminosity calculated using Equations~\ref{eq:c12_lagnIR}~and~\ref{eq:c12_fagn}. The full version of this table is available in the online publication.
\end{tablenotes}
\end{threeparttable}
\end{table*}
%\end{landscape}
%%%%%%%%%%%%%%%%%%%%%%%%%%%%%%%%%%%%%%%%%%%%%%%%%%%%%%%%%%%%%%%%

Many models and templates exist in the literature to fit the broadband SEDs of galaxies. We chose a model that allowed us to both decompose the SED into star-forming and AGN components, as well as provide estimates on the dust temperature and dust mass.

\subsection{Casey 2012 Model}
One of the most widely used models for fitting the FIR SED of galaxies is a single modified blackbody (MBB). The simple model consists of a normal, single temperature blackbody that represents isotropic dust emission combined with a frequency dependent opacity given that dust is not a perfect blackbody. In the optically thin limit, the opacity can be approximated as a powerlaw, $\tau_{\nu}=(\nu/\nu_{0})^{\beta}$. The form of the single modified blackbody for the flux density at each frequency is then

\begin{equation}
S(\nu) \propto \nu^{\beta}B_{\nu}(T_{\mathrm{d}})\,,
\end{equation}

\noindent where $\beta$ is the spectral emissivity index and $B_{\nu}(T_{\mathrm{d}})$ is the standard Planck blackbody function for an object with temperature $T_{\mathrm{d}}$. This simple model has been shown to fit well the prominent FIR bumps for large samples of star-forming galaxies  and provides estimates of the dust temperature, dust mass, and SFR \cite[e.g.][]{Calzetti:2000fk, Bianchi:2013jk, Cortese:2014qq}.

To calculate the dust mass, we must assume a particular dust absorption coefficient, $\kappa_{0}$ at a particular frequency, $\nu_{0}$. For this work, we assume $\kappa_{0}=0.192\,\mathrm{m^{2}\,kg^{-1}}$ and $\nu_{0}=857\,\mathrm{GHz}$ (i.e. 350 \micron) from \citet{Draine:2003gd}. However, as \citet{Bianchi:2013jk} shows, by assuming a specific $\kappa_{0}$, we must also fix the spectral emissivity index to the value used to measure $\kappa_{0}$. In this work, we fix $\beta=2.0$ to match the spectral emissivity index used by \citet{Draine:2003gd}.\footnote{We have tested the effect of allowing $\beta$ to be a free parameter by re-fitting sources, which were detected in all wavebands. We find a median $\beta=1.8\pm0.3$, consistent with our choice to fix $\beta=2.0$. Further we find that all of the derived parameters, especially the integrated luminosities, are consistent within measurement uncertainty with their values when $\beta$ is fixed at 2.0.} The final full form of the single MBB model is then

\begin{equation}\label{eq:greybody}
S_{\mathrm{MBB}}(\nu) = \frac{M_{\mathrm{d}}\kappa_{0}}{D_{\mathrm{L}}^2}\left(\frac{\nu}{\nu_{0}}\right)^{\beta}\frac{2h\nu^{3}}{c^{2}}\frac{1}{e^{{h\nu/kT_{\mathrm{d}}}}-1}\,,
\end{equation}

\noindent where $M_{\mathrm{d}}$ is the dust mass, $D_{\mathrm{L}}$ is the luminosity distance, $c$ is the speed of light, $h$ is the Planck constant, and $k$ is the Boltzmann constant. The two free parameters then are $M_{\mathrm{d}}$ and $T_{\mathrm{d}}$, the dust mass and dust temperature respectively.

The simple assumption that dust emission in the IR can be modeled with a single temperature blackbody works well for ``normal'' star-forming galaxies. However, for galaxies with large amounts of hot dust either due to a compact starburst or central AGN, this assumption can quickly break down. To account for this hot dust we also fit our sample using the model described in \citet[][hereafter C12]{Casey:2012jl}, which is the combination of a single MBB and an exponentially cutoff powerlaw. The C12 model takes the form

\begin{equation}\label{eq:casey}
S_{\mathrm{C12}}(\nu) = N_{\mathrm{pl}}\left(\frac{\nu}{\nu_{\mathrm{c}}}\right)^{-\alpha}e^{-\nu_{\mathrm{c}}/\nu} + S_{\mathrm{MBB}}(\nu)\,,
\end{equation}

\noindent where $\nu_{\mathrm{c}}$ represents the turnover frequency and $N_{\mathrm{PL}}$ is a normalization constant. C12 illustrated using the \textit{Great Origins All-Sky LIRG Survey} (GOALS) sample that this model provides better estimates of the cold dust temperature, dust mass, and IR luminosity compared to both a single temperature modified blackbody and template libraries.

The C12 model introduces three more free parameters ($N_{\mathrm{PL}}$, $\alpha$, and $\nu_{\mathrm{c}}$); however, within the implementation used by C12, $N_{\mathrm{PL}}$ and $\nu_{\mathrm{c}}$ are tied to the normalization of the modified blackbody component and dust temperature to produce a smoothly varying SED and reduce the number of free parameters from five to three. 

But, after early tests using this initial setup, we found that fixing $N_{\mathrm{PL}}$ and $\nu_{\mathrm{c}}$ as a function of the other parameters produced unreliable fits. This is because AGN SEDs from the MIR to FIR are not as smooth as those seen in (U)LIRGS, likely due to the disconnect between star-formation and AGN heating. Within starbursting galaxies both the hot and cold dust are related through the same heating process, i.e. star formation, while much of the MIR emission in AGN host galaxies originates from dust around the AGN with no strong connection to global star formation in the galaxy. Therefore, we chose to leave both $N_{\mathrm{PL}}$ and $\nu_{\mathrm{c}}$ as free parameters resulting in a total of five for the entire model.

We describe in detail within Appendix~\ref{sec:sed_bayes} the exact fitting method used to estimate each parameter. In short we use a Bayesian framework with a Monte Carlo Markov Chain (MCMC) to probe the posterior probability distribution function (PDF). We marginalize the resulting posterior PDF for each parameter and use the median as our best-fit value and the 16th and 84th percentiles to define our lower and upper uncertainties, respectively. 

{\color{red}For 35 sources in our sample, the SEDs contain less than four detected points.} These sources are undetected for several reasons. Either they are a) intrinsically faint, b) at relatively large distances, or c) located in a region with high foreground cirrus emission. This last reason causes problems because it produces high upper limits on the fluxes for the SPIRE wavebands, which give the appearance that a substantial amount of FIR emission could exist. Thus, we tested our modeling of these sources with three fixed dust temperatures: 15, 23, and 40K. 15 and 40 K are the extreme low and high temperatures we find for the whole sample, while 23 K is the median dust temperature. We fit only the 70 -- 500 \micron{} points/upper limits and let the dust mass increase until the model SED exceeded one of the points/upper limits. 

We find that using a dust temperature of either 23 or 40 K produces consistent MBB luminosities. Using a 15 K dust temperature results in MBB luminosities \textit{lower} than the 23 K and 40 K ones. Thus, it is reasonable to assume conservative upper limits on the MBB luminosity can be constrained with a higher dust temperature. The parameter most affected by the choice of temperature is the dust mass. Lower temperatures produce higher upper limits on the dust mass. In fact, moving from a 15 K temperature to a 40 K temperature changes the dust mass by two orders of magnitude. Given these tests, we chose to run our modeling of these 35 sources by fixing the dust temperature at 23 K, the median temperature of our whole sample. We acknowledge that for some of the undetected sources, we might be underestimating the upper limit on the dust mass. Since this subsample is only $\sim10$ per cent of the whole, we do not anticipate any large effect on the overall results.

We fit all 313 of the \herschel-BAT AGN using the C12 model and report the best fitting parameters in Table~\ref{tab:c12_params}. Best-fit model SEDs are shown along with the observed SEDs in Figure~\ref{fig:seds}. We also fit HRS galaxies using the exact same model so we can accurately compare the star-forming properties between non-AGN and AGN host galaxies. For the HRS sources with less than four detected points we fix the dust temperature at 21 K, the median dust temperature for the whole HRS sample. 

\subsection{Luminosities and AGN Fractions}\label{sec:lums_agn_frac}
In addition to the parameters associated with the model, we also calculated several luminosities and an AGN fraction. For this work, we define three luminosities: 1) $L_{\rm IR}$ will represent the total infrared luminosity determined by integrating the full SED from 8--1000 \micron, 2) $L_{\rm SF}$ will represent the 8--1000 \micron{} luminosity due to star formation in the host galaxy, 3) $L_{\rm AGN, IR}$ will represent the 8--1000 \micron{} luminosity due to AGN-heated dust. 

To determine $L_{\rm SF}$ and $L_{\rm AGN, IR}$ we first start with $L_{\rm PL}$ and $L_{\rm MBB}$, the 8--1000 \micron{} luminosities of the separate PL and MBB components from our SED decomposition. If we made the assumption that the PL component is completely dominated by the AGN in all sources then we could simply represent $L_{\rm PL}$ as the $L_{\rm AGN, IR}$ and $L_{\rm MBB}$ as $L_{\rm SF}$. However, we quickly recognized the well known effect that the PL component for some AGN host galaxies can have a strong contribution from dust heated by stars. Thus, a correction factor must be applied to convert $L_{\rm PL}$ and $L_{\rm MBB}$ into $L_{\rm AGN, IR}$ and $L_{\rm SF}$.

We make this correction using the results of the C12 modeling for the HRS sample. Since all of the galaxies in this sample have either no or low-luminosity AGN, the MIR emission is primarily the result of stochastically heated grains near star-forming regions. In Figure~\ref{fig:lmbb_lpl_ratio}, we show the distribution of $L_{\rm MBB}/L_{\rm PL}$, the ratio of the MBB component luminosity to the MIR powerlaw component. The ratio for the HRS sample is narrowly distributed around a single value, indicating that for our lower mass non-AGN comparison sample the energy contained in the PL and MBB components are tightly connected. We find a median $\log(L_{\rm MBB}/L_{\rm PL})=0.48 \pm 0.13$, which transforms to $L_{\rm PL,SF} \approx \frac{1}{3}L_{\rm MBB}$, where we now indicate the contribution to the PL component from star formation as  $L_{\rm PL,SF}$. We can calculate the AGN contribution for the \herschel-BAT AGN then by assuming the star-forming emission follows the same ratio.

\begin{align}
L_{\rm AGN,\,IR} &= L_{\rm PL} - L_{\rm PL, SF} = L_{\rm PL} - \frac{1}{3}L_{\rm MBB} \label{eq:c12_lagnIR}\,, \\
f_{\rm AGN} &= L_{\rm AGN, \, IR}/L_{\rm IR} \label{eq:c12_fagn}\,, \\
L_{\rm SF} &= L_{\rm MBB} + L_{\rm PL,SF} =  \frac{4}{3}L_{\rm MBB} \label{eq:c12_lsf}\,,
\end{align}

The uncertainty on the correction factor leads to an uncertainty on $f_{\rm AGN}$ using this method, which we estimate by measuring $f_{\rm AGN}$ for the HRS sample. Figure~\ref{fig:fagn_nonAGN} shows the distribution of ``$f_{\rm AGN}$'' for HRS. As expected the distribution is centered around 0. We estimate the spread of the distribution by calculating the standard deviation, finding an uncertainty of 0.1. Therefore, we add in quadrature an uncertainty of 0.1 to all of our $f_{\rm AGN}$ estimates. Any $f_{\rm AGN}$ below 0.1 are converted to upper limits with a value of 0.1. 

In Appendix~\ref{sec:comp_models}, we detail our comparison between our chosen C12 model and two other template-based models that aim to decompose the IR SEDs of AGN host galaxies. We find broad agreement between all three models in the measurement of these three luminosities and the AGN fraction. This comparison also provided additional systematic uncertainties on these parameters that we apply for the rest of this work. Further in Appendix~\ref{sec:pah_comp}, we analyze the relationship between our measured $L_{\rm SF}$ and polycyclic aromatic hydrocarbon luminosities from \textit{Spitzer}/IRS spectra and compare the relationship with a star-forming galaxy sample. We determine that the relationship between our measured $L_{\rm SF}$ and PAH luminosities is consistent with that seen in a purely star-forming sample.
 
 %This uncertainty on $f_{\rm AGN}$ also places a strict upper limit on the maximum $f_{\rm AGN}$ we are able to measure. With an uncertainty of 0.1, any measured $f_{\rm AGN}$ above 0.9 is converted to a lower limit 

 %Further we also assume that 0.9 is the maximum $f_{\rm AGN}$ able to be detected so any measured $f_{\rm AGN}$ above 0.9 are converted to lower limits. 

\begin{figure}
\includegraphics[width=\columnwidth]{figures/lmbb-lpl-ratio-hrs-mass-select}
\caption{KDEs of the $L_{\rm MBB}/L_{\rm PL}$ distribution for the \herschel-BAT AGN (blue), and HRS (red) samples. The HRS galaxies have a narrowly distributed ratio, while the BAT AGN span a wide range due to the AGN contribution. A color version of this figure is available in the online publication. \label{fig:lmbb_lpl_ratio}}
\end{figure}

\begin{figure}
\includegraphics[width=\columnwidth]{figures/fagn_nonAGN_hrs_mass_select}
\caption{$f_{\rm AGN}$ distribution for HRS. The standard deviation of this distribution quantifies the uncertainty on $f_{\rm AGN}$ associated with the correction factor used to calculate $f_{\rm AGN}$. A color version of this figure is available in the online publication. \label{fig:fagn_nonAGN}}
\end{figure}

\section{Results and Discussion}

\subsection{Correlation between $L_{\rm AGN, IR}$ and $L_{\rm 14-195\,keV}$}\label{sec:lagnIR_lbat_correlation}

\begin{figure}
\includegraphics{figures/lagnIR_lbat_correlation}
\caption{Correlation between $L_{\rm AGN, IR}$ and $L_{\rm 14-195\,keV}$ for the \herschel-BAT sample (black points). The solid blue line shows the result of Bayesian linear regression with the shaded region indicating the 95 per cent confidence interval. The strong correlation between the two luminosities provides evidence our SED decomposition performed well in measuring the IR luminosity associated with AGN heated dust.\label{fig:lagnIR_lbat}}
\end{figure}

In Figure~\ref{fig:lagnIR_lbat}, we plot the relationship between the 14--195 keV luminosity and $L_{\rm AGN,IR}$. As expected, we find a strong correlation between the two luminosities indicating we have measured $L_{\rm AGN,IR}$ accurately from the C12 model and our correction factor. To quantify the relationship, we performed linear regression using a \textsc{python} implementation of \textsc{linmix\_err}\footnote{\url{https://github.com/jmeyers314/linmix}} \citep{Kelly:2007lr}, a Bayesian method for linear regression that incorporates errors in both independent and dependent variables, as well as non-detections. \citet{Kelly:2007lr} showed through simulations that \textsc{linmix\_err} outperforms other popular linear regression methods such as ordinary least squares and \textsc{fit\_exy}. \textsc{linmix\_err} outputs marginalized posterior probability distributions for the slope, intercept, and intrinsic scatter between the two variables, as well as the correlation coefficient, $\rho$. Our reported best fit parameters here and throughout the rest of the Paper will be the median of the marginalized posterior probability distribution with an uncertainty equal to the standard deviation. The linear model assumed is of the form:

\begin{equation}\label{eqn:lin_mod}
Y = mX + b + \epsilon_{\rm int}\,,
\end{equation}

\noindent where $Y$ and $X$ are the dependent and independent variables, $m$ is the slope, $b$ is intercept, and $\epsilon_{\rm int}$ represents intrinsic scatter in the relationship beyond that related to measurement error. $\epsilon_{\rm int}$ is modeled as a Gaussian random variable with mean 0 and a variance of $\sigma^2$. 

We plot the best fit linear regression for the whole sample in Figure~\ref{fig:lagnIR_lbat} as a blue line with shading to indicate the 95 per cent confidence interval. The best-fit line corresponds to the following relationship:

\begin{equation}\label{eqn:lagnIR_lbat}
\log(L_{\rm AGN,IR}) =  (0.91\pm0.06)\log(L_{\rm 14-195\,keV}) - (4.03\pm2.6)\,,
\end{equation}

%%%%%%%%%%%%%%%%%%%%%%% TABLE: 14-195 keV vs LAGN_IR%%%%%%%%%%%%%%%%%%%%%%%%%%%%%
%\begin{table}
%\begin{threeparttable}
%\captionsetup{font=small,labelfont=bf,labelsep=period}
%\caption{Linear Regression Between $L_{\rm14-195\,keV}$ and  $L_{\rm AGN,IR}$ \label{tab:lbat_lagnIR_correlation}}
%\begin{tabular}{lcccc}
%\toprule 
% Sample & $m$ & $b$  & $\sigma_{\rm int}$ & $\rho$  \\
%\midrule
%All        & $0.75\pm0.05$ & $-22.5\pm2.3$ & $0.13\pm0.02$ & $0.76\pm0.03$ \\
%Type 1 &  $0.83\pm0.07$ & $-26.0\pm3.0$ & $0.11\pm0.02$ & $0.80\pm0.05$ \\ 
%Type 2 &  $0.65\pm0.08$ & $-18.2\pm3.5$ & $0.15\pm0.03$ & $0.69\pm0.06$ \\
%\bottomrule
%\end{tabular}
%\end{threeparttable}
%\end{table}
%%%%%%%%%%%%%%%%%%%%%%%%%%%%%%%%%%%%%%%%%%%%%%%%%%%%%%%%%%%%%%%%%%%%%%%%%


The correlation coefficient from \textsc{linmix\_err} is $0.8\pm0.03$ and the intrinsic scatter is 0.15 dex. The probability of zero correlation is only $10^{-77}$. Our measured slope of 0.90 is slightly flatter than the slope measured in \citet{Gandhi:2009kx} ($m=1.1$) using sub-arcsecond 12 \micron{} imaging for 16 nearby Seyferts. However, given our very simple method for determining the AGN MIR luminosity from broadband SED decomposition, a slope close to 1.0 and high correlation coefficient provides confidence that the decomposition has performed reasonably well. Further, Equation~\ref{eqn:lagnIR_lbat} will provide future studies with estimates of the full 8--1000 \micron{} AGN contribution and allow better estimates of IR-based SFRs for AGN, as long as X-ray data are available.

Because we are comparing luminosities and the \swift/BAT sample is flux-limited, the correlation we see could simply be due to a confounding variable, in this case distance. To investigate this possibility, we performed a partial correlation test with ASURV using distance as our third variable along with the 14--195 keV luminosity and AGN-related IR luminosity. The partial correlation test calculates Kendall's-$\tau$ rank correlation coefficient between all three variables, then estimates the intrinsic correlation coefficient after removing the dependence on the third variable (i.e. distance). 

We find $\tau = 0.30\pm0.03$ and reject the null hypothesis of zero partial correlation at a significance level of $10^{-23}$. While the value of the correlation coefficient is less than the correlation coefficient from \textsc{linmix\_err}, the partial correlation test does not account for scatter due to measurement error. 

To further show that our SED decomposition has accurately estimated the IR luminosity due to the AGN, we calculated the correlation between just the total 8--1000 \micron{} luminosity ($L_{\rm IR}$) and 14--195 keV luminosity using the same method. We find a slope of $0.41\pm0.05$ and a correlation coefficient of only $0.44\pm.05$. A z test between the $L_{\rm AGN,\,IR}$ correlation coefficient and the $L_{\rm IR}$ correlation coefficient indicates a zero probability that the $L_{\rm IR}$-$L_{\rm X}$ correlation is stronger than the $L_{\rm AGN,\,IR}$-$L_{\rm X}$ correlation. 

14--195 keV luminosities are only available for our low redshift sample, therefore Equation~\ref{eqn:lagnIR_lbat} is useless at higher redshifts. We can convert 14--195 keV luminosity to the more accessible 2--10 keV luminosity by making the simple assumption that the shape of the X-ray spectrum of AGN is a powerlaw with a photon index of 1.8 \citep[e.g.][]{Vasudevan:2013dz}. The ratio of $L_{14-195}/L_{2-10}$ is then $\sim2.7$. Switching to $L_{2-10}$ changes Equation~\ref{eqn:lagnIR_lbat} to 

\begin{equation}\label{eqn:lagnIR_2_10}
\log(L_{\rm AGN,IR}) =  (0.91\pm0.06)\log(L_{\rm 2-10\,keV}) - (3.64\pm2.6)
\end{equation}

\noindent $L_{\rm 2-10\,keV}$ assumes the luminosity has been absorption-corrected. Differences in the assumed photon index only change the intercept by at most 0.3 dex. 

\subsection{IR colors as a predictor of $f_{\rm AGN}$ and selecting AGN}

\begin{figure*}
\includegraphics[width=\textwidth]{figures/agnfrac_vs_flux_ratio}
\caption{Relationship between different IR colors and $f_{\rm AGN}$, the fraction of the 8--1000 \micron{} luminosity attributed to AGN heated dust. The IR colors (x-axis) are in logarithmic units. Upper limits on $f_{\rm AGN}$ are indicated as red arrows and are at 95 per cent confidence. We also display the 75th, 50th, and 25th percentile colors for the \herschel-BAT AGN as solid, dashed, and dotted vertical lines, respectively. Choosing a sample with colors \textit{above} these would result in 75, 50, and 25 per cent completeness for the \herschel-BAT AGN. \label{fig:agnfrac_vs_flux_ratio}}
\end{figure*}

We utilize our broad coverage of the IR SED for AGN host galaxies to test the viability of different IR colors for predicting the AGN IR luminosity fraction (i.e. $f_{\rm AGN}$). For large surveys and/or high redshift studies, coverage from 12--500 \micron{} might not be available or possible.

In Figure~\ref{fig:agnfrac_vs_flux_ratio}, we plot six MIR/FIR flux ratios (12/70, 12/160, 12/250, 22/70, 22/160, 22/250) against the measured $f_{\rm AGN}$ for the \herschel-BAT AGN. The flux ratios are all in log units to better visualize the relationships. Upper limits on $f_{\rm AGN}$ are shown as downward-pointing arrows. In the lower right corner of each subplot, we show representative error bars. We fit a linear model between each flux ratio and $f_{\rm AGN}$ with the following form

\begin{equation}
f_{\rm AGN} = m\log(F_{1}/F_{2}) + b + \epsilon_{\rm int}\,,
\end{equation}

\noindent where $F_{1}/F_{2}$ represents each flux ratio. To calculate the best fitting parameters, we again used \textsc{linmix\_err}. 

Table~\ref{tab:flux_ratio_fagn_linreg} outlines the best fit parameters for each relationship. Each flux ratio is highly correlated with $f_{\rm AGN}$ (all $\rho > 0.87$). {\color{red}Figure~\ref{fig:avg_sed} also reemphasizes this result as the MIR fluxes change more than the FIR ones a the X-ray luminosity changes.} The worst indicators are the flux ratios involving the 250 \micron{} flux with $\rho$ less than 0.9 and significant scatter as shown in Figure~\ref{fig:agnfrac_vs_flux_ratio}. The 250 \micron{} flux is located on the Rayleigh Jeans side of the blackbody and varies much more strongly as a function of the dust temperature due to the rapidly declining shape of the SED at long wavelengths compared to the 160 \micron{} flux, which is nearer to the peak of the MBB component. Thus, the 160 \micron{} flux is more stable with respect to the total MBB luminosity and $f_{\rm AGN}$.

Based on the slope of the linear regressions, the 22/70 \micron{}, with a slope of 0.8, overall seems to be the best flux ratio to use as an indicator of the AGN contribution to the IR SED. These two wavebands bracket the region where the intrinsic AGN SED is thought to turnover \citep[e.g.][]{Netzer:2007ve, Mullaney:2011yq} and seems to be the crucial transition region where the SED shifts from AGN-dominated to host galaxy dominated.  

%The ratios involving the 160 \micron{} flux work better as a predictor because the AGN component of the IR SED rarely contributes any emission to the total flux of the galaxy at 160 \micron{} compared to the star-forming component. While this is also true of the 250 \micron{} flux, the 160 \micron{} flux is nearer to the peak of the MBB component, thus, more closely related to the total luminosity.  Even though the total 70 \micron{} emission can have a large fraction of its emission contributed by the AGN (see next section), it still remains a decent predictor of $f_{\rm AGN}$ with $\rho = 0.82$ and $\rho=0.89$ for the 12/70 and 22/70 \micron{} ratios. 

%%%%%%%%%%%%%%%%%% TABLE OF LINEAR REGRESSION B/W FLUX RATIOS AND FAGN %%%%%%%%%%%%%%%%%%%%%%%%%%
\begin{table*}
\begin{threeparttable}
\captionsetup{font=small,labelfont=bf,labelsep=period}
\caption{Linear Regression Between Flux Ratios and $f_{\rm AGN}$\label{tab:flux_ratio_fagn_linreg}}
\begin{tabular}{lccccccc}
\toprule 
$F_{1}/F_{2}$ & $m$ & $b$  & $\sigma_{\rm int}$ & $\rho$ & 75th \%tile  & 50th \%tile  & 25th \%tile\\
&&&&&(\%tile for HRS)&(\%tile for HRS)&(\%tile for HRS)\\
\midrule
12/70      & $0.67\pm0.04$ & $1.02\pm0.03$ & $0.016\pm0.004$ & $0.89\pm0.02$ & -1.23 (90) & -0.97 (50) & -0.66 (21)\\
12/160    & $0.58\pm0.02$ & $1.05\pm0.03$ & $0.002\pm0.002$ & $0.99\pm0.01$ & -1.51 (31) & -1.21 (5.2) & -0.85 (2.2)\\
12/250    & $0.51\pm0.03$ & $0.84\pm0.03$ & $0.019\pm0.004$ & $0.87\pm0.03$ & -1.22 (28) &  -0.91 (5.2) & -0.57 (1.5)\\
22/70      & $0.79\pm0.03$ & $0.86\pm0.02$ & $0.001\pm0.001$ & $0.99\pm0.01$ & -0.89 (32) & -0.60 (16) & -0.32 (12)\\
22/160    & $0.58\pm0.02$ & $0.85\pm0.02$ & $0.0004\pm0.0005$ & $0.997\pm0.003$ & -1.14 (5.6) & 0.83 (1.2) & -0.45 (0)\\
22/250    & $0.49\pm0.03$ & $0.65\pm0.02$ & $0.016\pm0.004$ & $0.90\pm0.02$ & 0.87 (6.8) & -0.51 (0.02) & -0.18 (0) \\
\bottomrule
\end{tabular}
\end{threeparttable}
\end{table*}
%%%%%%%%%%%%%%%%%%%%%%%%%%%%%%%%%%%%%%%%%%%%%%%%%%%%%%%%%%%%%%%%%%%%%%%%%%%%%%

We further test how well each IR color exclusively selects AGN host galaxies. In Figure~\ref{fig:agnfrac_vs_flux_ratio} we plot the 75th, 50th, and 25th percentile colors for the \herschel-BAT sample as black solid (25th), dashed (50th), and dotted (75th) vertical lines. In Table~\ref{tab:flux_ratio_fagn_linreg}, the last three columns list the colors in logarithmic units for the 75th, 50th, and 25th percentiles. These colors mean that by selecting galaxies with a color above this one, the sample will be 75, 50, and 25 per cent complete based on the \herschel-BAT sample.

While these colors are useful for selecting AGN host galaxies, we also want to know what percentage of non-AGN galaxies would be selected with these color cuts as contamination for a sample. Therefore, we also calculated the percentile of HRS galaxies that would be selected. For this analysis, we used the complete HRS sample since it represents the natural field galaxy population, and we do not want to restrict to only high-mass non-AGN galaxies. The HRS sample, unfortunately, was not observed at 70 \micron, therefore to calculate the 12/70, and 22/70 colors, we simply used the 70 micron flux from the best-fit model. The percentiles for HRS are given in parentheses next to the values of the colors in Table~\ref{tab:flux_ratio_fagn_linreg}. 

We find that the best color to use in selecting an AGN-only sample would be the 22/160 color. Using the 75th percentile color (-1.14), we would only select 5.6 per cent of the HRS sample. To get only around 1 per cent of the HRS sample, we would have to use the 50th percentile 22/160 or 22/250 color; however this would only select half of a complete AGN sample. To select no non-AGN galaxies would require pushing to the 25th percentile colors and risk severe incompleteness and a selection bias of only AGN-dominated galaxies. In a forthcoming paper, we will evaluate these colors along with other MIR color selection techniques \cite[e.g.][]{Donley:2012qy, Stern:2012mz} to determine the biases and selection effects with regards to the SFR of the galaxy and strength of the AGN.

\subsection{The AGN contribution to the 70 \micron{} emission}\label{sec:agn_70_contribution}
Many studies examining the relationship between nuclear activity and star formation rely on single monochromatic luminosities as an indicator of star formation \cite[e.g.][]{Netzer:2007ve, Netzer:2009lr, Rosario:2012fr}. The most widely used indicator has been the 60 \micron{} luminosity given its availability from the all-sky \textit{IRAS} survey. 

With our \herschel-BAT sample and SED modeling we can test how often and at what level AGN emission contributes to different wavelength bands. We test the AGN contribution to the 70 \micron{} emission. Based on visual inspection of all of the SED fits, we found this waveband is the only \herschel{} band with a noticeable AGN contribution. All other longer wavelength bands were dominated by the MBB component. We restrict this analysis to only those sources with well-determined SEDs longwards of 70 \micron, excluding those objects with less than four detected SED points. We note that the majority of these objects are likely AGN dominated at 70 \micron; however, the uncertainty in their FIR SED, especially their dust temperature, results in a largely unconstrained calculation.

To determine the AGN contribution to the 70 \micron{} emission, we first estimated the PL component associated with star formation. We assumed the shape of the star-forming PL component followed the average power law slope ($\alpha=0.0$ and turnover wavelength ($\lambda_{\rm c} = 50$ \micron) from the HRS sample. We then adjusted the normalization such that the integrated luminosity would equal $1/3L_{\rm MBB}$. We combined the estimated star forming PL component with the measured MBB component to form each of the \herschel-BAT AGN's total star-forming component. The 70 \micron{} AGN contribution was then calculated as the excess 70 \micron{} emission leftover after subtracting the star-forming contribution from the total 70 \micron{} emission from the best-fit SED. We denote the fraction contributed by the AGN as $f_{\rm AGN, 70}$.

Figure~\ref{fig:70um_agn_contribution}, \textit{left} shows the distribution of $f_{\rm AGN, 70}$. We find that for 48 per cent of the sample,  $f_{\rm AGN, 70} < 0.2$ but for 24 per cent of the sample $f_{\rm AGN, 70} > 0.5$. Using the 70 \micron{} luminosity as a single SFR indicator would overestimate values by at least a factor of 2 for almost one-third of an AGN sample. 

In the right panel of Figure~\ref{fig:70um_agn_contribution} we show $f_{\rm AGN, 70}$ as a function of the 14--195 keV luminosity. The grey points indicate the individual points while the blue points with error bars are binned averages with a 68 per cent confidence interval determined through bootstrapping. It is abundantly clear that at $\log\,L_{\rm 14-195\,keV} < 42.5$, $f_{\rm AGN, 70}$ is largely negligible with only two sources with $f_{\rm AGN, 70} > 0.2$. These AGN are NGC 1052 and NGC 4941. Both are quite pointlike, possibly unresolved at 70 \micron, while at long wavelengths show large-scale extended emission.

Above $\log\,L_{\rm 14-195\,keV} = 42.5$, the AGN contribution at 70 \micron{} varies nearly uniformly between 0 and 1. This indicates that the AGN contribution is not a simple function of high AGN luminosity, rather it is due to the competing interplay between AGN emission and star-formation. This reinforces the need for SED decomposition to determine accurate SFRs when an AGN is present. We did repeat this process at 160 \micron, finding no source indicated an AGN contribution above the 5 per cent level.

%Our main conclusion from this section, though, is to be careful when using monochromatic luminosities as an indicator of the SFR, especially near 70 \micron. At longer wavelengths, we find single luminosities are quite reliable in regards to the AGN contribution.

\begin{figure*}
\includegraphics{figures/pacs_AGN_contribution}
\caption{\textit{left:} Histogram of the estimated 70 \micron{} contribution by AGN-heated dust. Each bar is a bin with a width of 0.1 and a height equal to the fraction of the total sample. \textit{right:} Relationship between the 14--195 keV luminosity and 70 \micron{} AGN contribution. Grey dots represent each individual source while the blue dots with error bars are averages within bins according to 14--195 keV luminosity. Error bars indicate the 68 percent confidence interval determined using bootstrap analysis.\label{fig:70um_agn_contribution}}
\end{figure*}

\subsection{$M_{\rm dust}$, $T_{\rm dust}$, and SFR of local AGN host galaxies}\label{sec:agn_sf_comparison}

%%%%%%%%%%%%%%%%%%%%%%%% FIGURE: Mdust, Tdust, SFR COMPARISON %%%%%%%%%%%
\begin{figure*}
\includegraphics[width=\textwidth]{figures/mdust_tdust_sfr_comparison_mass_select_hrs}
\caption{ Comparison of $M_{\rm dust}$, $T_{\rm dust}$, and SFR between the \herschel-BAT AGN (blue), and HRS mass-matched sample (red). The distributions with dashed lines indicate sources with only upper limits on these values. A color version of this figure is available in the online publication. \label{fig:mdust_tdust_sfr_comp}}
\end{figure*}
%%%%%%%%%%%%%%%%%%%%%%%%%%%%%%%%%%%%%%%%%%%%%%%%%%%%%%%%%

%%%%%%%%%%%%%%%%%%%%%%% TABLE: Mdust, Tdust, SFR Comparison %%%%%%%%%%%%%
\begin{table*}
\centering
\begin{threeparttable}
\captionsetup{font=small,labelfont=bf,labelsep=period}
\caption{Mean \mdust, \tdust, and SFR \label{tab:mean_mdust_tdust_sfr}}
\begin{tabular}{lcccccc}
\toprule 
Sample & $\log$ \mdust & \tdust & $\log$ SFR\\ %& $\log$ \mdust/\mstar & \tdust/$\log$ \mstar & $\log$ SFR/\mstar \\
 & [\msun]  & [K]  &  [\msun{} yr$^{-1}$] \\%&    &   & [yr$^{-1}$]\\
\midrule
\herschel-BAT & 7.36 (6.56--7.90) & 23.8 (19.4--27.5) & 0.23 (-0.57--0.81)\\ %& -3.22$\pm$0.05 & 2.21$\pm$0.03 & -10.35$\pm$0.06 \\ 
HRS & 7.08 (5.12--7.70) & 21.5 (19.4--24.6) & -0.23 (-1.57--0.36) \\%& -3.14$\pm$0.07 & 2.21$\pm$0.02 & -10.40$\pm$0.05 \\
%KINGFISH & 6.80$\pm$0.10 & 21.7$\pm$0.5 & -0.40$\pm$0.10 & -2.60$\pm$0.10 & 2.25$\pm$0.05 & -9.90$\pm$0.10 \\
\bottomrule
\end{tabular}
%\begin{tablenotes}
%\item blah blah blah
%\end{tablenotes}
\end{threeparttable}
\end{table*}
%%%%%%%%%%%%%%%%%%%%%%%%%%%%%%%%%%%%%%%%%%%%%%%%%%%%%%%%%

Figure~\ref{fig:mdust_tdust_sfr_comp} compares the distribution of \mdust, \tdust, and SFR for our three samples of galaxies. We calculated SFRs using the conversion from 8--1000 \micron{} star-forming luminosity using the following equation from \citet{Murphy:2011rt}

\begin{equation}\label{eq:sfr_ir}
\rm{SFR} = \frac{L_{\rm SF}}{2.57\times10^{43}}\,,
\end{equation}

\noindent The solid lines show the KDE for sources which have firm measurements, while the dashed lines show the KDE for the sources with only upper limits. Table~\ref{tab:mean_mdust_tdust_sfr} displays the medians for these three properties and each sample. The values inside parentheses indicate the 16th and 84th percentiles for each distribution. The median, 16th, and 84th percentiles were calculated using survival analysis and the KMPL as in Section~\ref{sec:ir_seds} to account for upper limits.

%While sources with only upper limits are not shown in Figure~\ref{fig:mdust_tdust_sfr_comp}, we incorporate them into our calculation of the mean. 

We find that the \herschel-BAT AGN have higher dust masses than the HRS sample, slightly higher dust temperatures, and higher SFRs. Because we have mass-matched the HRS sample to the \herschel-BAT sample, there should be no effects due to stellar mass. To confirm that the distributions are statistically different we run the Peto-Prentice two-sample test. The Peto-Prentice two sample test is comparable to the Kolmogorov-Smirnov (K-S) test, which calculates the probability that two separate samples were drawn from the same parent population. The Peto-Prentice test, however, allows for the inclusion of censored data. 

For the comparison of dust masses, we find the probability that the \herschel-BAT AGN and HRS galaxies are drawn from the same parent distribution is $1.8\times10^{-5}$. The probability for the dust temperatures is $4\times10^{-4}$, while the probability for the SFRs is $8.4\times10^{-11}$. All three probabilities are less than 0.05, the standard cutoff for declaring two samples are inconsistent with being drawn from the same parent population. 

The increased SFRs for the \herschel-BAT sample compared to the mass-selected HRS sample are consistent with the results of \citet{Koss:2011vn}. \citet{Koss:2011vn}, using optical colors, found that the \swift/BAT AGN displayed bluer colors compared to a mass-matched sample of Sloan Digital Sky Survey (SDSS) galaxies, indicative of increased levels of star formation. %Despite the bluer colors than inactive galaxies, the BAT AGN were still deemed to show ``intermediate'' colors, firmly placing them in the ``green valley'' in between the dominant star-forming and quiescent populations that form the bi-modality seen in the standard color-magnitude diagram \citep[e.g.][]{Strateva:2001tw, Nandra:2007rz}. Intermediate colors, however, can arise from dust reddening \citep{Cardamone:2010uq}, as well as a decrease in star formation. 

These results may seem contradictory with our previous work in \citet{Shimizu:2015xo} where, using the SFRs calculated here, we showed that AGN host galaxies are experiencing decreased levels of star formation given their stellar mass. However, the discrepancy lies in the comparison being done. In \citet{Shimizu:2015xo} we compare the \herschel-BAT AGN to the so-called star-forming ``main sequence,'' a tight nearly linear correlation between SFR and \mstar \citep{Brinchmann:2004bf, Salim:2007rm, Noeske:2007fr}, finding the AGN are falling off and transitioning towards quiescence. In this work, as well as in \citet{Koss:2011vn}, simple mass-matching is used to form comparison samples, which leads to selecting only high stellar mass non-AGN galaxies due to the high stellar masses of the BAT AGN. Based on previous studies of the color-magnitude relation, however, selection of high mass galaxies will result in a large quiescence fractionespecially at low redshifts  \citep{Kauffmann:2003lr, Wuyts:2011lh, Bell:2012qf}. Therefore, since the \herschel-BAT AGN show both intermediate optical colors and SFRs, it makes sense that a mass-selected non-AGN sample with indicate bluer colors and higher SFRs for AGN host galaxies given quiescent galaxies are largely red with low SFRs.

Further, looking into the details of our mass-selected HRS sample, we find that 47/122 (50 per cent) are morphologically classified as early type (E/S0), the dominant morphology for quiescent galaxies \citep{Wuyts:2011lh, Bell:2012qf}. The \herschel-BAT AGN, though, are largely late type spirals \citep{Koss:2011vn}, and even the few ellipticals in our sample show higher SFRs than expected for quiescent galaxies \citep{Shimizu:2015xo}. Thus, we can reconcile the results of \citet{Koss:2011vn}, \citet{Shimizu:2015xo}, and this work by understanding that mass-matching preferentially selects low star-forming, red galaxies in contrast to the intermediate star-forming, blue AGN.

The higher SFRs immediately explain the slightly larger dust masses observed for the \herschel-BAT AGN.\footnote{Excluding the objects with only upper limits on the dust mass, which are dependent on the assumed fixed dust temperature we used in the modeling (see Section~\ref{sec:sed_fit}) produces the same overall trend of higher dust masses in the \herschel-BAT AGN sample.} The \herschel-BAT AGN  have a median $\log$\mdust{} of 7.38 \msun{} vs. 7.0 \msun{} for the HRS. The 16th percentiles also extend to much lower dust masses for the HRS. \citet{Cortese:2012xy} showed using the full HRS sample that the dust-to-stellar mass ratio is anti-correlated with $NUV-r$ color, a proxy for specific SFR and decreases strongly when moving from late type to early type galaxies. \citet{da-Cunha:2010qf} also showed, using over 3000 galaxies from SDSS, a strong correlation between the cold dust mass of galaxies and their SFR. 

The increased dust temperatures observed in the \herschel-BAT AGN could also be explained by the increased SFRs. Dust temperatures are known to increase with both infrared luminosity and specific SFR \citep[e.g][]{Chapman:2003jk, Symeonidis:2013fe, Cortese:2014qq, Magnelli:2014sf}, although these studies did not extend to the low luminosities and specific SFR we observe in our samples. Because we know our sample contains an AGN, it is possible at least some of the increased dust temperatures could be due to AGN heating. Recently, \citet{Garcio-Gonzalez:2016bk} spatially decomposed the FIR images of a small sample of nearby Seyfert galaxies and found that nuclear emission shows markedly higher dust temperatures than the host galaxy (as probed through the 70/160 \micron{} color). Sources that are dominated by nuclear emission should then show higher temperatures that do not match with the overall correlation between SFR and $T_{\rm dust}$. 

In Figure~\ref{fig:sfr_tdust}, we show the relationship between SFR and $T_{\rm dust}$ for the \herschel-BAT AGN and HRS galaxies. The bulk of both samples follow in general a broad positive correlation. There exists a group of HRS galaxies at low SFR ($\log\,\rm{SFR} < -1.0$) that show a flat relationship with $T_{\rm dust}$. All but one of these galaxies are early type, which are known to show dust temperatures up to 30 K \citep{Temi:2007ty}. The encircled \herschel-BAT AGN at $T_{\rm dust} > 30$ K and $\log\,\rm{SFR} < 0.5$ seem to represent distinct outliers from both samples. These sources are 2MASX J09235371-3141305, 2MASXJ 15462424+6929102, ESO 103-035, ESO 417-G006, MCG --05-23-016, MCG --06-30-015, Mrk 841, and NGC 3516. Except for  2MASX J09235371-3141305, these sources also all have $f_{\rm AGN} > 0.5$, indicating the MIR PL component is quite dominant. ESO 103-035, MCG --05-23-016, and Mrk 841 show SEDs that peak around 20 \micron, highly indicative of an SED dominated by IR emission from the AGN. For these seven AGN, then, it is possible that the large dust temperatures are due to AGN heating, rather than star formation. 

Besides the seven sources listed, though, the bulk of the \herschel-BAT AGN seem to show \tdust{} largely consistent with star-formation heated dust and coincides with our results for \mdust{} and the SFR. 

%%%%%%%%%%%%%%%%%%%%%%%% FIGURE: TDUST vs SFR COMPARISON %%%%%%%%%%%
\begin{figure}
\includegraphics[width=\columnwidth]{figures/sfr_vs_tdust}
\caption{Correlation between dust temperature and SFR for the \herschel-BAT AGN (blue points) and HRS galaxies (red squares). Circled blue points indicate possible sources that are dominated by AGN-heated dust. \label{fig:sfr_tdust}}
\end{figure}
%%%%%%%%%%%%%%%%%%%%%%%%%%%%%%%%%%%%%%%%%%%%%%%%%%%%%%%%%

%In the lower row of Figure~\ref{fig:mdust_tdust_sfr_comp} and last three columns of Table~\ref{tab:mean_mdust_tdust_sfr} we show the distribution and mean values of \mdust, \tdust, and SFR after normalization to \mstar. The results change significantly after accounting for the high \mstar{} distribution of the \herschel-BAT AGN. Whereas before the \herschel-BAT AGN seemed to have higher \mdust, \tdust, and SFR compared to the HRS and KINGFISH samples, normalized to \mstar, they now are either equal to or lower than the non-AGN samples. All three samples now have nearly the same \tdust{} distribution, indicating the increased dust temperatures were simply a result of the increased stellar masses. There is no evidence for the AGN heating large scale dust in the galaxy to higher temperatures. 

%Particularly interesting is the significant difference between the KINGFISH and \herschel-BAT AGN. The HRS sample was mass selected so it includes both star-forming and quiescent galaxies while the KINGFISH sample was mainly selected as a star-forming sample. Compared to star-forming galaxies, the AGN host galaxies have on average a 0.6 dex (factor of 4) decrease in their dust-to-stellar mass ratio and a 0.45 dex (factor of 3) decrease in their specific SFR (sSFR = SFR/\mstar). In \citet{Shimizu:2015xo}, we extensively examined whether our AGN sample showed decreased specific SFR compared to the so-called ``Main Sequence'' of star formation. We concluded, using a multitude of comparison samples that AGN host galaxies do exhibit reduced levels of star formation for their given stellar mass, a possible indication that AGN feedback has occurred in these galaxies. 
 
%In this Paper, we find that not only is the specific SFR decreased in AGN host galaxies, but \mdust{} is also decreased. Assuming a constant dust-to-gas ratio, a reduced \mdust{} would indicate AGN host galaxies on average are gas deficient compared to similar mass star-forming galaxies. This would explain the reduced star formation since neutral gas provides the material for molecular gas from which stars form. %There are two ways the AGN could have played a role in reducing the gas fraction of the galaxy. One is through physically blowing the gas out with AGN-driven winds. The other is by heating the halo and shutting off accretion from the intergalactic medium. Discerning between the two processes however is beyond the scope of this paper.

%To determine whether the \herschel-BAT AGN host galaxy properties are similar to the HRS and KINGFISH galaxies, we run the Peto-Prentice two-sample test. The Peto-Prentice (P-P) two sample test is comparable to the Kolmogorov-Smirnov (K-S) test which calculates the probability that two separate samples were drawn from the same parent population. The P-P test, however, allows for the inclusion of censored data. 

%We input \mstar{} normalized properties for the \herschel-BAT AGN and run tests to compare the distribution with both the HRS and KINGFISH galaxies. For \mdust/\mstar{}, we find probabilities $<<0.01$ of the \herschel-BAT AGN being drawn from the same distribution as the HRS and KINGFISH galaxies. In contrast, we find probabilities of 0.99 and 0.66 that the \herschel-BAT AGN \tdust/$\log$\mstar{} distribution is the same as the one for the HRS and KINGFISH samples, respectively. Finally, we find probabilities of 0.20 and $<<0.01$ comparing the sSFR distributions between the \herschel-BAT AGN and HRS and KINGFISH samples. The standard cutoff probability to declare two samples consistent with being drawn from the same parent distribution is 0.05. 

%A probability of 0.20 between the sSFR distributions of the HRS and \herschel-BAT AGN indicates they could be drawn from the same distribtution. However, the HRS sample contains galaxies that have been stripped of their gas due to infall into a cluster. The ram-pressure stripping likely quenched SFR within the galaxy. If we remove these galaxies by only selecting HRS galaxies not in clusters, the probability drops to $<<0.01$, the same as the comparison between the \herschel-BAT AGN and the KINGFISH galaxies.

%Our survival analysis two-sample tests confirm what our previous visual and basic statistical analysis suggested. The \mdust/\mstar{} and sSFR distributions of AGN host galaxies are significantly different than those of normal star-forming galaxies while the \tdust{} distributions are identical after accounting for \mstar. 


% USE FOR SEPARATE PAPER
%\subsection{Comparison between Type 1s and 2}
%
%%%%%%%%%%%%%%%%%%% FIGURES COMPARING TYPE 1s and TYPE 2s %%%%%%%%%%%%%%%%%%%%%%%%%%
%\begin{figure*}
%\includegraphics[width=\textwidth]{figures/agn_type_comparison}
%\caption{Comparison of $M_{\rm dust}$, $T_{\rm dust}$, SFR, MIR powerlaw slope ($\alpha$), turnover wavelength ($\lambda_{\rm c}$), and $L_{\rm AGN,IR}$ between the \herschel-BAT AGN Type 1 (blue) and Type 2 (red) samples. Type 1 and Type 2 AGN have identical distributions in all properties and parameters except for $\alpha$ and $L_{\rm AGN,IR}$. A color version of this figure is available in the online publication. \label{fig:agn_type_comp}}
%\end{figure*}
%
%\begin{figure}
%\includegraphics[width=\columnwidth]{figures/fagn_ir_seyferts}
%\caption{Histogram of $f_{\rm AGN}$ for Type 1s (blue) and Type 2s (red). Type 2s on average have much lower $f_{\rm AGN}$ than Type 1s.\label{fig:frac_agn_seyferts}}
%\end{figure}
%
%%%%%%%%%%%%%%%%%%%%%%%%%%%%%%%%%%%%%%%%%%%%%%%%%%%%%%%%%%%%%%%%%%%%%%%%%%%
%
%%%%%%%%%%%%%%%%%%%%%%%% TABLE: Comparison of Types 1 and Types 2%%%%%%%%%%%%%%%%%%%%%%%%%%%%%
%\begin{table*}
%\centering
%\begin{threeparttable}
%\captionsetup{font=small,labelfont=bf,labelsep=period}
%\caption{Comparison of Star-Forming and SED Properties of Type 1 and Type 2 AGN \label{tab:agn_type_comp}}
%\begin{tabular}{lccccccccc}
%\toprule 
% & $\log$ \mdust & \tdust & $\log$ SFR & $\alpha$ & $\lambda_{\rm c}$ & $\log\,L_{\rm AGN,IR}$ \\
% & [\msun]  & [K]  &  [\msun{} yr$^{-1}$] &    &  \micron & [ergs s${^-1}$]\\
%\midrule
%Type 1 & 7.08$\pm$0.08 & 23.2$\pm$0.4 & -0.01$\pm$0.07 & 1.46$\pm$0.04 & 48.7$\pm$0.8 & 10.00$\pm$0.06 \\ 
%Type 2 & 7.33$\pm$0.05 & 23.5$\pm$0.4 & 0.21$\pm$0.04 & 1.73$\pm$0.04 & 49.4$\pm$0.7 & 9.71$\pm$0.07 \\
%Peto-Prentice Probability & 0.04 & 0.73 & 0.02 & $2.6\times10^{-5}$ & 0.73 & 0.003 \\
%\bottomrule
%\end{tabular}
%%\begin{tablenotes}
%%\item blah blah blah
%%\end{tablenotes}
%\end{threeparttable}
%\end{table*}
%%%%%%%%%%%%%%%%%%%%%%%%%%%%%%%%%%%%%%%%%%%%%%%%%%%%%%%%%%%%%%%%%%%%%%%%%%
%
%In this Section, we compare the properties of unabsorbed (Type 1) vs absorbed (Type 2) AGN. According to the unified model \citep{Antonucci:1993os,Urry:1995il}, Type 1 and Type 2 AGN are two manifestations of the same object. The only difference between the two is the viewing angle towards the central AGN. Due to a dusty structure, lines of sight at low angle are obscured causing the broad line region to be hidden while high angle lines of sight allow an unobstructed view of both the accretion disk and broad line region. Type 1 AGN therefore exhibit broad emission lines and low hydrogen absorption while the opposite is true of Type 2 AGN. 
%
%Because the differences in AGN types is related to the central obscuring structure, global star-forming properties of Type 1 and Type 2 AGN should be the same. In the top row of Figure~\ref{fig:agn_type_comp}, we compare the distribution of \mdust, \tdust, and SFR between Type 1s and Type 2s. Visually the distributions look identical. However, after including sources with upper limits on \mdust{} and SFR, we find differences between Type 1s and Type 2s. In Table~\ref{tab:agn_type_comp}, we show the average of these properties as well as the Peto-Prentice probability that Type 1s and Type 2s are drawn from the same distribution. The only property with a probability greater than 0.05 is \tdust, meaning that the cold dust temperature associated with star formation is the same in Type 1s and Type 2s. 
%
%Surprisingly, however, are the results of the Peto-Prentice test for \mdust{} and SFR. While Figure~\ref{agn_type_comparison} shows Type 1s and Type 2s having identical distributions, survival analysis indicates that Type 1s are skewed more strongly towards galaxies with lower \mdust{} and SFR. Type 1s have 0.25 dex lower average \mdust and 0.22 dex lower average SFR. This difference is caused by the population of sources with only upper limits on both \mdust{} and SFR. 25/35 of the AGN with upper limits are classified as Type 1. Further, the actual values of the upper limits for the Type 1 AGN compared to Type 2 AGN. Of the 25 Type 1 AGN with upper limits, 5 have $\log\,$\mdust $< 6.0$ and $\log\,$SFR $<-1.0$ compared to none of the Type 2 AGN. 
%
%These Type 1 AGN with little to no dust and no recent star formation seem to form a separate sample away from the main population of AGN host galaxies. If we restrict the samples to only those sources with $\log\,$\mdust $> 6.0$ and $\log\,$SFR $>-1.0$, the Peto-Prentice probability changes to 0.25 for \mdust{} and 0.09 for the SFR. The new probabilities are both above the standard threshold of 0.05, indicating they originate from the same parent population. These anomalous Type 1 AGN could be an interesting subsample for future study, but our main result is that for the bulk of Type 1 and Type 2 AGN, their host galaxies display the same star-forming properties in agreement with the unified model.
%
%Regarding the AGN related properties of the IR SED, we see differences between Type 1 and Type 2 AGN even before including sources with upper limits. Both the Type 1 $\alpha$ and $L_{\rm AGN,IR}$ distributions are visually different from Type 2 AGN with Type 2 AGN having higher $\alpha$ and lower $L_{\rm AGN,IR}$. Survival analysis and calculation of their means confirm the visual differences as shown in Table~\ref{tab:agn_type_comp}. Type 1 AGN have an average $\alpha$ of 1.46 compared to 1.73 for Type AGN and an average $\log\,L_{\rm AGN,IR}$ of 10.00 compared to 9.71. 
%
%Steeper power-law slopes for Type 2 AGN has been observed previously in studies of AGN using low resolution \textit{Spitzer}/IRS spectra. Both \citet{Buchanan:2006cq} and \citet{Wu:2009pt} reported flatter MIR spectral indices for Type 1 AGN. Explanations for the differences included a greater contribution of a starburst component for Type 2 AGN which show red MIR spectra \citet{Brandl:2006kx}. The difference in slopes can also be explained by a changing opacity from shorter to longer wavelengths. If the optical depth increases with wavelength and shorter wavelength emission originates near the inner edge of the ``torus'', then Type 2 AGN would also show steeper MIR slopes. At the present time we cannot make strong conclusions as to which is the correct explanation. We plan to analyze this more closely in a future publication which will combine the \herschel{} photometry with \textit{Spitzer}/IRS spectra to better constrain the starburst component of MIR spectra. 
%
%The difference in $\log\,L_{\rm AGN,IR}$ between Type 1 and Type 2 AGN can be explained by the different X-ray luminosity distributions of Type 1 and Type 2 \herschel-BAT AGN. It has been well documented  \citep[e.g.][]{Burlon:2011pi, Melendez:2014yu} that within the \swift/BAT AGN sample, Type 2 AGN are more prevalent at lower luminosities compared to Type 1 AGN. Given the strong correlation between the X-ray and IR luminosity of AGN (see Section~\ref{sec:lagnIR_lbat_correlation}), Type 2s should exhibit lower $\log\,L_{\rm AGN,IR}$.
%
%The combination of lower $L_{\rm AGN,IR}$ and relatively similar SFR distributions produces a strong difference in the $f_{\rm AGN}$ distribution for Type 2s and Type 1s as shown in Figure~\ref{fig:frac_agn_seyferts}. Type 2s have much lower $f_{\rm AGN}$ compared to Type 1 AGN which agrees with \citet{Melendez:2008pd} who found that Type 2s have colder 25/60 \micron{} colors. Our interpretation however is that the lower $f_{\rm AGN}$ is not indicative of larger SFRs in Type 2 AGN but rather lower AGN luminosities. 
%
%\newpage These results are in agreement with \citet{Diamond-Stanic:2012rw}~who found no statistically significant differences between the SFRs of Type 1 and Type 2 AGN on all physical scales using the 11.3 \micron{} PAH feature and 24 \micron{} continuum emission. Moreover, \citet{Imanishi:2004sf}, using the 3.3 \micron{} PAH feature as a measure of star-forming activity found no differences between Type 1 and Type 2 AGN; and \citet{Pereira-Santaella:2010zl}, using the [NeIII] emission line, determined Type 1 and Type 2 AGN exhibit the same star-forming activity. Thus, nearly all measures of SFR in the infrared (FIR, PAH features, ionization lines) seem to confirm that moderate luminosity Type 1 and Type 2 AGN are actively star-forming at the same level.
%
%Previous studies that indicated Type 2 AGN have enhanced star formation occurring in their host galaxies, such as \citet{Maiolino:1995nr} and \citet{Mouri:2002fk} reached these conclusions based on analysis of the colors and luminosity ratios which we have shown to be more indicative of the different luminosity distributions of Type 1 and Type 2 AGN rather than a difference in their SFRs. What this means is that apart from the interesting subsample of Type 1 AGN that exhibit extremely low SFRs and dust masses, studies which rely mainly on Type 2 AGN will not result in selection effects associated with their absolute star-forming activity, but will result in biases based on the host-galaxy dominance, especially in the IR.
 
\subsection{The Correlation Between SFR and AGN Luminosity}
Our main goal in this Paper is to assess the observational evidence for a connection between the AGN and star formation in its host galaxy in the local universe. To accomplish this, we analyze the correlation between the AGN strength and SFR. We assume the 14--195 keV luminosity ($L_{\rm 14-195 keV}$) linearly probes the bolometric AGN luminosity, as shown by \citet{Winter:2012yq}. $L_{\rm SF}$ measures the SFR using Equation~\ref{eq:sfr_ir}.

Figure~\ref{fig:sfr_lbat_correlation} plots the correlation between the SFR and $L_{\rm 14-195 keV}$ for the \herschel-BAT AGN. Black points with error bars indicate all of our detected sources, while red arrows show 95 per cent confidence upper limits for objects with less four SED points. We determine parameters for the following linear model:

\begin{equation}
\log {\rm SFR} = m\log L_{\rm 14-195\,keV} + b + \epsilon_{\rm int}\,,
\end{equation}

\noindent using \textsc{linmix\_err} as before. The blue line and shaded region in Figure~\ref{fig:sfr_lbat_correlation} represents the best-fit line and 95 percent confidence interval. We find a best fitting $m=0.18\pm0.07$, $b=-7.85\pm2.93$, and $\sigma^2=0.37\pm.04$ with a correlation coefficient of only 0.17$\pm$.06. Due to our large sample size, though, the probability for a null correlation is small with only a 0.25 per cent chance of measuring this correlation randomly.

%%%%%%%%%%%%%%%%%%%%%%% FIGURE: Correlation Between LBAT and SFR %%%%%%%%%%%%%%%%%%%%%%%%%%%%%
\begin{figure*}
\includegraphics{figures/sfr_lbat_correlation_nosy}
\caption{Correlation between AGN luminosity as probed by $L_{\rm 14-195\,keV}$ and SFR. Error bars on $L_{\rm14-195\,keV}$ correspond to the 90 percentile confidence interval from the 58 month BAT catalogue. Error bars on the SFR correspond to the 68 percentile confidence interval from our SED fitting. Downward pointing arrows plot 95 percent confidence upper limits on the SFR. The solid blue line with shading plots the best fit line and 95 percentile confidence interval for our AGN-SFR relationship from \textsc{linmix\_err}. No strong relationship exists between the AGN strength and global star formation. A color version of this figure is available in the online publication.\label{fig:sfr_lbat_correlation}}
\end{figure*}

%\begin{figure*}
%\includegraphics{figures/sfr_lbat_correlation_split_main_seqeunce}
%\caption{\label{fig:sfr_lbat_correlation_ms}}
%\end{figure*}
%%%%%%%%%%%%%%%%%%%%%%%%%%%%%%%%%%%%%%%%%%%%%%%%%%%%%%%%%%%%%%%%%%%%%%%%%%%

%%%%%%%%%%%%%%%%%%%%%%% FIGURE: Correlation Between LBAT and SFR %%%%%%%%%%%%%%%%%%%%%%%%%%%%%
\begin{figure*}
\includegraphics{figures/lbat_sfr_correlation_compare_lit}
\caption{Comparison between the measured SFR-$L_{\rm 14-195\,keV}$ from the \herschel-BAT sample (blue line with shaded region) and previous relationships from \citet{Netzer:2009lr} (dotted line) and \citet{Rosario:2012fr} (red dashed line). The points with error bars are the median SFRs binned by $L_{\rm 14-195\,keV}$ from the \herschel-BAT sample. Error bars represent the 68 per cent confidence interval.  A color version of this figure is available in the online publication.\label{fig:sfr_lbat_correlation_lit}}
\end{figure*}

%\begin{figure*}
%\includegraphics{figures/sfr_lbat_correlation_split_main_seqeunce}
%\caption{\label{fig:sfr_lbat_correlation_ms}}
%\end{figure*}
%%%%%%%%%%%%%%%%%%%%%%%%%%%%%%%%%%%%%%%%%%%%%%%%%%%%%%%%%%%%%%%%%%%%%%%%%%%

%None of the three samples show a strong correlation between the SFR and 14--195 keV luminosity. With slopes between 0.25--0.34 and correlation coefficients between 0.27 and 0.35, even though they are inconsistent with 0, the evidence for a connection between the strength of the AGN and global star formation is quite weak. The best-fit correlations also do not seem to follow any of the literature relationship. We attempted to fit the \herschel-BAT sample with a curving relationship such as the one found by \citet{Rosario:2012fr}, however all attempts ended up favoring a single slope relationship.

%Type 1 AGN do show a slightly stronger correlation than the whole sample and Type 2 AGN which is consistent with our findings in \citet{Melendez:2014yu} and \citet{Shimizu:2016qy} using the monochromatic \herschel{} luminosities. However, the significance of the differences are not more than 2$\sigma$ therefore we must conclude that Type 1 AGN show a similar relationship between their SFR and 14--195 keV luminosity.

%%%%%%%%%%%%%%%%%%%%%%% TABLE: AGN-SFR Connection%%%%%%%%%%%%%%%%%%%%%%%%%%%%%
%\begin{table}
%\begin{threeparttable}
%\captionsetup{font=small,labelfont=bf,labelsep=period}
%\caption{Linear Regression Between $L_{\rm14-195\,keV}$ and SFR \label{tab:lbat_sfr_correlation}}
%\begin{tabular}{lcccc}
%\toprule 
% Sample & $m$ & $b$  & $\sigma_{\rm int}$ & $\rho$  \\
%\midrule
%All        & $0.24\pm0.05$ & $-10.3\pm2.4$ & $0.21\pm0.02$ & $0.29\pm0.07$ \\
%Type 1 &  $0.32\pm0.09$ & $-13.7\pm3.9$ & $0.20\pm0.04$ & $0.35\pm0.10$ \\ 
%Type 2 &  $0.25\pm0.07$ & $-10.7\pm3.1$ & $0.25\pm0.03$ & $0.28\pm0.08$ \\
%NMS      & $0.27\pm0.09$ & $-11.3\pm3.8$ & $0.16\pm0.03$ & $0.38\pm0.12$\\
%BMS   & $0.43\pm0.11$ & $-18.9\pm5.0$ & $0.19\pm0.05$ & $0.52\pm0.13$ \\
%\bottomrule
%\end{tabular}
%\end{threeparttable}
%\end{table}
%%%%%%%%%%%%%%%%%%%%%%%%%%%%%%%%%%%%%%%%%%%%%%%%%%%%%%%%%%%%%%%%%%%%%%%%%

%As shown in \citet{Shimizu:2015xo}, the \herschel-BAT AGN span a large range in $\Delta\log$(SFR) which we defined as the logarithmic difference between the observed SFR and the SFR expected if the source lived on the star-forming main sequence with its stellar mass. Since AGN don't exclusively live either on the main sequence or off of it, its possible the large scatter we observe in Figure~\ref{fig:sfr_lbat_correlation} is due to the correlation changing as a function of $\Delta\log$(SFR). 

%To investigate this possibility, we split the sample up into galaxies ``Near the Main Sequence'' (NMS) and ``Below the Main Sequence'' (BMS). NMS sources correspond to $|\Delta\log$(SFR)$|< 0.64$ dex. These are galaxies that are within 2$\sigma$ of our measured main sequence in \citet{Shimizu:2015xo}. BMS galaxies have $\Delta\log$(SFR)$< -0.64$ dex and are more than 2$\sigma$ below the main sequence. 

%In Figure~\ref{fig:sfr_lbat_correlation_ms} we plot the split sample with NMS galaxies in blue and BMS galaxies in red. The total number of galaxies in both samples is significantly reduced (119 compared to 313 for the entire sample) because only a subsample of the whole \herschel-BAT AGN sample have reliable stellar masses. Using the same method as before, we calculated a best fit linear relation for both the NMS and BMS subsamples. The parameters are listed in Table~\ref{tab:lbat_sfr_correlation} and we plot the best fit line with a 95 percent confidence interval as blue and red lines with shading in Figure~\ref{fig:sfr_lbat_correlation_ms}. For reference, we also plot the relationship measured using the whole sample as a black line with grey shading. 

%The NMS sample doesn't show much change compared to the relationship measured using the entire sample with only a 0.02 change in the slope and 0.1 change in the correlation coefficient. There is a slight increase in the normalization as expected since we are using the upper half of the sample. For the BMS sample however, the slope has increased significantly by 0.18 and the correlation coefficient increased by 0.24. This is a possible indication that the strength of the relationship between AGN activity and star formation increases as galaxies become more quenched. Confirming this intriguing possibility will require more AGN with reliable masses which we are lacking in our sample. At the very least it seems plausible that the large scatter in the overall SFR-$L_{14-195\,keV}$ can be explained by the differing locations in comparison to the star forming main sequence.

%Our measured AGN-SFR relationship agrees mainly with the previous studies that found a flat or weak relationship \citep{Silverman:2009lq, Diamond-Stanic:2012rw, Mullaney:2012gf, Azadi:2015ve, Stanley:2015qy}. 

The scatter in the relationship is high, covering nearly two orders of magnitude in SFR for a given AGN luminosity. In Figure~\ref{fig:sfr_lbat_correlation_lit}, we compare our correlation with previous measured relationships at low redshift from \citet{Rosario:2012fr} (red dashed line) and \citet{Netzer:2009lr} (black dotted line). The blue points show the median $\log$ SFR in bins of 14--195 keV luminosity with error bars indicating the 16th and 84th percentiles. 

Our correlation does not agree with either \citet{Rosario:2012fr} or \citet{Netzer:2009lr}, showing a lower normalization and no upturn at high luminosities compared to the \citet{Rosario:2012fr} relationship and a much flatter slope compared to \citet{Netzer:2009lr}.The slope of the \citet{Netzer:2009lr} relationship is 0.8, which also coincides with the slope of the \citet{Rosario:2012fr} relationship at high luminosities. Both studies, however, used the 60 \micron{} luminosity as a measure of the SFR, which we showed in Section~\ref{sec:agn_70_contribution} can be strongly influenced by the AGN. \citet{Netzer:2009lr} also chose to only measure the relationship in AGN-dominated objects, leading to an even greater effect of the AGN contribution. 

At high luminosities, the upturn seen in the \citet{Rosario:2012fr} relationship could be due to Malmquist bias. Even though their sample was based on the \swift/BAT AGN, just as ours, no redshift cutoff was applied. Due to the flux-limited nature of the \swift/BAT catalogue, there is a strong selection bias where the highest luminosity objects are preferentially at high redshift, which will introduce a correlation based on distance rather than an inrinsic one. \citet{Rosario:2012fr} also only used \textit{IRAS} detected sources. The lower sensitivity of \textit{IRAS} compared to \herschel{} can explain the reduced normalization we see. %Combined with the work of \citet{Azadi:2015ve} and \citet{Stanley:2015qy}, we can firmly say that for all redshifts including the local universe, individual SFRs are not related to the current AGN luminosity.

It is however still possible that our sample did not extend to high enough AGN luminosity to reveal a trend. While the upturn can be explained by Malmquist bias, it can also be explained by the lack quasars ($L_{\rm AGN} > 10^{46}$ ergs s$^{-1}$). The upper error bars on our binned points do still extend to the \citet{Rosario:2012fr} relationship making the difference between our relationship and \citet{Rosario:2012fr} only $\sim1\sigma$ although the 95 per cent confidence interval on our relationship does not overlap with \citet{Rosario:2012fr}. Nevertheless, perhaps with more sources at higher AGN luminosity, an upturn could be revealed. The highest AGN luminosities are where we expect the AGN to have the greatest influence and the strongest link to current star formation as seen in merger simulations \citep[e.g.][]{Di-Matteo:2005lr}. \citet{Treister:2012rt} further showed that major mergers trigger only the most luminous AGN while moderate luminosity AGN seem to be triggered by secular process like disk instabilities. Therefore, perhaps the moderate luminosity AGN that comprise our sample is simply not probing the stage in an AGN host galaxy's lifetime where global star formation and nuclear activity are linked. 

In terms of the weak relationship we observe, two different explanations have been proposed to explain the lack of a relationship seen between the strength of the AGN and SFR. Recently, \citet{Diamond-Stanic:2012rw} suggested that the AGN only influences star formation in the nuclear regions of galaxies and find that when restricting their measurements to only the inner 1kpc, a stronger relationship appears. \citet{Esquej:2014vl}, probing even smaller scales ($r < 100$ pc) using MIR interferometry, also found a nearly linear relationship between the nuclear SFR and SMBH accretion rate, and  \citet{LaMassa:2013hb}, utilizing the fixed aperture of the SDSS fiber, found that only within 1.7 kpc a positive relationship between the SFR and accretion rate occurs. These are all in agreement with hydrodynamical simulations \citep{Hopkins:2010kx, Thacker:2014pd} that predict an increasingly linear relationship as the star formation size scale decreases. Work is currently ongoing to spatially decompose the SFRs of the \herschel-BAT sample to determine whether the relationship strengthens near the nucleus. %It could also explain the small differences we see when splitting the sample up by the location on the main sequence. If the BMS AGN have reduced star formation only on large scales, while the NMS have larger amounts of extended star formation, then our measurements would naturally be exploring different sizes. The reduced sizes in the BMS AGN would imply a stronger relationship, exactly what we observe in Figure~\ref{fig:sfr_lbat_correlation_ms}. Measurement of the FIR sizes and morphologies will be the subject of a future publication and we will explore the hypothesis that the star-forming region size is dependent on location with respect to the main sequence.

The second explanation involves the varying timescales associated with star formation and accretion onto the SMBH. Measuring the SFR from the IR luminosity results in an average SFR over nearly 100 Myr \citep{Kennicutt:2012it}, while the X-ray luminosity is more aligned with the instantaneous AGN luminosity especially given the observation that the X-ray emission likely originates very near to the SMBH \citep[e.g.][]{Chen:2011lr}. Therefore, if the AGN luminosity can vary over 100 Myr, while the SFR is relatively stable, this will cause any intrinsic relationship between the two to weaken. \citet{Hickox:2014yq} explored this using a simple model for the Eddington ratio distribution for an AGN and a linear relationship between the AGN luminosity and SFR, finding that a powerlaw Schechter function matches the observed weak AGN-SFR relationship. However, this was based on the upturn seen in the relationship from \citet{Rosario:2012fr}, which we clearly do not observe, suggesting that adjustments need to be made to conform with our observations.

Neither explanation, however, needs to invoke AGN feedback. An intrinsically positive correlation between the SFR and AGN luminosity, whether on small spatial scales or long timescales, can simply be explained by the availability of their common fuel, namely cold gas. Large amounts of cold gas will spur both high accretion rates and high SFRs. Certainly observations that X-ray detected AGN lie below the star-forming main sequence \citep{Shimizu:2015xo, Matsuoka:2015fk, Mullaney:2015wn} are suggestive that the AGN plays some role in quenching. AGN-driven outflows that are likely suppressing star formation have been observed in small samples \citep{Veilleux:2013qq, Cicone:2014ty, Tombesi:2015fj}; but in a recent study of 50 nearby \swift/BAT AGN, \citet{Stone:2016zl} found only four objects with evidence for a molecular outflow, strengthening the argument that AGN only contribute to driving gas out of the galaxy at high luminosity ($L_{\rm AGN} > 10^{45}$ ergs s$^{-1}$). Instead it is equally possible that at least some of our galaxies are going through passive quenching, whereby gas is simply being slowly depleted through star-formation and star-formation driven outflows after accretion onto the galaxy has slowed or been shut off \citep{Peng:2015yg}. This scenario agrees with the so-called ``bathtub'' model for galaxy evolution first proposed in \citet{Lilly:2013pd}. The fact we observe the majority of AGN host galaxies in transition would simply be a consequence that AGN activity is more likely in gas-rich galaxies \citep{Vito:2014eu}, but there is a delay between the peak of star formation and accretion onto the SMBH \citep{Davies:2007kx, Schawinski:2014cr}. This is supported by our results that AGN have higher dust masses along with higher SFRs than mass-matched field galaxies, assuming the dust-to-gas ratio is constant. On an individual galaxy basis, however, there does exist evidence that some form of AGN feedback is necessary.  Even though \citep{Peng:2015yg} conclude that slow quenching is likely the dominant quenching mechanism, they note that to obtain the $\alpha$-element enhancement observed in massive ellipticals, fast quenching must have occurred. They also suggest that AGN feedback could work in concert with slow quenching or even be the catalyst to jumpstart the quenching process.
 
%But as with the \citet{Rosario:2012fr} relationship which the \citet{Hickox:2014yq} model fits well, the \herschel-BAT data don't seem to follow the expected $z\sim0$ correlation (black dashed line in Figures~\ref{fig:sfr_lbat_correlation}. Instead, we find many AGN with low SFRs several dex below both the \citet{Rosario:2012fr} and \citet{Hickox:2014yq} relationship. This is due to the increased sensitivity realized by \herschel{} allowing measurements of the star-formation rate for nearly all of our sample and hard upper limits for the ones left undetected. Further, we necessarily do not rely on a single monochromatic luminosity as our estimate of the SFR as \citet{Rosario:2012fr} did. As we showed in Section~\ref{sec:agn_70_contribution}, doing so can cause significant overestimates of the SFR due to AGN contamination. This, combined with the low sensitivity of \textit{IRAS} data, could be the reason \citet{Rosario:2012fr} see a significant upturn in their relationship at high luminosities. However, its certainly possible we also have not probed to high enough AGN luminosity to be able to see a distinct upturn in the relationship. \citet{Lutz:2010kx} found the same flattening and upturn as \citet{Rosario:2012fr} in low and high redshift galaxies with the upturn occurring at $L_{\rm AGN} > 10^{45}$ ergs s$^{-1}$. Converting to a 14--195 keV luminosity using the relationship found by \citet{Winter:2012yq} results in $L_{\rm 14-195\,keV} > 10^{44.1}$ ergs s$^{-1}$. In our sample, there are 22 AGN above this threshold but only one above $10^{44.5}$ ergs s$^{-1}$. Therefore, our sample could just be lacking high luminosity AGN to see any significant change in the relationship. 

%It is likely both explanations contribute to our measured AGN-SFR relationship. Star-formation on global scales in AGN-host galaxies is clearly not strongly related to the strength of the AGN, but at moderate AGN luminosities, ultra-hard X-ray detected AGN occur in galaxies with SFRs extending over nearly 4 orders of magnitude. Combining our results in Section~\ref{sec:agn_sf_comparison} and \citet{Shimizu:2015xo}, it is clear that AGN host galaxies firmly lie in transition between star-forming and quiescence. The new result we find in this work is that the drop in specific SFR is coincident with a drop in dust mass, suggestive of gas depletion. 

%The remaining question then is the cause of the gas depletion. As we discuss in \citet{Shimizu:2015xo}, some form of AGN feedback could be responsible either through physical expulsion by massive AGN-powered outflows \citep[e.g.]{Cicone:2014ty, Tombesi:2015fj} or through heating of the intergalactic medium by AGN jets which causes a shutdown of gas accretion onto the galaxy such as that seen in brightest cluster galaxies \citep[e.g.]{Fabian:2003ek, McNamara:2007qy}. As suggested in the evolutionary scenario by \citet{Netzer:2009lr}, a single galaxy can go through multiple episodes of increased AGN luminosity combined with increased star-formation. This could be induced by mergers and interactions that funnel gas and dust to the nuclear regions, igniting both SMBH accretion and nuclear starbursts. \citet{Netzer:2009lr} proposes a path in the AGN-SFR plane (see his Figure 14) whereby at early times, the SFR increases first moving sources up in the diagram, then after the AGN is triggered movement is to the right increasing in AGN luminosity until feedback kicks in at the peak of AGN activity and star-formation, shutting down both. The decline could either be in parallel or with the AGN fading slower than star-formation. Since we have selected our sample using ultra-hard X-rays, we are likely seeing AGN host galaxies at all of these stages, further obscuring any long-term relationship between the two properties. The green dashed line in Figure~\ref{fig:sfr_lbat_correlation} shows the linear relationship from \citet{Netzer:2009lr} and we can see that the \herschel-BAT AGN straddle the line just as the ``loop'' path in his Figure~14 does. 

%The decreased dust masses and specific SFRs does seem to suggest that the majority of time spent by moderate luminosity X-ray detected AGN is in the declining phase. This could explain the lack of high luminosity AGN in our sample if most of them are associated with strong nuclear starbursts that are composed of Compton thick gas rendering them undetectable by \swift/BAT. \citet{Ellison:2016uq} suggested this scenario by looking at the specific SFR distribution for samples of AGN selected by different methods. They found that IR-selected AGN show enhanced star-formation compared to optically and radio-selected AGN and concluded that mergers were likely inducing starbursts as well as powerful, obscured AGN. After AGN feedback clears the galaxy of gas and dust, the AGN is more likely to be selected by X-ray and optical methods. 

%Alternatively, it is possible that AGN feedback is not necessary in the evolution of the galaxy. Recent numerical simulations suggest that even at quasar luminosities, the effect of AGN heating and outflows is only marginal, decreasing the SFR by only a few percent \citep{Gabor:2014fj, Roos:2015ty}. This would mean the fact we see AGN host galaxies with decreased dust mass and specific SFR distributions is simply coincidental. Perhaps nuclear activity is just more viable in galaxies with lowered global specific SFRs and dust masses or there is a delay in the triggering of an AGN after star-forming activity begins to shut down as suggested by \citet{Davies:2007kx} and \citet{Sani:2012vn}. Indeed, \citet{Kauffmann:2007kx} find that AGN are almost always associated with a young stellar population in bulge-dominated galaxies and the SMBH accretion rate is strongly anti-correlated with the age of the dominant stellar population in the bulge rather than the disk. The time delay could be caused by the difficulty for secular processes to funnel gas down to the pc-scales necessary for SMBH accretion. \citet{Kauffmann:2007kx} suggest that AGN are only found in galaxies with centrally concentrated gas, in line with our results on the high SFR surface densities of the \herschel-BAT sample presented in \citet{Mushotzky:2014ad}. 

It is clear then that the situation involving the connection between an AGN and star formation in its host galaxy is quite complicated. Using our well-constrained IR luminosities and ultra-hard X-ray luminosities, we find a lack of a relationship between the SFR and AGN strength. Because our sample is not extending to quasar luminosities, though, we might be missing the phase of an AGN's lifetime where the strongest link between star formation and AGN luminosity is thought to exist. Further, the AGN influence might only work over small physical scales and AGN variability might be smearing any intrinsic relationship. Future work will involve remedying two of these problems by combining our sample with low redshift quasar samples like the Palomar-Green quasar sample and utilizing the relatively high spatial resolution of the \herschel{} images to look for a connection at smaller scales. 

\section{Conclusions}
Using our high quality \herschel{} photometry from \citet{Melendez:2014yu} and \citet{Shimizu:2016qy} combined with archival \textit{WISE} 12 and 22 \micron{} photometry, we have constructed and modeled the SEDs for over 300 AGN. Our sample is unique given its nearly unbiased selection based on ultra-hard X-ray detection, as well as its local nature that eliminates possible biases and source confusion. The following is a summary of our results and conclusions.

\begin{enumerate}
\item After correcting for host galaxy contribution, we find a nearly linear correlation between the estimated 8--1000 \micron{} luminosity due to the AGN and the 14--195 keV luminosity, signifying our decompositions are accurate.

\item We determined relationships between various MIR and FIR colors as a proxy for the AGN contribution to the total 8--1000 \micron{} luminosity ($f_{\rm AGN}$), finding that the 22/70 and 22/160 colors follow the best relationship with $f_{\rm AGN}$. 

\item We investigated the AGN contribution to the 70 \micron{} emission of a galaxy and showed that for nearly 30 per cent of our sample, more than half of the 70 \micron{} flux is likely due to AGN heating.

\item We calculated a median dust mass of $10^{7.36}$ \msun{}, dust temperature of $\sim 24$ K, and SFR of 1.7 \msun{} yr$^{-1}$ for AGN host galaxies. In a comparison with a mass-matched sample of non-AGN, we find all three properties are systematically higher by a factor 2 for the dust mass, 2 K in dust temperature, and a factor of 3 in SFR. This can be explained as a consequence of high mass non-AGN samples largely being composed of quiescent early type galaxies.

\item Confident in our measures of the SFR, we find a nearly flat (slope $= 0.18\pm0.07$) relationship between the SFR and AGN luminosity. Our flat relationship is in contrast to previous low redshift studies that find either a nearly linear relationship or an upturn at high luminosities. We suspect this is due to either AGN contamination or Malmquist bias in their samples. If nuclear activity is related to global star formation, it is only on long timescales; however, another possible explanation is that the relationship strengthens only as the physical scales probed decrease.
\end{enumerate}

It is abundantly clear locally that AGN activity is not related to the global SFR as probed by the X-ray luminosity and FIR emission. However, we are working to determine whether spatial decomposition combined with our SED decomposition can illuminate the exact nature of the relationship between the AGN and nuclear star formation. This will also further test our simple decomposition methods and help expand our knowledge of the intrinsic AGN infrared SED into the FIR regime. Finally, we are also in the process of comparing the star forming properties for Type 1 and Type 2 objects to test if the obscuration of the AGN makes any significant difference.

\section*{Acknowledgements}
T.T.S. thanks Prof. Sylvain Veilleux for helpful discussions and comments that improved this work. TTS acknowledges support from a Graduate Dean's Dissertation Fellowship through the University of Maryland. We gratefully acknowledge support from NASA through Herschel contracts 1436909 (R.F.M., M.M., and T.T.S.), 1427288 (A.J.B.), and 1447980 (L.L.C.) This research has made use of the NASA/IPAC Extragalactic Database (NED), which is operated by the Jet Propulsion Laboratory, California Institute of Technology, under contract with the National Aeronautics and Space Administration. This publication makes use of data products from the Wide-field Infrared Survey Explorer, which is a joint project of the University of California, Los Angeles, and the Jet Propulsion Laboratory/California Institute of Technology, and NEOWISE, which is a project of the Jet Propulsion Laboratory/California Institute of Technology. WISE and NEOWISE are funded by the National Aeronautics and Space Administration. All figures in this publication were not possible without the Python package \textsc{MATPLOTLIB} \citep{Hunter:2007}. 
%%%%%%%%%%%%%%%%%%%%%%%%%%%%%%%%%%%%%%%%%%%%%%%%%%

%%%%%%%%%%%%%%%%%%%% REFERENCES %%%%%%%%%%%%%%%%%%

% The best way to enter references is to use BibTeX:
\bibliographystyle{mnras}
\bibliography{/Users/ttshimiz/Dropbox/Research/my_bib} % if your bibtex file is called example.bib

%%%%%%%%%%%%%%%%%%%%%%%%%%%%%%%%%%%%%%%%%%%%%%%%%%

%%%%%%%%%%%%%%%%% APPENDICES %%%%%%%%%%%%%%%%%%%%%

\appendix

\section{SED Parameter Estimation}\label{sec:sed_bayes}
%We used two different fitting methods to fit the SEDs of the galaxies: 1) a Bayesian framework including Monte Carlo Markov Chain (MCMC) analysis for parameter estimation and 2) maximum likelihood optimization. For the C12 model, we chose the Bayesian framework, and for the template based models (DecompIR and D14) we chose maximum likelihood optimization. The main reason for using two different methods to determine the best fit model SED is that the Bayesian framework does not work well with a set of discrete model templates, especially when the number of templates is low. Further, the only parameter in the template fitting that needs to be optimized is the normalization of the template to fit the observed SED. With a uniform prior on the one parameter, Bayesian methods essentially are reduced to maximum likelihood optimization.
To find the best fitting parameters in our SED modeling, instead of standard least squares analysis, we used a Bayesian framework along with Monte Carlo Markov Chains to probe the posterior probability distribution functions for each parameter. The Bayesian framework allows for robust estimates of the uncertainty and for explicit statements about prior knowledge of the parameters. It also makes it relatively easy to include information contained in the undetected photometry of the SED.

\subsection{Likelihood Representation}
The likelihood defines the probability of observing a set of data given a specific model. In SED fitting, this translates to the combined probability of measuring all the photometric data points in the observed SED given a model for the SED (whether based on templates or analytic models). The total likelihood can then be expressed as the product of the probabilities of observing each single photometric point:

\begin{equation}\label{eq:likelihood}
\mathcal{L}(F|M) = \prod_{i}P(F_i|M)\,,
\end{equation}

\noindent where $F$ is the set of photometric fluxes, $F_i$ and $M$ is the model. For our analysis, we assume the probability of our observations follows a Gaussian distribution with mean equal to $M$ and standard deviations equal to the measurement errors, $\sigma_{i}$.

\begin{equation}\label{eq:detected_prob}
P(F_i|M) = \frac{1}{\sqrt{2\pi\sigma_{i}^{2}}}\mathrm{exp}\left(\frac{-(F_i - M)^2}{2\sigma_i^2}\right)\,,
\end{equation}

Equation~\ref{eq:detected_prob} only defines the probability for detected fluxes. To use the information contained in the undetected photometry, $U_i$, we define a different probability under the assumption that all of the upper limits are 5$\sigma$. 

\begin{align}\label{eq:undetected_prob}
P(U_i|M) &= \int_{-\infty}^{U_i}\frac{1}{\sqrt{2\pi\sigma_{i}^{2}}}\mathrm{exp}\left(\frac{-(x - M)^2}{2\sigma_i^2}\right)dx \notag\\
&= \frac{1}{2}\left(1 + \mathrm{erf}\left[\frac{U_i - M}{\sigma_i\sqrt{2}}\right]\right)\,,
\end{align}

\noindent where $\sigma_i = \frac{U_i}{5}$ and erf is the standard error function. For numerical accuracy and simplicity, it is customary to minimize the negative log-likelihood. Supposing we have $N$ total SED points with $D$ detections and $D-N$ non-detections then the total negative log-likelihood combining Equations~\ref{eq:likelihood},~\ref{eq:detected_prob},~and~\ref{eq:undetected_prob} is:

\begin{align}\label{eq:log_like}
-\log\mathcal{L} = \frac{1}{2}&\sum_{i=0}^{D}\left[\log(2\pi\sigma_i^2) - \left(\frac{F_i - M}{\sigma_i}\right)^2\right] + \notag \\
&\sum_{j=0}^{D-N}\log\left[1 + \mathrm{erf}\left(\frac{U_j - M}{\sigma_j\sqrt{2}}\right)\right]\,,
\end{align}

It is important to recognize here how $M$ is calculated, no matter whether it represents a template or analytic model. Each data point in an SED is the observer-frame flux density measured over a defined wavelength range. Therefore, to determine the model flux densities we first redshifted the full rest-frame model SED into the observer frame using the known redshifts of all of our sources. This observer-frame SED was then convolved with each instrument filter transmission curve to produce model flux densities that can be accurately compared to the observed ones.

\subsection{Bayesian MCMC Analysis}
Within the Bayesian framework, the important probability is the probability of the model given the data at hand, i.e. the most probable SED model given the observed fluxes. This probability can be determined using Bayes theorem:

\begin{equation}\label{eq:bayes}
P(M|F) = \frac{P(F|M)P(M)}{P(F)}\,,
\end{equation}

\noindent and is known as the posterior probability distribution. $P(F|M)$ is proportional to the likelihood (Equation~\ref{eq:likelihood}), $P(M)$ codifies our prior knowledge about the model, and $P(F)$ is the model evidence and can be disregarded as a simple normalization term. 

The reader may notice that assuming a flat prior, $P(M) \propto 1$, reduces Equation~\ref{eq:bayes} to $P(M|F) \propto \mathcal{L}$ verifying our use of maximum likelihood in determining the best template models.

For the C12 model, we used flat priors for the dust temperature and dust mass. We placed conservative limits on both the dust temperature and $\log M_{\mathrm dust}$ to be between 1 and 100 K and 1 and 10 \msun respectively. Within the powerlaw component for the C12 model, we also used flat priors for the powerlaw slope between -5 and 5 and the log of the normalization between -10 and 10. Based on previous work attempting to measure the intrinsic AGN SED and modeling the dusty torus, we expect the SED to turnover anywhere in the range between 20-70 \micron. Thus, we imposed a Gaussian prior centered at 45 \micron{} with a standard deviation of 20 \micron. We found that imposing this prior resulted in better and more realistic fits to the SEDs.

We used the \textsc{PYTHON} package \textsc{emcee} \citep{Foreman-Mackey:2013lr} to perform MCMC and sample the posterior probability distribution function (Equation~\ref{eq:bayes}). \textsc{emcee} runs an implementation of the Affine-Invariant MCMC sampler from Goodman \& Weare 2010. Instead of one single MCMC chain, it samples the posterior PDF with multiple ``walkers'', each with their own chain. For our analysis, we used 50 walkers that each produced a 1000 step chain. To allow for each chain to stabilize and move away from the initial guesses for the parameters, we imposed a 200 step ``burn-in''. In total, this resulted in 40000 steps to define the full posterior PDF. 

To determine the best fit parameters, we first marginalized the posterior PDF over all other parameters and then calculated the median. All quoted uncertainties represent the 68 per cent confidence interval determined from the 16th and 84th percentile of the marginalized posterior PDF.

For sources with less than four detections in their SED, we fixed the dust temperature to 25 K since the peak of the FIR bump is no longer constrained. In these cases, we only obtained upper limits on the dust mass by calculating the upper 95th percentile of the marginalized PDF.

\section{Comparison between different models}\label{sec:comp_models}
In this section we compare the results for the total luminosity, IR AGN luminosity, and star forming luminosity between the C12 model and two other models to decompose the SED. We exclude from this analysis objects which were only detected by \herschel{} a less than one waveband given the strong uncertainties associated with their properties. We also exclude as before the 6 radio-loud objects and Mrk 3 (for lack of WISE data). 

\subsection{DecompIR model}
Besides analytic models, another popular method is the use of template SEDs. Templates are constructed based on well-sampled SEDs of large samples of galaxies and usually parameterized according to a known property such as infrared luminosity.

For galaxies known to host an AGN, recent studies have turned to the \textsc{DecompIR} \citep{Mullaney:2011yq} templates. \textsc{DecompIR} consists of five host galaxy templates that span the IR color and luminosity range of the original \citet{Brandl:2006kx} starburst galaxy templates. \citet{Mullaney:2011yq} constructed the AGN templates based on a subsample of the \swift/BAT AGN which had AGN dominated \textit{Spitzer}/IRS spectra determined by the equivalent width of the 11.3 \micron{} feature being $<0.03$ \micron. The \textit{Spitzer}/IRS spectra were combined with \textit{IRAS} photometry at 60 and 100 \micron{} to define the ``intrinsic'' AGN SED from 6--100 \micron. 

\citet{Mullaney:2011yq} created three different AGN templates: one based only on high AGN luminosity objects, low AGN luminosity objects, and a median of the entire sample. For this work, we only consider the median AGN template given our SEDs only contain two points in the MIR where AGN-related emission is expected to dominate. 

\subsection{Dale et al 2014 model}
The third model we chose to test on our sample is the semi-empirical templates from \citet[][hereafter D14]{Dale:2014yq}. These templates also contain two components, one for dust emission in the host galaxy and one for the AGN. The host galaxy components were built from an updated version of the \citet{Dale:2002ty} model. Each component represents an SED produced using a different value of $\alpha_{\mathrm SF}$, which is the powerlaw slope of the intensity distribution for the interstellar radiation field that is heating the dust. These SEDs contain a mixture of emission from PAHs, small stochastically heated grains, and thermally radiating large grains.

For the AGN component, D14 chose the median SED of the Palomar-Green quasars from \citet{Shi:2013vn} citing the care with which any star-forming component was removed and the prominence of several AGN related MIR features such as the [OIV] fine structure line and the broad 10 and 18 \micron{} silicate emission bumps. At long wavelengths the AGN template falls as a blackbody.

Instead of two separate templates for the AGN and host galaxy, D14 provided a single set of templates based on different combinations of $\alpha_{\mathrm SF}$ and $f_{\mathrm AGN}$, the fractional contribution of the AGN to the 5--20 \micron{} emission. In total there are 1365 templates that range in $\alpha_{\mathrm SF}=0.0625-4.0$ in 0.0625 intervals and $f_{\mathrm AGN} = 0-1$ in 0.05 intervals.

\subsection{SED Fitting for the template models}
For each template in a model set, Equation~\ref{eq:log_like} was minimized to determine the best fit normalization. We then chose the normalized template with the lowest $-\log\mathcal{L}$ as the best fitting model for a source. For the DecompIR set, this meant first simultaneously optimizing over a normalization for the AGN component and each host galaxy component, then choosing the combined template that resulted in the minimum $-\log\mathcal{L}$. For D14, this meant calculating $-\log\mathcal{L}$ over the entire set of $\alpha$ and $f_{AGN}$ templates.

Uncertainties using the maximum likelihood method were determined by generating 1000 simulated SEDs for each source. These data points in the simulated SEDs were calculated by assuming each detected point followed a Gaussian distribution with mean equal to the observed flux density and a standard deviation equal to the measured error. Each of the simulated SEDs were re-fit using the same method. The standard deviation on the set of best fit parameters from the simulated SEDs then was used as the uncertainty. In this way both statistical and systematic errors can be taken into account in assessing the reliability of our best fit parameters. 

\subsection{$L_{\rm IR}$ Comparison}
The measured property that should be most model independent is the total infrared luminosity, $L_{\rm IR}$. Assuming each model was able to fit well the broadband SEDs and reproduce the observed photometry, $L_{\rm IR}$ is simply the total integrated energy underneath the SED, irregardless of how the SED is decomposed. In Figure~\ref{fig:lir_total_comp}, we plot the correlations between the C12 model and DecompIR and D14 models for$L_{\rm IR}$.  

The median $\log\,L_{\rm IR}$ for the C12, DecompIR, and D14 models are all 10.39 \lsun with a spread of 0.5 dex. Clearly, based on Figure~\ref{fig:lir_total_comp}, each model well fits the broadband SED of our sample with a small scatter around the 1-to-1 correspondence line. The average difference in the luminosities from each model is only $\sim0.01$ dex with a standard deviation between 0.05 and 0.08 dex, slightly higher than the median statistical uncertainty for the C12 model of 0.03 dex.  

%%%%%%%%%%%%%%%%%%%%%%%% FIGURE: LIR COMPARISON %%%%%%%%%%%
\begin{figure*}
\includegraphics[width=0.75\textwidth]{figures/lir_total-comparison_nokde}
\caption{Correlations of total $L_{\rm IR}$ between the C12 model and DecompIR and D14. The red dashed line indicates a 1-to-1 correspondence.\label{fig:lir_total_comp}}
\end{figure*}
%%%%%%%%%%%%%%%%%%%%%%%%%%%%%%%%%%%%%%%%%%%%%%%%%%

%%%%%%%%%%%%%%%%%%%%%%%% FIGURE: LSF COMPARISON %%%%%%%%%%%
\begin{figure*}
\includegraphics[width=0.75\textwidth]{figures/lir_sf-comparison_nokde}
\caption{Same as Figure~\ref{fig:lir_total_comp} but for $L_{\rm SF}$ \label{fig:lir_sf_comp}}
\end{figure*}
%%%%%%%%%%%%%%%%%%%%%%%%%%%%%%%%%%%%%%%%%%%%%%%%%%

%%%%%%%%%%%%%%%%%%%%%%%% FIGURE: LAGN COMPARISON %%%%%%%%%%%
\begin{figure*}
\includegraphics[width=0.75\textwidth]{figures/lir_agn-comparison_nokde}
\caption{Same as Figure~\ref{fig:lir_total_comp} but for $L_{\rm AGN,\,IR}$ \label{fig:lir_agn_comp}}
\end{figure*}
%%%%%%%%%%%%%%%%%%%%%%%%%%%%%%%%%%%%%%%%%%%%%%%%%%


\subsection{$L_{\rm SF}$, $L_{\rm AGN, IR}$, and $f_{\rm AGN}$ Comparison}
Where the models begin to disagree more, is in the actual decomposition of the SED. In particular, we compare $L_{\rm SF}$, $L_{\rm AGN, IR}$, and$f_{\rm AGN}$. Calculating these parameters for the template models is relatively easy compared to the corrections we had to make for the C12 model. For the DecompIR template set, $L_{\rm IR}$, $L_{\rm SF}$, and $L_{\rm AGN, IR}$ were simply calculated from the best fit total, host galaxy, and AGN SEDs. For the D14 model set, $L_{\rm IR}$ was measured from the best fitting template and $L_{\rm SF}$ and $L_{\rm AGN, IR}$ were calculated based on the best fit $f_{\rm AGN}$.\footnote{\citet{Dale:2014yq} only provides $f_{\rm AGN}$ calculated between 5--15 \micron; however, D. Dale graciously provided the authors with $f_{\rm AGN}$ calculated between 8--1000 \micron{} through private communication in August 2015.} Uncertainties on all of these luminosities were determined with the same Monte Carlo method used to determine the uncertainties on the best fitting parameters.

Figures~\ref{fig:lir_sf_comp},~\ref{fig:lir_agn_comp},~and~\ref{fig:agn_frac_comp} display the relationship between each model's $L_{\rm SF}$, $L_{\rm AGN, IR}$, and, $f_{\rm AGN}$. Out of the three parameters, $L_{\rm SF}$ is most consistent between the three models. The C12, DecompIR, and D14 models have a median $L_{\rm SF}$ of 10.12, 10.12, and 10.19 \lsun{} respectively. The average difference between C12 and DecompIR is only 0.006 dex with a standard deviation of 0.16 dex while the average difference between C12 and D14 is -0.07 dex with a standard deviation of 0.11 dex. The DecompIR and D14 are measuring slightly higher star-forming luminosities. Based on these differences, we assign a conservative uncertainty of 0.2 dex for $L_{\rm SF}$ to account for model dependence.

The comparison for $L_{\rm AGN, IR}$ shows increased scatter compared to the two previous luminosities. The median $L_{\rm AGN, IR}$ for the three models is 10.04, 9.9, and 9.85 \lsun{} for the C12, DecompIR, and D14 models respectively. The average difference between C12 and DecompIR is 0.10 dex with a standard deviation of 0.20 dex while the average difference between C12 and D14 is 0.14 dex with a standard deviation of 0.20 dex. Since the DecompIR and D14 were measuring higher $L_{\rm SF}$, it makes sense that they are measuring lower $L_{\rm AGN, IR}$ given the total infrared luminosities between the three models was identical. It is likely that the more flexible C12 model, which allows the PL component to extend to longer wavelengths, creates the discrepancy between it and the template based models. Our implementation of DecompIR and D14 adheres to a strict single template for the intrinsic AGN SED. Adding the offset and spread in the differences in quadrature indicates an uncertainty of 0.25 dex in the determination of $L_{\rm AGN, IR}$. 

The relatively large uncertainty in $L_{\rm AGN, IR}$ then leads to a large scatter in $f_{\rm AGN}$ as shown in Figure~\ref{fig:agn_frac_comp}. The average difference between C12 and DecompIR is only 0.02; however, the standard deviation is 0.13. The average difference between C12 and D14 is 0.1 with a standard deviation of 0.11. Given the spread in $f_{\rm AGN}$ is comparable to the average statistical uncertainty (0.14) from our C12 modeling, we do not add any more uncertainty onto our estimates. 

The 15 per cent uncertainty between the three models represents our general lack of knowledge about the details of decomposing broadband SEDs of AGN host galaxies. The estimated $f_{\rm AGN}$ is highly dependent on the assumed models of both the host galaxy and AGN, and currently at best we can only constrain to within 15 per cent. This influences then the calculations of both the SFR (based on $L_{\rm SF}$) and the IR portion of AGN luminosity ($L_{\rm AGN, IR}$). Studies relying on calculating the SFR of AGN host galaxies using the infrared need to take these discrepancies into account or else risk over-interpreting results based on broad SED decomposition. This also highlights the need for high angular resolution studies of AGN, to determine the true shape of the AGN IR SED.

%%%%%%%%%%%%%%%%%%%%%%%% FIGURE: AGN FRACTION COMPARISON %%%%%%%%%%%%%%%
\begin{figure*}
\includegraphics[width=0.8\textwidth]{figures/agn_frac-comparison_nokde}
\caption{Correlations of $f_{\rm AGN}$ from the three SED models.  Error bars represent the 68 per cent confidence interval. Upper limits are indicated as red downward pointing arrows. The red dashed line indicates a 1-to-1 correspondence. A color version of this figure is available in the online publication. \label{fig:agn_frac_comp}}
\end{figure*}
%%%%%%%%%%%%%%%%%%%%%%%%%%%%%%%%%%%%%%%%%%%%%%%%%%%%%%%%%%%%%

\section{Comparison of $L_{\rm SF}$ with PAH measurements}\label{sec:pah_comp}
To test our estimates of the star-forming luminosity ($L_{\rm SF}$), we compare them with estimates of the star-forming luminosity calculated from the PAH 11.3 and 7.7 \micron{} luminosity. PAH features are thought to be caused by the vibrations of complex hydrocarbons \citep{Draine:1984fk, Puget:1989lr} and have been shown to be reliable tracers of the SFR \citep{Roussel:2001kx, Sargsyan:2009fj, Treyer:2010uq, Shipley:2016qy} even for AGN \citep{Diamond-Stanic:2010kl}. Thus, comparing our $L_{\rm SF}$ with the 11.3 and 7.7 \micron{} PAH luminosity provides a further check on our SED decomposition. 

To measure the 7.7 and 11.3 \micron{} luminosities we first searched the Cornell Atlas of Spitzer IRS Sources \citep[CASSIS;]{Lebouteiller:2011dk} for low-resolution Spitzer/IRS spectra of the \herschel-BAT AGN. We found 120/313 AGN on CASSIS and downloaded their reduced spectra using the default extraction method. Due to the different sizes of the Short-Low (SL) slit and Long-Low (LL) slit, the continuum of each can be offset mainly from extended emission not captured by the SL slit. Therefore, spectral orders were stitched together by fitting a line to the ends of each order and scaling to match the overlapping regions. SL2 and LL2 were first matched and scaled to SL1 and LL1 respectively. Then the combined SL1/SL2 spectra was matched to the combined LL1/LL2 spectra to produce the final full Spitzer/IRS low resolution spectra. 

We fit each spectra using small spectral windows centered on 7.7 and 11.3 \micron. Following \citet{Smith:2007lr} and the popular spectral fitting software package, PAHFIT, we modeled the PAH features using a Drude profile and any emission lines contained in the spectral windows as narrow Gaussians. The full-width-at-half-maximum (FWHM) and central wavelength of the emission lines were kept fixed at the values in Table 2 of \citet{Smith:2007lr}. The 7.7 and 11.3 \micron{} features are actually complexes with multiple subfeatures. PAHFIT models these with multiple Drude profiles chosen to reproduce the observed features and we follow this method using the values for the fractional contribution and FWHM as given in Table 3 of \citet{Smith:2007lr}. We model the continuum as a simple third-order polynomial. This allows the continuum to take on many shapes that is more necessary for AGN spectra, which can show broad silicate absorption and emission and varying continuum slopes \citep{Baum:2010kh, Spoon:2007oq, Wu:2009pt}. While silicate absorption is included within PAHFIT, silicate emission is absent prompting the need for own modeling. Finally, to determine the best-fit model, we again use the Bayesian methodology and MCMC described in Appendix~\ref{sec:sed_bayes}. This allows us to treat the continuum and other nearby features as nuisance parameters and robustly determine uncertainties on the 7.7 and 11.3 \micron{} luminosities.

Figure~\ref{fig:lsf_lpah} shows the correlation between $L_{\rm SF}$ and $L_{\rm PAH}$ for both the 7.7 (right) and 11.3 \micron{} (left) complexes. We chose to use the KINGFISH galaxies (red squares) as a comparison here rather than HRS given the readily available PAH measurements. $L_{\rm SF}$ was determined using the C12 model in the same way as the HRS sample and $L_{\rm PAH}$ is from \citet{Smith:2007lr}. The $L_{\rm PAH}$ for KINGFISH were measured using PAHFIT, but given the large similarity with our method, we do not expect large differences.

Based on Figure~\ref{fig:lsf_lpah}, both the KINGFISH and \herschel-BAT samples seem to follow broadly the same correlation indicating we are neither over nor under estimating $L_{\rm SF}$ from our SED decomposition for the \herschel-BAT AGN. Using \textsc{LINMIX}, we measured the relationship for both samples and both PAH luminosities. The median correlations with 95 per cent confidence intervals are shown in Figure~\ref{fig:lsf_lpah} with the blue and red solid lines and shaded regions corresponding to the \herschel-BAT AGN and KINGFISH samples respectively. For the \herschel-BAT AGN we find slopes of $1.26\pm0.1$ (11.3 \micron) and $1.32\pm0.11$ (7.7 \micron) with correlation coefficients of 0.92 and 0.93 while for the KINGFISH galaxies we find slopes of $1.02\pm0.15$ and $1.11\pm0.16$ with correlation coefficients of 0.82 for both. In all four cases the probability of a null correlation is extremely small with the largest probability being $10^{-10}$. The \herschel-BAT AGN show a slightly steeper relationship than the KINGFISH galaxies, however they are still within $\sim1.5\sigma$ of each other. The steepness of the correlation for the \herschel-BAT AGN could be driven by the large number of upper limits for the PAH luminosities. Indeed, when excluding these from the analysis we find slopes of $1.13\pm0.09$ and $1.15\pm0.12$, much closer to the KINGFISH galaxies. 

For the purposes of this work, Figure~\ref{fig:lsf_lpah} indicates that our SED decomposition of the \herschel-BAT AGN has produced reliable estimates of the star-forming luminosity consistent with measurements of the PAH 11.3 and 7.7 \micron{} luminosity. A more detailed analysis on the exact correlation between broadband IR luminosity and PAH emission for AGN is beyond the scope of this Paper and has no bearing on our results.


%For reference we also display the SFR-$L_{\rm PAH}$ relationship from \citet{Shipley:2016qy} where we have converted their SFR to $L_{\rm SF}$ using the following equation from \citet{Murphy:2011rt}:

%\begin{equation}\label{eq:sfr_ir}
%\rm{SFR} = \frac{L_{\rm SF}}{2.57\times10^{43}}
%\end{equation}

%The \citet{Shipley:2016qy} relationship follows approximately the same slope however has an offset towards high $L_{\rm PAH}$. This is likely due to the SFR indicator used in their work which was based on the combined H$\alpha$ and 24 \micron{} luminosity. These indicators are more sensitive to recent star-formation within the last 10 Myr whereas our $L_{\rm SF}$ based primarily on FIR emission extends to star-formation as long ago as 100 Myr \citep{Kennicutt:2012it} which would explain why most of our points, both KINGFISH and \herschel-BAT, lie below the \citet{Shipley:2016qy} line. Nevertheless, Figure~\ref{fig:lsf_lpah} shows that our derived $L_{\rm SF}$ will give consistent estimates of the SFR necessary for comparing non-AGN and AGN host galaxies without any threat of systematic bias.

\begin{figure*}
\includegraphics{figures/lsf-vs-lpah}
\caption{Correlation between $L_{\rm SF}$ and the 11.3 (left) and 7.7 \micron{} (right) luminosity. Blue points correspond to the \herschel-BAT AGN and red squares correspond to the KINGFISH galaxies. Sources with upper limits on $L_{\rm PAH}$ are shown as open triangles. Measured correlations between $L_{\rm SF}$ and the 11.3 and 7.7 \micron{} luminosity are shown as solid blue and red lines for the \herschel-BAT AGN and KINGFISH samples respectively. The shaded region around the median correlations indicate the 95 per cent confidence interval. The broad consistency of the correlation for both the KINGFISH and \herschel-BAT samples shows that SFRs determined from $L_{\rm SF}$ will be comparable without any systematic bias. \label{fig:lsf_lpah}}
\end{figure*}

\section{SED Figures}

\begin{figure*}
\centering
\includegraphics[width=\textwidth]{figures/sedfig1}
\caption{Observed frame 12--500 \micron{} SEDs for all of the \herschel-BAT sample. Black points plot the observed flux densities with downward-pointing red arrows indicating 5$\sigma$ upper limits. The solid blue line and shaded region shows the best-fit C12 model with a 95 percent confidence interval. The red squares are the model flux densities after convolving the best-fit model SED with each instrument's transmission curve. The orange and green dashed lines show the best-fit PL and MBB components, respectively. \label{fig:seds}}
\end{figure*}
%\begin{figure*}
\centering
\includegraphics[width=\textwidth]{figures/sedfig1}
\caption{12--500 \micron{} SEDs for all of the \herschel-BAT sample. Black points plot the observed flux densities with open red circles indicating 5$\sigma$ upper limits. The solid blue line and shaded region shows the best-fit C12 model with a 95 percent confidence interval. The red squares are the model flux densities after convolving the best-fit model SED with each instrument's transmission curve. The orange and green dashed lines show the best-fit PL and MBB components, respectively. \label{fig:seds}}
\end{figure*}

\begin{figure*}
\centering
\includegraphics[width=\textwidth]{figures/sedfig2}
\caption{}
\end{figure*}

\begin{figure*}
\centering
\includegraphics[width=\textwidth]{figures/sedfig3}
\caption{}
\end{figure*}

\begin{figure*}
\centering
\includegraphics[width=\textwidth]{figures/sedfig4}
\caption{}
\end{figure*}

\begin{figure*}
\centering
\includegraphics[width=\textwidth]{figures/sedfig5}
\caption{}
\end{figure*}

\begin{figure*}
\centering
\includegraphics[width=\textwidth]{figures/sedfig6}
\caption{}
\end{figure*}

\begin{figure*}
\centering
\includegraphics[width=\textwidth]{figures/sedfig7}
\caption{}
\end{figure*}

\begin{figure*}
\centering
\includegraphics[width=\textwidth]{figures/sedfig8}
\caption{}
\end{figure*}

\begin{figure*}
\centering
\includegraphics[width=\textwidth]{figures/sedfig9}
\caption{}
\end{figure*}

\begin{figure*}
\centering
\includegraphics[width=\textwidth]{figures/sedfig10}
\caption{}
\end{figure*}

\begin{figure*}
\centering
\includegraphics[width=\textwidth]{figures/sedfig11}
\caption{}
\end{figure*}

\begin{figure*}
\centering
\includegraphics[width=\textwidth]{figures/sedfig12}
\caption{}
\end{figure*}

\begin{figure*}
\centering
\includegraphics[width=\textwidth]{figures/sedfig13}
\caption{}
\end{figure*}

\begin{figure*}
\centering
\includegraphics[width=\textwidth]{figures/sedfig14}
\caption{}
\end{figure*}

\begin{figure*}
\centering
\includegraphics[width=\textwidth]{figures/sedfig15}
\caption{}
\end{figure*}

\begin{figure*}
\centering
\includegraphics[width=\textwidth]{figures/sedfig16}
\caption{}
\end{figure*}

\begin{figure*}
\centering
\includegraphics[width=\textwidth]{figures/sedfig17}
\caption{}
\end{figure*}

\begin{figure*}
\centering
\includegraphics[width=\textwidth]{figures/sedfig18}
\caption{}
\end{figure*}

\begin{figure*}
\centering
\includegraphics[width=\textwidth]{figures/sedfig19}
\caption{}
\end{figure*}

\begin{figure*}
\centering
\includegraphics[width=\textwidth]{figures/sedfig20}
\caption{}
\end{figure*}

\begin{figure*}
\centering
\includegraphics[width=\textwidth]{figures/sedfig21}
\caption{}
\end{figure*}

\begin{figure*}
\centering
\includegraphics[width=\textwidth]{figures/sedfig22}
\caption{}
\end{figure*}

\begin{figure*}
\centering
\includegraphics[width=\textwidth]{figures/sedfig23}
\caption{}
\end{figure*}

\begin{figure*}
\centering
\includegraphics[width=\textwidth]{figures/sedfig24}
\caption{}
\end{figure*}

\begin{figure*}
\centering
\includegraphics[width=\textwidth]{figures/sedfig25}
\caption{}
\end{figure*}

\begin{figure*}
\centering
\includegraphics[width=\textwidth]{figures/sedfig26}
\caption{}
\end{figure*}

%\section{Table of Best Fit Parameters}
%\documentclass[onecolumn]{mn2e}

\usepackage{longtable}
\usepackage{pdflscape}

\newcommand{\um}{$\mathrm{\mu}$m}
\newcommand{\msun}{M$_{\odot}$}
\newcommand{\lsun}{L$_{\odot}$}

\begin{document}

\setcounter{table}{0}
\begin{landscape}
{\footnotesize
\begin{center}
\begin{longtable}{lccccccccc}

\caption{Best Fit Casey 2012 Model Parameters and Luminosities}\\

\hline \multicolumn{1}{l}{Name} & \multicolumn{1}{c}{$\log\,M_{\rm dust}$} &\multicolumn{1}{c}{$T_{\rm dust}$} & \multicolumn{1}{c}{$\log\,N_{\rm PL}$} & \multicolumn{1}{c}{$\alpha$} & \multicolumn{1}{c}{$\lambda_{\rm c}$}  & \multicolumn{1}{c}{$\log\,L_{\rm IR}$} & \multicolumn{1}{c}{$\log\,L_{\rm MBB}$} & \multicolumn{1}{c}{$\log\,L_{\rm PL}$} &\multicolumn{1}{c}{$f_{\rm AGN,C12}$} \\
\multicolumn{1}{l}{} & \multicolumn{1}{c}{\msun} & \multicolumn{1}{c}{K} & \multicolumn{1}{c}{} & \multicolumn{1}{c}{} & \multicolumn{1}{c}{\um} & \multicolumn{1}{c}{\lsun} & \multicolumn{1}{c}{\lsun} & \multicolumn{1}{c}{\lsun} & \multicolumn{1}{c}{}\\
\hline
\endfirsthead

\multicolumn{10}{c}%
{ \tablename\ \thetable{} -- continued from previous page} \\
\hline \multicolumn{1}{l}{Name} & \multicolumn{1}{c}{$\log\,M_{\rm dust}$} &\multicolumn{1}{c}{$T_{\rm dust}$} & \multicolumn{1}{c}{$\log\,N_{\rm PL}$} & \multicolumn{1}{c}{$\alpha$} & \multicolumn{1}{c}{$\lambda_{\rm c}$}  & \multicolumn{1}{c}{$\log\,L_{\rm IR}$} & \multicolumn{1}{c}{$\log\,L_{\rm MBB}$} & \multicolumn{1}{c}{$\log\,L_{\rm PL}$} &\multicolumn{1}{c}{$f_{\rm AGN,C12}$} \\
\multicolumn{1}{l}{} & \multicolumn{1}{c}{\msun} & \multicolumn{1}{c}{K} & \multicolumn{1}{c}{} & \multicolumn{1}{c}{} & \multicolumn{1}{c}{\um} & \multicolumn{1}{c}{\lsun} & \multicolumn{1}{c}{\lsun} & \multicolumn{1}{c}{\lsun} & \multicolumn{1}{c}{}\\
\hline
\endhead

\hline \hline
\endfoot
1RXS J044154.5-082639 & $6.80_{-0.14}^{+0.21}$ & $27.98_{-4.21}^{+2.45}$ & $-0.61_{-0.19}^{+0.18}$ & $1.70_{-0.34}^{+0.46}$ &$47.71_{-13.96}^{+18.85}$ & $10.37_{-0.04}^{+0.03}$ & $9.88_{-0.25}^{+0.10}$ & $10.20_{-0.10}^{+0.10}$ & $0.57_{-0.14}^{+0.19}$ \\
1RXS J045205.0+493248 & $7.47_{-0.10}^{+0.12}$ & $23.27_{-1.61}^{+0.99}$ & $-0.31_{-0.13}^{+0.13}$ & $1.44_{-0.46}^{+0.59}$ &$39.95_{-13.09}^{+14.82}$ & $10.47_{-0.03}^{+0.04}$ & $10.07_{-0.09}^{+0.03}$ & $10.25_{-0.05}^{+0.09}$ & $0.47_{-0.06}^{+0.12}$ \\
2E 1739.1-1210 & $7.55_{-0.19}^{+0.22}$ & $25.11_{-3.00}^{+2.11}$ & $-0.09_{-0.17}^{+0.16}$ & $1.46_{-0.37}^{+0.51}$ &$45.22_{-14.16}^{+16.42}$ & $10.83_{-0.04}^{+0.03}$ & $10.35_{-0.14}^{+0.06}$ & $10.65_{-0.07}^{+0.09}$ & $0.55_{-0.09}^{+0.13}$ \\
2MASS J07594181-3843560 & $<6.58$ & ... & $-0.42_{-0.19}^{+0.14}$ & $1.08_{-0.46}^{+0.51}$ &$39.20_{-5.56}^{+7.93}$ & $<10.58$ & $<9.37$ & $10.55_{-0.05}^{+0.05}$ & $>0.92$ \\
2MASS J17485512-3254521 & $<6.14$ & ... & $-1.07_{-0.16}^{+0.21}$ & $1.45_{-0.47}^{+0.56}$ &$44.52_{-16.37}^{+18.07}$ & $<9.34$ & $<8.93$ & $9.13_{-0.05}^{+0.14}$ & $>0.48$ \\
2MASX J00253292+6821442 & $6.16_{-0.18}^{+0.39}$ & $25.34_{-5.06}^{+2.79}$ & $-0.25_{-0.16}^{+0.15}$ & $1.46_{-0.40}^{+0.53}$ &$45.40_{-15.93}^{+16.38}$ & $9.61_{-0.04}^{+0.05}$ & $8.98_{-0.18}^{+0.10}$ & $9.49_{-0.08}^{+0.09}$ & $0.69_{-0.11}^{+0.12}$ \\
2MASX J01064523+0638015 & $6.81_{-0.44}^{+0.69}$ & $21.56_{-7.50}^{+9.09}$ & $-0.30_{-0.14}^{+0.10}$ & $1.86_{-0.42}^{+0.53}$ &$44.58_{-9.10}^{+10.57}$ & $10.45_{-0.03}^{+0.06}$ & $9.21_{-0.43}^{+0.45}$ & $10.42_{-0.06}^{+0.07}$ & $0.92_{-0.13}^{+0.05}$ \\
2MASX J01073963-1139117 & $7.66_{-0.06}^{+0.09}$ & $25.52_{-1.96}^{+1.27}$ & $-0.18_{-0.18}^{+0.13}$ & $2.05_{-0.49}^{+0.58}$ &$41.53_{-10.35}^{+20.19}$ & $10.88_{-0.04}^{+0.02}$ & $10.50_{-0.11}^{+0.07}$ & $10.65_{-0.11}^{+0.07}$ & $0.45_{-0.14}^{+0.12}$ \\
2MASX J03305218+0538253 & $6.76_{-0.33}^{+0.69}$ & $28.20_{-8.76}^{+6.57}$ & $-0.07_{-0.17}^{+0.12}$ & $2.46_{-0.61}^{+0.78}$ &$33.77_{-9.67}^{+8.47}$ & $10.78_{-0.04}^{+0.09}$ & $9.86_{-0.29}^{+0.23}$ & $10.72_{-0.07}^{+0.11}$ & $0.84_{-0.11}^{+0.09}$ \\
2MASX J03342453-1513402 & $7.43_{-0.04}^{+0.05}$ & $26.61_{-1.09}^{+0.61}$ & $-0.48_{-0.18}^{+0.21}$ & $1.69_{-0.44}^{+0.58}$ &$43.00_{-16.10}^{+24.14}$ & $10.58_{-0.02}^{+0.03}$ & $10.38_{-0.07}^{+0.03}$ & $10.15_{-0.08}^{+0.14}$ & $0.16_{-0.07}^{+0.15}$ \\
2MASX J03502377-5018354 & $7.20_{-0.09}^{+0.12}$ & $27.00_{-2.09}^{+1.17}$ & $-0.68_{-0.28}^{+0.32}$ & $2.10_{-0.44}^{+0.51}$ &$49.32_{-19.27}^{+20.01}$ & $10.35_{-0.01}^{+0.04}$ & $10.19_{-0.09}^{+0.04}$ & $9.85_{-0.12}^{+0.21}$ & $<0.52$ \\
2MASX J03534246+3714077 & $6.99_{-0.04}^{+0.07}$ & $26.47_{-1.65}^{+0.73}$ & $-0.39_{-0.22}^{+0.34}$ & $1.73_{-0.42}^{+0.43}$ &$52.42_{-19.70}^{+29.97}$ & $10.10_{-0.00}^{+0.04}$ & $9.92_{-0.10}^{+0.04}$ & $9.62_{-0.07}^{+0.21}$ & $<0.68$ \\
2MASX J03540948+0249307 & $<6.81$ & ... & $-0.24_{-0.10}^{+0.09}$ & $1.68_{-0.35}^{+0.45}$ &$52.28_{-8.43}^{+12.61}$ & $<10.44$ & $<9.60$ & $10.37_{-0.05}^{+0.04}$ & $>0.81$ \\
2MASX J04234080+0408017 & $7.44_{-0.10}^{+0.17}$ & $26.29_{-3.61}^{+2.14}$ & $-0.01_{-0.16}^{+0.12}$ & $1.67_{-0.30}^{+0.55}$ &$50.48_{-12.35}^{+28.28}$ & $10.95_{-0.06}^{+0.01}$ & $10.36_{-0.22}^{+0.11}$ & $10.82_{-0.12}^{+0.04}$ & $0.66_{-0.14}^{+0.12}$ \\
2MASX J04440903+2813003 & $7.06_{-0.14}^{+0.11}$ & $24.22_{-0.83}^{+1.02}$ & $-0.47_{-0.18}^{+0.14}$ & $0.96_{-0.42}^{+0.58}$ &$45.07_{-16.82}^{+19.18}$ & $9.92_{-0.03}^{+0.03}$ & $9.77_{-0.04}^{+0.04}$ & $9.40_{-0.07}^{+0.07}$ & $<0.17$ \\
2MASX J05020903+0331499 & $6.45_{-0.22}^{+0.57}$ & $20.60_{-6.21}^{+3.06}$ & $-0.81_{-0.26}^{+0.16}$ & $1.62_{-0.30}^{+0.37}$ &$58.47_{-16.96}^{+16.68}$ & $9.23_{-0.05}^{+0.02}$ & $8.73_{-0.38}^{+0.16}$ & $9.06_{-0.13}^{+0.08}$ & $0.58_{-0.20}^{+0.23}$ \\
2MASX J05054575-2351139 & $7.29_{-0.39}^{+0.35}$ & $19.26_{-3.13}^{+4.76}$ & $-0.30_{-0.11}^{+0.10}$ & $1.56_{-0.35}^{+0.46}$ &$52.21_{-9.95}^{+13.78}$ & $10.37_{-0.03}^{+0.05}$ & $9.39_{-0.14}^{+0.20}$ & $10.33_{-0.05}^{+0.05}$ & $0.86_{-0.08}^{+0.04}$ \\
2MASX J05580206-3820043 & $<6.71$ & ... & $0.03_{-0.18}^{+0.16}$ & $0.71_{-0.36}^{+0.44}$ &$51.40_{-8.31}^{+11.66}$ & $<10.97$ & $<9.50$ & $10.96_{-0.04}^{+0.04}$ & $>0.96$ \\
2MASX J06411806+3249313 & $<6.32$ & ... & $-0.51_{-0.11}^{+0.11}$ & $1.43_{-0.36}^{+0.42}$ &$51.45_{-8.36}^{+12.18}$ & $<10.44$ & $<9.10$ & $10.42_{-0.04}^{+0.05}$ & $>0.94$ \\
2MASX J06561197-4919499 & $7.34_{-0.45}^{+0.33}$ & $19.56_{-2.98}^{+5.69}$ & $0.03_{-0.14}^{+0.10}$ & $1.93_{-0.48}^{+0.48}$ &$41.99_{-5.98}^{+9.57}$ & $10.78_{-0.04}^{+0.06}$ & $9.48_{-0.11}^{+0.22}$ & $10.76_{-0.06}^{+0.06}$ & $0.93_{-0.04}^{+0.02}$ \\
2MASX J07262635-3554214 & $<7.04$ & ... & $0.59_{-0.09}^{+0.08}$ & $1.95_{-0.39}^{+0.46}$ &$49.08_{-7.22}^{+11.29}$ & $<10.99$ & $<9.83$ & $10.96_{-0.04}^{+0.05}$ & $>0.91$ \\
2MASX J07595347+2323241 & $8.18_{-0.03}^{+0.03}$ & $24.36_{-0.36}^{+0.33}$ & $-0.28_{-0.15}^{+0.16}$ & $1.26_{-0.43}^{+0.52}$ &$44.50_{-16.21}^{+18.59}$ & $11.00_{-0.01}^{+0.02}$ & $10.90_{-0.02}^{+0.02}$ & $10.32_{-0.06}^{+0.10}$ & $<0.04$ \\
2MASX J08032736+0841523 & $<6.56$ & ... & $-1.13_{-0.19}^{+0.22}$ & $1.48_{-0.44}^{+0.47}$ &$45.87_{-15.33}^{+17.34}$ & $<9.94$ & $<9.35$ & $9.81_{-0.07}^{+0.13}$ & $>0.66$ \\
2MASX J09023729-4813339 & $<6.97$ & ... & $-0.68_{-0.20}^{+0.12}$ & $1.24_{-0.29}^{+0.36}$ &$56.59_{-16.97}^{+15.82}$ & $<10.28$ & $<9.75$ & $10.13_{-0.11}^{+0.05}$ & $>0.61$ \\
2MASX J09043699+5536025 & $6.72_{-0.23}^{+0.54}$ & $25.56_{-8.05}^{+4.92}$ & $-0.48_{-0.34}^{+0.15}$ & $2.18_{-0.35}^{+0.45}$ &$51.93_{-14.73}^{+14.12}$ & $10.15_{-0.05}^{+0.01}$ & $9.56_{-0.44}^{+0.24}$ & $10.02_{-0.20}^{+0.08}$ & $0.66_{-0.31}^{+0.22}$ \\
2MASX J09235371-3141305 & $6.37_{-0.17}^{+0.15}$ & $30.80_{-2.25}^{+2.18}$ & $-1.25_{-0.21}^{+0.30}$ & $1.63_{-0.43}^{+0.48}$ &$46.95_{-16.78}^{+19.81}$ & $9.93_{-0.01}^{+0.05}$ & $9.70_{-0.10}^{+0.06}$ & $9.55_{-0.08}^{+0.17}$ & $0.22_{-0.09}^{+0.20}$ \\
2MASX J09254750+6927532 & $7.12_{-0.15}^{+0.39}$ & $21.92_{-5.21}^{+2.54}$ & $-0.72_{-0.15}^{+0.14}$ & $0.99_{-0.34}^{+0.54}$ &$50.12_{-17.12}^{+17.56}$ & $10.30_{-0.04}^{+0.04}$ & $9.56_{-0.33}^{+0.14}$ & $10.22_{-0.07}^{+0.06}$ & $0.76_{-0.10}^{+0.13}$ \\
2MASX J09360622-6548336 & $<6.49$ & ... & $-1.71_{-0.22}^{+0.19}$ & $1.04_{-0.50}^{+0.64}$ &$45.48_{-18.45}^{+19.35}$ & $<9.55$ & $<9.27$ & $9.23_{-0.07}^{+0.10}$ & $>0.30$ \\
2MASX J09594263-3112581 & $7.44_{-0.23}^{+0.19}$ & $21.04_{-2.04}^{+2.96}$ & $-0.12_{-0.11}^{+0.12}$ & $1.28_{-0.33}^{+0.38}$ &$51.92_{-8.55}^{+12.14}$ & $10.70_{-0.03}^{+0.05}$ & $9.77_{-0.08}^{+0.11}$ & $10.65_{-0.04}^{+0.06}$ & $0.84_{-0.05}^{+0.03}$ \\
2MASX J10402231-4625264 & $7.29_{-0.04}^{+0.06}$ & $27.14_{-1.35}^{+0.82}$ & $-0.08_{-0.18}^{+0.20}$ & $1.76_{-0.42}^{+0.68}$ &$43.71_{-17.07}^{+20.95}$ & $10.55_{-0.02}^{+0.04}$ & $10.30_{-0.08}^{+0.04}$ & $10.20_{-0.08}^{+0.14}$ & $0.26_{-0.09}^{+0.16}$ \\
2MASX J11454045-1827149 & $6.99_{-0.07}^{+0.09}$ & $26.76_{-1.86}^{+1.31}$ & $-0.44_{-0.15}^{+0.15}$ & $1.19_{-0.36}^{+0.45}$ &$47.98_{-14.51}^{+19.49}$ & $10.44_{-0.03}^{+0.03}$ & $9.95_{-0.10}^{+0.06}$ & $10.28_{-0.06}^{+0.07}$ & $0.57_{-0.08}^{+0.10}$ \\
2MASX J12005792+0648226 & $7.65_{-0.04}^{+0.05}$ & $24.42_{-0.99}^{+0.68}$ & $-0.41_{-0.18}^{+0.18}$ & $1.50_{-0.38}^{+0.49}$ &$48.46_{-13.80}^{+17.16}$ & $10.63_{-0.02}^{+0.03}$ & $10.38_{-0.06}^{+0.03}$ & $10.28_{-0.08}^{+0.10}$ & $0.26_{-0.08}^{+0.12}$ \\
2MASX J12313717-4758019 & $7.54_{-0.03}^{+0.05}$ & $27.15_{-1.18}^{+0.57}$ & $-0.03_{-0.19}^{+0.21}$ & $2.03_{-0.49}^{+0.66}$ &$39.37_{-13.52}^{+21.42}$ & $10.76_{-0.02}^{+0.03}$ & $10.55_{-0.08}^{+0.03}$ & $10.34_{-0.09}^{+0.14}$ & $0.18_{-0.08}^{+0.16}$ \\
2MASX J12335145-2103448 & $6.50_{-0.08}^{+0.11}$ & $28.47_{-2.53}^{+1.66}$ & $-0.31_{-0.17}^{+0.14}$ & $1.83_{-0.42}^{+0.48}$ &$42.97_{-10.49}^{+14.87}$ & $10.10_{-0.04}^{+0.03}$ & $9.62_{-0.14}^{+0.08}$ & $9.92_{-0.08}^{+0.08}$ & $0.55_{-0.11}^{+0.14}$ \\
2MASX J12475784-5829599 & $<6.05$ & ... & $-1.05_{-0.18}^{+0.22}$ & $1.57_{-0.44}^{+0.52}$ &$46.91_{-16.19}^{+17.21}$ & $<9.50$ & $<8.84$ & $9.40_{-0.06}^{+0.13}$ & $>0.71$ \\
2MASX J13411287-1438407 & $7.85_{-0.15}^{+0.14}$ & $17.17_{-1.14}^{+1.31}$ & $-0.31_{-0.14}^{+0.13}$ & $1.14_{-0.36}^{+0.43}$ &$49.88_{-7.98}^{+10.82}$ & $10.67_{-0.03}^{+0.05}$ & $9.65_{-0.06}^{+0.06}$ & $10.63_{-0.04}^{+0.05}$ & $0.87_{-0.02}^{+0.02}$ \\
2MASX J13512953-1813468 & $<5.37$ & ... & $-1.14_{-0.19}^{+0.14}$ & $0.83_{-0.43}^{+0.52}$ &$45.78_{-17.64}^{+18.73}$ & $<8.93$ & $<8.16$ & $8.84_{-0.07}^{+0.06}$ & $>0.77$ \\
2MASX J14080674-3023537 & $<5.74$ & ... & $-0.82_{-0.15}^{+0.13}$ & $1.35_{-0.57}^{+0.67}$ &$37.13_{-12.17}^{+13.69}$ & $<9.63$ & $<8.52$ & $9.60_{-0.05}^{+0.08}$ & $>0.90$ \\
2MASX J14530794+2554327 & $<6.67$ & ... & $-1.72_{-0.21}^{+0.19}$ & $0.51_{-0.41}^{+0.48}$ &$47.72_{-17.27}^{+19.65}$ & $<9.83$ & $<9.46$ & $9.59_{-0.07}^{+0.05}$ & $>0.43$ \\
2MASX J15064412+0351444 & $6.78_{-0.12}^{+0.11}$ & $24.61_{-1.35}^{+1.06}$ & $-1.56_{-0.25}^{+0.32}$ & $1.60_{-0.50}^{+0.50}$ &$51.64_{-17.20}^{+18.16}$ & $9.67_{-0.02}^{+0.04}$ & $9.52_{-0.08}^{+0.04}$ & $9.11_{-0.08}^{+0.20}$ & $<0.39$ \\
2MASX J15115979-2119015 & $7.85_{-0.07}^{+0.09}$ & $27.56_{-2.16}^{+1.84}$ & $0.37_{-0.19}^{+0.14}$ & $1.85_{-0.30}^{+0.48}$ &$51.26_{-13.33}^{+19.38}$ & $11.33_{-0.04}^{+0.02}$ & $10.89_{-0.14}^{+0.10}$ & $11.13_{-0.12}^{+0.07}$ & $0.51_{-0.15}^{+0.14}$ \\
2MASX J15462424+6929102 & $6.24_{-0.30}^{+0.35}$ & $31.75_{-7.68}^{+4.71}$ & $-0.51_{-0.25}^{+0.17}$ & $2.66_{-0.67}^{+0.61}$ &$37.70_{-9.85}^{+18.44}$ & $10.18_{-0.06}^{+0.03}$ & $9.65_{-0.40}^{+0.16}$ & $10.03_{-0.15}^{+0.10}$ & $0.61_{-0.22}^{+0.23}$ \\
2MASX J16481523-3035037 & $<6.73$ & ... & $-0.91_{-0.21}^{+0.21}$ & $0.99_{-0.35}^{+0.34}$ &$62.11_{-14.29}^{+16.39}$ & $<9.97$ & $<9.51$ & $9.79_{-0.06}^{+0.08}$ & $>0.54$ \\
2MASX J18570768-7828212 & $7.42_{-0.18}^{+0.18}$ & $23.83_{-2.54}^{+2.98}$ & $-0.17_{-0.15}^{+0.11}$ & $1.42_{-0.33}^{+0.38}$ &$51.80_{-10.89}^{+14.00}$ & $10.77_{-0.03}^{+0.03}$ & $10.08_{-0.11}^{+0.13}$ & $10.67_{-0.07}^{+0.05}$ & $0.72_{-0.11}^{+0.07}$ \\
2MASX J19373299-0613046 & $7.12_{-0.03}^{+0.03}$ & $26.70_{-0.53}^{+0.41}$ & $0.39_{-0.14}^{+0.13}$ & $1.97_{-0.61}^{+0.75}$ &$32.15_{-9.71}^{+12.19}$ & $10.32_{-0.02}^{+0.04}$ & $10.08_{-0.03}^{+0.02}$ & $9.96_{-0.05}^{+0.11}$ & $0.24_{-0.04}^{+0.09}$ \\
2MASX J19380437-5109497 & $7.79_{-0.16}^{+0.13}$ & $17.60_{-1.15}^{+1.41}$ & $-0.74_{-0.12}^{+0.10}$ & $1.36_{-0.29}^{+0.38}$ &$55.25_{-9.88}^{+13.25}$ & $10.20_{-0.03}^{+0.03}$ & $9.66_{-0.06}^{+0.07}$ & $10.05_{-0.05}^{+0.04}$ & $0.61_{-0.07}^{+0.05}$ \\
2MASX J20005575-1810274 & $7.48_{-0.16}^{+0.16}$ & $25.31_{-2.81}^{+3.05}$ & $0.25_{-0.13}^{+0.11}$ & $1.21_{-0.30}^{+0.43}$ &$53.38_{-12.28}^{+17.36}$ & $11.12_{-0.04}^{+0.02}$ & $10.30_{-0.16}^{+0.14}$ & $11.05_{-0.07}^{+0.04}$ & $0.80_{-0.10}^{+0.06}$ \\
2MASX J20101740+4800214 & $6.97_{-0.09}^{+0.11}$ & $24.09_{-1.67}^{+0.96}$ & $-0.82_{-0.20}^{+0.24}$ & $1.70_{-0.41}^{+0.50}$ &$46.99_{-16.49}^{+16.94}$ & $9.89_{-0.02}^{+0.04}$ & $9.66_{-0.09}^{+0.03}$ & $9.51_{-0.08}^{+0.16}$ & $0.22_{-0.07}^{+0.18}$ \\
2MASX J20183871+4041003 & $<6.24$ & ... & $-0.43_{-0.18}^{+0.14}$ & $0.93_{-0.42}^{+0.56}$ &$44.93_{-17.35}^{+19.10}$ & $<9.76$ & $<9.02$ & $9.67_{-0.07}^{+0.07}$ & $>0.76$ \\
2MASX J21090996-0940147 & $7.36_{-0.14}^{+0.14}$ & $18.13_{-1.46}^{+1.59}$ & $-0.03_{-0.10}^{+0.10}$ & $1.21_{-0.29}^{+0.38}$ &$56.85_{-9.29}^{+13.27}$ & $10.48_{-0.03}^{+0.03}$ & $9.30_{-0.08}^{+0.09}$ & $10.45_{-0.04}^{+0.04}$ & $0.91_{-0.02}^{+0.02}$ \\
2MASX J21355399+4728217 & $7.35_{-0.15}^{+0.19}$ & $23.19_{-2.58}^{+2.04}$ & $-0.09_{-0.18}^{+0.14}$ & $1.58_{-0.32}^{+0.42}$ &$51.11_{-13.50}^{+14.51}$ & $10.42_{-0.03}^{+0.03}$ & $9.94_{-0.14}^{+0.09}$ & $10.24_{-0.08}^{+0.07}$ & $0.55_{-0.12}^{+0.13}$ \\
2MASX J23272195+1524375 & $9.16_{-0.28}^{+0.30}$ & $9.23_{-1.28}^{+1.48}$ & $-0.39_{-0.07}^{+0.07}$ & $1.17_{-0.10}^{+0.13}$ &$131.95_{-22.95}^{+16.95}$ & $10.41_{-0.02}^{+0.04}$ & $9.35_{-0.10}^{+0.11}$ & $10.37_{-0.03}^{+0.04}$ & $0.88_{-0.03}^{+0.02}$ \\
2MASX iJ1802473-145454 & $5.86_{-0.11}^{+0.11}$ & $25.00_{-1.24}^{+0.99}$ & $-0.22_{-0.18}^{+0.23}$ & $1.30_{-0.39}^{+0.42}$ &$50.15_{-16.44}^{+18.30}$ & $8.86_{-0.02}^{+0.03}$ & $8.65_{-0.06}^{+0.03}$ & $8.45_{-0.05}^{+0.13}$ & $0.18_{-0.05}^{+0.15}$ \\
2MFGC 02280 & $7.04_{-0.04}^{+0.05}$ & $27.10_{-0.75}^{+0.55}$ & $-0.37_{-0.18}^{+0.25}$ & $1.70_{-0.47}^{+0.64}$ &$43.73_{-16.70}^{+23.32}$ & $10.15_{-0.01}^{+0.03}$ & $10.04_{-0.04}^{+0.02}$ & $9.52_{-0.06}^{+0.16}$ & $<0.19$ \\
3C 111.0 & $9.98_{-0.02}^{+0.01}$ & $8.92_{-0.08}^{+0.10}$ & $-0.19_{-0.04}^{+0.04}$ & $1.05_{-0.10}^{+0.12}$ &$104.00_{-14.03}^{+11.96}$ & $10.84_{-0.03}^{+0.03}$ & $10.08_{-0.02}^{+0.01}$ & $10.76_{-0.03}^{+0.03}$ & $0.77_{-0.01}^{+0.02}$ \\
3C 120 & $9.80_{-0.15}^{+0.12}$ & $7.45_{-0.41}^{+0.56}$ & $0.63_{-0.03}^{+0.03}$ & $1.26_{-0.09}^{+0.09}$ &$132.76_{-5.85}^{+5.59}$ & $11.06_{-0.02}^{+0.03}$ & $9.42_{-0.03}^{+0.05}$ & $11.05_{-0.02}^{+0.03}$ & $0.97_{-0.00}^{+0.00}$ \\
4U 1344-60 & $<5.47$ & ... & $0.77_{-0.03}^{+0.03}$ & $2.05_{-0.19}^{+0.29}$ &$57.88_{-6.61}^{+5.24}$ & $<10.34$ & $<8.26$ & $10.33_{-0.03}^{+0.03}$ & $>0.99$ \\
6dF J0626586-370559 & $7.59_{-0.08}^{+0.08}$ & $21.89_{-1.21}^{+1.20}$ & $-0.57_{-0.18}^{+0.13}$ & $1.30_{-0.29}^{+0.38}$ &$56.73_{-13.78}^{+16.97}$ & $10.43_{-0.03}^{+0.02}$ & $10.03_{-0.08}^{+0.07}$ & $10.20_{-0.09}^{+0.05}$ & $0.46_{-0.11}^{+0.08}$ \\
6dF J2132022-334254 & $<6.27$ & ... & $-0.36_{-0.13}^{+0.10}$ & $1.90_{-0.48}^{+0.50}$ &$40.40_{-5.13}^{+7.92}$ & $<10.14$ & $<9.05$ & $10.10_{-0.05}^{+0.05}$ & $>0.89$ \\
ARK 241 & $8.06_{-0.26}^{+0.25}$ & $14.38_{-1.92}^{+2.38}$ & $-0.59_{-0.11}^{+0.11}$ & $1.21_{-0.30}^{+0.37}$ &$55.00_{-8.70}^{+12.77}$ & $10.34_{-0.03}^{+0.04}$ & $9.40_{-0.12}^{+0.15}$ & $10.29_{-0.04}^{+0.04}$ & $0.85_{-0.06}^{+0.04}$ \\
ARK 347 & $7.15_{-0.08}^{+0.07}$ & $22.10_{-1.12}^{+1.20}$ & $-0.12_{-0.10}^{+0.10}$ & $1.49_{-0.35}^{+0.43}$ &$51.85_{-8.93}^{+14.03}$ & $10.25_{-0.03}^{+0.03}$ & $9.62_{-0.06}^{+0.07}$ & $10.13_{-0.05}^{+0.05}$ & $0.69_{-0.06}^{+0.05}$ \\
ARP 102B & $6.48_{-0.24}^{+0.55}$ & $22.17_{-6.88}^{+4.59}$ & $-0.56_{-0.14}^{+0.11}$ & $1.51_{-0.35}^{+0.47}$ &$49.20_{-13.51}^{+13.04}$ & $9.83_{-0.04}^{+0.04}$ & $8.96_{-0.44}^{+0.25}$ & $9.77_{-0.07}^{+0.06}$ & $0.82_{-0.14}^{+0.11}$ \\
ARP 151 & $<5.70$ & ... & $-0.91_{-0.17}^{+0.12}$ & $1.14_{-0.50}^{+0.62}$ &$39.99_{-14.30}^{+15.58}$ & $<9.50$ & $<8.49$ & $9.46_{-0.06}^{+0.07}$ & $>0.87$ \\
AX J1737.4-2907 & $<6.92$ & ... & $0.18_{-0.11}^{+0.10}$ & $1.72_{-0.38}^{+0.46}$ &$49.82_{-8.49}^{+11.56}$ & $<10.42$ & $<9.71$ & $10.33_{-0.05}^{+0.05}$ & $>0.74$ \\
Ark 120 & $7.77_{-0.07}^{+0.10}$ & $23.37_{-1.55}^{+1.13}$ & $-0.11_{-0.19}^{+0.15}$ & $0.83_{-0.35}^{+0.35}$ &$55.87_{-15.70}^{+15.98}$ & $10.89_{-0.03}^{+0.03}$ & $10.38_{-0.08}^{+0.06}$ & $10.73_{-0.06}^{+0.06}$ & $0.58_{-0.08}^{+0.08}$ \\
CGCG 102-048 & $8.14_{-0.17}^{+0.16}$ & $12.50_{-0.97}^{+1.05}$ & $-0.88_{-0.08}^{+0.07}$ & $1.64_{-0.25}^{+0.34}$ &$64.28_{-10.88}^{+13.42}$ & $9.60_{-0.04}^{+0.03}$ & $9.12_{-0.09}^{+0.07}$ & $9.43_{-0.04}^{+0.03}$ & $0.56_{-0.06}^{+0.06}$ \\
CGCG 122-055 & $6.90_{-0.10}^{+0.14}$ & $25.46_{-2.67}^{+2.18}$ & $0.08_{-0.13}^{+0.11}$ & $1.97_{-0.43}^{+0.58}$ &$47.66_{-10.69}^{+26.32}$ & $10.31_{-0.05}^{+0.02}$ & $9.74_{-0.13}^{+0.12}$ & $10.17_{-0.09}^{+0.05}$ & $0.64_{-0.13}^{+0.09}$ \\
CGCG 229-015 & $7.21_{-0.10}^{+0.09}$ & $20.55_{-1.18}^{+1.34}$ & $-0.46_{-0.11}^{+0.10}$ & $1.83_{-0.42}^{+0.48}$ &$47.34_{-7.83}^{+12.89}$ & $10.05_{-0.03}^{+0.04}$ & $9.49_{-0.07}^{+0.08}$ & $9.91_{-0.05}^{+0.05}$ & $0.63_{-0.07}^{+0.06}$ \\
CGCG 300-062 & $7.71_{-0.13}^{+0.12}$ & $17.81_{-1.06}^{+1.24}$ & $-0.67_{-0.09}^{+0.07}$ & $1.63_{-0.26}^{+0.35}$ &$59.51_{-9.94}^{+12.92}$ & $10.04_{-0.03}^{+0.02}$ & $9.61_{-0.05}^{+0.05}$ & $9.84_{-0.05}^{+0.03}$ & $0.50_{-0.07}^{+0.05}$ \\
CGCG 312-012 & $6.82_{-0.06}^{+0.08}$ & $22.28_{-1.18}^{+0.73}$ & $-1.18_{-0.18}^{+0.18}$ & $1.22_{-0.41}^{+0.45}$ &$48.48_{-15.72}^{+17.94}$ & $9.61_{-0.03}^{+0.03}$ & $9.31_{-0.08}^{+0.04}$ & $9.30_{-0.07}^{+0.09}$ & $0.33_{-0.07}^{+0.12}$ \\
CGCG 319-007 & $7.71_{-0.04}^{+0.05}$ & $24.37_{-0.90}^{+0.52}$ & $-0.48_{-0.14}^{+0.15}$ & $1.76_{-0.56}^{+0.76}$ &$34.46_{-12.08}^{+14.65}$ & $10.72_{-0.02}^{+0.04}$ & $10.43_{-0.06}^{+0.03}$ & $10.40_{-0.05}^{+0.12}$ & $0.31_{-0.06}^{+0.13}$ \\
CGCG 341-006 & $7.64_{-0.09}^{+0.15}$ & $27.48_{-3.87}^{+2.48}$ & $0.18_{-0.24}^{+0.20}$ & $2.09_{-0.26}^{+0.47}$ &$55.44_{-17.47}^{+22.15}$ & $11.09_{-0.03}^{+0.02}$ & $10.67_{-0.24}^{+0.13}$ & $10.88_{-0.14}^{+0.11}$ & $0.49_{-0.20}^{+0.22}$ \\
CGCG 367-009 & $6.91_{-0.32}^{+0.55}$ & $19.90_{-4.30}^{+3.03}$ & $-0.78_{-0.15}^{+0.13}$ & $1.58_{-0.45}^{+0.55}$ &$42.77_{-13.13}^{+12.13}$ & $9.68_{-0.03}^{+0.06}$ & $9.10_{-0.13}^{+0.09}$ & $9.55_{-0.05}^{+0.09}$ & $0.65_{-0.08}^{+0.11}$ \\
CGCG 420-015 & $7.37_{-0.13}^{+0.12}$ & $23.96_{-1.96}^{+2.38}$ & $0.27_{-0.14}^{+0.12}$ & $1.74_{-0.51}^{+0.51}$ &$41.75_{-7.76}^{+11.13}$ & $10.82_{-0.03}^{+0.06}$ & $10.05_{-0.10}^{+0.11}$ & $10.75_{-0.04}^{+0.07}$ & $0.78_{-0.06}^{+0.06}$ \\
CGCG 468-002 NED01 & $7.26_{-0.09}^{+0.15}$ & $27.13_{-2.64}^{+1.48}$ & $0.24_{-0.27}^{+0.29}$ & $2.11_{-0.39}^{+0.49}$ &$50.01_{-18.21}^{+20.33}$ & $10.49_{-0.02}^{+0.04}$ & $10.26_{-0.13}^{+0.06}$ & $10.10_{-0.13}^{+0.20}$ & $0.21_{-0.13}^{+0.24}$ \\
CGCG 493-002 & $7.04_{-0.09}^{+0.10}$ & $23.51_{-1.53}^{+1.54}$ & $-0.05_{-0.12}^{+0.10}$ & $1.42_{-0.33}^{+0.46}$ &$53.18_{-10.48}^{+16.04}$ & $10.40_{-0.04}^{+0.03}$ & $9.67_{-0.08}^{+0.09}$ & $10.31_{-0.06}^{+0.04}$ & $0.75_{-0.06}^{+0.05}$ \\
CGCG 535-012 & $8.43_{-0.18}^{+0.18}$ & $13.23_{-1.38}^{+1.40}$ & $-0.39_{-0.07}^{+0.07}$ & $1.42_{-0.26}^{+0.33}$ &$60.27_{-9.32}^{+12.50}$ & $10.56_{-0.03}^{+0.04}$ & $9.56_{-0.12}^{+0.10}$ & $10.51_{-0.03}^{+0.04}$ & $0.87_{-0.03}^{+0.03}$ \\
Cen A & $7.27_{-0.02}^{+0.02}$ & $24.55_{-0.27}^{+0.25}$ & $1.32_{-0.18}^{+0.24}$ & $1.50_{-0.43}^{+0.48}$ &$48.05_{-16.23}^{+18.93}$ & $10.03_{-0.01}^{+0.02}$ & $10.01_{-0.02}^{+0.01}$ & $8.71_{-0.06}^{+0.15}$ & $<-0.23$ \\
ESO 005-G004 & $7.58_{-0.03}^{+0.03}$ & $21.62_{-0.35}^{+0.27}$ & $0.31_{-0.16}^{+0.19}$ & $1.02_{-0.38}^{+0.48}$ &$51.55_{-19.00}^{+18.51}$ & $10.10_{-0.01}^{+0.02}$ & $9.99_{-0.02}^{+0.01}$ & $9.45_{-0.05}^{+0.09}$ & $<0.08$ \\
ESO 031-G008 & $7.67_{-0.21}^{+0.20}$ & $16.31_{-1.83}^{+2.28}$ & $-0.78_{-0.14}^{+0.10}$ & $1.34_{-0.29}^{+0.36}$ &$57.03_{-10.98}^{+12.88}$ & $9.85_{-0.03}^{+0.03}$ & $9.35_{-0.12}^{+0.13}$ & $9.68_{-0.07}^{+0.04}$ & $0.58_{-0.14}^{+0.09}$ \\
ESO 033-G002 & $7.07_{-0.24}^{+0.27}$ & $22.60_{-3.25}^{+3.61}$ & $0.19_{-0.13}^{+0.10}$ & $1.46_{-0.33}^{+0.40}$ &$52.95_{-11.49}^{+14.21}$ & $10.35_{-0.03}^{+0.03}$ & $9.60_{-0.14}^{+0.15}$ & $10.26_{-0.07}^{+0.05}$ & $0.76_{-0.11}^{+0.07}$ \\
ESO 103-035 & $5.98_{-0.10}^{+0.13}$ & $36.45_{-4.10}^{+3.87}$ & $0.82_{-0.18}^{+0.11}$ & $2.94_{-0.69}^{+0.69}$ &$29.96_{-5.67}^{+5.93}$ & $10.58_{-0.06}^{+0.09}$ & $9.75_{-0.17}^{+0.16}$ & $10.51_{-0.09}^{+0.11}$ & $0.80_{-0.09}^{+0.08}$ \\
ESO 121-IG028 & $7.89_{-0.19}^{+0.21}$ & $16.50_{-2.00}^{+2.12}$ & $-0.95_{-0.16}^{+0.15}$ & $1.70_{-0.48}^{+0.64}$ &$39.81_{-13.11}^{+15.64}$ & $10.03_{-0.05}^{+0.09}$ & $9.60_{-0.15}^{+0.14}$ & $9.83_{-0.06}^{+0.11}$ & $0.51_{-0.12}^{+0.13}$ \\
ESO 137-34 & $7.42_{-0.06}^{+0.06}$ & $22.93_{-0.65}^{+0.53}$ & $0.12_{-0.17}^{+0.19}$ & $1.12_{-0.41}^{+0.43}$ &$49.75_{-16.37}^{+17.07}$ & $10.17_{-0.01}^{+0.03}$ & $9.98_{-0.03}^{+0.02}$ & $9.73_{-0.04}^{+0.10}$ & $0.14_{-0.04}^{+0.09}$ \\
ESO 139-G012 & $8.05_{-0.04}^{+0.05}$ & $18.37_{-0.49}^{+0.39}$ & $-0.47_{-0.15}^{+0.14}$ & $1.19_{-0.38}^{+0.54}$ &$47.32_{-15.25}^{+15.89}$ & $10.19_{-0.02}^{+0.02}$ & $10.03_{-0.03}^{+0.02}$ & $9.67_{-0.06}^{+0.07}$ & $<0.18$ \\
ESO 141-G055 & $7.90_{-0.06}^{+0.09}$ & $23.47_{-1.24}^{+0.79}$ & $-0.14_{-0.16}^{+0.13}$ & $1.14_{-0.41}^{+0.54}$ &$42.71_{-14.04}^{+15.16}$ & $10.93_{-0.03}^{+0.03}$ & $10.53_{-0.06}^{+0.03}$ & $10.72_{-0.06}^{+0.07}$ & $0.48_{-0.06}^{+0.09}$ \\
ESO 157-G023 & $8.20_{-0.07}^{+0.08}$ & $18.36_{-0.72}^{+0.60}$ & $-0.82_{-0.18}^{+0.12}$ & $1.01_{-0.46}^{+0.60}$ &$40.36_{-12.75}^{+12.37}$ & $10.51_{-0.03}^{+0.02}$ & $10.19_{-0.03}^{+0.02}$ & $10.24_{-0.06}^{+0.05}$ & $0.37_{-0.06}^{+0.05}$ \\
ESO 195-IG021 NED03 & $8.00_{-0.07}^{+0.10}$ & $22.50_{-1.74}^{+1.27}$ & $-0.32_{-0.23}^{+0.17}$ & $1.74_{-0.32}^{+0.40}$ &$57.95_{-15.50}^{+16.88}$ & $10.81_{-0.02}^{+0.02}$ & $10.51_{-0.12}^{+0.08}$ & $10.50_{-0.11}^{+0.09}$ & $0.33_{-0.14}^{+0.15}$ \\
ESO 197-G027 & $8.22_{-0.06}^{+0.07}$ & $23.24_{-1.04}^{+0.89}$ & $-0.20_{-0.22}^{+0.17}$ & $2.01_{-0.37}^{+0.52}$ &$47.48_{-13.29}^{+19.40}$ & $11.02_{-0.02}^{+0.02}$ & $10.81_{-0.06}^{+0.05}$ & $10.61_{-0.13}^{+0.10}$ & $<0.37$ \\
ESO 198-024 & $7.01_{-0.23}^{+0.41}$ & $22.76_{-5.52}^{+4.17}$ & $-0.65_{-0.18}^{+0.12}$ & $1.24_{-0.32}^{+0.37}$ &$55.32_{-13.74}^{+13.51}$ & $10.38_{-0.04}^{+0.04}$ & $9.56_{-0.33}^{+0.19}$ & $10.31_{-0.07}^{+0.06}$ & $0.80_{-0.11}^{+0.11}$ \\
ESO 209-G012 & $8.37_{-0.09}^{+0.09}$ & $21.23_{-1.02}^{+1.15}$ & $0.30_{-0.12}^{+0.09}$ & $1.88_{-0.40}^{+0.45}$ &$47.27_{-7.87}^{+11.09}$ & $11.18_{-0.03}^{+0.03}$ & $10.73_{-0.04}^{+0.05}$ & $10.98_{-0.06}^{+0.05}$ & $0.52_{-0.08}^{+0.05}$ \\
ESO 244-IG030 & $7.67_{-0.04}^{+0.04}$ & $25.50_{-0.55}^{+0.46}$ & $-0.41_{-0.18}^{+0.23}$ & $1.57_{-0.40}^{+0.48}$ &$48.56_{-17.33}^{+18.19}$ & $10.62_{-0.01}^{+0.03}$ & $10.51_{-0.03}^{+0.02}$ & $9.95_{-0.07}^{+0.14}$ & $<0.11$ \\
ESO 263-G013 & $<6.75$ & ... & $-0.24_{-0.11}^{+0.10}$ & $1.65_{-0.40}^{+0.45}$ &$49.35_{-7.84}^{+11.85}$ & $<10.40$ & $<9.54$ & $10.33_{-0.05}^{+0.05}$ & $>0.82$ \\
ESO 297-018 & $8.16_{-0.08}^{+0.07}$ & $18.84_{-0.79}^{+0.90}$ & $0.05_{-0.12}^{+0.08}$ & $1.61_{-0.23}^{+0.26}$ &$71.45_{-10.75}^{+12.98}$ & $10.54_{-0.03}^{+0.01}$ & $10.21_{-0.04}^{+0.04}$ & $10.26_{-0.07}^{+0.03}$ & $0.38_{-0.08}^{+0.05}$ \\
ESO 323-077 & $7.62_{-0.04}^{+0.05}$ & $27.38_{-1.10}^{+0.70}$ & $0.60_{-0.19}^{+0.20}$ & $1.55_{-0.39}^{+0.48}$ &$49.42_{-16.92}^{+20.03}$ & $10.88_{-0.02}^{+0.03}$ & $10.64_{-0.07}^{+0.04}$ & $10.50_{-0.08}^{+0.12}$ & $0.22_{-0.08}^{+0.13}$ \\
ESO 362-18 & $6.93_{-0.04}^{+0.05}$ & $26.12_{-0.91}^{+0.62}$ & $0.16_{-0.14}^{+0.13}$ & $2.01_{-0.51}^{+0.65}$ &$35.49_{-9.98}^{+11.14}$ & $10.15_{-0.02}^{+0.04}$ & $9.83_{-0.05}^{+0.03}$ & $9.87_{-0.06}^{+0.10}$ & $0.36_{-0.06}^{+0.11}$ \\
ESO 374-G044 & $7.79_{-0.15}^{+0.16}$ & $18.11_{-1.56}^{+1.65}$ & $0.04_{-0.10}^{+0.09}$ & $2.20_{-0.46}^{+0.50}$ &$40.81_{-5.16}^{+7.63}$ & $10.48_{-0.04}^{+0.05}$ & $9.74_{-0.10}^{+0.10}$ & $10.39_{-0.06}^{+0.06}$ & $0.76_{-0.06}^{+0.05}$ \\
ESO 383-18 & $6.56_{-0.10}^{+0.14}$ & $24.37_{-2.70}^{+2.13}$ & $0.10_{-0.14}^{+0.12}$ & $0.96_{-0.29}^{+0.52}$ &$55.83_{-15.39}^{+25.18}$ & $10.10_{-0.05}^{+0.02}$ & $9.28_{-0.17}^{+0.13}$ & $10.02_{-0.08}^{+0.03}$ & $0.80_{-0.09}^{+0.06}$ \\
ESO 399-20 & $7.75_{-0.06}^{+0.07}$ & $21.84_{-0.97}^{+0.81}$ & $-0.30_{-0.21}^{+0.18}$ & $1.21_{-0.34}^{+0.35}$ &$59.15_{-14.08}^{+14.45}$ & $10.46_{-0.02}^{+0.03}$ & $10.19_{-0.05}^{+0.04}$ & $10.12_{-0.07}^{+0.08}$ & $0.28_{-0.08}^{+0.10}$ \\
ESO 417-G006 & $5.72_{-0.13}^{+0.21}$ & $31.50_{-5.81}^{+2.56}$ & $-0.57_{-0.24}^{+0.27}$ & $1.93_{-0.37}^{+0.43}$ &$51.01_{-17.00}^{+22.04}$ & $9.50_{-0.02}^{+0.04}$ & $9.11_{-0.36}^{+0.11}$ & $9.27_{-0.11}^{+0.17}$ & $0.45_{-0.16}^{+0.32}$ \\
ESO 426-G002 & $7.74_{-0.21}^{+0.26}$ & $17.78_{-2.65}^{+2.27}$ & $-0.17_{-0.12}^{+0.12}$ & $1.77_{-0.45}^{+0.50}$ &$41.62_{-9.17}^{+10.29}$ & $10.20_{-0.03}^{+0.05}$ & $9.64_{-0.18}^{+0.13}$ & $10.06_{-0.05}^{+0.07}$ & $0.64_{-0.11}^{+0.13}$ \\
ESO 439-G009 & $8.07_{-0.07}^{+0.07}$ & $18.39_{-0.65}^{+0.69}$ & $0.12_{-0.08}^{+0.07}$ & $1.82_{-0.32}^{+0.38}$ &$54.07_{-8.15}^{+12.42}$ & $10.51_{-0.03}^{+0.03}$ & $10.06_{-0.03}^{+0.03}$ & $10.31_{-0.05}^{+0.04}$ & $0.52_{-0.05}^{+0.04}$ \\
ESO 464-G016 & $7.29_{-0.06}^{+0.09}$ & $26.56_{-1.63}^{+1.11}$ & $-0.51_{-0.25}^{+0.25}$ & $1.95_{-0.39}^{+0.39}$ &$53.62_{-15.33}^{+14.76}$ & $10.44_{-0.01}^{+0.03}$ & $10.23_{-0.07}^{+0.05}$ & $10.02_{-0.10}^{+0.15}$ & $<0.47$ \\
ESO 479-G031 & $6.26_{-0.15}^{+0.25}$ & $25.68_{-4.57}^{+1.84}$ & $-1.10_{-0.23}^{+0.33}$ & $1.47_{-0.46}^{+0.46}$ &$54.24_{-21.43}^{+18.45}$ & $9.45_{-0.02}^{+0.06}$ & $9.11_{-0.29}^{+0.06}$ & $9.19_{-0.05}^{+0.20}$ & $0.39_{-0.08}^{+0.32}$ \\
ESO 490-IG026 & $7.79_{-0.10}^{+0.10}$ & $21.67_{-1.36}^{+1.43}$ & $0.65_{-0.05}^{+0.04}$ & $2.18_{-0.28}^{+0.37}$ &$54.70_{-8.31}^{+11.04}$ & $10.88_{-0.03}^{+0.02}$ & $10.20_{-0.08}^{+0.07}$ & $10.78_{-0.05}^{+0.03}$ & $0.72_{-0.06}^{+0.04}$ \\
ESO 499-G041 & $6.88_{-0.07}^{+0.08}$ & $23.17_{-1.43}^{+0.94}$ & $-0.29_{-0.20}^{+0.20}$ & $1.41_{-0.38}^{+0.38}$ &$53.51_{-15.45}^{+17.45}$ & $9.78_{-0.02}^{+0.03}$ & $9.47_{-0.10}^{+0.05}$ & $9.49_{-0.08}^{+0.11}$ & $0.34_{-0.09}^{+0.16}$ \\
ESO 506-G027 & $8.21_{-0.07}^{+0.07}$ & $16.96_{-0.62}^{+0.64}$ & $0.09_{-0.06}^{+0.06}$ & $1.38_{-0.23}^{+0.28}$ &$63.38_{-9.11}^{+12.08}$ & $10.56_{-0.03}^{+0.02}$ & $9.98_{-0.03}^{+0.03}$ & $10.43_{-0.03}^{+0.03}$ & $0.64_{-0.03}^{+0.03}$ \\
ESO 509-G038 & $7.37_{-0.08}^{+0.08}$ & $22.86_{-1.43}^{+1.47}$ & $-0.17_{-0.19}^{+0.12}$ & $1.74_{-0.38}^{+0.46}$ &$49.99_{-11.51}^{+16.48}$ & $10.34_{-0.04}^{+0.03}$ & $9.92_{-0.09}^{+0.10}$ & $10.14_{-0.12}^{+0.06}$ & $0.50_{-0.15}^{+0.09}$ \\
ESO 509-IG066NED01 & $7.87_{-0.05}^{+0.06}$ & $24.25_{-1.19}^{+0.73}$ & $0.06_{-0.17}^{+0.14}$ & $2.79_{-0.64}^{+0.65}$ &$32.25_{-7.53}^{+9.56}$ & $10.88_{-0.03}^{+0.04}$ & $10.58_{-0.08}^{+0.03}$ & $10.57_{-0.08}^{+0.12}$ & $0.33_{-0.08}^{+0.14}$ \\
ESO 511-G030 & $8.34_{-0.10}^{+0.09}$ & $16.94_{-0.69}^{+0.82}$ & $-0.67_{-0.17}^{+0.19}$ & $1.04_{-0.41}^{+0.56}$ &$47.61_{-18.30}^{+19.26}$ & $10.28_{-0.02}^{+0.05}$ & $10.11_{-0.03}^{+0.04}$ & $9.77_{-0.05}^{+0.10}$ & $<0.21$ \\
ESO 533-G050 & $7.95_{-0.07}^{+0.08}$ & $17.94_{-0.64}^{+0.56}$ & $-1.21_{-0.17}^{+0.20}$ & $1.34_{-0.41}^{+0.48}$ &$48.01_{-16.18}^{+17.87}$ & $9.97_{-0.02}^{+0.03}$ & $9.88_{-0.03}^{+0.02}$ & $9.26_{-0.05}^{+0.12}$ & $<0.04$ \\
ESO 548-G081 & $7.17_{-0.03}^{+0.03}$ & $24.40_{-0.34}^{+0.33}$ & $-0.51_{-0.24}^{+0.21}$ & $0.15_{-0.41}^{+0.52}$ &$41.71_{-14.23}^{+18.88}$ & $10.21_{-0.03}^{+0.02}$ & $9.90_{-0.02}^{+0.02}$ & $9.93_{-0.06}^{+0.04}$ & $0.35_{-0.05}^{+0.04}$ \\
ESO 549-G049 & $7.91_{-0.04}^{+0.11}$ & $26.82_{-3.30}^{+0.72}$ & $0.19_{-0.27}^{+0.66}$ & $1.90_{-0.43}^{+0.46}$ &$56.67_{-26.20}^{+35.53}$ & $11.01_{--0.00}^{+0.06}$ & $10.88_{-0.23}^{+0.04}$ & $10.43_{-0.06}^{+0.44}$ & $<0.69$ \\
ESO 553-G022 & $7.93_{-0.07}^{+0.08}$ & $19.48_{-0.79}^{+0.58}$ & $-1.22_{-0.16}^{+0.18}$ & $1.16_{-0.40}^{+0.44}$ &$47.40_{-15.83}^{+17.31}$ & $10.23_{-0.03}^{+0.03}$ & $10.07_{-0.05}^{+0.03}$ & $9.73_{-0.05}^{+0.10}$ & $<0.25$ \\
ESO 553-G043 & $<6.17$ & ... & $-0.28_{-0.11}^{+0.09}$ & $2.07_{-0.46}^{+0.50}$ &$43.30_{-5.78}^{+9.06}$ & $<10.09$ & $<8.96$ & $10.06_{-0.05}^{+0.06}$ & $>0.90$ \\
ESO 565-G019 & $7.21_{-0.03}^{+0.05}$ & $28.52_{-1.60}^{+0.72}$ & $0.17_{-0.23}^{+0.31}$ & $2.02_{-0.39}^{+0.49}$ &$47.10_{-16.00}^{+31.24}$ & $10.51_{-0.02}^{+0.03}$ & $10.34_{-0.10}^{+0.03}$ & $10.02_{-0.10}^{+0.19}$ & $<0.52$ \\
ESO 578-G009 & $7.88_{-0.07}^{+0.08}$ & $21.68_{-0.97}^{+0.68}$ & $-0.57_{-0.19}^{+0.17}$ & $1.96_{-0.48}^{+0.63}$ &$40.33_{-13.96}^{+13.84}$ & $10.48_{-0.02}^{+0.03}$ & $10.30_{-0.06}^{+0.03}$ & $10.02_{-0.09}^{+0.13}$ & $<0.32$ \\
Fairall 1146 & $7.64_{-0.09}^{+0.11}$ & $25.59_{-1.79}^{+1.61}$ & $0.28_{-0.17}^{+0.12}$ & $1.68_{-0.41}^{+0.46}$ &$43.72_{-9.68}^{+13.73}$ & $10.99_{-0.04}^{+0.03}$ & $10.49_{-0.08}^{+0.08}$ & $10.82_{-0.09}^{+0.07}$ & $0.58_{-0.12}^{+0.09}$ \\
Fairall 272 & $7.22_{-0.03}^{+0.03}$ & $25.95_{-0.49}^{+0.42}$ & $-0.64_{-0.19}^{+0.16}$ & $0.85_{-0.40}^{+0.45}$ &$49.38_{-16.86}^{+18.86}$ & $10.29_{-0.02}^{+0.02}$ & $10.10_{-0.03}^{+0.02}$ & $9.84_{-0.06}^{+0.07}$ & $0.13_{-0.05}^{+0.06}$ \\
Fairall 49 & $7.28_{-0.05}^{+0.08}$ & $29.39_{-2.25}^{+1.31}$ & $0.60_{-0.16}^{+0.14}$ & $1.51_{-0.39}^{+0.67}$ &$43.99_{-13.33}^{+25.30}$ & $10.96_{-0.04}^{+0.03}$ & $10.49_{-0.12}^{+0.07}$ & $10.78_{-0.09}^{+0.07}$ & $0.55_{-0.10}^{+0.11}$ \\
Fairall 51 & $7.56_{-0.11}^{+0.11}$ & $19.24_{-1.24}^{+1.28}$ & $0.72_{-0.05}^{+0.05}$ & $1.68_{-0.25}^{+0.36}$ &$60.08_{-9.64}^{+12.64}$ & $10.52_{-0.03}^{+0.03}$ & $9.67_{-0.07}^{+0.06}$ & $10.46_{-0.03}^{+0.03}$ & $0.81_{-0.03}^{+0.03}$ \\
Fairall 9 & $7.40_{-0.04}^{+0.06}$ & $28.74_{-1.31}^{+0.83}$ & $-0.01_{-0.14}^{+0.10}$ & $1.25_{-0.57}^{+0.80}$ &$31.96_{-10.84}^{+12.84}$ & $11.19_{-0.04}^{+0.05}$ & $10.55_{-0.07}^{+0.03}$ & $11.08_{-0.05}^{+0.07}$ & $0.69_{-0.04}^{+0.06}$ \\
{[HB89]} 0241+622 & $9.91_{-0.23}^{+0.07}$ & $6.92_{-0.19}^{+0.70}$ & $-0.01_{-0.04}^{+0.04}$ & $0.45_{-0.09}^{+0.12}$ &$141.21_{-14.57}^{+7.16}$ & $11.21_{-0.05}^{+0.04}$ & $9.34_{-0.02}^{+0.05}$ & $11.20_{-0.05}^{+0.05}$ & $0.98_{-0.00}^{+0.00}$ \\
IC 0486 & $7.64_{-0.08}^{+0.09}$ & $24.09_{-1.47}^{+1.23}$ & $0.03_{-0.20}^{+0.15}$ & $1.82_{-0.35}^{+0.48}$ &$49.28_{-12.77}^{+17.23}$ & $10.64_{-0.03}^{+0.02}$ & $10.33_{-0.08}^{+0.06}$ & $10.35_{-0.11}^{+0.08}$ & $0.35_{-0.12}^{+0.12}$ \\
IC 1657 & $8.05_{-0.07}^{+0.08}$ & $16.77_{-0.85}^{+0.79}$ & $0.83_{-0.06}^{+0.06}$ & $2.86_{-0.21}^{+0.30}$ &$67.68_{-9.27}^{+8.65}$ & $10.28_{-0.01}^{+0.02}$ & $9.79_{-0.06}^{+0.06}$ & $10.11_{-0.03}^{+0.04}$ & $0.57_{-0.05}^{+0.06}$ \\
IC 1816 & $7.34_{-0.03}^{+0.04}$ & $25.99_{-0.61}^{+0.47}$ & $0.06_{-0.15}^{+0.15}$ & $2.20_{-0.58}^{+0.64}$ &$33.36_{-8.75}^{+12.58}$ & $10.44_{-0.02}^{+0.04}$ & $10.23_{-0.04}^{+0.02}$ & $10.01_{-0.06}^{+0.12}$ & $0.17_{-0.05}^{+0.11}$ \\
IC 2461 & $7.09_{-0.10}^{+0.08}$ & $19.29_{-1.01}^{+1.38}$ & $0.36_{-0.25}^{+0.12}$ & $2.31_{-0.47}^{+0.29}$ &$70.80_{-9.63}^{+10.85}$ & $9.46_{-0.02}^{+0.01}$ & $9.20_{-0.07}^{+0.09}$ & $9.10_{-0.15}^{+0.07}$ & $<0.42$ \\
IC 2637 & $7.92_{-0.03}^{+0.03}$ & $26.73_{-0.38}^{+0.34}$ & $-0.36_{-0.15}^{+0.16}$ & $1.28_{-0.43}^{+0.59}$ &$42.28_{-15.56}^{+19.35}$ & $10.97_{-0.01}^{+0.02}$ & $10.88_{-0.02}^{+0.02}$ & $10.24_{-0.05}^{+0.11}$ & $<0.04$ \\
IC 2921 & $7.54_{-0.29}^{+0.39}$ & $18.35_{-3.84}^{+4.02}$ & $-0.58_{-0.14}^{+0.12}$ & $1.23_{-0.37}^{+0.40}$ &$49.93_{-9.85}^{+13.70}$ & $10.43_{-0.03}^{+0.05}$ & $9.52_{-0.23}^{+0.22}$ & $10.38_{-0.05}^{+0.06}$ & $0.84_{-0.11}^{+0.07}$ \\
IC 4329A & $6.91_{-0.17}^{+0.15}$ & $25.71_{-2.75}^{+3.48}$ & $0.76_{-0.14}^{+0.12}$ & $1.25_{-0.39}^{+0.48}$ &$46.68_{-8.77}^{+11.01}$ & $10.86_{-0.04}^{+0.05}$ & $9.77_{-0.15}^{+0.17}$ & $10.82_{-0.05}^{+0.05}$ & $0.89_{-0.05}^{+0.03}$ \\
IC 4518A & $7.59_{-0.16}^{+0.37}$ & $26.06_{-5.26}^{+2.78}$ & $0.82_{-0.34}^{+0.23}$ & $2.56_{-0.43}^{+0.59}$ &$45.09_{-15.65}^{+17.22}$ & $10.83_{-0.04}^{+0.03}$ & $10.48_{-0.20}^{+0.11}$ & $10.57_{-0.18}^{+0.15}$ & $0.40_{-0.22}^{+0.25}$ \\
IC 4709 & $7.35_{-0.11}^{+0.10}$ & $20.35_{-1.10}^{+1.33}$ & $0.10_{-0.09}^{+0.07}$ & $1.84_{-0.32}^{+0.40}$ &$52.87_{-8.34}^{+11.42}$ & $10.14_{-0.03}^{+0.03}$ & $9.60_{-0.05}^{+0.06}$ & $9.99_{-0.06}^{+0.04}$ & $0.61_{-0.07}^{+0.04}$ \\
IC 5063 & $7.61_{-0.09}^{+0.08}$ & $20.11_{-0.93}^{+1.10}$ & $1.18_{-0.10}^{+0.07}$ & $2.23_{-0.48}^{+0.47}$ &$43.60_{-5.44}^{+9.12}$ & $10.75_{-0.04}^{+0.05}$ & $9.83_{-0.05}^{+0.07}$ & $10.69_{-0.05}^{+0.06}$ & $0.84_{-0.03}^{+0.02}$ \\
IGR J11366-6002 & $7.01_{-0.08}^{+0.10}$ & $25.35_{-1.60}^{+1.01}$ & $-0.11_{-0.23}^{+0.24}$ & $1.89_{-0.40}^{+0.47}$ &$47.47_{-15.61}^{+17.74}$ & $10.04_{-0.02}^{+0.04}$ & $9.83_{-0.08}^{+0.04}$ & $9.63_{-0.10}^{+0.16}$ & $0.18_{-0.10}^{+0.18}$ \\
IGR J23308+7120 & $7.62_{-0.06}^{+0.06}$ & $24.21_{-0.69}^{+0.60}$ & $-0.88_{-0.15}^{+0.20}$ & $1.55_{-0.45}^{+0.63}$ &$42.37_{-16.96}^{+19.97}$ & $10.45_{-0.01}^{+0.03}$ & $10.33_{-0.03}^{+0.02}$ & $9.85_{-0.06}^{+0.14}$ & $<0.17$ \\
IIS Z010 & $6.69_{-0.09}^{+0.13}$ & $25.13_{-2.16}^{+1.35}$ & $-0.50_{-0.13}^{+0.11}$ & $1.52_{-0.53}^{+0.71}$ &$33.35_{-10.44}^{+9.22}$ & $10.30_{-0.03}^{+0.07}$ & $9.49_{-0.11}^{+0.05}$ & $10.23_{-0.04}^{+0.08}$ & $0.79_{-0.03}^{+0.06}$ \\
II Zw083 & $7.55_{-0.08}^{+0.10}$ & $25.56_{-1.71}^{+1.68}$ & $0.41_{-0.14}^{+0.12}$ & $1.80_{-0.49}^{+0.62}$ &$38.77_{-9.19}^{+10.32}$ & $11.01_{-0.03}^{+0.06}$ & $10.39_{-0.09}^{+0.08}$ & $10.89_{-0.06}^{+0.08}$ & $0.68_{-0.08}^{+0.08}$ \\
IRAS 03219+4031 & $7.69_{-0.06}^{+0.09}$ & $29.40_{-2.21}^{+1.18}$ & $-0.01_{-0.27}^{+0.21}$ & $3.32_{-0.67}^{+0.67}$ &$34.14_{-8.49}^{+13.73}$ & $11.12_{-0.04}^{+0.02}$ & $10.90_{-0.12}^{+0.05}$ & $10.73_{-0.16}^{+0.13}$ & $<0.57$ \\
IRAS 04124-0803 & $7.08_{-0.09}^{+0.09}$ & $29.46_{-2.12}^{+2.12}$ & $0.12_{-0.13}^{+0.13}$ & $1.43_{-0.44}^{+0.53}$ &$41.43_{-9.48}^{+12.14}$ & $11.01_{-0.03}^{+0.05}$ & $10.29_{-0.11}^{+0.10}$ & $10.92_{-0.05}^{+0.07}$ & $0.74_{-0.07}^{+0.07}$ \\
IRAS 05078+1626 & $6.79_{-0.31}^{+0.58}$ & $26.67_{-7.29}^{+5.92}$ & $0.58_{-0.16}^{+0.12}$ & $2.49_{-0.58}^{+0.61}$ &$36.68_{-9.45}^{+8.57}$ & $10.57_{-0.03}^{+0.08}$ & $9.74_{-0.25}^{+0.23}$ & $10.51_{-0.07}^{+0.10}$ & $0.80_{-0.14}^{+0.10}$ \\
IRAS 05218-1212 & $7.01_{-0.08}^{+0.12}$ & $30.22_{-2.60}^{+1.66}$ & $-0.13_{-0.14}^{+0.13}$ & $1.70_{-0.66}^{+0.67}$ &$34.53_{-9.59}^{+12.54}$ & $10.97_{-0.03}^{+0.07}$ & $10.29_{-0.13}^{+0.07}$ & $10.87_{-0.04}^{+0.09}$ & $0.72_{-0.05}^{+0.09}$ \\
IRAS 05589+2828 & $7.34_{-0.14}^{+0.20}$ & $25.67_{-4.16}^{+3.50}$ & $0.27_{-0.15}^{+0.09}$ & $1.24_{-0.26}^{+0.46}$ &$57.54_{-17.42}^{+26.58}$ & $11.01_{-0.06}^{+0.01}$ & $10.20_{-0.25}^{+0.19}$ & $10.93_{-0.10}^{+0.03}$ & $0.79_{-0.14}^{+0.09}$ \\
KAZ 320 & $6.86_{-0.07}^{+0.11}$ & $28.77_{-2.51}^{+1.68}$ & $-0.30_{-0.14}^{+0.14}$ & $1.55_{-0.42}^{+0.53}$ &$43.59_{-13.49}^{+18.03}$ & $10.52_{-0.03}^{+0.04}$ & $10.01_{-0.13}^{+0.08}$ & $10.36_{-0.06}^{+0.08}$ & $0.59_{-0.09}^{+0.12}$ \\
KUG 1141+371 & $7.68_{-0.47}^{+0.47}$ & $14.70_{-2.98}^{+4.04}$ & $-0.91_{-0.13}^{+0.11}$ & $1.39_{-0.32}^{+0.37}$ &$55.19_{-9.75}^{+13.26}$ & $9.90_{-0.04}^{+0.05}$ & $9.08_{-0.16}^{+0.20}$ & $9.83_{-0.05}^{+0.04}$ & $0.80_{-0.10}^{+0.06}$ \\
KUG 1208+386 & $6.62_{-0.28}^{+0.36}$ & $21.54_{-4.77}^{+5.52}$ & $-0.33_{-0.17}^{+0.11}$ & $1.30_{-0.30}^{+0.36}$ &$56.69_{-12.34}^{+13.92}$ & $10.03_{-0.03}^{+0.04}$ & $9.02_{-0.31}^{+0.31}$ & $9.99_{-0.07}^{+0.05}$ & $0.87_{-0.14}^{+0.07}$ \\
LCRS B034324.7-394349 & $7.61_{-0.11}^{+0.11}$ & $19.65_{-1.30}^{+1.44}$ & $-0.51_{-0.11}^{+0.11}$ & $0.92_{-0.26}^{+0.32}$ &$59.71_{-9.78}^{+13.62}$ & $10.59_{-0.03}^{+0.04}$ & $9.76_{-0.07}^{+0.07}$ & $10.51_{-0.05}^{+0.04}$ & $0.80_{-0.04}^{+0.03}$ \\
LCRS B232242.2-384320 & $7.72_{-0.03}^{+0.03}$ & $25.01_{-0.47}^{+0.38}$ & $-0.75_{-0.16}^{+0.20}$ & $1.44_{-0.44}^{+0.57}$ &$43.69_{-15.73}^{+22.60}$ & $10.62_{-0.01}^{+0.02}$ & $10.51_{-0.03}^{+0.02}$ & $9.98_{-0.07}^{+0.13}$ & $<0.15$ \\
LEDA 138501 & $<6.54$ & ... & $-1.22_{-0.19}^{+0.15}$ & $0.88_{-0.41}^{+0.51}$ &$47.06_{-18.52}^{+18.22}$ & $<10.07$ & $<9.33$ & $9.98_{-0.06}^{+0.08}$ & $>0.76$ \\
LEDA 170194 & $7.69_{-0.13}^{+0.13}$ & $22.21_{-1.76}^{+1.76}$ & $-0.27_{-0.21}^{+0.13}$ & $1.56_{-0.31}^{+0.37}$ &$56.67_{-13.42}^{+15.86}$ & $10.58_{-0.03}^{+0.02}$ & $10.17_{-0.08}^{+0.08}$ & $10.37_{-0.11}^{+0.05}$ & $0.48_{-0.14}^{+0.09}$ \\
LEDA 214543 & $7.42_{-0.08}^{+0.08}$ & $20.60_{-0.99}^{+1.07}$ & $-0.70_{-0.14}^{+0.12}$ & $1.40_{-0.36}^{+0.47}$ &$48.86_{-10.19}^{+14.11}$ & $10.12_{-0.03}^{+0.03}$ & $9.70_{-0.06}^{+0.05}$ & $9.91_{-0.06}^{+0.06}$ & $0.49_{-0.07}^{+0.07}$ \\
LEDA 38038 & $7.18_{-0.05}^{+0.08}$ & $31.35_{-2.49}^{+1.31}$ & $0.42_{-0.17}^{+0.15}$ & $2.04_{-0.63}^{+0.66}$ &$36.60_{-10.64}^{+17.34}$ & $11.01_{-0.03}^{+0.04}$ & $10.56_{-0.15}^{+0.06}$ & $10.82_{-0.07}^{+0.10}$ & $0.53_{-0.10}^{+0.15}$ \\
M 106 & $7.30_{-0.02}^{+0.02}$ & $21.92_{-0.17}^{+0.16}$ & $0.61_{-0.20}^{+0.17}$ & $0.53_{-0.40}^{+0.49}$ &$48.10_{-17.08}^{+19.30}$ & $9.82_{-0.01}^{+0.01}$ & $9.74_{-0.01}^{+0.01}$ & $9.05_{-0.07}^{+0.04}$ & $<-0.08$ \\
MCG +00-09-042 & $7.59_{-0.03}^{+0.03}$ & $29.42_{-0.55}^{+0.50}$ & $-0.19_{-0.15}^{+0.22}$ & $1.55_{-0.49}^{+0.60}$ &$43.88_{-16.51}^{+24.92}$ & $10.88_{-0.01}^{+0.03}$ & $10.80_{-0.02}^{+0.02}$ & $10.13_{-0.05}^{+0.15}$ & $<0.07$ \\
MCG +01-57-016 & $7.46_{-0.13}^{+0.13}$ & $23.46_{-2.18}^{+2.31}$ & $0.20_{-0.20}^{+0.11}$ & $1.90_{-0.33}^{+0.37}$ &$54.71_{-12.23}^{+13.41}$ & $10.57_{-0.03}^{+0.02}$ & $10.08_{-0.13}^{+0.12}$ & $10.40_{-0.10}^{+0.06}$ & $0.57_{-0.15}^{+0.11}$ \\
MCG +02-21-013 & $7.99_{-0.05}^{+0.05}$ & $23.11_{-0.57}^{+0.48}$ & $-0.65_{-0.17}^{+0.24}$ & $1.42_{-0.42}^{+0.49}$ &$50.28_{-17.91}^{+18.71}$ & $10.68_{-0.01}^{+0.03}$ & $10.58_{-0.03}^{+0.02}$ & $10.01_{-0.05}^{+0.14}$ & $<0.13$ \\
MCG +02-57-002 & $7.94_{-0.15}^{+0.14}$ & $15.20_{-1.44}^{+1.72}$ & $0.19_{-0.05}^{+0.04}$ & $1.86_{-0.11}^{+0.14}$ &$90.24_{-11.61}^{+9.92}$ & $10.41_{-0.01}^{+0.03}$ & $9.43_{-0.13}^{+0.13}$ & $10.36_{-0.02}^{+0.03}$ & $0.86_{-0.04}^{+0.04}$ \\
MCG +04-22-042 & $8.12_{-0.11}^{+0.09}$ & $15.87_{-0.62}^{+0.76}$ & $-0.27_{-0.15}^{+0.13}$ & $1.15_{-0.38}^{+0.41}$ &$47.57_{-6.48}^{+10.11}$ & $10.53_{-0.04}^{+0.04}$ & $9.72_{-0.03}^{+0.03}$ & $10.45_{-0.05}^{+0.04}$ & $0.79_{-0.03}^{+0.02}$ \\
MCG +04-48-002 & $7.68_{-0.03}^{+0.03}$ & $27.90_{-0.48}^{+0.43}$ & $0.18_{-0.16}^{+0.20}$ & $1.30_{-0.43}^{+0.53}$ &$47.11_{-17.99}^{+18.79}$ & $10.84_{-0.01}^{+0.02}$ & $10.75_{-0.02}^{+0.02}$ & $10.10_{-0.05}^{+0.13}$ & $<0.02$ \\
MCG +05-03-013 & $8.18_{-0.06}^{+0.08}$ & $22.10_{-1.21}^{+0.85}$ & $-0.18_{-0.26}^{+0.22}$ & $1.96_{-0.34}^{+0.42}$ &$55.40_{-15.16}^{+15.71}$ & $10.81_{-0.02}^{+0.02}$ & $10.64_{-0.07}^{+0.05}$ & $10.33_{-0.13}^{+0.13}$ & $<0.34$ \\
MCG +05-28-032 & $7.25_{-0.04}^{+0.05}$ & $26.36_{-0.89}^{+0.57}$ & $-0.38_{-0.16}^{+0.20}$ & $1.68_{-0.44}^{+0.54}$ &$42.04_{-13.41}^{+20.66}$ & $10.36_{-0.02}^{+0.03}$ & $10.18_{-0.05}^{+0.02}$ & $9.89_{-0.07}^{+0.12}$ & $0.12_{-0.06}^{+0.12}$ \\
MCG +06-16-028 & $7.07_{-0.03}^{+0.04}$ & $28.87_{-0.96}^{+0.62}$ & $0.17_{-0.15}^{+0.17}$ & $1.43_{-0.41}^{+0.54}$ &$42.86_{-13.64}^{+19.10}$ & $10.51_{-0.02}^{+0.03}$ & $10.24_{-0.05}^{+0.03}$ & $10.17_{-0.07}^{+0.10}$ & $0.29_{-0.07}^{+0.11}$ \\
MCG +06-24-008 & $7.57_{-0.02}^{+0.03}$ & $24.61_{-0.34}^{+0.34}$ & $-0.92_{-0.16}^{+0.14}$ & $1.14_{-0.45}^{+0.57}$ &$41.94_{-14.96}^{+19.87}$ & $10.40_{-0.01}^{+0.02}$ & $10.32_{-0.02}^{+0.02}$ & $9.62_{-0.07}^{+0.08}$ & $<-0.04$ \\
MCG +06-49-019 & $7.61_{-0.12}^{+0.11}$ & $17.76_{-0.94}^{+1.13}$ & $-0.61_{-0.15}^{+0.10}$ & $1.53_{-0.29}^{+0.34}$ &$60.78_{-10.82}^{+13.14}$ & $9.82_{-0.03}^{+0.02}$ & $9.50_{-0.05}^{+0.05}$ & $9.54_{-0.08}^{+0.04}$ & $0.36_{-0.09}^{+0.06}$ \\
MCG +08-11-011 & $8.24_{-0.08}^{+0.07}$ & $18.88_{-0.73}^{+0.81}$ & $0.88_{-0.08}^{+0.08}$ & $1.72_{-0.33}^{+0.40}$ &$51.58_{-7.47}^{+10.29}$ & $11.06_{-0.03}^{+0.04}$ & $10.30_{-0.03}^{+0.04}$ & $10.98_{-0.04}^{+0.04}$ & $0.77_{-0.02}^{+0.02}$ \\
MCG +11-11-032 & $7.77_{-0.16}^{+0.13}$ & $18.38_{-1.09}^{+1.55}$ & $-0.69_{-0.13}^{+0.09}$ & $1.29_{-0.27}^{+0.29}$ &$63.38_{-10.22}^{+13.10}$ & $10.20_{-0.03}^{+0.02}$ & $9.75_{-0.04}^{+0.06}$ & $10.02_{-0.07}^{+0.03}$ & $0.53_{-0.08}^{+0.04}$ \\
MCG +12-10-067 & $8.25_{-0.09}^{+0.09}$ & $20.39_{-1.01}^{+1.05}$ & $0.14_{-0.12}^{+0.08}$ & $1.82_{-0.26}^{+0.35}$ &$59.41_{-10.72}^{+14.06}$ & $10.86_{-0.03}^{+0.01}$ & $10.51_{-0.06}^{+0.05}$ & $10.60_{-0.07}^{+0.04}$ & $0.40_{-0.09}^{+0.07}$ \\
MCG -01-05-047 & $8.34_{-0.07}^{+0.06}$ & $18.78_{-0.81}^{+0.79}$ & $0.61_{-0.11}^{+0.10}$ & $2.04_{-0.23}^{+0.25}$ &$74.65_{-11.60}^{+12.12}$ & $10.66_{-0.02}^{+0.02}$ & $10.38_{-0.06}^{+0.04}$ & $10.33_{-0.06}^{+0.06}$ & $0.29_{-0.07}^{+0.08}$ \\
MCG -01-09-045 & $6.83_{-0.19}^{+0.19}$ & $19.12_{-1.66}^{+1.92}$ & $-1.50_{-0.19}^{+0.17}$ & $0.97_{-0.44}^{+0.53}$ &$46.13_{-16.45}^{+19.71}$ & $9.05_{-0.04}^{+0.05}$ & $8.92_{-0.05}^{+0.06}$ & $8.48_{-0.07}^{+0.08}$ & $<0.13$ \\
MCG -01-13-025 & $6.20_{-0.20}^{+0.21}$ & $24.20_{-2.46}^{+1.64}$ & $-0.88_{-0.17}^{+0.14}$ & $1.02_{-0.42}^{+0.61}$ &$44.05_{-17.06}^{+18.96}$ & $9.42_{-0.04}^{+0.03}$ & $8.90_{-0.14}^{+0.04}$ & $9.27_{-0.06}^{+0.07}$ & $0.60_{-0.06}^{+0.11}$ \\
MCG -01-24-012 & $7.73_{-0.08}^{+0.08}$ & $18.65_{-1.03}^{+1.06}$ & $0.24_{-0.09}^{+0.07}$ & $2.28_{-0.46}^{+0.49}$ &$43.76_{-6.00}^{+9.25}$ & $10.35_{-0.04}^{+0.04}$ & $9.76_{-0.07}^{+0.06}$ & $10.22_{-0.05}^{+0.06}$ & $0.66_{-0.05}^{+0.05}$ \\
MCG -01-30-041 & $7.24_{-0.04}^{+0.06}$ & $26.55_{-1.20}^{+0.74}$ & $-0.09_{-0.24}^{+0.24}$ & $1.89_{-0.42}^{+0.46}$ &$49.19_{-17.37}^{+18.58}$ & $10.37_{-0.01}^{+0.03}$ & $10.19_{-0.07}^{+0.04}$ & $9.90_{-0.10}^{+0.16}$ & $<0.39$ \\
MCG -01-33-063 & $7.93_{-0.06}^{+0.07}$ & $19.70_{-0.81}^{+0.58}$ & $-0.84_{-0.27}^{+0.31}$ & $1.23_{-0.43}^{+0.44}$ &$64.63_{-17.21}^{+15.40}$ & $10.21_{-0.01}^{+0.03}$ & $10.09_{-0.04}^{+0.03}$ & $9.59_{-0.07}^{+0.14}$ & $<0.16$ \\
MCG -01-40-001 & $8.03_{-0.10}^{+0.09}$ & $20.18_{-1.12}^{+1.19}$ & $0.61_{-0.05}^{+0.05}$ & $2.78_{-0.29}^{+0.42}$ &$54.32_{-8.39}^{+11.65}$ & $10.73_{-0.02}^{+0.02}$ & $10.26_{-0.06}^{+0.05}$ & $10.55_{-0.04}^{+0.04}$ & $0.55_{-0.06}^{+0.06}$ \\
MCG -02-02-095 & $<6.24$ & ... & $-0.62_{-0.20}^{+0.10}$ & $1.96_{-0.39}^{+0.54}$ &$49.09_{-11.82}^{+12.81}$ & $<9.52$ & $<9.02$ & $9.35_{-0.11}^{+0.06}$ & $>0.58$ \\
MCG -02-08-014 & $7.50_{-0.12}^{+0.11}$ & $16.91_{-0.96}^{+1.09}$ & $-0.08_{-0.05}^{+0.04}$ & $1.81_{-0.24}^{+0.29}$ &$62.47_{-9.01}^{+12.89}$ & $9.88_{-0.02}^{+0.02}$ & $9.27_{-0.05}^{+0.05}$ & $9.75_{-0.03}^{+0.03}$ & $0.67_{-0.04}^{+0.03}$ \\
MCG -02-08-038 & $7.61_{-0.24}^{+0.22}$ & $19.24_{-1.93}^{+2.37}$ & $-0.53_{-0.14}^{+0.12}$ & $1.42_{-0.37}^{+0.44}$ &$47.77_{-10.97}^{+12.18}$ & $10.25_{-0.03}^{+0.04}$ & $9.72_{-0.07}^{+0.07}$ & $10.10_{-0.05}^{+0.07}$ & $0.61_{-0.07}^{+0.07}$ \\
MCG -02-12-050 & $8.14_{-0.08}^{+0.08}$ & $20.87_{-1.02}^{+1.06}$ & $-0.11_{-0.17}^{+0.12}$ & $1.79_{-0.28}^{+0.31}$ &$61.70_{-11.53}^{+14.45}$ & $10.74_{-0.02}^{+0.02}$ & $10.45_{-0.05}^{+0.06}$ & $10.42_{-0.08}^{+0.06}$ & $0.31_{-0.10}^{+0.08}$ \\
MCG -02-14-009 & $7.58_{-0.14}^{+0.26}$ & $22.05_{-2.68}^{+1.52}$ & $-0.32_{-0.15}^{+0.17}$ & $1.30_{-0.40}^{+0.54}$ &$46.99_{-17.45}^{+17.35}$ & $10.45_{-0.02}^{+0.05}$ & $10.04_{-0.08}^{+0.05}$ & $10.24_{-0.06}^{+0.10}$ & $0.48_{-0.08}^{+0.12}$ \\
MCG -03-04-072 & $8.05_{-0.10}^{+0.09}$ & $16.72_{-0.80}^{+0.84}$ & $-0.50_{-0.09}^{+0.10}$ & $1.22_{-0.28}^{+0.38}$ &$57.51_{-9.69}^{+13.17}$ & $10.54_{-0.03}^{+0.03}$ & $9.78_{-0.04}^{+0.04}$ & $10.46_{-0.04}^{+0.04}$ & $0.77_{-0.03}^{+0.03}$ \\
MCG -03-34-064 & $7.08_{-0.03}^{+0.06}$ & $32.50_{-1.82}^{+0.75}$ & $0.82_{-0.15}^{+0.15}$ & $2.43_{-0.91}^{+0.82}$ &$29.79_{-8.40}^{+15.93}$ & $10.97_{-0.03}^{+0.04}$ & $10.55_{-0.11}^{+0.03}$ & $10.76_{-0.06}^{+0.12}$ & $0.49_{-0.06}^{+0.14}$ \\
MCG -05-23-016 & $5.43_{-0.06}^{+0.09}$ & $39.66_{-3.71}^{+1.95}$ & $0.69_{-0.14}^{+0.12}$ & $2.41_{-0.60}^{+0.75}$ &$28.88_{-7.58}^{+6.78}$ & $10.14_{-0.03}^{+0.08}$ & $9.41_{-0.17}^{+0.07}$ & $10.05_{-0.04}^{+0.10}$ & $0.75_{-0.05}^{+0.10}$ \\
MCG -06-30-015 & $5.80_{-0.05}^{+0.08}$ & $32.01_{-2.35}^{+1.26}$ & $0.29_{-0.11}^{+0.11}$ & $1.56_{-0.52}^{+0.72}$ &$36.54_{-12.12}^{+14.59}$ & $9.81_{-0.03}^{+0.05}$ & $9.23_{-0.12}^{+0.06}$ & $9.68_{-0.05}^{+0.08}$ & $0.65_{-0.06}^{+0.10}$ \\
MCG -07-03-007 & $7.18_{-0.07}^{+0.08}$ & $24.44_{-1.35}^{+1.37}$ & $-0.26_{-0.15}^{+0.12}$ & $2.20_{-0.59}^{+0.59}$ &$37.44_{-9.54}^{+11.09}$ & $10.36_{-0.02}^{+0.06}$ & $9.91_{-0.07}^{+0.07}$ & $10.16_{-0.06}^{+0.10}$ & $0.52_{-0.09}^{+0.10}$ \\
Mrk 10 & $8.41_{-0.06}^{+0.06}$ & $17.95_{-0.62}^{+0.62}$ & $0.15_{-0.06}^{+0.05}$ & $1.98_{-0.23}^{+0.33}$ &$63.18_{-10.65}^{+13.31}$ & $10.68_{-0.02}^{+0.01}$ & $10.33_{-0.04}^{+0.03}$ & $10.42_{-0.05}^{+0.03}$ & $0.40_{-0.05}^{+0.04}$ \\
Mrk 1018 & $8.36_{-0.15}^{+0.13}$ & $13.11_{-0.71}^{+0.80}$ & $-0.68_{-0.13}^{+0.12}$ & $0.90_{-0.28}^{+0.34}$ &$57.32_{-9.41}^{+11.93}$ & $10.41_{-0.04}^{+0.04}$ & $9.46_{-0.05}^{+0.04}$ & $10.35_{-0.04}^{+0.04}$ & $0.85_{-0.02}^{+0.02}$ \\
Mrk 1210 & $6.58_{-0.13}^{+0.12}$ & $28.09_{-2.71}^{+3.09}$ & $0.82_{-0.16}^{+0.09}$ & $3.15_{-0.61}^{+0.60}$ &$30.46_{-4.67}^{+4.67}$ & $10.56_{-0.06}^{+0.07}$ & $9.67_{-0.15}^{+0.15}$ & $10.50_{-0.09}^{+0.08}$ & $0.83_{-0.09}^{+0.06}$ \\
Mrk 1310 & $6.75_{-0.08}^{+0.11}$ & $22.78_{-1.49}^{+0.94}$ & $-0.96_{-0.22}^{+0.22}$ & $1.12_{-0.41}^{+0.44}$ &$54.67_{-18.34}^{+18.39}$ & $9.59_{-0.02}^{+0.03}$ & $9.29_{-0.08}^{+0.04}$ & $9.29_{-0.06}^{+0.10}$ & $0.33_{-0.07}^{+0.13}$ \\
Mrk 1392 & $7.86_{-0.07}^{+0.08}$ & $21.99_{-1.07}^{+0.92}$ & $-0.20_{-0.15}^{+0.13}$ & $1.90_{-0.51}^{+0.58}$ &$37.07_{-9.46}^{+9.90}$ & $10.69_{-0.02}^{+0.05}$ & $10.32_{-0.05}^{+0.04}$ & $10.46_{-0.06}^{+0.09}$ & $0.44_{-0.07}^{+0.09}$ \\
Mrk 18 & $6.90_{-0.15}^{+0.23}$ & $24.64_{-5.04}^{+4.19}$ & $0.67_{-0.60}^{+0.21}$ & $2.39_{-0.40}^{+0.20}$ &$78.11_{-32.22}^{+8.24}$ & $10.10_{-0.03}^{+0.03}$ & $9.65_{-0.37}^{+0.26}$ & $9.90_{-0.33}^{+0.14}$ & $<0.87$ \\
Mrk 198 & $7.12_{-0.03}^{+0.04}$ & $27.61_{-1.00}^{+0.61}$ & $-0.44_{-0.17}^{+0.23}$ & $1.34_{-0.40}^{+0.45}$ &$49.41_{-18.01}^{+23.85}$ & $10.37_{-0.01}^{+0.04}$ & $10.16_{-0.06}^{+0.03}$ & $9.94_{-0.06}^{+0.14}$ & $0.17_{-0.06}^{+0.15}$ \\
Mrk 202 & $6.61_{-0.08}^{+0.08}$ & $25.54_{-1.07}^{+0.83}$ & $-0.99_{-0.16}^{+0.20}$ & $1.27_{-0.41}^{+0.50}$ &$48.21_{-17.53}^{+17.34}$ & $9.69_{-0.02}^{+0.03}$ & $9.46_{-0.05}^{+0.03}$ & $9.31_{-0.05}^{+0.11}$ & $0.22_{-0.05}^{+0.11}$ \\
Mrk 279 & $7.29_{-0.06}^{+0.08}$ & $27.07_{-1.79}^{+1.50}$ & $0.10_{-0.16}^{+0.14}$ & $1.77_{-0.47}^{+0.56}$ &$42.22_{-12.22}^{+13.47}$ & $10.77_{-0.03}^{+0.05}$ & $10.28_{-0.10}^{+0.08}$ & $10.59_{-0.07}^{+0.10}$ & $0.56_{-0.11}^{+0.12}$ \\
Mrk 290 & $6.37_{-0.17}^{+0.20}$ & $27.83_{-3.33}^{+3.61}$ & $-0.27_{-0.13}^{+0.13}$ & $1.63_{-0.50}^{+0.59}$ &$39.84_{-9.00}^{+9.85}$ & $10.32_{-0.02}^{+0.06}$ & $9.43_{-0.14}^{+0.16}$ & $10.26_{-0.04}^{+0.08}$ & $0.83_{-0.07}^{+0.06}$ \\
Mrk 3 & ... & ... & ... & ... &... & ... & ... & ... & ... \\
Mrk 335 & $6.46_{-0.20}^{+0.18}$ & $26.11_{-2.94}^{+3.93}$ & $-0.11_{-0.11}^{+0.13}$ & $1.02_{-0.29}^{+0.41}$ &$52.82_{-10.05}^{+12.85}$ & $10.47_{-0.04}^{+0.05}$ & $9.36_{-0.13}^{+0.17}$ & $10.44_{-0.05}^{+0.05}$ & $0.90_{-0.05}^{+0.03}$ \\
Mrk 348 & $7.02_{-0.23}^{+0.40}$ & $23.23_{-4.29}^{+3.44}$ & $0.33_{-0.14}^{+0.13}$ & $1.56_{-0.45}^{+0.52}$ &$43.16_{-12.99}^{+13.08}$ & $10.34_{-0.02}^{+0.06}$ & $9.61_{-0.13}^{+0.14}$ & $10.25_{-0.06}^{+0.08}$ & $0.75_{-0.10}^{+0.08}$ \\
Mrk 352 & $<5.55$ & ... & $-1.47_{-0.20}^{+0.15}$ & $0.81_{-0.42}^{+0.56}$ &$46.52_{-17.41}^{+18.39}$ & $<8.86$ & $<8.34$ & $8.70_{-0.08}^{+0.05}$ & $>0.60$ \\
Mrk 359 & $6.92_{-0.04}^{+0.06}$ & $28.80_{-1.64}^{+0.77}$ & $-0.03_{-0.18}^{+0.23}$ & $1.66_{-0.39}^{+0.56}$ &$44.71_{-15.52}^{+27.01}$ & $10.33_{-0.02}^{+0.03}$ & $10.07_{-0.09}^{+0.04}$ & $9.98_{-0.08}^{+0.12}$ & $0.26_{-0.08}^{+0.16}$ \\
Mrk 417 & $6.51_{-0.14}^{+0.21}$ & $26.78_{-3.89}^{+2.73}$ & $-0.37_{-0.15}^{+0.12}$ & $1.71_{-0.54}^{+0.58}$ &$38.00_{-10.09}^{+10.52}$ & $10.31_{-0.03}^{+0.07}$ & $9.47_{-0.21}^{+0.13}$ & $10.24_{-0.04}^{+0.09}$ & $0.80_{-0.06}^{+0.09}$ \\
Mrk 477 & $7.14_{-0.06}^{+0.09}$ & $32.57_{-2.55}^{+1.75}$ & $0.28_{-0.19}^{+0.13}$ & $2.63_{-0.58}^{+0.63}$ &$34.43_{-8.41}^{+9.77}$ & $11.06_{-0.04}^{+0.04}$ & $10.62_{-0.12}^{+0.08}$ & $10.86_{-0.10}^{+0.09}$ & $0.52_{-0.13}^{+0.14}$ \\
Mrk 50 & $<6.40$ & ... & $-1.31_{-0.23}^{+0.14}$ & $0.76_{-0.54}^{+0.67}$ &$35.75_{-11.87}^{+15.01}$ & $<9.55$ & $<9.19$ & $9.30_{-0.07}^{+0.05}$ & $>0.42$ \\
Mrk 509 & $7.35_{-0.04}^{+0.06}$ & $30.65_{-1.57}^{+0.95}$ & $0.23_{-0.15}^{+0.14}$ & $1.56_{-0.52}^{+0.62}$ &$37.56_{-10.75}^{+17.21}$ & $11.11_{-0.03}^{+0.04}$ & $10.66_{-0.08}^{+0.05}$ & $10.92_{-0.06}^{+0.08}$ & $0.53_{-0.07}^{+0.10}$ \\
Mrk 590 & $8.21_{-0.06}^{+0.06}$ & $18.84_{-0.63}^{+0.67}$ & $-0.02_{-0.11}^{+0.10}$ & $1.51_{-0.32}^{+0.38}$ &$53.98_{-9.05}^{+12.22}$ & $10.61_{-0.02}^{+0.02}$ & $10.26_{-0.04}^{+0.03}$ & $10.36_{-0.05}^{+0.04}$ & $0.41_{-0.06}^{+0.05}$ \\
Mrk 595 & $7.28_{-0.15}^{+0.16}$ & $22.85_{-2.16}^{+2.19}$ & $-0.22_{-0.20}^{+0.12}$ & $1.74_{-0.28}^{+0.39}$ &$56.31_{-13.24}^{+14.99}$ & $10.28_{-0.04}^{+0.02}$ & $9.83_{-0.12}^{+0.09}$ & $10.09_{-0.11}^{+0.06}$ & $0.53_{-0.14}^{+0.11}$ \\
Mrk 6 & $7.24_{-0.14}^{+0.12}$ & $22.77_{-1.61}^{+2.09}$ & $0.41_{-0.11}^{+0.10}$ & $1.67_{-0.37}^{+0.45}$ &$49.40_{-8.10}^{+11.64}$ & $10.54_{-0.04}^{+0.04}$ & $9.78_{-0.08}^{+0.09}$ & $10.46_{-0.05}^{+0.05}$ & $0.77_{-0.06}^{+0.05}$ \\
Mrk 618 & $7.98_{-0.05}^{+0.08}$ & $27.49_{-1.78}^{+1.07}$ & $0.38_{-0.19}^{+0.20}$ & $1.53_{-0.37}^{+0.47}$ &$49.69_{-17.87}^{+23.59}$ & $11.33_{-0.02}^{+0.04}$ & $11.02_{-0.11}^{+0.05}$ & $11.04_{-0.07}^{+0.13}$ & $0.35_{-0.08}^{+0.17}$ \\
Mrk 653 & $7.88_{-0.12}^{+0.10}$ & $19.69_{-1.01}^{+1.27}$ & $-0.51_{-0.13}^{+0.11}$ & $1.60_{-0.43}^{+0.46}$ &$44.45_{-7.56}^{+11.67}$ & $10.51_{-0.03}^{+0.04}$ & $10.04_{-0.04}^{+0.05}$ & $10.33_{-0.05}^{+0.06}$ & $0.55_{-0.06}^{+0.06}$ \\
Mrk 704 & $6.81_{-0.20}^{+0.24}$ & $26.01_{-3.56}^{+3.89}$ & $0.08_{-0.16}^{+0.12}$ & $1.13_{-0.46}^{+0.54}$ &$41.72_{-10.65}^{+11.56}$ & $10.77_{-0.04}^{+0.05}$ & $9.70_{-0.16}^{+0.18}$ & $10.73_{-0.05}^{+0.06}$ & $0.89_{-0.06}^{+0.04}$ \\
Mrk 728 & $<6.43$ & ... & $-0.95_{-0.18}^{+0.08}$ & $1.66_{-0.29}^{+0.36}$ &$59.45_{-11.57}^{+14.50}$ & $<9.76$ & $<9.22$ & $9.62_{-0.11}^{+0.03}$ & $>0.62$ \\
Mrk 739E & $7.84_{-0.03}^{+0.03}$ & $25.92_{-0.42}^{+0.34}$ & $-0.34_{-0.16}^{+0.14}$ & $1.03_{-0.41}^{+0.61}$ &$43.99_{-15.90}^{+22.50}$ & $10.88_{-0.02}^{+0.02}$ & $10.72_{-0.02}^{+0.02}$ & $10.36_{-0.06}^{+0.08}$ & $<0.23$ \\
Mrk 766 & $6.89_{-0.05}^{+0.12}$ & $30.42_{-3.89}^{+1.54}$ & $0.72_{-0.19}^{+0.19}$ & $2.32_{-0.58}^{+0.72}$ &$39.70_{-11.29}^{+37.32}$ & $10.58_{-0.04}^{+0.02}$ & $10.19_{-0.26}^{+0.08}$ & $10.36_{-0.11}^{+0.11}$ & $0.46_{-0.13}^{+0.23}$ \\
Mrk 79 & $7.83_{-0.09}^{+0.09}$ & $22.12_{-1.19}^{+1.37}$ & $0.42_{-0.13}^{+0.11}$ & $1.53_{-0.37}^{+0.42}$ &$49.11_{-10.12}^{+13.74}$ & $10.82_{-0.02}^{+0.04}$ & $10.30_{-0.07}^{+0.07}$ & $10.66_{-0.05}^{+0.06}$ & $0.60_{-0.07}^{+0.07}$ \\
Mrk 817 & $7.51_{-0.07}^{+0.09}$ & $28.46_{-2.50}^{+2.11}$ & $0.57_{-0.17}^{+0.12}$ & $2.06_{-0.59}^{+0.69}$ &$40.92_{-10.51}^{+18.52}$ & $11.18_{-0.04}^{+0.04}$ & $10.64_{-0.14}^{+0.12}$ & $11.04_{-0.08}^{+0.08}$ & $0.62_{-0.13}^{+0.12}$ \\
Mrk 841 & $6.44_{-0.13}^{+0.18}$ & $34.21_{-4.41}^{+3.67}$ & $0.12_{-0.13}^{+0.12}$ & $1.97_{-0.56}^{+0.57}$ &$35.03_{-7.78}^{+9.44}$ & $10.85_{-0.03}^{+0.07}$ & $10.05_{-0.18}^{+0.14}$ & $10.78_{-0.05}^{+0.08}$ & $0.79_{-0.08}^{+0.08}$ \\
Mrk 885 & $7.44_{-0.04}^{+0.04}$ & $23.36_{-0.51}^{+0.41}$ & $-0.83_{-0.17}^{+0.19}$ & $1.52_{-0.41}^{+0.46}$ &$44.60_{-14.27}^{+17.98}$ & $10.17_{-0.01}^{+0.02}$ & $10.05_{-0.04}^{+0.02}$ & $9.56_{-0.06}^{+0.13}$ & $<0.15$ \\
Mrk 926 & $7.66_{-0.08}^{+0.10}$ & $26.11_{-1.86}^{+1.37}$ & $-0.13_{-0.17}^{+0.15}$ & $1.24_{-0.34}^{+0.38}$ &$49.67_{-13.63}^{+15.45}$ & $11.05_{-0.03}^{+0.03}$ & $10.56_{-0.11}^{+0.07}$ & $10.87_{-0.07}^{+0.07}$ & $0.56_{-0.09}^{+0.11}$ \\
Mrk 975 & $7.94_{-0.04}^{+0.06}$ & $25.93_{-1.26}^{+0.74}$ & $-0.07_{-0.15}^{+0.13}$ & $1.42_{-0.44}^{+0.61}$ &$40.35_{-13.04}^{+18.62}$ & $11.20_{-0.03}^{+0.03}$ & $10.82_{-0.08}^{+0.04}$ & $10.97_{-0.08}^{+0.08}$ & $0.45_{-0.08}^{+0.11}$ \\
NGC 1052 & $6.36_{-0.16}^{+0.14}$ & $18.16_{-1.34}^{+1.71}$ & $0.37_{-0.06}^{+0.07}$ & $1.57_{-0.27}^{+0.35}$ &$58.77_{-9.00}^{+12.16}$ & $9.21_{-0.03}^{+0.03}$ & $8.31_{-0.08}^{+0.08}$ & $9.15_{-0.04}^{+0.04}$ & $0.83_{-0.04}^{+0.03}$ \\
NGC 1106 & $7.42_{-0.07}^{+0.06}$ & $21.12_{-0.88}^{+0.98}$ & $0.35_{-0.09}^{+0.07}$ & $2.22_{-0.40}^{+0.48}$ &$47.88_{-7.65}^{+10.81}$ & $10.23_{-0.03}^{+0.03}$ & $9.76_{-0.06}^{+0.06}$ & $10.04_{-0.06}^{+0.05}$ & $0.54_{-0.07}^{+0.06}$ \\
NGC 1125 & $6.74_{-0.11}^{+0.16}$ & $29.30_{-3.20}^{+1.94}$ & $0.56_{-0.23}^{+0.18}$ & $3.04_{-0.64}^{+0.63}$ &$36.11_{-9.46}^{+11.63}$ & $10.27_{-0.04}^{+0.04}$ & $9.94_{-0.13}^{+0.07}$ & $10.00_{-0.12}^{+0.12}$ & $0.37_{-0.14}^{+0.19}$ \\
NGC 1194 & $7.56_{-0.19}^{+0.15}$ & $15.00_{-0.86}^{+1.18}$ & $0.33_{-0.09}^{+0.09}$ & $1.50_{-0.31}^{+0.39}$ &$54.43_{-8.22}^{+10.95}$ & $10.16_{-0.03}^{+0.04}$ & $9.01_{-0.03}^{+0.03}$ & $10.13_{-0.03}^{+0.04}$ & $0.90_{-0.01}^{+0.01}$ \\
NGC 1365 & $8.20_{-0.04}^{+0.05}$ & $23.96_{-0.91}^{+0.63}$ & $1.86_{-0.27}^{+0.24}$ & $2.06_{-0.41}^{+0.40}$ &$54.32_{-14.89}^{+17.69}$ & $11.02_{-0.01}^{+0.02}$ & $10.88_{-0.06}^{+0.04}$ & $10.46_{-0.13}^{+0.15}$ & $<0.28$ \\
NGC 2110 & $6.91_{-0.03}^{+0.04}$ & $28.02_{-0.78}^{+0.53}$ & $0.34_{-0.15}^{+0.18}$ & $1.37_{-0.42}^{+0.56}$ &$44.91_{-17.31}^{+21.27}$ & $10.21_{-0.01}^{+0.03}$ & $9.99_{-0.04}^{+0.02}$ & $9.79_{-0.06}^{+0.11}$ & $0.18_{-0.05}^{+0.11}$ \\
NGC 235A & $7.46_{-0.05}^{+0.07}$ & $27.73_{-1.52}^{+0.81}$ & $0.21_{-0.22}^{+0.22}$ & $2.71_{-0.65}^{+0.79}$ &$34.94_{-10.52}^{+16.85}$ & $10.73_{-0.03}^{+0.03}$ & $10.52_{-0.08}^{+0.03}$ & $10.32_{-0.10}^{+0.15}$ & $0.18_{-0.08}^{+0.16}$ \\
NGC 2655 & $7.05_{-0.15}^{+0.42}$ & $20.61_{-4.28}^{+1.75}$ & $0.22_{-0.70}^{+0.42}$ & $1.44_{-1.19}^{+0.85}$ &$67.21_{-17.33}^{+13.34}$ & $9.57_{-0.03}^{+0.01}$ & $9.33_{-0.20}^{+0.06}$ & $9.19_{-0.13}^{+0.15}$ & $0.23_{-0.14}^{+0.26}$ \\
NGC 2885 & $7.84_{-0.13}^{+0.12}$ & $18.03_{-1.08}^{+1.35}$ & $-0.35_{-0.11}^{+0.07}$ & $1.72_{-0.24}^{+0.31}$ &$63.37_{-10.79}^{+13.38}$ & $10.15_{-0.03}^{+0.01}$ & $9.78_{-0.06}^{+0.06}$ & $9.91_{-0.07}^{+0.03}$ & $0.43_{-0.10}^{+0.06}$ \\
NGC 2992 & $7.26_{-0.04}^{+0.05}$ & $26.31_{-1.25}^{+0.62}$ & $0.64_{-0.24}^{+0.29}$ & $1.97_{-0.39}^{+0.49}$ &$47.76_{-18.23}^{+23.04}$ & $10.33_{-0.01}^{+0.03}$ & $10.18_{-0.08}^{+0.03}$ & $9.79_{-0.11}^{+0.18}$ & $<0.38$ \\
NGC 3035 & $7.63_{-0.04}^{+0.05}$ & $20.88_{-0.61}^{+0.49}$ & $-0.35_{-0.22}^{+0.20}$ & $1.70_{-0.36}^{+0.43}$ &$52.46_{-15.83}^{+16.95}$ & $10.06_{-0.01}^{+0.02}$ & $9.94_{-0.03}^{+0.02}$ & $9.45_{-0.09}^{+0.11}$ & $<0.13$ \\
NGC 3079 & $8.14_{-0.02}^{+0.02}$ & $24.74_{-0.26}^{+0.27}$ & $0.87_{-0.18}^{+0.21}$ & $1.06_{-0.41}^{+0.50}$ &$51.30_{-18.09}^{+18.18}$ & $10.97_{-0.01}^{+0.02}$ & $10.90_{-0.01}^{+0.01}$ & $10.13_{-0.06}^{+0.10}$ & $<-0.06$ \\
NGC 3081 & $7.31_{-0.07}^{+0.07}$ & $20.00_{-0.88}^{+0.85}$ & $0.77_{-0.06}^{+0.06}$ & $1.98_{-0.34}^{+0.41}$ &$54.29_{-8.71}^{+13.47}$ & $10.05_{-0.03}^{+0.03}$ & $9.51_{-0.05}^{+0.05}$ & $9.90_{-0.04}^{+0.03}$ & $0.62_{-0.04}^{+0.04}$ \\
NGC 3227 & $7.61_{-0.03}^{+0.04}$ & $23.07_{-0.69}^{+0.48}$ & $0.80_{-0.18}^{+0.18}$ & $1.55_{-0.39}^{+0.46}$ &$48.77_{-15.47}^{+16.09}$ & $10.37_{-0.01}^{+0.03}$ & $10.18_{-0.05}^{+0.02}$ & $9.91_{-0.06}^{+0.12}$ & $0.13_{-0.05}^{+0.11}$ \\
NGC 3281 & $7.23_{-0.03}^{+0.04}$ & $28.51_{-1.02}^{+0.65}$ & $0.84_{-0.15}^{+0.14}$ & $2.12_{-0.59}^{+0.65}$ &$34.51_{-9.50}^{+12.67}$ & $10.68_{-0.02}^{+0.04}$ & $10.36_{-0.06}^{+0.03}$ & $10.39_{-0.06}^{+0.10}$ & $0.36_{-0.06}^{+0.11}$ \\
NGC 3393 & $7.84_{-0.08}^{+0.08}$ & $19.03_{-0.74}^{+0.85}$ & $0.58_{-0.07}^{+0.07}$ & $2.11_{-0.38}^{+0.44}$ &$50.11_{-7.45}^{+12.05}$ & $10.35_{-0.03}^{+0.03}$ & $9.91_{-0.04}^{+0.04}$ & $10.15_{-0.05}^{+0.04}$ & $0.51_{-0.06}^{+0.04}$ \\
NGC 3431 & $7.89_{-0.09}^{+0.08}$ & $18.09_{-0.81}^{+0.91}$ & $0.09_{-0.10}^{+0.08}$ & $1.81_{-0.23}^{+0.27}$ &$70.06_{-11.17}^{+13.38}$ & $10.18_{-0.02}^{+0.02}$ & $9.84_{-0.05}^{+0.05}$ & $9.92_{-0.07}^{+0.04}$ & $0.40_{-0.08}^{+0.06}$ \\
NGC 3516 & $6.15_{-0.06}^{+0.07}$ & $31.72_{-1.84}^{+1.09}$ & $0.31_{-0.13}^{+0.14}$ & $1.58_{-0.51}^{+0.68}$ &$36.85_{-11.22}^{+16.55}$ & $10.01_{-0.03}^{+0.04}$ & $9.56_{-0.10}^{+0.04}$ & $9.82_{-0.05}^{+0.09}$ & $0.53_{-0.06}^{+0.12}$ \\
NGC 3718 & $7.03_{-0.08}^{+0.59}$ & $17.95_{-4.14}^{+0.54}$ & $-0.81_{-0.35}^{+0.98}$ & $0.33_{-0.62}^{+1.70}$ &$63.53_{-23.47}^{+20.13}$ & $9.07_{-0.03}^{+0.01}$ & $8.95_{-0.09}^{+0.01}$ & $8.43_{-0.05}^{+0.17}$ & $<0.18$ \\
NGC 3783 & $7.45_{-0.06}^{+0.06}$ & $20.65_{-0.75}^{+0.83}$ & $0.91_{-0.10}^{+0.09}$ & $1.69_{-0.37}^{+0.39}$ &$47.93_{-5.97}^{+9.83}$ & $10.42_{-0.03}^{+0.04}$ & $9.74_{-0.04}^{+0.05}$ & $10.31_{-0.05}^{+0.05}$ & $0.72_{-0.04}^{+0.04}$ \\
NGC 3786 & $6.88_{-0.14}^{+0.11}$ & $26.15_{-1.08}^{+1.30}$ & $-0.28_{-0.17}^{+0.17}$ & $1.13_{-0.43}^{+0.48}$ &$47.98_{-16.62}^{+18.02}$ & $9.92_{-0.02}^{+0.04}$ & $9.78_{-0.05}^{+0.03}$ & $9.34_{-0.03}^{+0.10}$ & $<0.16$ \\
NGC 4051 & $7.61_{-0.03}^{+0.04}$ & $21.50_{-0.48}^{+0.36}$ & $0.76_{-0.16}^{+0.18}$ & $1.26_{-0.38}^{+0.43}$ &$49.57_{-16.13}^{+16.56}$ & $10.15_{-0.01}^{+0.02}$ & $10.01_{-0.03}^{+0.02}$ & $9.59_{-0.05}^{+0.10}$ & $<0.15$ \\
NGC 4102 & $7.29_{-0.03}^{+0.04}$ & $28.07_{-1.02}^{+0.57}$ & $1.42_{-0.23}^{+0.22}$ & $2.61_{-0.49}^{+0.66}$ &$38.21_{-11.50}^{+17.38}$ & $10.53_{-0.02}^{+0.02}$ & $10.38_{-0.06}^{+0.03}$ & $10.00_{-0.12}^{+0.14}$ & $<0.33$ \\
NGC 4138 & $6.63_{-0.03}^{+0.03}$ & $21.53_{-0.31}^{+0.28}$ & $-0.53_{-0.23}^{+0.18}$ & $0.34_{-0.40}^{+0.49}$ &$50.48_{-19.21}^{+18.12}$ & $9.14_{-0.02}^{+0.01}$ & $9.03_{-0.02}^{+0.02}$ & $8.51_{-0.07}^{+0.04}$ & $<0.02$ \\
NGC 4151 & $6.40_{-0.08}^{+0.08}$ & $24.10_{-1.21}^{+1.33}$ & $1.16_{-0.14}^{+0.12}$ & $1.79_{-0.48}^{+0.57}$ &$40.36_{-7.92}^{+9.25}$ & $9.65_{-0.03}^{+0.05}$ & $9.09_{-0.06}^{+0.07}$ & $9.51_{-0.06}^{+0.07}$ & $0.64_{-0.07}^{+0.07}$ \\
NGC 4180 & $7.23_{-0.03}^{+0.02}$ & $24.91_{-0.29}^{+0.30}$ & $-0.20_{-0.17}^{+0.14}$ & $0.88_{-0.39}^{+0.52}$ &$47.36_{-17.40}^{+18.40}$ & $10.09_{-0.01}^{+0.01}$ & $10.00_{-0.02}^{+0.02}$ & $9.32_{-0.06}^{+0.07}$ & $<-0.05$ \\
NGC 4235 & $6.68_{-0.06}^{+0.09}$ & $20.62_{-1.25}^{+0.67}$ & $-0.60_{-0.22}^{+0.26}$ & $0.94_{-0.43}^{+0.44}$ &$60.40_{-19.27}^{+16.67}$ & $9.21_{-0.02}^{+0.03}$ & $8.96_{-0.08}^{+0.04}$ & $8.85_{-0.05}^{+0.11}$ & $0.25_{-0.06}^{+0.13}$ \\
NGC 424 & $7.36_{-0.08}^{+0.08}$ & $19.43_{-1.03}^{+1.09}$ & $0.66_{-0.09}^{+0.10}$ & $1.11_{-0.26}^{+0.36}$ &$59.24_{-9.59}^{+12.94}$ & $10.50_{-0.03}^{+0.03}$ & $9.48_{-0.07}^{+0.07}$ & $10.46_{-0.03}^{+0.04}$ & $0.87_{-0.02}^{+0.02}$ \\
NGC 4388 & $7.03_{-0.04}^{+0.05}$ & $24.33_{-0.93}^{+0.80}$ & $1.03_{-0.19}^{+0.14}$ & $2.10_{-0.46}^{+0.58}$ &$43.44_{-12.15}^{+15.23}$ & $10.00_{-0.03}^{+0.03}$ & $9.75_{-0.06}^{+0.04}$ & $9.64_{-0.10}^{+0.09}$ & $0.25_{-0.10}^{+0.11}$ \\
NGC 4507 & $7.66_{-0.06}^{+0.06}$ & $21.00_{-0.86}^{+0.83}$ & $0.98_{-0.06}^{+0.05}$ & $1.79_{-0.24}^{+0.35}$ &$59.42_{-9.37}^{+11.06}$ & $10.64_{-0.03}^{+0.02}$ & $10.00_{-0.05}^{+0.05}$ & $10.53_{-0.04}^{+0.03}$ & $0.70_{-0.04}^{+0.03}$ \\
NGC 4619 & $8.02_{-0.03}^{+0.03}$ & $22.49_{-0.31}^{+0.29}$ & $-0.74_{-0.21}^{+0.20}$ & $0.49_{-0.39}^{+0.48}$ &$53.57_{-19.31}^{+17.61}$ & $10.63_{-0.02}^{+0.01}$ & $10.53_{-0.02}^{+0.01}$ & $9.96_{-0.07}^{+0.05}$ & $<0.01$ \\
NGC 4748 & $7.29_{-0.05}^{+0.05}$ & $24.50_{-0.86}^{+0.64}$ & $0.04_{-0.19}^{+0.16}$ & $2.20_{-0.55}^{+0.63}$ &$38.93_{-11.58}^{+13.94}$ & $10.23_{-0.02}^{+0.03}$ & $10.02_{-0.05}^{+0.03}$ & $9.82_{-0.08}^{+0.12}$ & $0.18_{-0.08}^{+0.12}$ \\
NGC 4939 & $8.54_{-0.05}^{+0.05}$ & $14.81_{-0.37}^{+0.37}$ & $0.66_{-0.04}^{+0.04}$ & $1.79_{-0.20}^{+0.26}$ &$72.41_{-10.81}^{+12.00}$ & $10.30_{-0.02}^{+0.02}$ & $9.96_{-0.03}^{+0.02}$ & $10.02_{-0.03}^{+0.03}$ & $0.38_{-0.03}^{+0.03}$ \\
NGC 4941 & $7.38_{-0.06}^{+0.06}$ & $14.30_{-0.36}^{+0.40}$ & $0.39_{-0.05}^{+0.06}$ & $1.50_{-0.24}^{+0.29}$ &$65.13_{-9.78}^{+12.74}$ & $9.11_{-0.02}^{+0.02}$ & $8.71_{-0.02}^{+0.02}$ & $8.88_{-0.03}^{+0.03}$ & $0.46_{-0.03}^{+0.03}$ \\
NGC 4992 & $7.77_{-0.13}^{+0.12}$ & $17.81_{-0.99}^{+1.27}$ & $-0.38_{-0.12}^{+0.11}$ & $1.30_{-0.31}^{+0.40}$ &$54.30_{-9.12}^{+12.43}$ & $10.19_{-0.03}^{+0.03}$ & $9.67_{-0.04}^{+0.06}$ & $10.03_{-0.05}^{+0.04}$ & $0.59_{-0.06}^{+0.05}$ \\
NGC 5033 & $8.12_{-0.12}^{+0.06}$ & $18.35_{-0.75}^{+1.48}$ & $1.50_{-0.26}^{+0.09}$ & $2.35_{-0.56}^{+0.35}$ &$69.51_{-11.00}^{+11.08}$ & $10.34_{-0.02}^{+0.01}$ & $10.10_{-0.05}^{+0.11}$ & $9.96_{-0.19}^{+0.05}$ & $<0.34$ \\
NGC 5106 & $8.13_{-0.03}^{+0.03}$ & $26.05_{-0.36}^{+0.36}$ & $-0.38_{-0.16}^{+0.18}$ & $1.26_{-0.43}^{+0.49}$ &$46.13_{-16.11}^{+20.30}$ & $11.10_{-0.01}^{+0.02}$ & $11.03_{-0.02}^{+0.02}$ & $10.30_{-0.05}^{+0.11}$ & $<-0.03$ \\
NGC 513 & $7.59_{-0.03}^{+0.03}$ & $26.79_{-0.40}^{+0.36}$ & $-0.37_{-0.15}^{+0.17}$ & $1.14_{-0.43}^{+0.54}$ &$43.52_{-15.40}^{+19.78}$ & $10.65_{-0.01}^{+0.02}$ & $10.56_{-0.02}^{+0.02}$ & $9.92_{-0.06}^{+0.10}$ & $<0.02$ \\
NGC 5231 & $7.58_{-0.04}^{+0.04}$ & $23.04_{-0.56}^{+0.43}$ & $-0.57_{-0.18}^{+0.19}$ & $1.35_{-0.40}^{+0.47}$ &$48.24_{-16.07}^{+17.48}$ & $10.29_{-0.01}^{+0.02}$ & $10.16_{-0.03}^{+0.02}$ & $9.72_{-0.07}^{+0.11}$ & $<0.18$ \\
NGC 5252 & $7.12_{-0.13}^{+0.18}$ & $23.78_{-2.42}^{+1.40}$ & $-0.47_{-0.21}^{+0.20}$ & $0.91_{-0.41}^{+0.41}$ &$55.65_{-18.53}^{+17.86}$ & $10.21_{-0.03}^{+0.03}$ & $9.77_{-0.11}^{+0.05}$ & $10.02_{-0.06}^{+0.09}$ & $0.52_{-0.07}^{+0.13}$ \\
NGC 526A & $6.95_{-0.23}^{+0.22}$ & $20.54_{-2.33}^{+2.72}$ & $-0.08_{-0.16}^{+0.12}$ & $1.34_{-0.45}^{+0.51}$ &$42.81_{-7.56}^{+10.24}$ & $10.17_{-0.04}^{+0.05}$ & $9.22_{-0.11}^{+0.11}$ & $10.12_{-0.04}^{+0.06}$ & $0.85_{-0.05}^{+0.04}$ \\
NGC 5273 & $5.46_{-0.06}^{+0.07}$ & $27.20_{-1.14}^{+0.80}$ & $-0.57_{-0.20}^{+0.26}$ & $1.32_{-0.43}^{+0.47}$ &$53.76_{-18.47}^{+17.19}$ & $8.65_{-0.01}^{+0.04}$ & $8.47_{-0.06}^{+0.03}$ & $8.19_{-0.05}^{+0.15}$ & $0.13_{-0.05}^{+0.15}$ \\
NGC 5290 & $7.54_{-0.03}^{+0.03}$ & $20.49_{-0.40}^{+0.38}$ & $-0.47_{-0.22}^{+0.18}$ & $0.40_{-0.40}^{+0.46}$ &$50.21_{-17.31}^{+17.56}$ & $9.91_{-0.03}^{+0.02}$ & $9.81_{-0.03}^{+0.03}$ & $9.24_{-0.06}^{+0.04}$ & $<0.00$ \\
NGC 5506 & $6.73_{-0.06}^{+0.06}$ & $25.56_{-1.15}^{+1.12}$ & $1.17_{-0.11}^{+0.09}$ & $1.83_{-0.38}^{+0.44}$ &$49.21_{-7.46}^{+13.16}$ & $10.16_{-0.04}^{+0.03}$ & $9.57_{-0.07}^{+0.06}$ & $10.03_{-0.06}^{+0.04}$ & $0.65_{-0.06}^{+0.05}$ \\
NGC 5548 & $7.28_{-0.08}^{+0.08}$ & $23.90_{-1.28}^{+1.29}$ & $0.29_{-0.14}^{+0.11}$ & $1.68_{-0.40}^{+0.44}$ &$46.89_{-8.92}^{+12.24}$ & $10.44_{-0.03}^{+0.04}$ & $9.95_{-0.07}^{+0.07}$ & $10.28_{-0.07}^{+0.06}$ & $0.58_{-0.09}^{+0.08}$ \\
NGC 5610 & $7.87_{-0.07}^{+0.07}$ & $23.87_{-1.38}^{+1.26}$ & $0.83_{-0.18}^{+0.13}$ & $2.40_{-0.28}^{+0.26}$ &$61.17_{-10.13}^{+10.52}$ & $10.83_{-0.02}^{+0.02}$ & $10.53_{-0.08}^{+0.07}$ & $10.53_{-0.10}^{+0.09}$ & $0.33_{-0.13}^{+0.13}$ \\
NGC 5674 & $8.09_{-0.03}^{+0.03}$ & $23.31_{-0.32}^{+0.30}$ & $-0.65_{-0.20}^{+0.19}$ & $0.44_{-0.39}^{+0.53}$ &$48.19_{-18.90}^{+19.69}$ & $10.80_{-0.02}^{+0.01}$ & $10.69_{-0.02}^{+0.01}$ & $10.14_{-0.06}^{+0.04}$ & $<0.01$ \\
NGC 5683 & $<6.85$ & ... & $-0.59_{-0.18}^{+0.09}$ & $1.65_{-0.32}^{+0.40}$ &$55.16_{-11.46}^{+13.73}$ & $<10.17$ & $<9.64$ & $10.01_{-0.09}^{+0.04}$ & $>0.61$ \\
NGC 5728 & $7.33_{-0.06}^{+0.07}$ & $23.55_{-1.31}^{+1.05}$ & $0.91_{-0.31}^{+0.21}$ & $2.03_{-0.45}^{+0.39}$ &$63.18_{-11.32}^{+11.56}$ & $10.16_{-0.02}^{+0.02}$ & $9.96_{-0.07}^{+0.06}$ & $9.74_{-0.15}^{+0.12}$ & $<0.40$ \\
NGC 5899 & $7.99_{-0.04}^{+0.05}$ & $21.40_{-0.71}^{+0.45}$ & $0.29_{-0.25}^{+0.33}$ & $1.23_{-0.44}^{+0.47}$ &$63.82_{-17.47}^{+14.92}$ & $10.50_{-0.01}^{+0.02}$ & $10.38_{-0.04}^{+0.02}$ & $9.89_{-0.05}^{+0.15}$ & $<0.19$ \\
NGC 5995 & $8.06_{-0.04}^{+0.05}$ & $25.26_{-0.96}^{+0.83}$ & $0.54_{-0.19}^{+0.14}$ & $1.56_{-0.34}^{+0.45}$ &$50.36_{-12.61}^{+16.36}$ & $11.18_{-0.03}^{+0.02}$ & $10.88_{-0.05}^{+0.05}$ & $10.88_{-0.10}^{+0.07}$ & $0.33_{-0.10}^{+0.09}$ \\
NGC 6221 & $7.64_{-0.03}^{+0.04}$ & $24.65_{-0.78}^{+0.60}$ & $1.48_{-0.24}^{+0.22}$ & $1.73_{-0.38}^{+0.40}$ &$54.54_{-15.37}^{+15.83}$ & $10.55_{-0.01}^{+0.03}$ & $10.39_{-0.06}^{+0.04}$ & $10.05_{-0.10}^{+0.13}$ & $<0.30$ \\
NGC 6240 & $8.27_{-0.04}^{+0.11}$ & $30.66_{-3.88}^{+1.07}$ & $1.21_{-0.30}^{+0.45}$ & $2.81_{-0.44}^{+0.87}$ &$40.29_{-12.95}^{+34.89}$ & $11.78_{-0.03}^{+0.02}$ & $11.59_{-0.24}^{+0.05}$ & $11.34_{-0.18}^{+0.24}$ & $<0.68$ \\
NGC 6300 & $7.57_{-0.05}^{+0.05}$ & $20.22_{-0.77}^{+0.76}$ & $1.36_{-0.18}^{+0.12}$ & $1.67_{-0.32}^{+0.30}$ &$65.46_{-10.31}^{+12.61}$ & $10.07_{-0.02}^{+0.01}$ & $9.81_{-0.05}^{+0.04}$ & $9.73_{-0.08}^{+0.05}$ & $0.27_{-0.09}^{+0.08}$ \\
NGC 6552 & $7.44_{-0.05}^{+0.07}$ & $29.39_{-1.74}^{+1.15}$ & $0.52_{-0.19}^{+0.14}$ & $3.25_{-0.67}^{+0.60}$ &$31.86_{-6.91}^{+7.76}$ & $11.01_{-0.05}^{+0.04}$ & $10.65_{-0.10}^{+0.06}$ & $10.77_{-0.11}^{+0.11}$ & $0.43_{-0.12}^{+0.15}$ \\
NGC 6814 & $7.63_{-0.04}^{+0.05}$ & $21.08_{-0.55}^{+0.37}$ & $0.35_{-0.21}^{+0.30}$ & $1.03_{-0.41}^{+0.49}$ &$58.78_{-18.83}^{+16.89}$ & $10.09_{-0.01}^{+0.02}$ & $9.98_{-0.03}^{+0.02}$ & $9.45_{-0.05}^{+0.13}$ & $<0.14$ \\
NGC 6860 & $7.55_{-0.05}^{+0.06}$ & $22.75_{-0.84}^{+0.63}$ & $0.01_{-0.17}^{+0.17}$ & $0.99_{-0.36}^{+0.40}$ &$52.98_{-16.16}^{+16.77}$ & $10.39_{-0.02}^{+0.03}$ & $10.09_{-0.05}^{+0.03}$ & $10.09_{-0.05}^{+0.08}$ & $0.33_{-0.05}^{+0.09}$ \\
NGC 7172 & $7.49_{-0.02}^{+0.02}$ & $24.42_{-0.30}^{+0.29}$ & $0.11_{-0.17}^{+0.17}$ & $1.05_{-0.41}^{+0.48}$ &$47.34_{-15.87}^{+17.33}$ & $10.31_{-0.01}^{+0.02}$ & $10.21_{-0.02}^{+0.02}$ & $9.63_{-0.04}^{+0.08}$ & $<0.03$ \\
NGC 7213 & $6.92_{-0.07}^{+0.07}$ & $20.26_{-0.82}^{+0.86}$ & $0.54_{-0.24}^{+0.14}$ & $1.24_{-0.38}^{+0.34}$ &$68.36_{-12.12}^{+13.52}$ & $9.46_{-0.02}^{+0.01}$ & $9.15_{-0.05}^{+0.05}$ & $9.16_{-0.08}^{+0.05}$ & $0.33_{-0.09}^{+0.07}$ \\
NGC 7465 & $6.73_{-0.04}^{+0.04}$ & $26.53_{-0.56}^{+0.52}$ & $-0.08_{-0.19}^{+0.23}$ & $1.24_{-0.42}^{+0.47}$ &$52.28_{-17.34}^{+17.45}$ & $9.78_{-0.01}^{+0.03}$ & $9.68_{-0.02}^{+0.02}$ & $9.13_{-0.05}^{+0.13}$ & $<0.10$ \\
NGC 7469 & $8.29_{-0.07}^{+0.08}$ & $26.77_{-1.62}^{+1.53}$ & $1.53_{-0.20}^{+0.13}$ & $2.69_{-0.41}^{+0.51}$ &$44.98_{-9.22}^{+14.81}$ & $11.56_{-0.03}^{+0.02}$ & $11.25_{-0.09}^{+0.08}$ & $11.27_{-0.12}^{+0.09}$ & $0.35_{-0.14}^{+0.13}$ \\
NGC 7479 & $8.14_{-0.06}^{+0.06}$ & $19.62_{-0.69}^{+0.73}$ & $1.44_{-0.06}^{+0.05}$ & $2.20_{-0.27}^{+0.38}$ &$57.72_{-8.94}^{+11.36}$ & $10.69_{-0.02}^{+0.02}$ & $10.29_{-0.04}^{+0.04}$ & $10.47_{-0.05}^{+0.03}$ & $0.47_{-0.06}^{+0.05}$ \\
NGC 7582 & $7.57_{-0.03}^{+0.04}$ & $27.05_{-0.93}^{+0.53}$ & $1.44_{-0.23}^{+0.25}$ & $2.09_{-0.43}^{+0.59}$ &$43.57_{-14.94}^{+20.54}$ & $10.70_{-0.02}^{+0.03}$ & $10.57_{-0.06}^{+0.03}$ & $10.14_{-0.10}^{+0.16}$ & $<0.29$ \\
NGC 7603 & $7.98_{-0.03}^{+0.04}$ & $24.02_{-0.54}^{+0.40}$ & $-0.20_{-0.22}^{+0.17}$ & $0.46_{-0.40}^{+0.45}$ &$49.75_{-18.14}^{+19.20}$ & $10.99_{-0.03}^{+0.02}$ & $10.66_{-0.03}^{+0.02}$ & $10.72_{-0.07}^{+0.04}$ & $0.38_{-0.06}^{+0.05}$ \\
NGC 7679 & $7.68_{-0.03}^{+0.03}$ & $29.20_{-0.59}^{+0.46}$ & $0.29_{-0.17}^{+0.21}$ & $1.51_{-0.42}^{+0.52}$ &$45.13_{-15.06}^{+23.12}$ & $10.99_{-0.01}^{+0.03}$ & $10.87_{-0.03}^{+0.02}$ & $10.34_{-0.07}^{+0.13}$ & $<0.15$ \\
NGC 788 & $7.63_{-0.07}^{+0.07}$ & $14.94_{-0.43}^{+0.46}$ & $0.24_{-0.07}^{+0.08}$ & $1.70_{-0.30}^{+0.38}$ &$52.66_{-7.20}^{+8.73}$ & $10.03_{-0.03}^{+0.04}$ & $9.07_{-0.02}^{+0.02}$ & $9.98_{-0.03}^{+0.04}$ & $0.85_{-0.01}^{+0.01}$ \\
NGC 931 & $8.28_{-0.06}^{+0.06}$ & $18.59_{-0.65}^{+0.67}$ & $0.72_{-0.07}^{+0.06}$ & $1.48_{-0.23}^{+0.32}$ &$62.94_{-9.99}^{+13.26}$ & $10.81_{-0.03}^{+0.02}$ & $10.30_{-0.04}^{+0.03}$ & $10.66_{-0.05}^{+0.03}$ & $0.60_{-0.05}^{+0.04}$ \\
NGC 985 & $8.32_{-0.07}^{+0.08}$ & $21.73_{-1.15}^{+1.08}$ & $0.40_{-0.08}^{+0.07}$ & $1.66_{-0.28}^{+0.39}$ &$56.13_{-10.04}^{+14.54}$ & $11.30_{-0.04}^{+0.02}$ & $10.74_{-0.06}^{+0.05}$ & $11.15_{-0.06}^{+0.03}$ & $0.63_{-0.06}^{+0.05}$ \\
PG 2304+042 & $<5.90$ & ... & $-1.16_{-0.15}^{+0.12}$ & $1.35_{-0.46}^{+0.65}$ &$37.94_{-12.32}^{+12.57}$ & $<9.80$ & $<8.69$ & $9.77_{-0.06}^{+0.08}$ & $>0.90$ \\
PICTOR A & $9.78_{-0.10}^{+0.09}$ & $8.39_{-0.32}^{+0.38}$ & $-0.52_{-0.05}^{+0.06}$ & $0.69_{-0.10}^{+0.13}$ &$130.08_{-17.31}^{+14.43}$ & $10.42_{-0.03}^{+0.04}$ & $9.72_{-0.02}^{+0.02}$ & $10.32_{-0.04}^{+0.04}$ & $0.73_{-0.02}^{+0.02}$ \\
PKS 2331-240 & $9.95_{-0.07}^{+0.04}$ & $8.34_{-0.16}^{+0.27}$ & $-0.32_{-0.06}^{+0.05}$ & $1.23_{-0.11}^{+0.12}$ &$137.74_{-10.52}^{+9.34}$ & $10.54_{-0.03}^{+0.03}$ & $9.87_{-0.02}^{+0.02}$ & $10.43_{-0.04}^{+0.03}$ & $0.72_{-0.02}^{+0.02}$ \\
SBS 0915+556 & $<6.51$ & ... & $-0.47_{-0.10}^{+0.09}$ & $1.66_{-0.35}^{+0.43}$ &$49.65_{-7.65}^{+11.32}$ & $<10.47$ & $<9.29$ & $10.44_{-0.04}^{+0.04}$ & $>0.91$ \\
SBS 1301+540 & $7.64_{-0.20}^{+0.22}$ & $13.85_{-1.51}^{+1.53}$ & $-1.32_{-0.19}^{+0.17}$ & $0.94_{-0.42}^{+0.45}$ &$48.39_{-17.33}^{+18.85}$ & $9.52_{-0.05}^{+0.07}$ & $8.89_{-0.10}^{+0.09}$ & $9.41_{-0.06}^{+0.09}$ & $0.69_{-0.06}^{+0.07}$ \\
SDSS J104326.47+110524.2 & $<6.87$ & ... & $-0.81_{-0.15}^{+0.09}$ & $1.81_{-0.34}^{+0.44}$ &$52.74_{-10.56}^{+13.60}$ & $<10.16$ & $<9.65$ & $10.00_{-0.09}^{+0.04}$ & $>0.59$ \\
SWIFT J212745.6+565636 & $<5.37$ & ... & $0.07_{-0.19}^{+0.13}$ & $1.16_{-0.47}^{+0.46}$ &$42.27_{-4.92}^{+8.84}$ & $<10.11$ & $<8.16$ & $10.11_{-0.05}^{+0.04}$ & $>0.99$ \\
UGC 01479 & $7.50_{-0.06}^{+0.07}$ & $23.90_{-1.26}^{+0.91}$ & $0.08_{-0.31}^{+0.23}$ & $2.09_{-0.39}^{+0.37}$ &$57.96_{-12.84}^{+13.43}$ & $10.33_{-0.02}^{+0.02}$ & $10.17_{-0.07}^{+0.05}$ & $9.84_{-0.15}^{+0.13}$ & $<0.34$ \\
UGC 03142 & $7.60_{-0.05}^{+0.05}$ & $24.61_{-0.49}^{+0.48}$ & $-0.43_{-0.16}^{+0.13}$ & $1.04_{-0.41}^{+0.54}$ &$43.40_{-15.44}^{+19.02}$ & $10.50_{-0.02}^{+0.02}$ & $10.35_{-0.03}^{+0.02}$ & $9.98_{-0.07}^{+0.07}$ & $<0.17$ \\
UGC 03478 & $7.66_{-0.05}^{+0.06}$ & $21.13_{-0.76}^{+0.70}$ & $0.09_{-0.21}^{+0.16}$ & $1.46_{-0.33}^{+0.34}$ &$59.57_{-13.42}^{+16.50}$ & $10.22_{-0.02}^{+0.02}$ & $10.00_{-0.05}^{+0.04}$ & $9.82_{-0.09}^{+0.07}$ & $0.19_{-0.09}^{+0.08}$ \\
UGC 03601 & $6.95_{-0.05}^{+0.06}$ & $23.71_{-1.08}^{+0.66}$ & $-0.56_{-0.18}^{+0.19}$ & $1.70_{-0.46}^{+0.60}$ &$43.26_{-14.95}^{+16.80}$ & $9.83_{-0.02}^{+0.04}$ & $9.60_{-0.06}^{+0.03}$ & $9.44_{-0.06}^{+0.12}$ & $0.21_{-0.07}^{+0.13}$ \\
UGC 03995A & $7.33_{-0.18}^{+0.24}$ & $23.79_{-2.80}^{+2.25}$ & $-0.20_{-0.17}^{+0.18}$ & $1.06_{-0.42}^{+0.52}$ &$48.55_{-17.97}^{+20.23}$ & $10.25_{-0.04}^{+0.06}$ & $9.99_{-0.11}^{+0.08}$ & $9.91_{-0.05}^{+0.10}$ & $0.27_{-0.07}^{+0.14}$ \\
UGC 05881 & $7.48_{-0.08}^{+0.11}$ & $24.81_{-2.05}^{+1.61}$ & $0.34_{-0.25}^{+0.19}$ & $2.30_{-0.30}^{+0.28}$ &$60.28_{-12.72}^{+12.52}$ & $10.54_{-0.02}^{+0.02}$ & $10.25_{-0.12}^{+0.09}$ & $10.24_{-0.13}^{+0.13}$ & $<0.63$ \\
UGC 06728 & $<5.04$ & ... & $-0.50_{-0.10}^{+0.10}$ & $1.23_{-0.27}^{+0.39}$ &$58.20_{-9.75}^{+12.65}$ & $<8.79$ & $<7.82$ & $8.74_{-0.04}^{+0.04}$ & $>0.86$ \\
UGC 07064 & $7.80_{-0.04}^{+0.05}$ & $25.05_{-0.76}^{+0.51}$ & $-0.00_{-0.15}^{+0.18}$ & $1.93_{-0.50}^{+0.61}$ &$38.27_{-11.43}^{+17.46}$ & $10.77_{-0.02}^{+0.03}$ & $10.59_{-0.04}^{+0.02}$ & $10.31_{-0.07}^{+0.12}$ & $0.12_{-0.06}^{+0.11}$ \\
UGC 08327NED02 & $7.14_{-0.13}^{+0.14}$ & $33.02_{-4.20}^{+2.30}$ & $0.17_{-0.22}^{+0.27}$ & $2.05_{-0.46}^{+0.71}$ &$41.76_{-14.74}^{+32.70}$ & $11.01_{-0.03}^{+0.04}$ & $10.65_{-0.28}^{+0.08}$ & $10.76_{-0.12}^{+0.16}$ & $0.42_{-0.15}^{+0.28}$ \\
UGC 10593 & $7.57_{-0.08}^{+0.11}$ & $23.19_{-1.70}^{+1.16}$ & $-0.31_{-0.23}^{+0.20}$ & $1.68_{-0.36}^{+0.41}$ &$53.94_{-14.87}^{+16.49}$ & $10.44_{-0.02}^{+0.03}$ & $10.16_{-0.09}^{+0.05}$ & $10.10_{-0.10}^{+0.12}$ & $0.29_{-0.11}^{+0.15}$ \\
UGC 11185NED02 & $7.31_{-0.13}^{+0.22}$ & $26.60_{-3.76}^{+2.50}$ & $-0.27_{-0.30}^{+0.22}$ & $2.31_{-0.35}^{+0.51}$ &$47.79_{-14.83}^{+21.50}$ & $10.60_{-0.04}^{+0.03}$ & $10.26_{-0.16}^{+0.11}$ & $10.33_{-0.19}^{+0.12}$ & $0.39_{-0.23}^{+0.20}$ \\
UGC 12237 & $8.14_{-0.11}^{+0.11}$ & $17.77_{-0.96}^{+1.08}$ & $0.16_{-0.05}^{+0.04}$ & $2.48_{-0.28}^{+0.36}$ &$56.89_{-8.71}^{+11.98}$ & $10.50_{-0.03}^{+0.02}$ & $10.04_{-0.05}^{+0.05}$ & $10.32_{-0.04}^{+0.03}$ & $0.54_{-0.05}^{+0.05}$ \\
UGC 12282 & $7.96_{-0.06}^{+0.09}$ & $20.58_{-1.12}^{+0.64}$ & $-0.21_{-0.28}^{+0.34}$ & $1.25_{-0.47}^{+0.47}$ &$64.59_{-18.12}^{+15.07}$ & $10.39_{-0.01}^{+0.02}$ & $10.24_{-0.06}^{+0.02}$ & $9.83_{-0.07}^{+0.15}$ & $<0.26$ \\
UGC 12741 & $7.21_{-0.04}^{+0.05}$ & $22.86_{-0.74}^{+0.50}$ & $-0.72_{-0.20}^{+0.26}$ & $1.59_{-0.41}^{+0.47}$ &$51.42_{-18.49}^{+18.01}$ & $9.89_{-0.01}^{+0.03}$ & $9.77_{-0.05}^{+0.02}$ & $9.28_{-0.06}^{+0.17}$ & $<0.21$ \\
UM614 & $<6.57$ & ... & $-0.79_{-0.12}^{+0.12}$ & $2.01_{-0.62}^{+0.79}$ &$28.36_{-8.58}^{+8.11}$ & $<9.95$ & $<9.36$ & $9.83_{-0.03}^{+0.10}$ & $>0.66$ \\
VII Zw073 & $7.69_{-0.04}^{+0.08}$ & $30.13_{-2.33}^{+1.01}$ & $0.22_{-0.27}^{+0.22}$ & $2.78_{-0.73}^{+0.85}$ &$35.01_{-11.81}^{+30.58}$ & $11.22_{-0.04}^{+0.02}$ & $10.96_{-0.15}^{+0.05}$ & $10.87_{-0.13}^{+0.14}$ & $0.26_{-0.13}^{+0.20}$ \\
WKK 1263 & $7.07_{-0.04}^{+0.06}$ & $27.60_{-1.33}^{+0.66}$ & $-0.19_{-0.15}^{+0.16}$ & $1.84_{-0.53}^{+0.64}$ &$35.80_{-10.96}^{+18.10}$ & $10.43_{-0.03}^{+0.04}$ & $10.11_{-0.08}^{+0.03}$ & $10.13_{-0.08}^{+0.11}$ & $0.35_{-0.07}^{+0.14}$ \\
WKK 4374 & $<6.73$ & ... & $-0.20_{-0.25}^{+0.11}$ & $1.57_{-0.31}^{+0.39}$ &$53.98_{-14.69}^{+15.31}$ & $<10.00$ & $<9.51$ & $9.82_{-0.14}^{+0.04}$ & $>0.56$ \\
WKK 4438 & $7.22_{-0.12}^{+0.15}$ & $23.79_{-2.02}^{+1.26}$ & $0.04_{-0.17}^{+0.16}$ & $1.84_{-0.45}^{+0.63}$ &$40.64_{-12.89}^{+15.11}$ & $10.22_{-0.03}^{+0.04}$ & $9.88_{-0.09}^{+0.04}$ & $9.96_{-0.07}^{+0.11}$ & $0.40_{-0.08}^{+0.14}$ \\
WKK 6092 & $<5.43$ & ... & $-0.21_{-0.08}^{+0.08}$ & $1.61_{-0.30}^{+0.39}$ &$56.81_{-9.14}^{+12.46}$ & $<9.68$ & $<8.21$ & $9.66_{-0.04}^{+0.04}$ & $>0.95$ \\
WKK 6471 & $7.79_{-0.09}^{+0.11}$ & $19.40_{-0.76}^{+0.60}$ & $-1.09_{-0.19}^{+0.14}$ & $0.75_{-0.42}^{+0.51}$ &$43.29_{-15.07}^{+17.57}$ & $10.19_{-0.04}^{+0.02}$ & $9.92_{-0.05}^{+0.03}$ & $9.86_{-0.07}^{+0.05}$ & $0.29_{-0.06}^{+0.06}$ \\
\end{longtable}
\end{center}
}
\end{landscape}
\end{document}
%%%%%%%%%%%%%%%%%%%%%% TABLE OF DECOMPIR PARAMETERS %%%%%%%%%%%%%%%%%%%%%%%%%%
%\begin{table*}
%\centering
%\begin{threeparttable}
%\captionsetup{font=small,labelfont=bf,labelsep=period}
%\caption{Best Fit DecompIR Model Parameters, Luminosities, and AGN Fractions \label{tab:decompir_params}}
%\begin{tabular}{lccccc}
%\toprule 
%Name & Host Galaxy & $\log\,L_{\rm IR}$ & $\log\,L_{\rm SF}$ & $\log\,L_{\rm AGN, IR}$ & $f_{\rm AGN}$ \\
% & Template & [\lsun] & [\lsun] & [\lsun] & \\
%\midrule
%1RXSJ044154.5-082639 & SB2 & $10.39_{-0.02}^{+0.03}$ & $10.05_{-0.04}^{+0.08}$ & $10.12_{-0.08}^{+0.06}$ &$0.54_{-0.09}^{+0.05}$ \\
%1RXSJ045205.0+493248 & SB1 & $10.50_{-0.02}^{+0.02}$ & $10.10_{-0.04}^{+0.03}$ & $10.28_{-0.05}^{+0.05}$ &$0.60_{-0.04}^{+0.04}$ \\
%2E1739.1-1210 & SB1 & $10.89_{-0.03}^{+0.02}$ & $10.43_{-0.05}^{+0.04}$ & $10.71_{-0.06}^{+0.04}$ &$0.65_{-0.04}^{+0.04}$ \\
%2MASSJ07594181-3843560 & SB5 & $<10.24$ & $<8.94$ & $>10.21$ &$>0.95$ \\
%2MASSJ17485512-3254521 & SB1 & $<9.53$ & $<9.25$ & $>9.06$ &$>0.42$ \\
%2MASXJ00253292+6821442 & SB1 & $9.68_{-0.02}^{+0.02}$ & $9.01_{-0.04}^{+0.04}$ & $9.58_{-0.04}^{+0.03}$ &$0.79_{-0.03}^{+0.02}$ \\
%2MASXJ01064523+0638015 & SB3 & $10.48_{-0.03}^{+0.04}$ & $9.69_{-0.34}^{+0.07}$ & $10.40_{-0.04}^{+0.08}$ &$0.84_{-0.04}^{+0.09}$ \\
%2MASXJ01073963-1139117 & SB5 & $10.90_{-0.02}^{+0.02}$ & $10.70_{-0.02}^{+0.02}$ & $10.47_{-0.09}^{+0.05}$ &$0.37_{-0.05}^{+0.03}$ \\
%2MASXJ03305218+0538253 & SB1 & $10.81_{-0.03}^{+0.02}$ & $9.70_{-0.08}^{+0.18}$ & $10.78_{-0.05}^{+0.03}$ &$0.92_{-0.04}^{+0.01}$ \\
%2MASXJ03342453-1513402 & SB5 & $10.65_{-0.02}^{+0.02}$ & $10.53_{-0.01}^{+0.02}$ & $10.04_{-0.11}^{+0.07}$ &$0.24_{-0.05}^{+0.03}$ \\
%\bottomrule
%\end{tabular}
%\begin{tablenotes}
%\item Notes -- \textit{Column 1:} Name of the source. \textit{Column 2:} Best fit host galaxy template \textit{Column 3:} Log of the total infrared luminosity from 8--1000 \micron{} in solar units. \textit{Column 4:} Log of the MBB component luminosity in solar units. \textit{Column 5:} Log of the powerlaw component luminosity in solar units. \textit{Column 6:} Fractional contribution of the AGN to the total infrared luminosity. The full version of this table is available in the online publication.
%\end{tablenotes}
%\end{threeparttable}
%\end{table*}
%%%%%%%%%%%%%%%%%%%%%%%%%%%%%%%%%%%%%%%%%%%%%%%%%%%%%%%%%%%%%%%%%
%
%%%%%%%%%%%%%%%%%%%%%% TABLE OF D14 PARAMETERS %%%%%%%%%%%%%%%%%%%%%%%%%%
%\begin{table*}
%\centering
%\begin{threeparttable}
%\captionsetup{font=small,labelfont=bf,labelsep=period}
%\caption{Best Fit D14 Model Parameters, Luminosities, and AGN Fractions \label{tab:d14_params}}
%\begin{tabular}{lcccccc}
%\toprule 
%Name & $\alpha$ & $f_{\rm AGN, MIR}$ & $\log\,L_{\rm IR}$ & $\log\,L_{\rm SF}$ & $\log\,L_{\rm AGN, IR}$ & $f_{\rm AGN}$ \\
% &  &  & [\lsun] & [\lsun] & [\lsun] & \\
%\midrule
%1RXSJ044154.5-082639 & $1.6250$ & $0.85$ & $10.63_{-0.06}^{+0.02}$ & $10.36_{-0.03}^{+0.02}$ & $10.29_{-0.14}^{+0.04}$ &$0.47_{-0.07}^{+0.02}$ \\
%1RXSJ045205.0+493248 & $2.2500$ & $0.85$ & $10.81_{-0.08}^{+0.01}$ & $10.41_{-0.01}^{+0.03}$ & $10.58_{-0.14}^{+0.02}$ &$0.60_{-0.09}^{+0.01}$ \\
%2E1739.1-1210 & $2.0000$ & $0.85$ & $11.12_{-0.05}^{+0.04}$ & $10.77_{-0.02}^{+0.02}$ & $10.87_{-0.10}^{+0.06}$ &$0.56_{-0.06}^{+0.02}$ \\
%2MASSJ07594181-3843560 & $1.5000$ & $1.00$ & $<10.95$ & $<9.65$ & $>10.93$ &$>0.95$ \\
%2MASSJ17485512-3254521 & $1.0000$ & $0.95$ & $<9.72$ & $<9.44$ & $>9.18$ &$>0.39$ \\
%2MASXJ00253292+6821442 & $1.9375$ & $0.90$ & $9.93_{-0.03}^{+0.02}$ & $9.47_{-0.02}^{+0.02}$ & $9.75_{-0.03}^{+0.03}$ &$0.65_{-0.01}^{+0.01}$ \\
%2MASXJ01064523+0638015 & $1.5000$ & $0.95$ & $10.77_{-0.02}^{+0.03}$ & $10.22_{-0.03}^{+0.03}$ & $10.63_{-0.02}^{+0.04}$ &$0.72_{-0.01}^{+0.02}$ \\
%2MASXJ01073963-1139117 & $1.8750$ & $0.70$ & $11.03_{-0.03}^{+0.05}$ & $10.86_{-0.01}^{+0.03}$ & $10.53_{-0.11}^{+0.11}$ &$0.32_{-0.06}^{+0.06}$ \\
%2MASXJ03305218+0538253 & $1.5625$ & $0.95$ & $11.07_{-0.02}^{+0.03}$ & $10.49_{-0.04}^{+0.03}$ & $10.93_{-0.03}^{+0.04}$ &$0.73_{-0.02}^{+0.02}$ \\
%2MASXJ03342453-1513402 & $1.8125$ & $0.50$ & $10.73_{-0.03}^{+0.04}$ & $10.66_{-0.03}^{+0.02}$ & $9.93_{-0.09}^{+0.19}$ &$0.16_{-0.02}^{+0.06}$ \\
%\bottomrule
%\end{tabular}
%\begin{tablenotes}
%\item Notes -- \textit{Column 1:} Name of the source. \textit{Column 2:} Best fit slope of the intensity distribution for the interstellar radiation field. \textit{Column 3:}  Best fit MIR AGN fraction. \textit{Column 4:} Log of the total infrared luminosity from 8--1000 \micron{} in solar units. \textit{Column 5:} Log of the MBB component luminosity in solar units. \textit{Column 6:} Log of the powerlaw component luminosity in solar units. \textit{Column 7:} Fractional contribution of the AGN to the total infrared luminosity. The full version of this table is available in the online publication.\end{tablenotes}
%\end{threeparttable}
%\end{table*}
%%%%%%%%%%%%%%%%%%%%%%%%%%%%%%%%%%%%%%%%%%%%%%%%%%%%%%%%%%%%%%%%%

%%%%%%%%%%%%%%%%%%%%%%%%%%%%%%%%%%%%%%%%%%%%%%%%%%


% Don't change these lines
\bsp	% typesetting comment
\label{lastpage}
\end{document}